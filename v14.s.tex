% heading und wdh
%
\sS{Satz (Eigenschaften des Logarithmus)}
\begin{enumerate}
\item{$log(1)=0$}
\item{$log(x·y)=log(x)+log(y)$}
\item{$\lim\limits_{x→0}log(x)=-∞$}
\item{$\lim\limits_{x→∞}log(x)=∞$}
\end{enumerate}
\bew
Folgt aus Eigenschaften von $exp$, Details: Übung
\sss{Erinnerung:}
also $a>0,\ n\eZ,\ m\eN$ ist $a^{\frac{n}{m}}:=\sqrt[m]{a^n}$
%Lemma 6.19
%Chritopher
\sS{Definition}
Sei $a>0,\ x\eR$ setze $a^x:=exp(x·log(a))$\\*
\sS{Bemerkung}
Die Regeln der Potenzrechnung gelten:
$$a^{x+y}=a^x·a^y,\qquad a^{x·y}=(a^x)^y$$
\bew
$$a^{x+y}=exp((x+y)·log(a))=exp(x·log(a))·exp(y·log(a))=a^x·a^y$$
$$a^{x·y}=exp(x·y·log(a))=(a^x)^y=exp(y·log(exp(x·log(a))))\footnote{$log\circ exp= id$}=exp(y·x·log(a))$$\qed
\bem
Eulersche Zahl
$$e:=exp(1)=2{,}7…\footnote{$log(e)=1$}\qquad e^x=exp(x·log(e))=exp(x)$$
%Definition und Gleichmäßige Stetigkeit
\sS{Definition:}
Eine Funktion $f:D→\R$ heißt gleichmäßig stetig wenn gilt:\\*
Für jedes $ε>0$ gibt es ein $δ>0$ so dass für alle $x,y\in D$ mit $|x-y|<δ$ gilt:
$$|f(x)-f(y)|<ε$$
\sss{Wesentlich:}
$δ$ hängt nur von $ε$, nicht von $x$ ab.
\bsp
$D=(0,1)\qquad f:D→\R,\ f(x)=\frac{1}{x}$\\*
GRAPH $f$ stetig, aber nicht gleichmäßig stetig
\bew
Wähle $ε=1$. Angenommen es gibt $δ>0$ mit $|x-y|<δ\Rarr |f(x)-f(y)<1$
Wähle: $x=\frac{1}{n},\ y\frac{1}{n+1}$ so dass $\frac{1}{n·(n+1)}<δ$
$$|x-y|=|\frac{1}{n}-\frac{1}{n+1}|=|\frac{n+1-n}{n·(n+1)}|=\frac{1}{n·(n+1)}$$
Dann $$|f(x)-f(y)|=|n-(n+1)|=1$$
Das Zeigt: $δ$ existiert nicht.
%Satz und beweis
%
% christopher
%
\chapter{Komplexe Zahlen und Trigonometrie}
Der Körper \C{} der komplexen Zahlen\\*
Mangel von \R{}: Die Gleichung $x^2=-1$ hat keine Lösung\\*
\sS{Definition}
Es sei $\C=\R^2=\{(x,y)|x,y\eR\}$ mit folgender Addition und Multiplikation:
$$(x,y)+(x',y'):=(x+x',y+y')$$
$$(x,y)·(x',y'):=(x·x'-y·y',x·y'-y·x')$$
Addition gleich der Vektoraddition in \R^2 GRAPH
\sS{Satz}
\C{} ist ein Körper mit Null (0,0) und Eins (1,0)
\bew
Überprüfe Körperaxiome
\ssss{Exemplarisch:}
\begin{enumerate}
\item{Assotiativgesetz der Multiplikation\\*
Gegeben $(x,y),\ (x',y'),\ (x'',y'')\eC$
$$((x,y)·(x',y'))·(x'',y'')=(x·x'-y·y',x·y'-x·y'+y·x')(x'',y'')=(x·x'·x''-y·y'·x''-x·y'·y''-y·x'·y'',x·x'·y''-y·y'·y''+x·y'·x''+y·x'·x'')$$
$$(x,y)((x',y')(x'',y'')=…\text{ erhalte gleiches Ergebnis}$$}
%christopher \item
\end{enumerate}
\sss{Definition:}
$i:=(0,1)$ (imaginäre Einheit)\\*
Dann ist $$i^2=(0,1)·(0,1)=(-1,0) \Rarr i^2+1=0$$
\bem
Für $x,x'\eR$ gilt:
$$(x,0)+(x',0)=(x+x',0)$$
$$(x,0)·(x',0)=(x·x',0)$$
Die Abbildung $\R→\C,\ x\mapsto(x,0)$ ist injektiv und verträglich mit $+,·$\\*
$\leadsto$ Fasse \R{} mittels diese Abbildung als Teilmenge von \C{} auf, einschließlich der Körperstruktur\\
Dann $i^2=-1$.\\*
Für
$$(x,y)\eC$ gilt $(x,y)=(x,0)+(0,y)=(x,0)+(0,1)·(y,0)=x+i·y$$
Jede komplexe Zahl $z$ hat eine eindeutige Darstellung $z=x+i·y$ mit $x,y\eR$.\\
\sss{Idee}
\C{} entsteht aus \R{} durch Hinzunahme einer Zahl $i$ mit $i^2=-1$\\
Interpretation der Multiplikation in \C:\\*
$$(x+i·y)(x'+i·y')=(x·x'+x·i·y'+i·y·x'+i·y·i·y')=(x·x'-y·y')+i(x·y'+y·x')$$
%Definition*
\bem
\begin{enumerate}
\item{$$z·\={z}=(x+i·y)(x-i·y)=x^2+y^2+i(-x·y+y·x)=x^2+y^2=|z|^2$$}
\item{Insbesondere gilt $z·\={z}\eR$ und $z·\={z}\geq 0$}
\item{$|z|=\sqrt{x^2+y^2=}=\sqrt{z·\={z}}$ (sinnvoll wegen 2)}
\end{enumerate}
