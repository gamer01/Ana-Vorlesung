\begin{tikzpicture}[domain=-1.2:1.2, scale=2]
    \draw[very thin,color=gray] (-1.49,-1.49) grid (1.49,1.49);
    \draw[->] (-1.2,0) -- (1.2,0) node[right] {$x$};
    \draw[->] (0,-1.2) -- (0,1.2) node[above] {$i$};
	\draw (0,0) circle (1cm);
	% Sin und Cos einzeichnung fehlt.
\end{tikzpicture}

Potenzreihen von $sin$ und $cos$:
Für $x \in \R$ gilt:
$$cos(x) + i \cdot sin(x) = exp(i \cdot x) = \frac{1}{0!}+\frac{i \cdot x}+{1!}+\frac{(i \cdot x)^2}{1!}+{2!}+\frac{(i \cdot x)^3}{3!}+...$$
$$=(\frac{1}{0!} + \frac{(i \cdot x)^2}{2!} + \frac{(i \cdot x)^4}{4!}) + \frac{(i \cdot x)^6}{6!} + ...) + i (\frac{(i \cdot x)}{1!} + \frac{(i \cdot x)^3}{3!}) + \frac{(i \cdot x)^5}{5!}) + ...)$$
\sS{Satz}
Für $x \in R$ gilt:
$$cos(x) = \ds\sum_{k \geq 0} \frac{(-1)^k}{(2k)!} \cdot x^{2k},\ sin(x) =\sum_{k \geq 0} \frac{(-1)^k}{(2k +1)!} \cdot x^{2k+1} $$
\sS{Bemerkung}
Siehe Übung\\*
$$cos(x)-cos(y)=2sin \ldots$$

% Stefan 

% Graph
\tikz[scale=2,domain=-0.49:2.5, samples=200,prefix=plots/,smooth]{
      \draw[very thin, color=gray!50] (-0.49,-0.25) grid (2.49,2.49);
      \draw[->] (-0.5,0) -- (2.5,0) node[right] {$x$};
      \draw[->] (0,-0.5) -- (0,2.5) node[above] {$y$};
      \clip (-0.49,-0.5) rectangle (2.5,2.5);
      \draw[color=red] plot[id=x] function{x} node[below]{\footnotesize $f_1(x) =x$};
      \draw[color=blue] plot[id=abs,sharp plot] function{sin(x)} node {\footnotesize $f_2(x) = sin(x)$};
      \draw[color=cyan] plot[id=x3d6] function{x - (x**3 / 6)} node [below] {\footnotesize $f_2(x) = x - \frac{x^3}{6}$};
}

\sS*{Lemma} es gilt $cos(2) < 0$ und $cos(1) > 0$
\bew
	Es gilt $cos(2) = \sum (-1)^n \cdot b_n$, $b_n = \frac{2^{2n}}{(2n)!}$. Für $n \geq 1$ gilt: 
	$$\frac{b_{n+1}}{b_n} = \frac{2^2}{(2n+1)(2n+2)} < 1$$
	Also $b_1 > b_2 > b_3 > b_4 >...$ \\*
	\\*
	Somit:\\
	$cos(2) = b_0 - b_1 + b_2 - b_3 + b_4 - ...$\\*
	$= b_0 - b_1 + b_2 \underbrace{-(b_3 + b_4)}_{<0} \underbrace{-(b_5 + b_6)}_{< 0}... < b_0 - b_1 + b_2$ \\*
	$= 1 - 2 + \frac{2}{3} = - \frac{1}{3}$ \qed\\
	Analog $cos(1) > 1-\frac{1}{2}$
	
	%Stefan Lemma 7.29
	
% Graph
 \tikz[scale=2,domain=-0.49:2.5, samples=200,prefix=plots/,smooth]{
      \draw[very thin, color=gray!50] (-0.2,-0.25) grid (2,1.2);
      \draw[->] (-0.5,0) -- (2,0) node[right] {$x$};
      \draw[->] (0,-0.5) -- (0,1.2) node[above] {$y$};
      \draw[color=red] plot[id=x] function{x/1.4} node[below]{\footnotesize $cos$};
      \draw[color=red] plot[id=x] function{1 - x/(1.4)} node[below]{\footnotesize $sin$};
}

%Stefan

Es gilt:\\*
\begin{tabular}{l|c|c|c|c|c|l}
$x$ & $0$ & $\frac{\pi}{2}$ & $\pi$ & $\frac{3 \cdot \pi}{2}$ &\\\hline
$cos(x)$ & $1$ & $0$ & $-1$ & $0$ & $1$ & \\\hline
$sin(x)$ & $0$ & $1$ & $0$ & $-1$ & $0$ & \\\hline
\end{tabular}
\hfill\\*
dazu:
\begin{enumerate}
\item{$sin(x)^2 + cos(x)^2 = 1$\\*
$sin(\frac{\pi}{2})^2 = 1$ also $sin(\frac{\pi}{2} = \pm 1$ aber $sin(\frac{\pi}{2}) > 0$\\*
d.h.:\\*
$e^{i \frac{\pi}{2}} = cos(\frac{\pi}{2}) + i \cdot sin(\frac{\pi}{2}) = i$}
\item{$$e^{i\cdot \pi} = (e^{i\cdot \frac{\pi}{2}})^2 = i^2 = -1 = cos(\pi) + i \cdot sin(\pi)$$}
\item{$$e^{i\frac{3 \cdot \pi}{2}} = e^{i\pi} \cdot e^{i\frac{\pi}{2}} = -1 \cdot i = -i = cos(\frac{3\pi}{2}) + i \cdot sin(\frac{3\pi}{2})$$}
\item{...}
\end{enumerate}

% Stefan

\bew
	Sei $x = cos (\frac{\pi}{3}),\ y = sin(\frac{\pi}{3}), z = x + i\cdot y = e^{i\frac{\pi}{3}}$\\*
	Dann gilt:\\*
	$$z^2 = e^{2 \cdot i\frac{\pi}{3}} = e^{i\pi \cdot - \frac{\pi}{3}} = -1 \cdot e^{i\frac{\pi}{3}} = -\bar{z}$$\\*
	Also $(x + iy)^2 = -x + iy$ d.h. $x^2 - y^2 = -x$, $2xy = y$ und $x^2 + y^2 = 1$\\*
	Auflösen liefert Beh.