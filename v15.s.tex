% Kopfzeile beim Kapitelanfang:
\fancypagestyle{plain}{
%Kopfzeile links bzw. innen
\fancyhead[L]{\calligra\Large Vorlesung Nr. 15}
%Kopfzeile rechts bzw. außen
\fancyhead[R]{\calligra\Large 29.11.2012}
}
%Kopfzeile links bzw. innen
\fancyhead[L]{\calligra\Large Vorlesung Nr. 15}
%Kopfzeile rechts bzw. außen
\fancyhead[R]{\calligra\Large 29.11.2012}
% **************************************************
%
\wdh
$\C=\R^2$ mit 
$$(x,y)+(x',y')=(x+x',y+y')$$
$$(x,y)·(x',y')=(x·x'-y·y',x·y'-y·x')$$
$$(x,y)=x+i·y\qquad i=(0,1)\qquad i^2=(-1)$$
SKIZZE\\
$$z=x+i·y \eC\ \Rarr\ Re(z)=x \quad Im(z)=y$$
$\bar{z}=x-i·y$
$$z·\bar{z}=(x+i·y)(x-i·y)=x^2-(i·y)^2=x^2-i^2·y^2=x^2·y^2$$
Abstand von 0 nach $z$ ist:
$$|z|=\sqrt{x^2+y^2}=\sqrt{z·\bar{z}}\footnote{\ul{Betrag} von $z$}$$
$$|2+i|=\sqrt{2^2+i^2}=\sqrt{5}$$%muss Wurzel 3 rauskommen
%Berechnung
\sS{Lemma}
Sei $z,w\eC$ dann gilt:
\begin{enumerate}
\item{$$\bar{zw}=\bar{z}·\bar{w},\ \bar{z+w}=\bar{z}+\bar{w}$$}
\item{$z+\bar{z}=2 Re(z),\ z-\bar{z}=2i·Im(z)$\\*
$Re(z)=\frac{z+\bar{z}}{z}\qquad Re(z)=\frac{z-\bar{z}}{2i}$}
\end{enumerate}
%
\bew
\begin{enumerate}
\setcounter{enumi}{1}
\item{$$z=x+i·y$$
$$z+\bar{z}=(x+i·y)+(x-i·y)=2x=2Re(z)$$
$$z-\bar{z}=(x+i·y)-(x-i·y)=2iy=2i·Im(z)$$}
\end{enumerate}
%\Bew{1}
\sS{Satz}
Sei $z,w\eC$
\begin{enumerate}
\item{$|z|\geq 0$, und $|z|=0$ \equiv{} $z=0$ (klar)}
\item{$|z·w|=|z|·|w|,\ |\bar{z}|=|z|$}
\item{$|z+w|\leq|z|+|w|$ (Dreiecksungleichung)
SKIZZE}
\end{enumerate}
\bew
\begin{enumerate}
\setcounter{enumi}{1}
\item{$$|z·w|=\sqrt{zw·\bar{zw}}\underset{7.3}{=}\sqrt{z·w·\bar{z}·\bar{w}}=\sqrt{z·\bar{z}·w·\bar{w}}=\sqrt{z·\bar{z}}\sqrt{w·\bar{w}}=|z|·|w|$$
beide reell $\geq 0$
Nachtrag rechts unten 3. tafel}
\item{$$|z+w|=(z+w)·(\bar{z+w})\underset{7.3}{=}(z+w)·(\bar{z}+\bar{w})=z·\bar{z}+z·\bar{w}+w·\bar{z}+w·\bar{w}=\footnote{$\bar{z·\bar{w}}=\bar{z}·\bar{\bar{w}}=\bar{z}·w=w·\bar{z}$}z·\bar{z}+\underbrace{z·\bar{w}+\bar{z·\bar{w}}}_{2·Re(z·\bar{w})}+w·\bar{w}$$
$$=z·\bar{z}+2·Re(z·\bar{w})+w·\bar{w}\leq |z|^2+2·|z|·|w|+|w|^2=(|z|+|w|)^2\footnote{Für jedes $u=c+d·i\eC$ gilt $|u|=\sqrt{c^2+d^2}\geq \sqrt{c^2}=|c|=|Re(u)|$}$$
$$\sqrt{}\ \Rarr\ |z+w|=|z|+|w|$$\qed}
%BEMERKUNG}

\end{enumerate}
%Definition

\sS{Satz:}
Sei $c_n=a_n+i*b_n,\quad c=a+ib$\\*
Es gilt $c_n→c\ \equ \ a_n→a$ und $b_n→b$
\bew
\begin{description}
\item["\Rarr"]{Es gilt:
$$|a_n-a|=|Re(c_n-c)|\leq |c_n-c|$$
$$|b_n-b|=|Im(c_n-c)|\leq |c_n-c|$$
also gilt: $|c_n-c|<ε\ \Rarr \ |a_n-a|<ε$ und  $|b_n-b|<ε$ Somit gilt die "\Rarr"}
%\Rückrichtung LARR
\end{description}
\sS{Definition}
Eine Folge komplexer Zahlen $(c_n)_{n\geq 0}$ heißt Cauchy-Folge, wenn für jedes $ε>0$ ein $N\eN$ existiert, so dass gilt:\\*
Für alle $n,m\geq N$ gilt $|c_n-c_m|<ε$
\sS{Satz}
Sei $c_n=a_n+ib_n$\\*
$(c_n)$ ist Cauchy-Folge \equ{} $(a_n)$ und $(b_n)$ sind Cauchy-Folgen
\bew
Genau wie Beweis von 7.6 verwende:
$$|a_n-a_m|\leq |c_n-c_m|$$
$$|b_n-b_m|\leq |c_n-c_m|$$
$$|c_n-c_m|\leq |a_n-a_m|+|b_n-b_m|$$\qed
%Satz 7.9
\sS{Satz}
Wenn $c_n→c, c'_n→c'$ konvergente Folgen komplexer Zahlen sind, dann gilt:
\begin{enumerate}
\item{$c_n+c_m→c+c'$}
\item{$c_n·c_m→c·c'$}
\item{Wenn $c\neq 0$ dann $c_n\neq 0$ für fast alle $n$ und $\frac{1}{c_n}→\frac{1}{c}$}
\end{enumerate}
\bew
Analog zum Fall reeller Folgen\qed
\sS{Definition}
Eine Reihe komplexer Zahlen\\*
$\ds\sum_{n=0}^{∞}c_n$ heißt \ul{absolut konvergent, wenn die Reihe $\ds\sum_{n=0}^{∞}|c_n|$ konvergent ist}
%Satz 7.12
\sS{Satz}
Sei $c_n\eC$ für \nN\\*
\begin{enumerate}
\item{Majorantenkriterium:\\*
Wenn reelle Zahlen $a_n$ existieren, so dass $|c_n|\leq |a_n|$ und $\sum a_n$ konvergiert, dann konvergiert auch $\sum c_n$ absolut}
\item{Quotientenkriterium:\\*
Wenn eine reelle zahl $b\eR$ existiert mit $0\leq b<1$, so dass $|c_{n+1}\leq b·|c_n|$ für fast alle \nN\\*
Dann konvergiert $\ds\sum_{n\geq 0} c_n$ absolut}
\end{enumerate}
\sS{Satz}
Seien $\sum c_n$ und $\sum d_n$ zwei konvergente Reihen komplexer Zahlen, $\sum c_n=c,\ \sum d_n=d$\\*
Wenn eine der Reihen absolut konvergiert, konvergiert auch das Cauchy-Produkt mit Grenzwert $c·d$