% Kopfzeile beim Kapitelanfang:
\fancypagestyle{plain}{
%Kopfzeile links bzw. innen
\fancyhead[L]{\calligra\Large Vorlesung Nr. 13}
%Kopfzeile rechts bzw. außen
\fancyhead[R]{\calligra\Large 22.11.2012}
}
%Kopfzeile links bzw. innen
\fancyhead[L]{\calligra\Large Vorlesung Nr. 13}
%Kopfzeile rechts bzw. außen
\fancyhead[R]{\calligra\Large 22.11.2012}
% **************************************************
%
\wdh
Zwischenwertsatz\\*
Sei $a\leq b,\ f:[a,b]→\R$ stetig\\*
Sei $y\eR$ zwischen $f(a)$ und $f(b)$ d.h. $f(a)\leq y\leq f(b)$ oder $f(a)\geq y\geq f(b)$\\*
Dann gibt es ein $x\in[a,b]$ mit $f(x)=y$ SKIZZE
\Bew {Intervallschachtelung}
Starte mit $[a_0,b_0]=[a,b]$\\*
Definiere unendliche Kette von Intervallen\\*
$[a_0,b_0]\subseteqq [a_1,b_1]\subseteqq [a_2,b_2]\subseteqq …$\\*
So dass $[b_n-a_n]=2^{-n}[b_0,a_0]$ und $y$ zwischen $f(a_n)$ und $f(b_n)$ liegt.\\*
Annahme: $f(a)\leq y\leq f(b)$ (Anderer Fall $f(a)\geq y\geq f(b)$ analog)\\*
Angenommen, $[a_n,b_n]$ ist konstruiert so dass $[_n-a_n=2^{-n}(b_0,a_0)$ und $f(a_n)\leq y\leq f(b_n)$\\*
Betrachte $m:=\frac{a_n+b_n}{2}$, Wenn $f(m)\geq y$ dann setze $[a_{n+1},b_{n+1}]:=[a_n,m]$\\*
Wenn $f(m)<y$ dann setze $[a_{n+1},b_{n+1}]:=[m,b_n]$\\*
Dann gilt in beiden Fällen:
$$b_{n+1}-a_{n+1}=\frac{1}{2}(b_n-a_n)=2^{-1}·2^{-n}(b_0-a_0)=2^{-n-1}(b_0-a_0)$$
und $f(a_{n+1})\leq y\leq f(b_{n+1})$
% IDEE christopher
\bem
Weil $a\leq a_n\leq b$ gilt $a\leq x\leq b$ d.h. $x\in[a,b]$\qed
\uS{Anwendung}
\sS{Satz}
Sei \nN{} \ul{ungerade}, $f:\R→\R$
$$f(x)=x^n+a_{n-1}·x^{n-1}+…+a_0$$
Dann hat $f$ eine Nullstelle, d.h. es gibt $x\eR$ mit $f(x)=0$
\bew
Für $x\neq 0$ betrachte
$$g(x)=\frac{1}{x^n}f(x)=1+\frac{a_{n-1}}{x}+\frac{a_{n-2}}{x^2}+…+\frac{a_0}{x^n}$$
Für $x→∞$ ist $g(x)→1$\\*
Für $x→-∞$ ist $g(x)→1$\\*
D.h. es gibt $a\eR$ mit $a>0$ und
$$x\geq a \Rarr g(x)>0$$
$$x\geq -a \Rarr g(x)>0$$
% Also… Christopher
\bsp
$f(x)=x^3-x+20$ GRAPH
\bsp
Bedingung "$n$ ungerade" ist wesentlich, denn $f(x)=x^2+1$ hat keine Nullstelle
% Satz + bem
\bsp
$$f:(0,1)→\R,\ f(x)=\frac{1}{x}$$
$$f(D)=(1,∞)$$ GRAPH
\bsp
$$f:(-1,1)→\R,\ f(x)=x^2$$
$$f(D)=[0,1)$$ GRAPH
\bew
$f:D→\R$ stetig\\*
Sei $a:=inf(f(D))\eR\cup\{-∞\}$\\*
\phantom{Sei }$a:=inf(f(D))\eR\cup\{-∞\}$\\*
Angenommen $y\eR$ mit $a<y<b$ d.h. $x\in(a,b)$\\*
Es gibt $x_1,x_2\in D$ mit $a<f(x_1)<y<f(x_2)<b$\\*
Zwischenwertsatz \Rarr{} es gibt $x$ zwischen $x_1,x_2$\\*
(\Rarr $x\in D$ weil $D$ Intervall) mit $f(x)=y$\\*
Also $(a,b) \subseteq f(D)$\\*
Dann ist $f(D)$ eines der Intervalle $(a,b),[a,b),(a,b],\underset{\overset{\uparrow}{\text{nur wenn }a\neq -∞, b\neq ∞}}{[a,b]}$\qed
% whatever
\bew
Die Abbildung $f:D→D'$ ist
\begin{itemize}
\item{surjektiv nach Definition von $D'$}
\item{streng monoton \Rarr{} injektiv}
\item{also bijektiv. Somit existiert $f^{-1}:D'→D$}
\end{itemize}
\sss{Annahme}
$f$ streng monoton wachsend (fallend analog)
\beh
$f^{-1}$ ist streng monoton wachsend, d.h. gegeben $x_1,x_2\in D'$ mit $x_1<x_2$ zeige:
$$f^{-1}(x_1)<f^{-1}(x_2)$$
Angenommen $f^{-1}(x_1)\geq f^{-1}(x_2)$ \Rarr{} $f$ monoton wachsend \Rarr{} $x_1=f(f^{-1}(x_1))\geq f(f^{-1}(x_2))=x_2$ \Rarr{} Widerspruch\\*
also $f^{-1}(x_1)<f^{-1}(x_2)$ \Rarr{} $f^{-1}$ streng monoton wachsend
% beh
\bew
$$x+δ_1\lew x-δ<z<x+δ\leq x+δ-2$$
$f^{-1}$ streng monoton wachsend \Rarr
$$f^{-1}(x)-ε'=f^{-1}(x-δ_1)<f^{-1}(z)<f^{-1}(x+δ-2)=y+ε'=f^{-1}(x)+ε'$$
$$\Rarr |f^{-1}(z)-f^{-1}(x)|<ε'\leq ε \text{\Rarr{} Behauptung}$$
Somit $f^{-1}$ stetig in $x$\\*
Falls $x$ Randpunkt: Betrachte $[x,x+δ]$ bzw. $[x-δ,x]$ wieder analog\qed
% 2 Anwendungen
\uS{Logarithmus und allgemeine Potenzen}
\sS{Satz}
Die Exponentialfunktion $exp:\R→\R_{>0}=(0,∞)$ ist stetig, streng monoton wachsend und $exp(\R)=\R_{>0}$
\bew
Bekannt: $exp$ stetig, streng monoton wachsend.
Für $x>0$ ist $$exp(x)=1+x+\frac{x^2}{2}+…\geq 1+x$$
also gilt $$\lim_{x→∞} exp(x)=∞$$
$$exp(-x)=\frac{1}{exp(x)}\Rarr \lim_{x→-∞}exp(x)=\lim_{x→∞}\frac{1}{exp(x)}$$
Somit $exp(\R)=(0,∞)$\qed
Folge mittels 6.15: $exp:\R→\R_{>0}$ ist bijektiv und die Umkehrfunktion $exp^{-1}:=log:\R_{>0}→\R$ ist stetig, streng monoton wachsend, bijektiv \footnote{$exp^{-1}=log$ heißt Logarithmusfunktion} konkret: $log(x)=y \equ x=exp(y)$ GRAPH