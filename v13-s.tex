% Kopfzeile beim Kapitelanfang:
\fancypagestyle{plain}{
%Kopfzeile links bzw. innen
\fancyhead[L]{\calligra\Large Vorlesung Nr. 13}
%Kopfzeile rechts bzw. außen
\fancyhead[R]{\calligra\Large 22.11.2012}
}
%Kopfzeile links bzw. innen
\fancyhead[L]{\calligra\Large Vorlesung Nr. 13}
%Kopfzeile rechts bzw. außen
\fancyhead[R]{\calligra\Large 22.11.2012}
% **************************************************
%
\wdh
Zwischenwertsatz\\*
Sei $a\leq b,\ f:[a,b]→\R$ stetig\\*
Sei $y\eR$ zwischen $f(a)$ und $f(b)$ d.h. $f(a)\leq y\leq f(b)$ oder $f(a)\geq y\geq f(b)$\\*
Dann gibt es ein $x\in[a,b]$ mit $f(x)=y$ SKIZZE
\Bew {Intervallschachtelung}
Starte mit $[a_0,b_0]=[a,b]$\\*
Definiere unendliche Kette von Intervallen\\*
$[a_0,b_0]\subseteqq [a_1,b_1]\subseteqq [a_2,b_2]\subseteqq …$\\*
So dass $[b_n-a_n]=2^{-n}[b_0,a_0]$ und $y$ zwischen $f(a_n)$ und $f(b_n)$ liegt.\\*
Annahme: $f(a)\leq y\leq f(b)$ (Anderer Fall $f(a)\geq y\geq f(b)$ analog)\\*
Angenommen, $[a_n,b_n]$ ist konstruiert so dass $[_n-a_n=2^{-n}(b_0,a_0)$ und $f(a_n)\leq y\leq f(b_n)$\\*
Betrachte $m:=\frac{a_n+b_n}{2}$, Wenn $f(m)\geq y$ dann setze $[a_{n+1},b_{n+1}]:=[a_n,m]$\\*
Wenn $f(m)<y$ dann setze $[a_{n+1},b_{n+1}]:=[m,b_n]$\\*
Dann gilt in beiden Fällen:
$$b_{n+1}-a_{n+1}=\frac{1}{2}(b_n-a_n)=2^{-1}·2^{-n}(b_0-a_0)=2^{-n-1}(b_0-a_0)$$
und $f(a_{n+1})\leq y\leq f(b_{n+1})$
% IDEE christopher
\bem
Weil $a\leq a_n\leq b$ gilt $a\leq x\leq b$ d.h. $x\in[a,b]$\qed
\uS{Anwendung}
\sS{Satz}
Sei \nN{} \ul{ungerade}, $f:\R→\R$
$$f(x)=x^n+a_{n-1}·x^{n-1}+…+a_0$$
Dann hat $f$ eine Nullstelle, d.h. es gibt $x\eR$ mit $f(x)=0$
\bew
Für $x\neq 0$ betrachte
$$g(x)=\frac{1}{x^n}f(x)=1+\frac{a_{n-1}}{x}+\frac{a_{n-2}}{x^2}+…+\frac{a_0}{x^n}$$
Für $x→∞$ ist $g(x)→1$\\*
Für $x→-∞$ ist $g(x)→1$\\*
D.h. es gibt $a\eR$ mit $a>0$ und
$$x\geq a \Rarr g(x)>0$$
$$x\geq -a \Rarr g(x)>0$$
% Also… Christopher