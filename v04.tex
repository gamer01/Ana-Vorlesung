% Kopfzeile beim Kapitelanfang:
\fancypagestyle{plain}{
%Kopfzeile links bzw. innen
\fancyhead[L]{\calligra\Large Vorlesung Nr. 4}
%Kopfzeile rechts bzw. außen
\fancyhead[R]{\calligra\Large 18.10.2012}
}
%Kopfzeile links bzw. innen
\fancyhead[L]{\calligra\Large Vorlesung Nr. 4}
%Kopfzeile rechts bzw. außen
\fancyhead[R]{\calligra\Large 18.10.2012}
% *****************************************
\wdh
Angeordneter Körper:\\*
Menge $K$ mit $+, ·, <$\\*
so dass gewisse Eigenschaften erfüllt sind
\bsp
\Q{} sind ein angeordneter Körper\\*
Sei $K$ angeordneter Körper, $M\subseteq K$ Teilmenge $a\in K$ ist obere Schranke von $M$, wenn $U\subseteq a$, d.h.: $x\leq a\qquad ?x\in M$\\*
$a\in K$ ist kleinste obere Schranke, wenn\\*
\begin{enumerate}
\item{$M\leq a$}
\item{Wenn $b < a$, dann \ul{nicht} $M\leq b$}
\end{enumerate}
\vspace*{-9.5ex}\hspace*{15.5em}
$\left.
\begin{array}{l}
{}\vspace*{2ex}\\*{}
\end{array}
\right\}$
\vspace*{-5ex}Bezeichnung $a=sup(M)$
\vspace*{5ex}
%
\bsp
$K=\Q\qquad M=\{-\frac{1}{n}|n\in\N\}=\{-1,-\frac{1}{2},-\frac{1}{3},?\}$
\sss{Behauptung}
$sup(M)=0$
\bew
\begin{enumerate}
\item {Zeige: $M \leq 0$, d.h.: $\frac{1}{n}<0$ für alle $n\in\N$\ok}
\item {Wenn $b=\Q,\ b<0$, dann nicht $M\leq b$}
\end{enumerate}
Schreibe $b=\frac{m}{n},\ m\in\Z, n\in\N$\\*[1ex]
$b<0$ heißt $m<0,\ m\leq -1$\\*[1ex]
$b=\frac{m}{n} \leq \frac{-1}{n} \leq \frac{-1}{n+1}\in M$\\*[1ex]
\Rarr{} $M\not\leq b$ (nicht $M\leq b$)\qed

\uS{Vollständigkeit}
\sS{Definition Vollständigkeit}
Ein angeordneter Körper $K$ heißt Dedekind-vollständig, wenn jede nach oben beschränkte Teilmenge von $K$ eine kleinste obere Schranke hat (die Element $K$ ist).

\sS{Satz}
Es gibt genau einen Dedekind-vollständigen, angeordneten Körper $K$\\*
Dieser heißt Körper der reellen Zahlen\\*
\ul{Bezeichnung:} \R\\*
(Beweis ausgelassen)

\sS{Satz}
Die Teilmenge \N{} von \R{} ist unbeschränkt
\bew
(verwende nur die Axiome)\\*
Indirekter Beweis: Angenommen, \N{} ist beschränkt\\*
\desc{Vollständigkeit:}{\N{} hat eine kleinste obere Schranke $a\in\R$\\*
Es gilt $a-1<a \Rarr{} a-1$ ist kleinste obere Schranke von $\N\ n\leq a\qquad ∀ n\in\N\\*
\Rarr{} n+1\leq a\qquad ∀ n\in\N\\*
\Rarr{} n\leq a-1\qquad ∀n\in\N$ Widerspruch!\\*
Also Annahme falsch, d.h. \N{} ist unbeschränkt\qed}
\begin{tabular}{lcl}
beschränkt &=& nach oben beschränkt und nach unten beschränkt\\*
unbeschränkt &=& nicht nach oben beschränkt oder nicht nach unten beschränkt
\end{tabular}
%
\sS{Folgerung (Prinzip des Archimedes)}
Seien $x,y\in\R,\ x>0$, Dann gibt es $n\in\N$ mit $n·x>y$\\*
SKIZZE % SKIZZE
%
\bew
$n·x>y\ \equ\ n>\frac{y}{x}$ (weil $x>0$)\\*
\N{} unbeschränkt und nicht nach oben beschränkt \Rarr{} $\frac{y}{x}$ ist keine obere Schranke von \N\\*
\Rarr{} es gibt $n\in\N$ mit $n>\frac{y}{x}$\qed

\sS{Folgerung}
Sei $x\in\R,\ x>0$, dann gibt es $n\in\N$ mit $\frac{1}{n}<x$\\*
SKIZZE % SKIZZE
\bew
$\frac{1}{n}<x\ \equ\ 1<n·x\ \equ\ \frac{1}{x}<n$ (weil $x$ positiv)\\*
$\frac{1}{x}$ keine obere Schranke von \N{} \Rarr{} es gibt \nN{} mit $\frac{1}{x}<n$\qed

\sS{Satz}
Seien $x,y\eR$ mit $x<y$\\*
Dann gibt es $a\eQ$ mit $x<a<y$, man sagt \Q{} liegen dicht in \R\\*
SKIZZE % SKIZZE
\bew
$y-x>0$ Wähle \nN{} mit $\frac{1}{n}<y-x$\\*
Ansatz: $a=\frac{m}{n}$ mit $m\eZ$\\*
Sei $M:=\{m\eZ|x<\frac{m}{n}\}=\{m\eZ|nx<m\}$\\*
$M$ ist nach unten beschränkt und nicht leer (wegen Archimedes)\\*
$M$ hat Minimum\\*
Sei $m=min(M)$\\*
$m\in M \Rarr x<\frac{m}{n}$\\*
$m-1\not\in M \Rarr x\geq\frac{m-1}{n}$\\*
$y-\frac{m}{n} =y-x+x-\frac{m}{n}>\frac{1}{n}+x-\frac{m}{n}=x-\frac{m-1}{n}\geq0$\\*
$y>\frac{m}{n}$\qed

\uS{Wurzeln}
\sS{Satz}
Es gibt kein $a\eQ$ mit $a^2=2$\\*
\bew
Angenommen $a\frac{m}{n}\eQ,\ a^2=2,\ m,\nN$\\*
Kürze den Bruch $\Rarr \frac{m}{n}$ teilerfremd\\*
$$a^2=2\ \Rarr\ \frac{m^2}{n^2}=2 \ \Rarr\ m^2=2n^2 \ \Rarr\ m^2 \text{ gerade } \ \Rarr\ m \text{ gerade } \ \Rarr\ m=2q,\ q\eN$$
$$(2q)^2=2n^2 \ \Rarr\ 4q^2=2n^2 \ \Rarr\ 2q^2=n^2 \ \Rarr\  n^2 \text{ gerade } \Rarr n \text{ gerade }$$
Widerspruch zur Annahme $m,n$ teilerfremd\qed\\*
SKIZZE WURZEL 2 \Rarr\ $\sqrt{2}$ sollte existieren % SKIZZE
\bem
Wenn \nN, keine Quadratzahl, dann gibt es kein $a\eQ$ mit $a^2=n$ (ähnlicher Beweis)

\sS{Satz}
Sei $x\eR, x\geq 0, \nN$\\*
Dann gibt es \ul{genau ein} $y\eR, x\geq 0$ mit $y^n=x$\\*
Bezeichnung: $x=\sqrt[n]{y}$
\bew
später\\*
\ul{Ansatz:} $sup\{a\eQ\mid a^n\leq x\}=:y$ (sup existiert weil \R{} Dedekind-vollständig)

\sS{Definition Potentzrechnung}
Sei $x\eR,\ x>0\qquad \frac{m}{n}\eQ$\\*
\nN, $m\eZ\qquad x^{\frac{m}{n}}=\sqrt[n]{x^m}\qquad x^{\frac{1}{n}}=\sqrt[n]{x}$
\sss{Potenzrechnung:}
$\ds x^{(a+b)}=x^a·x^b,\ x^{a·b}=(x^a)^b$\hfill für $x\eR,\ x>0,\ a,b\eQ$
\bem
Später wir definiert: $x^a$ für $x\eR,\ x>0,\ a\eR$

\chapter{Folgen und Reihen reeller Zahlen}
Grundbegriff der Analysis: Konvergenz
\bsp
Wenn \nN{} immer größer wird, geht $\frac{1}{n}$ immer näher an Null.\\*
Sei $\N_0=\{0,1,2,3,4…\}$

\sS{Definition Folge}
Eine Folge reeller Zahlen ist eine Abbildung $\N_0→\R$ d.h. jeder natürliche Zahl $n\geq 0$ wird eine reelle Zahl $a_n$ zugeordnet.\\*
\notat{$(a_n)_{\nN_0}$ oder $(a_n)_{\N\geq 0}$ oder $(a_0,a_1,a_2,a_3,…)$}
\sss{Variante:}
Folgen, die bei $k\in\Z$ anfangen: $(a_n)_{\N\geq 0}=(a_k,a_{k+1},a_{k+2},…)$
\bsp
\begin{enumerate}
\item{konstante Folge: $a_n=a,\ a\eR$ fest: $(a,a,a,a,a,…)$}
\item{$a_n=\frac{1}{n}$ für $n\geq 1$\\*
$(a_n)_{n\geq 1}=(1,\frac{1}{2},\frac{1}{3},…)$}
\item{$a_n=(-1)^n\qquad n>0$\\*
$(1,-1,1,-1,1,-1,1,-1,1,-1,…$}
\item{$(\frac{n}{n+1})_{n\geq 0}n=(0,\frac{1}{2},\frac{2}{3},\frac{3}{4},\frac{4}{5},…$}
\end{enumerate}

\sS{Definition Konvergenz}
Sei $(a_n)_{n\geq 0}$ eine Folge reeller Zahlen
\begin{enumerate}
\item{Eine Folge $(a_n)$ konvergiert gegen $a\eR$ wenn gilt: Für jedes $\e>0$ gibt es ein $N\eN$, so dass $|a_n-a|<\e$ für jedes \nN{} mit $n\geq N$\\*
Dann heißt $a$ Grenzwert der Folge $(a_n)$\\*
\notat{$\ds\lim_{n→∞}a_n=a$ oder $a_n→a$ für $n→∞$}}
\item{Die Folge $(a_n)$ heißt \ul{Nullfolge}, wenn $a_n→a$ für $n→∞$}
\item{Die Folge $(a_n)$ ist divergent, wenn sie keinen Grenzwert hat.}
\end{enumerate}
%
\bsp
\begin{enumerate}
\item{$a_n=\frac{1}{n}$ für $n\geq 1$
\sss{Behauptung:} $a_n→a$ für $n→∞$ SKIZZE % SKIZZE
\bew
Sei $\e>0$ wähle $N=0$ für $n\geq N$ gilt $|a_k-a|=0<\e$\qed}
\item{$a_n=(-1)^n=(1,-1,1,-1,1,-1,…)$ SKIZZE % SKIZZE
\sss{Behauptung:} $(a_n)$ ist divergent. 
\bew
Angenommen, $a\eR$ ist Grenzwert der Folge.\\*
Wähle $\e=1$. Es gibt $N\eN$ mit $|a_n-a|<1$ für alle $n\geq N$\\*
Wenn $n$ gerade: $a_n=1\qquad |1-a|<1$
Wenn $n$ ungerade: $a_n=-1\qquad |-1-a|<1 \Rarr |1+a|<1$\\*
$2=|2|=|1-a+1+a|\leq |1-a|+|1+a|<2 \Rarr 2<2$\\*
Widerspruch: Also ist $(a_n)$ divergent\qed}
\end{enumerate}