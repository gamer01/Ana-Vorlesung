$((cos (x))^2)' = f(cos (x))' = f'(cos (x)) \cdot cos'(x) = 2 \cdot cos(x) \cdot sin(x)$
\sS{Satz Quotientenregel}
Seien $f, g: I \to \R$ in $x_0$ differenzierbar, $g(x) \neq 0$ für alle $x \in I$.\\*
Dann ist $\frac{f}{g}: I \to \R$, $\frac{f}{g}(x) := \frac{f(x)}{g(x)}$ differenzierbar in $x_0$\\*
$$(\frac{f}{g})'(x) = \frac{f'(x_0)\cdot g(x_0) - f(x_0) \cdot g'(x_0)}{g(x_0^2)}$$
\bew
Fall $f = 1$:\\*
Kettenregel:
$\frac{1}{g} = \frac{1}{x} \cdot g$\\*
Sei $h(x) = \frac{1}{x}\ \ \frac{1}{g}(x) = h(g(x))$ \\*
$(\frac{1}{g})'(x) = h'(g(x)) \cdot g'(x) = - \frac{1}{g(x^2)} \cdot g'(x) \approx Beh.$\\*
Insbesondere: $(\frac{1}{g})' = -\frac{1}{g^2} \cdot g' = -\frac{g'}{g^2}$
Allgemeiner Fall:
$\frac{f}{g} = f \cdot \frac{1}{g}$\\*
Produktregel \Rarr{} $(\frac{f}{g})' = (f \cdot \frac{1}{g})' = f'\frac{1}{g} + f \cdot (\frac{1}{g})'$\\*
$= \frac{f' \cdot g}{g^2} - f\cdot \frac{g'}{g^2} = \frac{f'\cdot g - f \cdot g'}{g^2} \qed$\\*

% Stefan 	BSP
%			Satz 8.7 Ableitung der Umkehrfunktion

% Graph für Stefan (8.7)
\begin{tikzpicture}[domain=-1:4,prefix=plots/, smooth]
\draw[very thin,color=gray] (-0.99,-0.49) grid (3.99,2.99);
\draw[->] (-1,0) -- (4,0) node[right] {$x$};
\draw[->] (0,-1) -- (0,3) node[above] {$y$};
% Some fancy function
% Tangente zu dem Fancy Graphen bei x_0
% \draw[color=blue] plot[id=cos1] function{cos(x)} node[below, midway] {\footnotesize $f_2(x) = cos(x)$};
\end{tikzpicture}\\*

\ul{Folge} $log: \R_{>0}\to \R$ ist differenzierbar, $log'(x) = \frac{1}{x}$\\*
\bew
$log(x) = exp(x)^{-1}$ Umkehrfunktion $exp'(x) = exp(x) \neq 0$ für alle $x$\\*
\Rarr{} 8.7 anwendbar. Sei $y = exp(x)$, $x = log(y)$.
$log'(y) = \frac{1}{exp'(x)} = \frac{1}{exp(x)} = \frac{1}{y}\qed$

%Stefan 	Anwendung

\uS{Höhere Ableitungen}
Idee: Wenn $f: I \to \R$ differenzierbar\\*
\approx $f':I \to \R$ Fkt\\*
Wenn $f'$ diffbar $ \approx (f')' = f'' = \frac{d^2f}{dx^2}$\\*
2. Ableitung weiter:
$f'' = f^{(2)}$\\*
$f^{(n+1)} = (f^{n})'$ wenn differenzierbar\\*
$f^{(n)}$: n-te Ableitung von $f$.
\bsp
$$(x^5)^{(2)} = ((x^5)')' = (5x^4)' = 20x^3$$
$$cos'' = -sin' = -cos$$
$$sin'' = -cos' = -sin$$

% Stefan 	8.9

\uS{Lokale Extrema und Mittelwertsatz}
\sS{Definition Lokales Maximum}
Sei $f: I \to \R$ Fkt.\\*
f hat ein \ul{lokales Maximum} in $x_0 \in I$ wenn gilt:\\*
$\case{
\text{es gibt ein $\e > 0$ s.d.}\\
\text{Für alle $x \in I$ mit $|x-x_0| < \e$}\\
\text{gilt $f(x) \leq f(x_0)$}
}$
Analog: Lokales Minimum.
\bem
Lokale Minima von $f$ = lokale Maxima von $-f$

% Stefan 	Satz 8.11

\sS{Satz Rolle}
Sei $a < b$, $f:[a, b] \to \R$ stetig auf $(a, b)$ differenzierbar.\\*
Sei $f(a) = f(b)$.\\*
Dann gibt es ein $x_0 \in (a, b)$ mit $f'(x_0) = 0$\\*
% Graph
\bew
Wenn $f$ konstant, d.h. $f(x) = f(a)$ für alle $x \in (a, b),$ dann $f'(x) = 0 $ für alle $x$ \Rarr{} Satz stimmt.\\*
Sei $f$ nicht konstant, gibt es $x_1 \in (a, b)$ mit $f(x_1) \neq f(a)$\\*
Angenommen $f(x_1) > f(a)$  (sonst Betrag $-f$)\\*
sei $x_0 \in I$ mit $f(x_0) \geq f(x)$ für alle $x \in I$\\*
$f(x_0) \geq f(x_1) > f(a) = f(b) \Rarr x_0 \neq a,\ x_0 \neq b$\\*
$f$ hat in $x_0$ ein \ul{lokales} Maximum \overset{8.19}{\Rarr{}} $f'(x_0) = 0$