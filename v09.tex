% Kopfzeile beim Kapitelanfang:
\fancypagestyle{plain}{
%Kopfzeile links bzw. innen
\fancyhead[L]{\calligra\Large Vorlesung Nr. 9}
%Kopfzeile rechts bzw. außen
\fancyhead[R]{\calligra\Large 08.11.2012}
}
%Kopfzeile links bzw. innen
\fancyhead[L]{\calligra\Large Vorlesung Nr. 9}
%Kopfzeile rechts bzw. außen
\fancyhead[R]{\calligra\Large 08.11.2012}
% **************************************************
%
\wdh
Eine folge reeller Zahlen $(a_n)$ ist eine Cauchy-Folge wenn gilt:\\
Für jedes $\e>0$ gibt es ein $n\in\N$ so dass für $m,n\geq\N$ gilt $|a_n-a_m|<\e$\\
$(a_n)$ konvergiert \equ $(a_n)$ ist Cauchy-Folge\\
Für Reihen: $\ds \sum_{k=0}^{∞}a_k$ konvergiert \equ Für jedes $\e>0$ gibt es ein $N\in\N$ so dass für $m,n\geq\N$ mit $m\geq n$ ist $\ds \left|\sum_{k=n}^m a_n\right|<\e$

\uS{Absolute Konvergenz}

\sS{Definition}
Eine Reihe $\ds\sum_{k=0}^{∞} a_k$ mit $a_k\in\R$ heißt absolut konvergent wenn die Reihe $\ds\sum_{k=0}^{∞} |a_k|$ konvergiert

\sS{Satz}
Jede absolut konvergente Reihe konvergiert

\bew
Verwende Cauchy-Kriterium für Reihen\\
Sei $\ds\sum_{k=0}^{∞} a_k $ absolut von konvergent.\\
\Rarr Für jedes $\e>0$ gibt es $N\in\N$ mit:\\
Für $n\geq m\geq N$ gilt $\ds\sum_{k=m}^n |a_k| < \e \Rarr \left|\sum_{k=m}^n a_k\right| \underset{\overset{\uparrow}{Dreiecksungleichung}}{\leq} \sum_{k=m}^n |a_k| < \e \Rarr \sum_{k=m}^n a_k konvergiert$\qed

\bem
Umkehrung gilt nicht.
$\ds\sum_{k=1}^{∞} (-1)^k \frac{1}{k} = -1+\frac{1}{2}+\frac{1}{3}+\frac{1}{4}+...$\\
konvergiert (Leibnitz)\\
denn $\ds\sum_{k=1}^{∞} \left|(-1)^k \frac{1}{k}\right| = \sum_{k=1}^{∞} \frac{1}{k}$ divergiert

\sS{Definition}
Eine Reihe $\ds\sum_{k=0}^{∞} b_k$ heißt Majorante der Reihe $\ds\sum_{k=0}^{∞} a_k$, wenn $|a_k|\leq b_k$ für alle k\\
(schon gewesen wenn $a_k\geq 0$)

\sS{Satz (Majorantenkriterium)}
Wenn eine Reihe eine konvergente Majorante hat, dann konvergiert sie absolut.
\underline{Beweis} von Satz 4.5\qed

\uS{Umordnung von Reihen}
\sS{Definition}
Eine Umordnung einer Reihe $\ds\sum_{k=0}^{∞} a_k$ ist eine Reihe der Form $\ds\sum_{k=0}^{∞} a_{n_k}$ wobei $(n_0,n_1,n_2…)$ eine Folge natürlicher Zahlen ist, in der jedes $n\in\N_0$ genau einmal vorkommt.\\

\sS{Satz}
Jede Umordnung einer \underline{absolut} konvergenten Reihe ist wieder absolut konvergent und hat den gleichen Grenzwert.\\
Im Gegensatz dazu gilt:\\
\sS{Satz}
Sei $\ds\sum_{k=0}^{∞} a_k$ eine konvergente, nicht absolut konvergente, Reihe. Für jedes $c\in\R\cup\{-∞,∞\}$ hat $\sum a_k$ eine Umordnung, die gegen c konvergiert.

\bsp
Eine Reihe $\dfrac{1}{2}-\dfrac{1}{2}+\dfrac{1}{3}-\dfrac{1}{3}+\dfrac{1}{4}-\dfrac{1}{4}+\dfrac{1}{5}-\dfrac{1}{5}+…$
konvergiert gegen 0. Konvergiert aber nicht absolut:\\
Folge: $(\dfrac{1}{2},0,\dfrac{1}{3},0,\dfrac{1}{4},0,…→0)\quad\ds\sum_{k=1}^{∞} 2·1/k=∞$\\
Produziere Umordnung, die gegen ∞ konvergiert:\\
\[\dfrac{1}{2}-\dfrac{1}{2}+\underbrace{\dfrac{1}{3}+\dfrac{1}{4}}_{\geq\dfrac{1}{4}+\dfrac{1}{4}=\dfrac{1}{2}}-\dfrac{1}{3}+\underbrace{\dfrac{1}{5}+\dfrac{1}{6}+\dfrac{1}{7}+\dfrac{1}{8}}_{\geq\dfrac{1}{2}}-\dfrac{1}{4}+\underbrace{\dfrac{1}{5}+…+\dfrac{1}{16}}_{\geq\dfrac{1}{2}}-\dfrac{1}{5}+…\]\\
\[\leq\underbrace{\dfrac{1}{2}-\dfrac{1}{2}}_{\text{\large{0}}}+\underbrace{\dfrac{1}{2}-\dfrac{1}{3}}_{\dfrac{1}{6}}+\underbrace{\dfrac{1}{2}-\dfrac{1}{4}}_{<\qquad\dfrac{1}{4}\qquad<}+\underbrace{\dfrac{1}{2}-\dfrac{1}{5}}_{\dfrac{3}{10}}+…=∞\]\\
Beweise von 4.24, 4.25 eventuell später.

\uS{Produkte von Reihen}
Frage: was ist $\ds\left(\sum_{k=0}^{∞} a_k\right)·\left(\sum_{k=0}^{∞} b_k\right) ?$

\sS{Definition}
Das Cauchy-Produkt von zwei reihen $\ds\sum_{k=0}^{∞} a_k$ und $\ds\sum_{k=0}^{∞} b_k$ ist eine Reihe $\ds\sum_{k=0}^{∞} c_k$ mit $\ds c_n :=\sum_{k=0}^{∞} a_k·b_{n-k}=a_0·b_n+a_1·b_{n-1}+a_2·b_{n-2}+…+a_n·b_0$\\
2-dimensionale Anordnung der $a_k·b_l$ 
% Ed's heft
\sS{Satz}
Seien $\ds\sum_{k=0}^{∞} a_k$ und $\ds\sum_{k=0}^{∞} b_k$ konvergente Reihen, mindestens eine von ihnen absolut konvergent. Dann konvergiert ihr Cauchy-Produkt $\ds\sum_{k=0}^{∞} c_k$. Wenn $\ds\sum_{k=0}^{∞} a_k = a, \ds\sum_{k=0}^{∞} b_k = b$ $\ds\sum_{k=0}^{∞} c_k = a·b$



\Bew{von 4.27}
Sei $\sum a_k$ absolut konvergent, $\sum b_k$ konvergent, so zeige $\sum c_k\ $ konvergent, $\ds c_n :=\sum_{k=0}^{∞} a_k·b_{n-k}$
Schreibe:
$s_n=a_0+…+a_n$\\
$t_n=b_0+…+b_n$\\
$u_n=c_0+…+c_n$\\
$s_n→a$,$t_n→b$ (*)\\[8pt]
Zeige $u_n→a·b$\\[4pt]
(*)\Rarr $s_n·b→a·b$ Zeige $s_n·b-u_n→0$\\[4pt]
$u_n=a_0·b_0+(a_0·b_1+a_1·b_0)+(a_0·b_2+a_1·b_1+a_2·b_0)+…+a_n·b_0=a_1·t_{n-1}+a_2·t_{n-2}+…+a_n·t_0$\\[4pt]
$s_n·b=a_0·b+a_1·b+a_2·b+a_3·b+…+a_n·b$\\[4pt]
$s_n·b-u=a_0·(b-t_n)+a_1·(b-t_{n-1})+a_2·(b-t_{n-2})+a_3·(b-t_{n-3})+…+a_n·(b-t_0)\underset{?}{→}0$\\[8pt]
Sei $C\in\R$ mit $|b|\leq C$ und $|b-t_n|\leq C$ für alle n\\
Sei $\ds\sum_{k=0}^{∞} |a_n| = a^*.$\\
Gegeben sei $\ds\e>0$. Wähle $N\in\N$ so dass $C·(|a_N|+|a_{N+1}|+|a_{N+2}|+…)<\frac{\e}{2}$\\
(geht weil $\sum|a_k|$ konvergiert)\\
und $|b-t_n|<\frac{\e}{2a^*}^{(2)}$ für alle $n\geq N$\\
(geht weil $b-t_n→0$ für alle $m→∞$)
\bem
Wenn $a^*=0$ dann $a_n=0$ für alle k. Dann alles klar.
Für alle $n\geq 2N$ gilt:\\
$|a_0(b-t_n)+a_1(b-t_{n-1})+…+a_n(b-t_0)|\leq |a_0|·|(b-t_n)|+|a_1|·|(b-t_{n-1})|+…+|a_n|·|(b-t_0)|$\\[4pt]
$\leq(|a_0|+|a_1|+|a_2|+…|a_N|)·\underset{\overset{\uparrow}{wegen\left(2\right)}}{\frac{\e}{2a^*}} +(|a_{N+1}|+|a_{N+2}|+|a_{N+3}|+…|a_n|) · C \leq a^* ·\frac{\e}{2a^*}+\frac{\e}{2}=\frac{\e}{2}+\frac{\e}{2}=\e$\\
Also gilt: $s_n-u→0$ für $n→∞$\qed\\
\underline{Zusatz:} Wenn $\sum a_k$ und $\sum b_k$ beide absolut konvertieren, dann auch das Cauchy-Produkt $\sum c_k$\\
\bew
Sei $\sum a_k^*$ das Cauchy-Produkt von  $\sum |a_k|$ und  $\sum |b_k|$. Beide konvergieren \Rarr $\sum_n c_n^*$ konvergiert\\[8pt]
d.h. $c_n^*=|a_0·b_{n}|+|a_1·b_{n-1}|+…+|a_n·b_{0}|\geq|a_0·b_{n}+a_1·b_{n-1}+…+a_n·b_{0}|=|c_n|$\\[8pt]
Also $\sum_n c_n^*$ ist konvergente Majorante von $\sum_n c_n$ \Rarr $\sum_n c_n$ konvergent absolut\qed

\bsp
Die Reihe $\ds\sum_{k=0}^{∞} a_k = 1-+\dfrac{1}{\sqrt{2}}+\dfrac{1}{\sqrt{3}}-\dfrac{1}{\sqrt{4}}+\dfrac{1}{\sqrt{5}}-…\quad $ konvergiert (Leibnitz)\\
Das Cauchy-Produkt der Reihe von $\sum a_k$ und $\sum a_k$ konvergiert nicht.\\

\sS{Beispiel}
Für jedes $x\in\R$ ist die Exponentialreihe $\ds exp(x)=\sum_{k=0}^{∞} \frac{x^k}{k!}$ absolut konvergent.\\
Es gilt \fbox{$exp(x)-exp(y)=exp(x+y)$} Funktionalgleichung der Exponentialfunktion.\\
\bew
Betrag von $\ds \sum_{k=0}^{∞} |\frac{x^k}{k!}|=\sum_{k=0}^{∞} \frac{|x|^k}{k!}=exp(|x|)$ konvergiert (bekannt, Quotientenkriterium)\\
Berechne Cauchy-Produkt $\ds exp(x)·exp(y)=\sum_{k=0}^{∞} c_k$\\
$\ds c_k = \frac{x^0}{0!}·\frac{x^n}{n!}+\frac{x^1}{1!}·\frac{y^{n-1}}{(n-1)!}+…+\frac{x^n}{n!}·\frac{y^0}{0!}=\dfrac{1}{n!}·\left(\dfrac{n!}{0!·n!}·x^0y^n+\dfrac{n!}{1!·(n-1)!}·x^1y^{n-1}+…+\dfrac{n!}{n!·0!}·x^ny^0+\right)=\dfrac{1}{n!}\sum_{k=0}^{n}\dfrac{n!}{k!·(n-k)!}x^ky^{n-k} =\dfrac{1}{n!}\sum_{k=0}^{n}\binom{n}{k}x^ky^{n-k}\underset{binomische Formel}{=}\dfrac{1}{n!}(x+y)^n\Rarr\sum_{k=0}^{∞}c_k=exp(x+y)$\qed