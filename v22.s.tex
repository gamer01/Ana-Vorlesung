% Kopfzeile beim Kapitelanfang:
\fancypagestyle{plain}{
%Kopfzeile links bzw. innen
\fancyhead[L]{\calligra\Large Vorlesung Nr. 21}
%Kopfzeile rechts bzw. außen
\fancyhead[R]{\calligra\Large 07.01.2013}
}
%Kopfzeile links bzw. innen
\fancyhead[L]{\calligra\Large Vorlesung Nr. 21}
%Kopfzeile rechts bzw. außen
\fancyhead[R]{\calligra\Large 07.01.2013}
% **************************************************
%
\chapter{Integration}
\sss{Idee} Sei $f:[a,b]→\R_{\geq 0}$\\*
GRAPH\\*
$\int_a^b f(x)dx=Fläche$ zwischen Graphen von $f$ und $x$-Achse\\*
Wenn allgemeiner $f:[a,b]→\R$\\*
GRAPH\\*
dann zählen Flächen unterhalb der $x$-Achse negativ
$$\int_a^b f(x)dx =F_1-F_2+F_3$$
\sss{Fragen}
Formale Definition des Intervalls? Welche Funktionen sind interpretierbar? Eigenschaften, Berechnung des Integrals.

\uS{Treppenfunktion}
\sS{Definition der Treppenfunktion}
Sei $a,b \eR$, $a<b$
\enum{
\item Eine Funktion $f:[a,b]→\R$ heißt Treppenfunktion, wenn es eine Unterteilung $a=x_0<x_1<x_2<x_3<…<x_n=b$ gibt, so dass $f$ auf $(x_{i-1},x_{i})$ konstant ist, dass heißt $f(x)=c_i$ für alle $x$ mit $ x_{i-1}<x<x_{i} $\\*
Graph
\item In diesem Fall definiere
$$\int_a^b f(x)dx=\sum_{i=1}^n c_i(x_{i}-x_{i-1})$$
"Summe der Rechtecke"
\bem
Die Definition eines Integrals für die Treppenfuntkion ist Unabhängig vin der Unterteilung
\bsp
GRAPH (ohne formalen Beweis)
}

\sS{Lemma}
Seien $f,g:[a,b]→\R$ Treppenfuntionen\\*
Dann gilt:
\enum{
\item $\ds\int_a^b (f+g)(x)dx=\int_a^b f(x)dx+\int_a^b g(x)dx$
\item $\ds c\eR \int_a^b c·f(x)dx=c·\int_a^b f(x)dx$
\item Wenn $f\leq g$, dass heißt $f(x)\leq g(x)\ ∀x$, dann $\ds \int_a^b f(x)dx\leq \int_a^b g(x)dx$\\*
GRAPH (ohne formalen Beweiß)
}

%christopher riemansches Integral
\sS{Definition des Riemannschen Integral}
Sei $f:[a,b]→\R$ beschränkte Funktion\\*
Unterintegral:
$$sup \left\{\int_a^b g(x)dx\mid g:[a,b]\text{ Treppenfunktion mit }g\leq f\right\}=:\int_a^b{}_* f(x)dx$$
Oberintegral:
$$inf \left\{\int_a^b h(x)dx\mid h:[a,b]\text{ Treppenfunktion mit }f\leq h\right\}=:\int_a^b{}^* f(x)dx$$
(\ul{Idee} Wenn $\int_a^b f(x)$ definiert, sollte $ \int_a^b{}_* f(x)dx\leq \int_a^b f(x)dx \leq \int_a^b{}^* f(x)dx $)
\sss{Definition}
$f$ heißt integrierbar, wenn $\int_a^b{}_* f(x)dx=\int_a^b{}^* f(x)dx$\\*
Dann setzte $\int_a^b f(x)dx:=\int_a^b{}_* f(x)dx$

%cchristopher bemerkung
%Eigenschaften des Integrals

