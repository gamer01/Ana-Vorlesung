%\begin{tikzpicture}[domain=-1:2.5]
%    \draw[very thin,color=gray] (-1,-0.5) grid (2.5,1.5);
%    \draw[->] (-1,0) -- (2.5,0) node[right] {$x$};
%    \draw[->] (0,-0.5) -- (0,1.5) node[above] {$i$};
%    \draw node at(2,1) {$2+i$};
%    \draw node at (0.8, 0.5) {$|2i|$}
%    \draw (0,0) -- (2,1);
%\end{tikzpicture}
%
%%graph 2
%
%\begin{tikzpicture}[domain=-1:2.5]
%    \draw[very thin,color=gray] (-1,-0.5) grid (2.5,1.5);
%    \draw[->] (-1,0) -- (2.5,0) node[right] {$x$};
%    \draw[->] (0,-1) -- (0,1.5) node[above] {$i$};
%    \draw node at(0.4,1) {$w$};
%    \draw node at (1.2, -0.5) {$z$}
%    \draw node at (1.6, 0.5) {$z+w$}
%    \draw (0,0) -- (0.4,1);
%    \draw (0,0) -- (1.2,-0.5);
%    \draw (0.4,1) -- (1.2,-0.5);
%\end{tikzpicture}
%
% Kopfzeile beim Kapitelanfang:
\fancypagestyle{plain}{
%Kopfzeile links bzw. innen
\fancyhead[L]{\calligra\Large Vorlesung Nr. 15}
%Kopfzeile rechts bzw. außen
\fancyhead[R]{\calligra\Large 29.11.2012}
}
%Kopfzeile links bzw. innen
\fancyhead[L]{\calligra\Large Vorlesung Nr. 15}
%Kopfzeile rechts bzw. außen
\fancyhead[R]{\calligra\Large 29.11.2012}
% **************************************************
%
\wdh
$\C=\R^2$ mit 
$$(x,y)+(x',y')=(x+x',y+y')$$
$$(x,y)·(x',y')=(x·x'-y·y',x·y'-y·x')$$
$$(x,y)=x+i·y\qquad i=(0,1)\qquad i^2=(-1)$$
SKIZZE\\
$$z=x+i·y \eC\ \Rarr\ Re(z)=x \quad Im(z)=y$$
$\overline{z}=x-i·y$
$$z·\overline{z}=(x+i·y)(x-i·y)=x^2-(i·y)^2=x^2-i^2·y^2=x^2·y^2$$
Abstand von 0 nach $z$ ist:
$$|z|=\sqrt{x^2+y^2}=\sqrt{z·\overline{z}}\footnote{\ul{Betrag} von $z$}$$
$$|2+i|=\sqrt{2^2+i^2}=\sqrt{5}$$%muss Wurzel 3 rauskommen

Berechnung von $z^{-1}$\\*
$$\frac{1}{z} = \frac{z}{\overline{z} \cdot z}$$ % Pfeil auf z {Reelle Zahl}
$$\frac{1}{2+i} = \frac{2 - i}{(2+i)(2-i)} = \frac{2-1}{5} = \frac{2}{5} = \frac{i}{5}$$
\sS{Lemma}
Sei $z,w\eC$ dann gilt:
\begin{enumerate}
\item{$$\bar{zw}=\bar{z}·\bar{w},\ \bar{z+w}=\bar{z}+\bar{w}$$}
\item{$z+\bar{z}=2 Re(z),\ z-\bar{z}=2i·Im(z)$\\*
$Re(z)=\frac{z+\bar{z}}{z}\qquad Re(z)=\frac{z-\bar{z}}{2i}$}
\end{enumerate}
%
\bew
\begin{enumerate}

\end{enumerate}

\bew
\begin{enumerate}
\setcounter{enumi}{1}
\item{$$z=x+i·y$$
$$z+\bar{z}=(x+i·y)+(x-i·y)=2x=2Re(z)$$
$$z-\bar{z}=(x+i·y)-(x-i·y)=2iy=2i·Im(z)$$}
\setcounter{enumi}{1}
\item{$z = x+iy,\ w = a + ib$\\*
$$\overline{z+w} = \ol{x+iy + a + ib} = \ol{x+a + i(y+b)} = x + a - i(y + b) = (x - iy)+(a + ib) = \ol{z} + \ol{w}$$
$$\ol{zw} = \ol{(x+iy)(a + ib) = \ol{(ax - by) + i(ay + bx)} = (ax - by) - i(ay + bx) = |x|}$$
$$\ol{z} \cdot \ol{w} = \ol{(x+iy)} \cdot \ol{a+ib} = (x-iy) \codt (a+ib) = ax - (-y)(-b) + i(a\cdot (-y) + (-b) \cdot x)$$
$$= ax - by -i(ay + bx) =  |x|$$}
\end{enumerate}

\bem
Aus 3) folgt: \\*
$$z = x+iy \Rarr |z| \leq |iy| = |x| + |y|$$
Erinnerung:\\*
$$|z| \leq |x|,\ |z| \leq |y| \text{ (aus der NR)}$$
\uS{Folgen und Reihen komplexer Zahlen}
\sS{Definition}
Sei $(c_n)_{n\geq 0}$ eine Folge komplexer Zahlen, $c \in \C$\\*
Die Folge $c_n$ konvergiert gegen $c$ wenn gilt:\\*
Für jedes $\e > 0$ gibt es ein $N \in \N$ so dass für jedes $n \geq \N$ gilt $|c - c_n| < \e$
\notat{ \ds\lim_{n \to \infty} c_n = c \\
		oder $c_n \to c \ für n \to \infty$}

%Stefan

\bem
\begin{enumerate}
\item{Es gilt $c_n \to c \equ \ol{c_n} \to \ol{c}$}
\item{Wenn $c_n \to c$ dann $|c_n| \to |c|$}
\end{enumerate}
\bew
\begin{enumerate}
%% Er war einsam aber schneller.
\item{$|\ol{c_n} - \ol{c}| = |\ol{c_n - c}| = |c_n - c| \Rarr beh$}
\item{Übung}
\end{enumerate}

%Stefan

\item[$",\Larr"'$]{Verwende $|c_n - c| \leq |a_n - a| + |b_n - b| (*)$\\*
Gegeben $\e > 0$\\*
Es gibt $N \in \N$ so dass für jedes $n \geq N$:\\
$$|a_n - a| < \frac{\e}{2}, \ |b_n - b| < \frac{\e}{2}$$
Dann gilt für $n \geq N$:\\*
$|c_n - c| \leq \frac{\e}{2} + \frac{\e}{2} = \e$\\*}

% Stefan
% 7.7 Definition
% 7.8 Satz

\sS{Satz}
Eine Folge komplexer Zahlen $(c_n)$ konvergiert \equ{} $(c_n)$ ist Cauchy-Folge
\bew
	Sei $c_n = a_n + i \cdot b_n$\\*
	$(c_n)$ konvergiert $\underset{7.6}{\equ} \ (a_n$ und $(b_n)$ konvergieren $\equ \ (a_n)$ und $(b_n)$ sind Couchy-Folgen $\underset{7.8}{\equ}$ $(c_n)$ ist Cauchy-Folge \qed

% Stefan
% 7.10 Satz

\sS{Satz}
Wenn die Reihe $\sum_{n = 0}^{\infty} c_n$ absolut konvergent ist, dann ist die konvergent.
\bew
Sei $c_n = a_n + i \cdot b_n$ $$|c_n| \geq |a_n|\ \ |c_n| \geq |b_n|$$
$\sum |c_n|$ konvergent \Rarr $\sum |a|,\ \sum |b_n|$ konvergent. (Majorantenkriterium)\\*
d.h.: $\sum a_n,\ \sum |b_n|$ konvergiert absolut\\*
$\Rarr \sum a_n,\ \sum b_n$ konvergiert.\\*
$\overset{7.7}{\Rarr} \sum c_n$ konvergent \qed

\ul{Zusatz}\\*
Angenommen $\sum c_n$ konvergiert absolut, dann: 
$$|\sum_{n \geq 0} c_n| = \sum_{n \geq 0} c_n$$
(Dreiecksungleichung für $\infty$ viele Summanden)
\bew
Gewöhnliche Dreiecksungleichung \Rarr
$$|c_0 + c_1 + ... + c_n| \leq |c_0| + |c_1 + ... + c_n|$$
$$\text{(Partialsummen)} \leq ... \leq |c_0| + |c_1| + ... + |c_n| (*)$$
Wenn $c = \sum c_n$, dann $c = \ds\lim_{n \to \infty} (c_0 + c_1 + ... + c_n)$ (Definition des Grenzwertes einer Reihe)\\*
$\Rarr |c| = \ds\lim_{n \to \infty} |c_0 + c_1 + ... + c_n| = |\sum_{n = 0}^{\infty} c_n| \leq \lim_{n \to \infty} |c_0| + |c_1| + ... + |c_n| = \sum_{n = 0}^{\infty} |c_n|$