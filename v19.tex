% Kopfzeile beim Kapitelanfang:
\fancypagestyle{plain}{
%Kopfzeile links bzw. innen
\fancyhead[L]{\calligra\Large Vorlesung Nr. 19}
%Kopfzeile rechts bzw. außen
\fancyhead[R]{\calligra\Large 17.12.2012}
}
%Kopfzeile links bzw. innen
\fancyhead[L]{\calligra\Large Vorlesung Nr. 19}
%Kopfzeile rechts bzw. außen
\fancyhead[R]{\calligra\Large 17.12.2012}
% **************************************************
%
\chapter{Differenzialrechnung}
\sS{Definition Differenzialrechnung}
Sei $I$ ein Intervall:\\*
Eine Funktion $f:I→\R $ heißt \ul{$x_0\in I$ differenzierbar}, wenn der Grenzwert existiert $$f'(x_0):=\lim_{x→x_0}\frac{f(x)-f(x_0)}{x-x_0}$$ und $f'(x_0)$ heißt \ul{Ableitung} von $f$ in $x_0$\\*
$f$ heißt differnzierbar, wenn $f$ in jedem $x_0\in I$ differenzierbar ist.
\sss{Andere Bezeichnung}
$$f'(x_0)=\frac{df}{dx}(x_0)=Df(x_0).$$
\sss{Geometrische Interpretation}
Der Differenzialquotient $$ \frac{f(x)-f(x_0)}{(x-x_0)} $$ ist Steigung der Geraden durch die Punkte $ (x,f(x)),\ (x_0,f(x_0)) $ (Sekante)\\
\begin{tikzpicture}[scale=2,domain=-0.5:2, samples=200,prefix=plots/,smooth]
\draw[very thin, color=gray!50] (0,0) grid (2,4.49);
\draw[->] (-0.5,0) -- (2.5,0) node[right] {$x$};
\draw[->] (0,-0.5) -- (0,4.5) node[above] {$y$};
\draw[color=black] plot[id=19.1] function{x*x} node[right] {$f_1(x) =x^2$};
\draw[color=blue] (0,0) -- (1.5, 0) node[midway,below] {$x-x_0$};
\draw[color=red] (1.5,0) -- (1.5, 2.25) node[midway,right] {$f(x)-f(x_0)$};
\draw[color=black] (0,0) -- (1.5, 2.25);
\end{tikzpicture}\\*
$f'(x_0)$ (wenn existiert) ist die Steigung der \ul{Tangente} an $Γ_f$ im Punkt $(x_0,f(x_0))$
\bem
\desc{Schreibe}{$x=x_0+h$\\$h=x-x_0$}
$$\leadsto\ f'(x_0):=\lim_{h→0}\frac{f(x_0+h)-f(x_0)}{h}$$
\bsp
\enum{
\item{$f:\R→\R,\ f(x)=c$ konstante Funktion
$$f'(x_0):=\lim_{x→x_0}\frac{c-c}{\underbrace{x-x_0}_0}=0$$}
\item{$f:\R→\R,\ f(x)=a·x, a\eR$
$$f'(x_0):=\lim_{x→x_0}\frac{a·x-a·x_0}{x-x_0}=\lim_{x→x_0}a=a\ \Rarr f\text{ differenzierbar}$$}
\item{$f:\R→\R,\ f(x)=x^2$
$$f'(x_0):=\lim_{h→0}\frac{(x_0+h)^2-x_0^2}{h}=\lim_{h→0}\frac{2x_0h+h^2}{h}=\lim_{h→0}\frac{2x_0+h}{h}=2x_0\ \Rarr f\text{ differenzierbar}$$}
\item{$f:\R \bs\{0\}→\R,\ f(x)=\frac{1}{x}$ Sei $x_0≠0$
	\alg{f'(x_0):=\lim_{x→x_0}\frac{\frac{1}{x}+\frac{1}{x_0}}{x-x_0}=\lim_{x→x_0}\frac{\frac{x_0-x}{x·x_0}}{x-x_0}=\lim_{x→x_0}\frac{\cancel{x_0-x}}{\cancel{(x-x_0)}·x·x_0}&=\lim_{x→x_0}\frac{-1}{x·x_0}=-\frac{1}{x_0^2}\ \Rarr f\text{ differenzierbar},\\*
	&\left(\frac{1}{x}\right)'=-\frac{1}{x^2}}
}
\item{$f: \R\to\R,\ f(x) = |x|$\\*
\begin{tikzpicture}[scale=2,domain=-2:2, samples=200,prefix=plots/,smooth]
\draw[very thin, color=gray!50] (-2,0) grid (2,2.5);
\draw[->] (-2.5,0) -- (2.5,0) node[right] {$x$};
\draw[->] (0,-0.5) -- (0,2.5) node[above] {$y$};
\draw[color=black] plot[id=19.2] function{abs(x)} node[right] (1,1) {\footnotesize $f_2(x) =|x|$};
\end{tikzpicture}\\
$x_0 = 0$
$$f'(x_0):=\lim_{h→0}\frac{|h| - |0|}{h} = \frac{|h|}{h} \text{ existiert nicht, denn }\case{1 &h>0\\
-1 &h<0}$$
\Rarr{} $f$ ist nicht in $0$ differenzierbar.
}
\item{$exp: \R \to \R$ bekannt aus Satz 7.36\\*
Sei $x_0 \in \R$.\\*
$exp(x_0) = \lim_{h \to 0} \frac{exp(x_0 +h) - exp(x_0)}{h}$\\*
$\lim_{x \to 0} \frac{exp{x} - 1}{x} = 1 = \frac{exp(x) -exp(0)}{x-0}$
Das heißt: $exp'(0) = 1$. Insbesondere ist $exp$ differenzierbar in $0$}
\item{$sin:\R\bs\{0\}→\R$ Sei $x_0\eR$
\alg{\lim_{h→0}\frac{sin(x_0+h)-sin(x_0)}{h}&=\frac{1}{h}(sin(x_0)·cos(h)+cos(x_0)·sin(h)-sin(x_0))\\*
	&=\frac{1}{h}·sin(x_0)·(cos(h)-1)+cos(x_0)·\frac{sin(h)}{h}\\*
	&=sin(x_0)·\underbrace{\frac{(cos(h)-1)}{h}}_{→0\ für\ h→0}+cos(x_0)·\underbrace{\frac{sin(h)}{h}}_{→1\ für\ h→0} \text{ (Korollar zu Satz 7.36)}
}
Somit $sin'(x_0) = \lim_{h \to 0} \frac{sin(x_0 + h) - sin(x_0)}{h} = cos(x_0)$
\Rarr{} $sin$ ist differenzierbar, $sin' = cos$.}
\item{$cos: \R \to \R $ analog... cos' = -sin\\
\begin{tikzpicture}[domain=-2:7,prefix=plots/, smooth]
\draw[very thin,color=gray!50] (-2,-1.2) grid (7,1.2);
\draw[->] (-2,0) -- (7,0) node[right] {$x$};
\draw[->] (0,-1.2) -- (0,1.2) node[above] {$i$};
\draw[color=red] plot[id=sin1] function{sin(x)} node[above, pos=0.5] {\footnotesize $f_1(x) = sin(x)$};
\draw[color=blue] plot[id=cos1] function{cos(x)} node[below, midway] {\footnotesize $f_2(x) = cos(x)$};
\end{tikzpicture}\\*
$ sin' = cos $ \\*
$ cos' = -sin $}
}

\sS{Lemma}
Eine Funktion $f:I→\R$ ist genau dann in $x_0$ differenzierbar, wenn eine andere Funktion $\phi:I→\R$ existiert, sodass
\enum{
	\item{$f(x)-f(x_0)=\phi(x)·(x-x_0)$ für alle $x\in I$}
	\item{$\phi$ ist stetig in $x_0$}
}
Dann gilt $\phi(x_0) =f'(x_0)$
\bew
Definiere notwendig
$$\phi_0:I\bs\{x_0\}→\R,\ \phi(x)=\frac{f(x)-f(x_0)}{x-x_0}$$
Folgenstetigkeit: $ \phi_0$ hat eine Fortsetzung $\phi:I→\R$, die in $x_0$ stetig ist\ \equ\ $\ds\lim_{x→x_0}\phi_0(x)$ existiert, dann ist 
$$\phi(x_0)=\lim_{x→x_0}\phi_0(x)\ \equ\ \lim_{x→x_0}\frac{f(x)-f(x_0)}{x-x_0}$$
existiert, dann ist $\phi(x_0)=f'(x_0)\qed$

\sS{Satz}
Sei $f I \to \R eine Funktion$
\begin{enumerate}
\item{$f$ in $x_0$ differenzierbar \Rarr{} $f$ in $x_0$ stetig.}
\item{$f$ differenzierbar \Rarr{} $f$ stetig.}
\end{enumerate}
\bew
\begin{enumerate}
\item{Sei \phi wie im Lemma \Rarr{} $f(x) = f(x_0) + \phi(x)(x-x_0)$\\*
$\phi$ stetig in $x_0$ \Rarr{} $f$ stetig in $x_0$ \qed}
\item{folg aus 1.)}
\end{enumerate}

\uS{Berechnung der Ableitung}
\sS{Satz (Zusammengesetzte Ableitungen)}
Seien $f,\ g: I \to \R$ differenzierbar in $x_0 \in I$, dann sind auch $f + g$, $a \cdot f$, $f \cdot g: I \to \R$ in $x_0$ differenzierbar. ($a \in R$) und:
\begin{enumerate}
\item{$(f + g)'(x_0) = f'(x_0) + g'(x_0)$}
\item{$(a \cdot f)'(x_0) = a \cdot f'(x_0)$}
\item{$(f \cdot g)'(x_0) = f'(x_0) \cdot g(x_0) + f(x_0) \cdot g'(x_0)$}
\end{enumerate}
\bew
Zeige 3), 1) und 2) analog.
\alg{\lim_{x \to x_0} \frac{f(x) \cdot g(x) - f(x_0) \cdot g(x_0)}{x - x_0} &= \lim_{x \to x_0} \frac{f(x) \cdot g(x) - f(x_0) \cdot g(x_0)}{x - x_0} + \frac{f(x) \cdot g(x_0) - f(x_0) \cdot g(x_0)}{x - x_0}\\*
&= \lim_{x \to x_0} \left( \frac{f(x) \cdot g(x) - f(x_0) \cdot g(x_0)}{x - x_0} \right) + \lim_{x \to x_0}\left( \frac{f(x) \cdot g(x_0) - f(x_0) \cdot g(x_0)}{x - x_0} \right)\\*
&=f(x_0)g'(x_0) + f'(x_0)g(x_0)}
Weil $f$ stetig in $x_0$ und nach Definition der Ableitung. \qed{}\\*
\ul{Folge:} Für $n \geq 1$ $(x^n)' = n \cdot x^{n-1}$
\bew
mit vollständiger Induktion:
\ind{$(x^1) = 1 =1 \cdot x^0$}
{$(x^n+1) = n\cdot x^{n-1}$
$$(x^{n+1})' = x' \cdot x^n + x \cdot (x^n)' = 1 + x^n + x \cdot n \cdot x^{n-1} = (n+1) x^n$$}

\sS{Satz Kettenregel}
Sei $I,J$ Intervalle, $f:I→\R,\ g:J→\R$ Funktionen\\*
mit $f(I)\subseteq J \leadsto g\circ f:I→\R$ ist definiert\\*
$I\stackrel{f}{\longrightarrow}J\stackrel{g}{\longrightarrow}\R$\\*
$x_0\longmapsto f(x_0)$
\desc{Wenn}{$f$ in $x_0$ differenzierbar und\\$g$ in $f(x_0)$ differenzierbar,}
dann ist $g\circ f$ in $x_0$ differenzierbar,
und $(g\circ f)'(x_0)=g'(f(x_0))·f'(x_0)$\\
\ul{Beweisidee}
$$\frac{g(f(x)) - g(f(x_0))}{x-x_0} = \frac{g(f(x)) - g(f(x_0))}{f(x)-f(x_0)} \cdot \frac{f(x) - f(x_0)}{x-x_0}$$