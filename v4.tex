\wdh
Angeordneter Körper:\\
Menge $K$ mit $+, ·, <$\\
so dass gewisse Eigenschaften erfüllt sind
\bsp
\Q{} sind ein angeordneter Körper\\
Sei $K$ angeordneter Körper, $M\subseteq K$ Teilmenge $a\in K$ ist obere Schranke von $M$, wenn $U\subseteq a$, d.h.: $x\leq a\qquad ?x\in M$\\
$a\in K$ ist kleinste obere Schranke, wenn\\
\begin{enumerate}
\item{$M\leq a$}
\item{Wenn $b < a$, dann \underline{nicht} $M\leq b$}
\end{enumerate}
\vspace*{-9.5ex}\hspace*{15.5em}
$\left.
\begin{array}{l}
{}\vspace*{2ex}\\{}
\end{array}
\right\}$
\vspace*{-5ex}Bezeichnung $a=sup(M)$
\vspace*{5ex}
%
\bsp
$K=\Q\qquad M=\{-\frac{1}{n}|n\in\N\}=\{-1,-\frac{1}{2},-\frac{1}{3},?\}$\\
\sss{Behauptung}
$sup(M)=0$
\bew
\begin{enumerate}
\item {Zeige: $M \leq 0$, d.h.: $\frac{1}{n}<0$ für alle $n\in\N$\ok}
\item {Wenn $b=\Q,\ b<0$, dann nicht $M\leq b$}
\end{enumerate}
Schreibe $b=\frac{m}{n},\ m\in\Z, n\in\N$\\[1ex]
$b<0$ heißt $m<0,\ m\leq -1$\\[1ex]
$b=\frac{m}{n} \leq \frac{-1}{n} \leq \frac{-1}{n+1}\in M$\\[1ex]
\Rarr{} $M\not\leq b$ (nicht $M\leq b$)\qed
%
\uS{Vollständigkeit}
\Def
Ein angeordneter Körper $K$ heißt Dedekind-vollständig, wenn jede nach oben beschränkte Teilmenge von $K$ eine kleinste obere Schranke hat.
\Satz
Es gibt genau einen Dedekind-vollständigen, angeordneten Körper $K$\\
Dieser heißt Körper der reellen Zahlen\\
\underline{Bezeichnung:} \R\\
(Beweis ausgelassen)
%
\Satz
Die Teilmenge \N{} von \R{} ist unbeschränkt
\bew
(verwende nur die Axiome)\\
Indirekter Beweis: Angenommen, \N{} ist beschränkt\\
\einruck{Vollständigkeit:}{\vspace{3.65ex}\N{} hat eine kleinste obere Schranke $a\in\R$\\
Es gilt $a-1<a \Rarr{} a-1$ ist kleinste obere Schranke von $\N\ n\leq a\qquad ∀ n\in\N\\
\Rarr{} n+1\leq a\qquad ∀ n\in\N\\
\Rarr{} n\leq a-1\qquad∀n\in\N$ Widerspruch!\\
Also Annahme falsch, d.h. \N{} ist unbeschränkt\qed}
\begin{tabular}{lcl}
beschränkt &=& nach oben beschränkt und nach unten beschränkt\\
unbeschränkt &=& nicht nach oben beschränkt oder nicht nach unten beschränkt
\end{tabular}
%
\sS{Folgerung (Prinzip des Archimedes)}
Seien $x,y\in\R,\ x>0$, Dann gibt es $n\in\N$ mit $n·x>y$\\
SKIZZE % SKIZZE
%
\bew
$nx>y \equ n>\frac{y}{x}$ (weil $x>0$)\\
\N{} unbeschränkt und nicht nach oben beschränkt \Rarr{} $\frac{y}{x}$ ist keine obere Schranke von \N\\
\Rarr{} es gibt $n\in\N$ mit $n>\frac{y}{x}$\qed
%
\sS{Folgerung}
Sei $x\in\R,\ x>0$ Dann gibt es $n\in\N$ mit $\frac{1}{n}<x$\\
SKIZZE % SKIZZE
\bew
$\frac{1}{n}<x \equ 1<n·x \equ \frac{1}{x}<n$ (weil $x$ positiv)\\
$\frac{1}{x}$ keine obere Schranke von \N{} \Rarr{} es gibt \nN{} mit $\frac{1}{x}<n$\qed
%
\Satz
Seien $x,y\eR$ mit $x<y$\\
Dann gibt es $a\eQ$ mit $x<a<y$, man sagt \Q{} liegen dicht in \R\\
SKIZZE % SKIZZE
\bew
$y-x>0$ Wähle \nN{} mit $\frac{1}{n}<y-x$\\
Ansatz: $a=\frac{m}{n}$ mit $m\eZ$\\
Sei $M:=\{m\eZ|x<\frac{m}{n}\}=\{m\eZ|nx<m\}$\\
$M$ ist nach unten beschränkt und nicht leer (wegen Archimedes)\\
$M$ hat Minimum\\
Sei $m=min(M)$\\
$m\in M \Rarr x<\frac{m}{n}$\\
$m-1\not\in M \Rarr x\geq\frac{m-1}{n}$\\
$y-\frac{m}{n} =y-x+x-\frac{m}{n}>\frac{1}{n}+x-\frac{m}{n}=x-\frac{m-1}{n}\geq0$\\
$y>\frac{m}{n}$\qed
%
\uS{Wurzeln}
\Satz
Es gibt kein $a\eQ$ mit $a^2=2$\\
\bew
Angenommen $a\frac{m}{n}\eQ,\ a^2=2,\ m,\nN$\\
Kürze den Bruch $\Rarr \frac{m}{n}$ teilerfremd\\
$$a^2=2\Rarr \frac{m^2}{n^2}=2 \Rarr m^2=2n^2 \Rarr m^2 \text{ gerade } \Rarr m \text{ gerade } \Rarr m=2q,\ q\eN$$
$$(2q)^2=2n^2 \Rarr 4q^2=2n^2 \Rarr 2q^2=n^2 \Rarr  n^2 \text{ gerade } \Rarr n \text{ gerade }$$
Widerspruch zur Annahme $m,n$ teilerfremd\qed\\
SKIZZE WURZEL 2 \Rarr $\sqrt{2}$ sollte existieren % SKIZZE
\bem
Wenn \nN, keine Quadratzahl, dann gibt es kein $a\eQ$ mit $a^2=n$ (ähnlicher Beweis)
%
\Satz
Sei $x\eR, x\geq 0, \nN$\\
Dann gibt es \underline{genau ein} $y\eR, x\geq 0$ mit $y^n=x$\\
Bezeichnung: $x=\sqrt[n]{x}$
\bew
später\\
\underline{Ansatz:} $sup\{a\eQ|a^n\leq x\}=:y$ (sup existiert weil \R{} Dedekind-vollständig)
%
\Def
Sei $x\eR,\ x>0\qquad \frac{m}{n}\eQ$\\
\nN, $m\eZ\qquad x^{\frac{m}{n}}=\sqrt[n]{x^m}\qquad x^{\frac{1}{n}}=\sqrt[n]{x}$\\
\sss{Potenzrechnung:} $$x^{(a+b)}=x^a·x^b,\ x^{a·b}=(x^a)^b$$\hfill für $x\eR,\ x>0,\ a,b\eQ$\\
\bem
Später wir definiert: $x^a$ für $x\eR,\ x>0,\ a\eR$
\chapter{Folgen und Reihen reeller Zahlen}

