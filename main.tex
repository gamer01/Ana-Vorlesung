\documentclass[a4paper,10pt]{scrreprt}
\usepackage[utf8]{inputenc}
\usepackage{amsmath,amssymb}
\usepackage{textcomp}
\usepackage[T1]{fontenc}
\usepackage{breqn}
\usepackage{dsfont}
\usepackage{calligra}
\usepackage{cancel}
\usepackage{tipa}

%Beschreibaren Seitenbereich definieren
\usepackage[left=2.5cm,right=2.5cm,top=2cm,bottom=2cm]{geometry}

%Kopf- und Fußzeile einfügen
\usepackage{fancyhdr}
\pagestyle{fancy}

%Kopfzeile Linie oben
\renewcommand{\headrulewidth}{0.5pt}

%Fußzeile mittig
\fancyfoot[C]{\thepage}

%Linie unten
\renewcommand{\footrulewidth}{0.5pt}

% *************************
% *** costum characters ***
% *************************

% Unicode Character definition
\DeclareUnicodeCharacter{221E}{\text{$\infty$}}
\DeclareUnicodeCharacter{00B7}{\text{$\cdot$}}
\DeclareUnicodeCharacter{2192}{\text{$\to$}}
\DeclareUnicodeCharacter{2026}{\text{\ldots}}
\DeclareUnicodeCharacter{00D7}{\text{$\times$}}


% ***********************
% *** costum commands ***
% ***********************

% kürzerer command um neue mathemathische commands (ohne parameter) zu erstellen
\newcommand{\mcmd}[2]{\newcommand{ #1 }{\text{$#2$}}} 
%kürzerer command um neue text commands (ohne parameter) zu erstellen
\newcommand{\tcmd}[2]{\newcommand{ #1 }{\text{#2}}} 
%kürzerer command um neue commands (ohne parameter, ohne environment) zu erstellen
\newcommand{\cmd}{\newcommand}


% Mengen
\mcmd{\N}{\mathds{N}}
\mcmd{\Z}{\mathds{Z}}
\mcmd{\Q}{\mathds{Q}}
\mcmd{\R}{\mathds{R}}
\mcmd{\C}{\mathds{C}}

% griechische Buchstaben
\mcmd{\z}{\zeta}
\mcmd{\e}{\varepsilon}

% Pfeile
\mcmd{\equ}{\Leftrightarrow}
\mcmd{\Rarr}{\Rightarrow}
\mcmd{\Larr}{\Leftarrow}

% Symbole
\cmd{\qed}{\hfill\text{$\blacksquare$}}
\mcmd{\bs}{\backslash}
\mcmd{\ba}{\backslash}

% Formelkürzel
\cmd{\bino}[2]{\text{$\ds\binom{#1}{#2} = \frac{#1!}{#2!\cdot(#1-#2)!}$}}

% Überschriften
\cmd{\sS}[1]{\section{#1}}
\cmd{\Def}{\sS{Definition:}}
\cmd{\Satz}{\sS{Satz:}}
\cmd{\uS}[1]{\section*{\underline{#1}}}
\cmd{\wdh}{\section*{Wiederholung}}
\cmd{\sss}[1]{\vspace{-4mm}\subsection*{\underline{#1}}}
\cmd{\bsp}{\sss{Beispiel:}}
\cmd{\Bsp}[1]{\vspace{-4mm}\subsection*{\underline{Beispiel:} #1}}
\cmd{\bem}{\sss{Bemerkung:}}
\cmd{\bew}{\sss{Beweis:}}
\cmd{\Bew}[1]{\vspace{-4mm}\subsection*{\underline{Beweis} #1:}}
\cmd{\anm}{\sss{Anmerkung:}}
\cmd{\ssss}[1]{\vspace{4mm}{\bf\underline{#1}}\\}

% Textkürzel
\cmd{\ok}{\marginpar{\checkmark}}
\cmd{\einruck}[2]{#1\vspace{-12pt}\begin{addmargin}{.05\textwidth}#2\end{addmargin}}
\cmd{\ind}[3]{\einruck{IA:}{#1\ok}\\\einruck{IV:}{#2}\\\einruck{IS: }{$n\to n+1$#3\qed}}
\cmd{\notat}[1]{\einruck{\em Notation:}{\em#1}}

% Environments
\cmd{\ds}{\displaystyle}


% Author, title…
\title{Analysis Vorlesung}
\author{Stefan Heid, Christopher Jordan}
\date{\today}

% Less detailed TOC
\setcounter{tocdepth}{1}

\renewcommand*{\contentsname}{Inhaltsverzeichnis}

\begin{document}
%\maketitle
%\tableofcontents
%\newpage
%%Kopfzeile links bzw. innen
\fancyhead[L]{\calligra\Large Vorlesung Nr. 1}
%Kopfzeile rechts bzw. außen
\fancyhead[R]{\calligra\Large 8.10.2012}
% **************************************************
\chapter{Mengen}
\Def
\begin{enumerate}
\item Eine Menge ist eine Ansammlung verschiedener Objekte
\item Die Objekte in einer Menge heißen \underline{Elemente}\\
%
\notat{
a $\in$ M heißt a ist Element der Menge M\\
a ${\not\in}$ M heißt a ist kein Element der Menge M}
%
\item Sei M eine Menge. Eine Menge U heißt Teilmenge von M, von der jedes Element von U auch Element von M ist\\
%
\notat{
U $\subseteq$ M heißt U ist Teilmenge von M\\
U ${\not\subseteq}$ M heißt U ist keine Teilmenge von M}
\end{enumerate}
%
\sS{Beispiele}
\begin{enumerate}
\item {\einruck{Sei}{M die Menge aller Studierenden in L1\\W  die Menge aller weiblichen Studierenden in L1\\F die Menge aller Frauen}
Dann gilt: W $\subseteq$ M, W $\subseteq$ F, M ${\not\subseteq}$ F, F ${\not\subseteq}$ M}
\item {Die Menge der natürlichen Zahlen
$\N = \{1,2,3,4 …\}$
G sei die Menge der geraden natürlichen Zahlen
$G := \{n \in \N | $n ist gerade$\} = \{2m | m \in \mathds{N}\} = \{2,4,6,8 …\}$
Es gilt G $\subseteq \N, \N \subseteq$ G}
\item {Die Menge der ganzen Zahlen
$\Z = \{0,1,-1,2,-2,3,-3, …\}$}
\item {Die Menge der rationalen Zahlen
$\mathds{Q} = \{a/b | a, b \in \mathds{Z}, b \neq 0\}$}
\item {Die Menge ohne Element heißt die leere Menge
Symbol: $\emptyset = \{\}$}
\end{enumerate}
%
\bem
\begin{enumerate}
\item Für jede Menge M gilt $\setminus \subseteq M$
\item $\N \subseteq \Z \subseteq \Q$
\end{enumerate}

\sS{Definition: Sei M eine Menge und U,V $\subseteq$ M Teilmengen}
\begin{enumerate}
\item Die Vereinbarung von U und V ist $U \cup V := \{x \in M \mid x \in U oder x \in V\}$
\item Der Durchschnitt von U und V ist $U \cap V := \{x \in M \mid x \in U oder x \in V\}$
U und V heißen disjunkt, wenn $U \cap V = \emptyset$
\item Die Differenzmenge von U und V ist $U \setminus V := \{x \in U \mid x \in V\}$
\item Das Komplement von U ist $U^C = M \setminus U = \{x \in M \mid x {\not\in} U\}$
%
\einruck{Bsp: }{Sei M = N \\
$\{1,3\} \cup \{3,5\} = \{1,3,5\}$\\
$\{1,3\} \cap \{3,5\} = \{3\}$\\
$\{1,3\} \cap \{2,4,7\} = \emptyset \leftarrow$ disjunkt\\
$\{1,2,3\} \setminus \{3,4,5\} = \{1,2\}$\\
$\{1,3,5\}^C = \{2,4,6,7,8,…\}$}
\end{enumerate}
%
\sS{Satz (de Morjensche Regeln)}
Sei M eine Menge, U,V $\subseteq$ M Teilmengen\\
Dann:
\begin{enumerate}
\item $(U \cap V)^C = U^C \cup V^C$
\item $(U \cup V)^C = U^C \cap V^C$
\end{enumerate}
%
\bew
\begin{enumerate}
\item Sei x$ \in M\\$Es gilt: x $\in (U \cap V)^C \Leftrightarrow x {\not\in} U \cap V \Leftrightarrow x {\not\in} $U oder x$ {\not\in} $V$\Leftrightarrow x \in U^C$ oder x$\in V^C \Leftrightarrow x\in U^C \cup V^C$
\item Sei x$ \in M\\$Es gilt: x $\in (U \cup V)^C \Leftrightarrow x {\not\in} U \cup V \Leftrightarrow x {\not\in} $U und x$ {\not\in} $V$\Leftrightarrow x \in U^C$ und x$\in V^C \Leftrightarrow x\in U^C \cap V^C$
\end{enumerate}
%
\section{Prinzip der Vollständigen Induktion}
Für jedes $n \in \N$ sei eine Aussage A(n) gegeben\\
Ziel: Beweisen, Dass A(n) für jedes $n \in \N$ mehr ist dafür reicht es zu zeigen
\begin{enumerate}
\item Induktionsanfang (IA): A(1) ist wahr
\item Induktionsschrit (IS): Wenn für ein $n \in \N$ A(n) wahr ist, dann ist auch A(n+1) wahr
\end{enumerate}
%
\Satz
Für jede natürliche Zahl n gilt: $\ds 1+2+3+4+5+…+n=\frac{n(n+1)}{2}$\\
Probe:\\
\begin{tabular}{r|c|c|c|c}
n & 1 & 2 & 3 & 4\\ \hline\hline
1+2+3...+n & 1 & 3 & 6 & 10\\ \hline
$\ds \frac{n(n+1)}{2}$ & 1 & 3 & 6 & 10\\
\end{tabular}
\sss{Beweis des Satzes mit Induktion}
Abkürzung: $S(n) := 1+2+3+…+n$
Aussage: A(n): $\ds S(n) = \frac{n(n+1)}{2}$
\begin{enumerate}
\item {Induktionsanfang (IA): n=1 $S(1) = 1 = \dfrac{1·2}{2}$\marginpar{ok!}}
\item {Induktionsschritt (IS): $n → n+1$\\
Annahme: A(n) gilt: $\ds S(n) = \frac{n(n+1)}{2}$\\
Zu zeigen: A(n+1) gilt: $S(n+1)=\frac{(n+1)\cdot(n+2)}{2}$\\
$\ds S(n+1)=S(n)+n+1=\frac{n(n+1)}{2}+\frac{2(n+1)}{2}=\frac{(n+2)(n+1)}{2}$\\
Das beendet den Beweis} \qed
\end{enumerate}
Zur Vereinfachung der Notation:\\
Seien $a_1,a_2,a_3,...,a_n$ Zahlen $n \in \N$\\
Setze: $\sum_{k=1}^n a_k := a_1+a_2+a_3+…+a_n$\\
\begin{tabbing}
Allgemeiner: \=Sei $l,m \in \mathds{N}$, $l \le m \le n$\\
\>$\sum_{k=l}^m a_k = a_l+a_{l+1}+…+a_m$\\
\end{tabbing}
Aussage des Satzes:
\[ \sum_{k=1}^n k = \frac{n(n+1)}{2}\]
\hfill\underline{Kombinatorik} (mathematisches Zählen)

\Def
Seien A, B Mengen. Das kartesische Produkt von A und B ist definiert als $A × B := \{(a,b)|a\in A, b \in B\}$ Die Elemente von $A × B$ heißen geordnete Paare\\
Bsp.: $\{1,7\}\times \{2,3\}=\{(1,2),(1,3),(7,2),(7,3)\}$\\
Allgemeiner: Gegeben seien Mengen
$A_1,…,A_k$ mit $k \in \N$. Das kartesische Produkt von $A_1,…,A_k$ ist $A_1\times …\times A_k = \{(a_1,…,a_k)|a\in A, $für $i=1,…,k\}$\\
Elemente von $A_1 × … × A_k$ heißen k-Tupel\\
Falls $A_1=A_2=…=A_k=A$, schreibe $\underbrace{A\times…\times A}_{k-mal}=A^k$

\section{Definition}
Eine Menge A ist endlich, wenn A nur endlich viele Elemente hat. Dann bezeichnet
$\#A = \{|A|\}$ die Anzahl der Elemente von A und somit dessen Kardinalit\"at
oder M\"achtigkeit. Wenn A nicht endlich ist, so schreibe: $\# A= \infty$\\
Bsp.: $\#\emptyset = 0, \#\mathds{N}=\infty, \# \{1,3,5\} = 3$

\section{Bemerkung}
\begin{enumerate}
\item Sei A endliche Menge. $U,V\subseteq A$ disjunkte Teilmengen\\
Dann $\#(U\cup V)=\# U + \# V$ 
\item Seien $A_1,...,A_k$ endliche Mengen $k \in \mathds{N}$\\
Dann: $\#(A_1 \times ... \times A_k)=(\#A_1)(\#A_2)...(\#A_k)$
\end{enumerate}

\section{Definition}
\begin{enumerate}
\item Für $n\in \mathds{N}$ setze $n!=1\cdot 2\cdot 3\cdot ... \cdot n=\prod_{k=i}^n k$
Setze $0!=1$
\item Für $k,n\in \Z$ mit $0\le k \le n$ sei ${n \choose k} := \frac{n!}{k!\cdot(n-1)!}$ $\leftarrow$ Binomialkoeffizient\\
\begin{tabular}{r|c|c|c|c|c|c|c}
n & 0 & 1 & 2 & 3 & 4 & 5 & 6\\ \hline
n! & 1 & 1 & 2 & 6 & 24 & 120 & 720
\end{tabular}\\
\bsp
${5 \choose 2} := \frac{5!}{2!\cdot 3!} = \frac{5\cdot 4 \cdot\cancel{3 \cdot 2 \cdot 1}}{2\cdot 1\cdot \cancel{3\cdot 2}\cdot 1 } = \frac{20}{2}=10$\\
Bemerkung: ${ n \choose 0 }= 1 = {n \choose n}$
\end{enumerate}\newpage
%Kopfzeile links bzw. innen
\fancyhead[L]{\calligra\Large Vorlesung Nr. 2}
%Kopfzeile rechts bzw. außen
\fancyhead[R]{\calligra\Large 11.10.2012}
% **************************************************
%
\wdh
Sei M Menge.\\
Wenn M endlich: $\#M=Anzahl$ $Elemente\in M$\\
Wenn M unendlich: $\#M=\infty$\\
Für $n\in \N:=\{1,2,3,\ldots\}$\\
$$n!=1 · 2 · 3 · 4 · … · n \qquad 0!=1$$
Binomialkoeffizient: Für $0\leq k\leq n$\\
$$\bino{n}{k} \qquad\qquad\qquad \bino{n}{0}=\bino{n}{n}=1$$
%
\subsection{Lemma}
Für $0<k< n$ gilt:
$$\binom{n}{k} = \binom{n -1}{1} + \binom{n-1}{k}$$\\
%
\bew 
$$\binom{n-1}{k-1}+ \binom{n-1}{k}=\frac{(n-1)!}{(k-1)!·(n-k)!} +\frac{(n-1)!}{(k-1)!·(n-1-k)!} = \frac{k(n-1)!+(n-k)\cdot(n-1)!}{k! (n-k)!}=\frac{n(n-1)!}{k!(n-k)!}$$
%
\sS{Geometrische Anordnung (Pascalsches Dreieck)}
\parbox{0.5\textwidth}{\centering
$\binom{0}{0}$\\
$\binom{1}{0} \binom{1}{1}$\\
$\binom{2}{0} \binom{2}{1} \binom{2}{2}$\\
$\binom{3}{0} \binom{3}{1} \binom{3}{2} \binom{3}{3}$\\}
\parbox{0.5\textwidth}{\centering
1\\
1 1\\
1 2 1\\
1 3 3 1\\}\\[5mm]
Folge $\binom{n}{k}\in \N$ für alle $0\leq k\leq<n$
%
%
\Satz
Sei A endliche Menge. $\#A=n$\\[4pt]
Sei $k\in\Z$ mit $0\leq k\leq n$\\[4pt]
$P_k(A):=\{U\subseteq A| \#U=k\}$ (Menge aller k-elementigen Teilmengen von A)\\[4pt]
Dann gilt $\#P_k(A)=\binom{n}{k}$\\
\bsp
$A=\{1,2,3,4\}$ $n=4$ $k=2$\\[4pt]
2-elementige Teilmengen von A:
$\{1,2\}, \{1,3\}, \{1,4\}, \{2,3\}, \{2,4\}, \{3,4\} \to 6\qquad \binom{4}{2}=6$ \ok
%
\bew
Vorüberlegung: Sei $k=0 \vee k=n$\\
$P_0(A)=1=\binom{n}{0}$ $\#P_n(A)=1=\binom{n}{n}$\ok\\
Jetzt: Induktionsbeweis nach n\\[4pt]
\ind{$n=0$ Dann $k=0$}{Sei $\#A=n+1 \Rarr 0 \leq k \leq (n+1)$
Falls $k = 0\vee k = n + 1$\\
Sei also: $o < k < n + 1$\\
Wähle $a\in A$\\
Sei $B=A\bs\{a\}$\\
Dann $A=B\cup\{a\}, \#B=n$\\
Man kann die Wahl einer k-elementigen Teilmenge von A so strukturieren
\begin{enumerate}
\item Entscheiden, ob $a\in U \vee a\notin U$
\item\begin{enumerate}
\item Wenn $a\notin U$: Wähle k Elemente aus B
\item Wenn $a\in U$: Wähle k-1 Elemente aus B
\end{enumerate}
\end{enumerate}
$$\Rarr\ \#P_k(A)=\#P_k(B)+\#P_{k-1} (B) \stackrel{IV}{=} \binom{n}{k} + \binom{e}{ -1} \stackrel{1.11}{=} \binom{n+1}{k}$$
}
%
\sS{Satz (Binomische Formel)}
Seien $a,b$ Zahlen, $n\in\N$\\
Dann $(a+b)^n=a^n+\binom{n}{1} a^{n-1} b+\binom{n}{2}a^{n-2}b^2+…+b^n$
%
\bsp
$(a+b)^4=a^4+4a^3b+6a^2b^2+4ab^3+b^4$\\
$(a+b)^2=a^2+2ab+b^2$
%
\bew
Schreibe $(a+b)^n=\underbrace{(a+b)(a+b)(a+b)(a+b)…(a+b)}_{n-Faktoren}$
%
\sss{Ausmultiplizieren}
Halte Terme der Form $a^{n-k}b^k$ mit $0\leq k\leq n$\\
Häufigkeit von $a^{n-k}b^k$ = Anzahl der Möglichkeiten aus n-Faktoren k mal b zu wählen.\\
Das ist $\binom{n}{k}$ (Satz 1.13)
%
\sss{Folgerung}
Setze $a=b=1\qquad a^{n-k}b^k=1$\\
$(a+b)^n=2^n=\binom{n}{0}+\binom{n}{1}+\binom{n}{2}+…+\binom{n}{n}$\\
%
\bsp
$1+4+6+4+1=16=2^4$
%
\sS{Definition}
Sei A endliche Menge\\
Eine Anordnung von A ist ein n-Tupel\\
$(a_1,a_2,a_3,a_4,…,a_n)$ mit $a\in A$ für alle i und $a_i\neq a_j$ wenn $i\neq j$\\

\bsp
Anordnung von $\{1,2,3\}=(1,2,3)(1,3,2)(2,1,3)(2,3,1)(3,1,2)(3,2,1)→6$
%
\sS{Satz}
Sei $A$ endliche Menge, $\#A=n\geq 1$\\
Dann ist die Anzahl der Anordnungen von $A$ gleich $n!$
\bew
Induktion nach n
\ind{n=1}{Sei $\#A=n+1$\\
Wahl einer Anordnung von $A$ kann man so unterteilen:\\
\begin{enumerate}
\item{Wähle 1 Element $a_1\in A$ (n+1 Möglichkeiten)}
\item{Wähle Anordnungen von $A\bs\{a_1\}$\\
$\#(A\bs\{a_1\})=n$ \Rarr $n!$ Möglichkeiten bei 2\\
Insgesamt $(n+1)·n!=(n+1)!$}
\end{enumerate}}
%
\bem
(Zusammenhang zwischen Anordnung und Teilmengen)\\
Sei $A$ endliche Menge, $\#A=n,\ 0\leq k\leq n$\\
Sei $(a_1,…,a_n)$ Anordnung von $A$\\
$\leadsto$ Teilmenge $U:=\{a_1,…,a_n\}$\\
Dann $U\subseteq A,\ \#U=k\qquad U\in P_k(A)$\\
Jedes $U\in P_k(A)$ entsteht so, aber mehrfach:\\
\[\underset{\overset{\uparrow}{Anordnungen\ von\ U}}{k!}·\underset{\overset{\uparrow}{Anordnungen\ von\ A\backslash U}}{(n-k)!}-mal\]
$\#$ Anordnungen von $A=n!=\#P_k(A)·k!(n-k)!\Rarr\#P_k(A)=\frac{n!}{k!·(n-k)!)}=\binom{n}{k}$\\
%
\chapter{Die reellen Zahlen}
Was sind die reellen Zahlen?\\
Präzise Konstruktion ist umfangreich, daher Axiomatischer Zugang\\
Beschreibung der reellen Zahlen durch ihre Eigenschaften (Axiome):\\
\begin{enumerate}
\item{Grundrechenarten → Körper}
\item{Ungleichungen → angeordneter Körper}
\item{Lückenlosigkeit → Vollständigkeit}
\end{enumerate}
%
\uS{Körper}
%2.1
\Def
Ein Körper ist eine Menge $K$ mit 2 Rechenoperationen:\\
Addition (+) und Multiplikation (·), so dass folgende 9 Eigenschaften erfüllt sind:\\[8pt]
\underline{Addition}\\[-15pt]
\begin{enumerate}
\item{$(a+b)+c=a+(b+c)$ für alle $a,b,c\in K$ (Assotiativgesetz)}
\item{$a+b=b+a$ für alle $a,b\in K$ (Kommutativgesetz)}
\item{Es gibt ein $0\in K$ so dass $0+a=a$}
\item{Für jedes $a\in K$ gibt es ein $b\in K$ mit $a+b=0$}
\bem
$0\in K$ ist eindeutig
\bew
Wenn $0'\in K$ mit $0'+a=a$, dann $0=0'+0=0+0'=0'$\qed
\bem
Das $b$ in 4. ist auch eindeutig.\\
\notat{$b=-a$ (Negatives von $a$)}
\bew
Angenommen $b'+a=0$\\
$b=b+0=b+(a+b')=(b+a)+b'=0+b'=b'$\qed
\end{enumerate}
\underline{Multiplikation}\\[-15pt]
\begin{enumerate}
\setcounter{enumi}{4}
\item{$a(b·c)=(a·b)c\qquad ∀a,b,c\in K$}
\item{$a·b=b·a\qquad ∀a,b\in K$}
\item{Es gibt ein $1\in K$ mit $1\neq 0$, so dass $1·a=a\qquad ∀a\in K$}
\item{Für alle $a\in K,\ a\neq 0$, gibt es ein $b\in K$ mit $a·b=1$}
\bem
$1\in K$ ist eindeutig, $b$ in 8. ist eindeutig\\
Beziehung $b=a^{-1}$
\bew
Wie eben\qed
\item{$a(a+c)=a·b+a·c\qquad ∀a,b,c\in K$ (Distributivgesetz)}
\end{enumerate}
Weitere Bezeichnungen:\\
$a-b:=a+(-b),\ \frac{a}{b}=a·b^{-1}$, wenn $b≠0$
\bem
Die üblichen Rechenregeln folgen aus diesen Axiomen 1.-9.
\bsp
$$-(-a)=a,\ a(b-c)=a·b+a·c,\ a(-b)=-(a·b)$$
%
\sS{Beispiele}
\Q ist ein Körper\\
\Z ist kein Körper (8. nicht erfüllt)
%
\sS{Beispiel}
$\mathbb{F}_z=\{0,1\}$\\
\underline{Definitionen von + und · :}\\[8pt]
\parbox{.2\textwidth}{\begin{tabular}{c|cc}
+&0&1\\[2pt]\hline
0&0&1\\[2pt]
1&1&0
\end{tabular}}
\parbox{.2\textwidth}{\begin{tabular}{c|cc}
·&0&1\\[2pt]\hline
0&0&0\\[2pt]
1&0&1
\end{tabular}}
$1+1=0$\\[4pt]
\fbox{\underline{Übung:} Prüfe alle Körperaxiome}\\
\bem
Sei $K$ \underline{endlicher} Körper\\
Dann gilt $\#K=p^r$ wobei $p$ Primzahl, $r\in\N$\\
Für jede solche Zahl $q=p^r$ gibt es genau einen Körper

\newpage
%\chapter{Angeordneter Körper}
%\chapter{Folgen}
%\chapter{Konvergenzsätze}
%%Kopfzeile links bzw. innen
\fancyhead[L]{\calligra {\Large Vorlesung Nr. 6}}
%Kopfzeile rechts bzw. außen
\fancyhead[R]{\calligra 25.10.2012}



\setcounter{chapter}{3}
\setcounter{section}{9}

\subsection*{Wiederholung / Ergänzung}
Eine Folge reeler Zahlen $(a_n)$ konvergiert uneigentlich gegen ∞ wenn gilt:\\
Für jedes $C \in \R$ gilbt es ein $n \in\N$ mit $a_n > C$ für jedes $n \in\N$\\
\\
$(a_n)$ konvergiert uneigentlich gegen $- \infty$ wenn $(-a_n)$ gegen $\infty$ konvergiert.\\

\notat{
$a_n \to \infty \qquad \text{ für } n \to \infty$\\
$a_n \to - \infty \qquad \text{ für } n \to \infty$
}

\bsp
$a_n = n^2 \to \infty$\\
$a_n = -n^2 \to -\infty$\\
$a_n = (-1)^n \cdot n^2$\\
$(0, -1, 4, -9)$ konvergiert weder gegen $\infty$ noch gegen $ - \infty$

\sss{Rechenregeln:}
Angenommen $(a_n), (b_n)$ sind konvergente Folgen.\\
\begin{enumerate}
\item{$(a_n + b_n) \to a + b$}
\item{$(a_n \cdot b_n) \to ab$}
\item{$\ds\frac{1}{b_n} \to \frac{1}{b}$}
\item{$c \cdot a_n \to c \cdot a$}
\item{$a_n - b_n \to a - b$}
\item{$\ds\frac{a_n}{b_n} \to \frac{a}{b}$}
\end{enumerate}

\noindent \underline{Beweis 6):}\\
3) $\Rightarrow \displaystyle\frac{1}{b_n} \to \displaystyle\frac{1}{b}$\\
$\displaystyle\frac{a_n}{b_n} = a_n \cdot \displaystyle\frac{1}{b}$\\
2) $\Rightarrow a_n \cdot displaystyle\frac{1}{b_n} \to a \cdot \displaystyle\frac{1}{b} = \displaystyle\frac{a}{b} \phantom{XXX} q.e.d.$\\ \\
\underline{Beispiel}\\\\

\begin{tabular}{l|c|c|c|c|c|r}
$n$   & 0 & 1 & 2 & 3 & 10 & 100\\\hline
$a_n$ & 0 & 0 & $\frac{2}{9}$ & $\frac{6}{19}$ & $\frac{90}{201}$ & $\frac{9900}{20001}$ \\
\end{tabular}
\vspace{5mm}\\
Vermutung: $a_n \to \displaystyle\frac{1}{2}$ für $n \to \infty$\\
Rechenregel 6 anwenden:\\
\begin{itemize}
\item[1.]{Versuch:\\
$a_n = \frac{b_n}{c_n}$\\ 
\\
$b_n = n^2 -n; c_n = 2n^2 + 1$\\
$(b_n) und (c_n)$ sind divergend. Schlecht.}
\item[2.]{Versuch:\\
$\displaystyle\frac{n^2 - n}{2n^2 + 1} = \displaystyle\frac{n^2(1 - \displaystyle\frac{1}{n})}{n^2(2 + \displaystyle\frac{1}{n^2}}$ für $n \geq 1$ \\ \\
$= \displaystyle\frac{1 - \frac{1}{n}}{2 + \frac{1}{n^2}} = \displaystyle\frac{b_n}{c_n}$ mit $b_n := 1 - \displaystyle\frac{1}{n}; c_n = 2 + \displaystyle\frac{1}{n^2}$\\ \\
$\displaystyle\frac{1}{n} \to 0$ für $n \to \infty $
\\ \\ $\Rightarrow 1 - \displaystyle\frac{1}{n} \to 1 - 0 = 1$ für $n \to \infty$
\\ \\ $\Rightarrow 2 + \displaystyle\frac{1}{n^2} \to 2 + 0 = 2$ für $n \to \infty$}
\end{itemize}

$\Rightarrow a_n \to \frac{1}{2}$ für $n \to \infty$\\

\section{Satz}
Seien $a_n \to a$, $b_n \to b$ zwei konvergente Folgen reeler Zahlen.\\
wenn $a_n \leq b_n$ für unendlich viele $n \in \mathbb{N}$ dann ist $a \leq b$. \\
\underline{Beweis:}\\
Angenommen: $a > b$\\

Wähle $\epsilon := \displaystyle\frac{a - b}{2} > 0$\\
Es gibt $N \in \mathbb{N}$ so dass:
$
\left.
\begin{array}{ll}
\mid a_n - a \mid  < \epsilon \\
\mid b_n - b \mid  < \epsilon
\end{array} \right\rbrace$ für $n \geq N$\\
$\Rightarrow a_n > a - \epsilon$\\ \\
$= a - \frac{a - b}{2} = \displaystyle\frac{a + b}{2} = b + \displaystyle\frac{a - b}{2}\\
\\
= b + \epsilon > b_n \Rightarrow a_n > b_n$ für $n \geq \mathbb{N}$\\
Widerspruch zur Annahme.\\
$a_n \leq b_n$ für unendlich viele $n \in \mathbb{N} \hfill q.e.d.$

\section{Definition: Reihen}
Sei $(a_n)_{n \geq 0}$ eine Folge reeler Zahlen.\\
Bilde eine Folge:
\begin{align*}
s_0 &= a_0\\
s_1 &= a_0 + a_1\\
s_2 &= a_0 + a_1 + a_2\\
\vdots
s_n &= a_0 + a_1 + a_n = \sum\limits_{k = 0}^{n} a_k
\end{align*}
Die Folge $(s_n)_{n \geq 0}$ heißt Reihe mit den Gliedern $a_n$.\\
$s_n$ heißen die \underline{Partialsummen} der Reihe.\\
Bezeichnung:\\
$\sum\limits_{k = 0}^{\infty} a_k$ oder $a_0 + a_1 + a_2 + a_3 + ...$\\ \\
Wenn $s_n \to s \in \mathbb{R}$ für $n \to \infty$ dann schreiben wir:\\
$\sum\limits_{k = 0}^{\infty} a_k = s$\\
Summe der Reihe.\\
\\
\underline{Achtung:} Symbol $\sum\limits_{k = 0}^{\infty} a_k$ hat \underline{zwei} Bedeutungen:
\begin{enumerate}
\item{die Folge $(s_n)$} \\
oder 
\item{deren Grenzwert}
\end{enumerate}
\noindent \underline{Beispiele:}\\
\begin{enumerate}
\item{$\sum\limits_{k = 1}^{\infty} 1 = 1+1+1+...$\\
ist die Folge $(1, 2, 3, 4,...) = (n + 1)_{n \in \mathbb{N}_{0}}$}
\item{$\sum\limits_{k = 1}^{\infty} k = 0 + 1 + 2 + 3+ ...$ \\
ist die Folge $(1, 3, 6, 10,...) = (\displaystyle\frac{n(n - 1)}{2})_{n \in \mathbb{N}}$ }
\item{$\sum\limits_{k = 1}^{\infty} \displaystyle\frac{1}{k(k+1)} = \displaystyle\frac{1}{2} + \displaystyle\frac{1}{6} + \displaystyle\frac{1}{12} + ...$\\
ist die Folge $(\displaystyle\frac{1}{2}, \displaystyle\frac{2}{3}, \displaystyle\frac{3}{4})$}
\end{enumerate}
\vspace{5mm}
Vorüberlegung:\\
$\displaystyle\frac{1}{k(k+1)} = \displaystyle\frac{(k+1) - k}{k(k+1)} = \frac{1}{k} - \displaystyle\frac{1}{k + 1}$\\ \\
$s_n := \sum\limits_{k = 1}^{\infty} \displaystyle\frac{1}{k(k+1)}
= (\displaystyle\frac{1}{1} - \displaystyle\frac{1}{2}) + (\displaystyle\frac{1}{2} - \displaystyle\frac{1}{3}) + ... + (\displaystyle\frac{1}{n} - \displaystyle\frac{1}{n + 1})\\ \\
= 1 - \displaystyle\frac{1}{n + 1}\\ $ Teleskopsumme \\ \\
$\displaystyle\frac{1}{n + 1} \to 0$ für $n \to \infty$\\ \\
Summe der Reihe:\\ \\
$\sum\limits_{k = 1}^{\infty} \displaystyle\frac{1}{k(k+1)} = \lim_{n \to \infty}(1 - \displaystyle\frac{1}{n + 1}) = 1 \phantom{XXX} q.e.d.$\\ 
\\
\underline{Bemerkung:} Jede Folge kann man auch als Reihe Schreiben. (Differenzen bilden)\\
z.B.: die Folge der Primzahlen:\\
$(2, 3, 5, 7, 11, 13, 17, 19)$\\
ist die Reihe:\\
$(2 + 1 + 2+ 4+2+4+2+...)$\\
Goldbachsche Vermutung: in dieser Reihe kommt die Zahl 2 unendlich oft vor.\\
\section{Satz, Die geometrische Reihe}
Sei $x \in \mathbb{R}$\\
a) $ \sum\limits_{k = 0}^{\infty} x^k = 1 + x^1 + x^2 + x^3 + ... = \frac{1}{1-x} \text{ wenn } \mid x \mid < 1$\\
b) $ \sum\limits_{k = 0}^{\infty} x^k \text{ divergiert wenn } \mid x \mid \geq 1$\\
\begin{itemize}
\item[a] {wenn $|x| < 1$\\
dann folgt $\sum{k=0}{\infty} a_k = \displaystyle\lim_{n \to \infty}(\frac{1}{1 - x} - \frac{x}{1-x} \cdot x^n) = \frac{1}{1 - x}$}
\item[b]{wenn $|x| > 1$\\
dann $(x^n)$ divergent $\Rightarrow (\frac{x}{1-x} \cdot x^n)$ divergent\\
denn $\frac{x}{1-x} \neq 0$\\
$\Rightarrow (\frac{?}{?})$}
\end{itemize}
Beweis:\\
\begin{quote}
$x = 1 \phantom{xxx} \sum_{k = 0}^{\infty} x^k = (1 + 1 + 1 +...)\text{ divergiert, ok}\\
\text{Sei nun }x \neq\\
\text{Bekannt aus der Übung: } \displaystyle\sum_{k = 0}^{\infty} x^k = 1 + x + x^2 +x^3 ... +x^n = \displaystyle\frac{1 -x^{n+1}}{1 - x} = \displaystyle\frac{1}{1 - x} - \displaystyle\frac{x}{1 - x} \cdot x^n \\ $
\end{quote}
Potenzenwachstum\\
$x^n \to 0$ für $ n \to \infty$ \underline{wenn} $|x| < 1$\\
$(x^n)$ divergiert, wenn $(|x| \geq 1 \text{ und } x \neq 1)$\\

\section{Satz}
Wenn die Reihe $\displaystyle\sum\limits_{k=0}^{\infty} a_k $ kovergiert, dann ist $(a_n)_{n \in \mathbb{N}}$ eine Nullfolge.\\ \\
\underline{Beweis:} Gegeben sei $\epsilon > 0$\\
Sei $a = \displaystyle\sum\limits_{k = 0}^{\infty} a_k = $ $\displaystyle\lim_{n \to \infty}(s_n)$ mit $s_n = a_0 + ... + a_n$\\
Es gibt $ N \ in \mathbb{N}$ mit $|s_n - a| < \displaystyle\frac{\epsilon}{2}$ für $n \geq N$\\
$|a_n| = |s_n - s_{n-1}|$\\
\phantom{$|a_n| $} = $|s_n - a + a - s_{n-1}|$\\
\phantom{$|a_n| $} $\leq |s_n - a| + |a - s_{n-1}| < \displaystyle\frac{\epsilon}{2} + \displaystyle\frac{\epsilon}{2} = \epsilon$\\
für $n \geq N + 1$\\
$\Rightarrow a_n \to 0$ für $n \to \infty$\\

\section{Satz, die harmonische Reihe}
$\displaystyle\sum\limits_{k = 1}^{\infty} \frac{1}{k}= 1 + \frac{1}{2} + \frac{1}{3} + ...$ divergiert\\
\\
\underline{Beweisidee:}\\
\\
$\phantom{= }1 + \displaystyle\frac{1}{2} + \displaystyle\frac{1}{3} + \displaystyle\frac{1}{4} + \displaystyle\frac{1}{5} + \displaystyle\frac{1}{6} + \displaystyle\frac{1}{7} + \displaystyle\frac{1}{8} + \displaystyle\frac{1}{9} + ...$\\
$\phantom{\geq }1 + \displaystyle\frac{1}{2} + \displaystyle\frac{1}{4} + \displaystyle\frac{1}{4} + \displaystyle\frac{1}{8} + \displaystyle\frac{1}{8} + \displaystyle\frac{1}{8} + \displaystyle\frac{1}{8} + \displaystyle\frac{1}{16} + ...$\\
$\phantom{= }1 + \displaystyle\frac{1}{2} + \displaystyle\frac{2}{4} + \displaystyle\frac{4}{8} + \displaystyle\frac{8}{16} + ...$\\
$\phantom{= }1 + \displaystyle\frac{1}{2} + \displaystyle\frac{1}{2} + \displaystyle\frac{1}{2} + \displaystyle\frac{1}{2} + ... = \infty$\\
\end{document}














\newpage
%% Kopfzeile beim Kapitelanfang:
\fancypagestyle{plain}{
%Kopfzeile links bzw. innen
\fancyhead[L]{\calligra\Large Vorlesung Nr. 7}
%Kopfzeile rechts bzw. außen
\fancyhead[R]{\calligra\Large 29.10.2012}
}
%Kopfzeile links bzw. innen
\fancyhead[L]{\calligra\Large Vorlesung Nr. 7}
%Kopfzeile rechts bzw. außen
\fancyhead[R]{\calligra\Large 29.10.2012}
%
% set chapters end sections
%\setcounter{chapter}3
%
Sei $(n_n)$ eine Folge reeler Zahlen.\\
Die Reihe mit den Gliedern $a_n$ ist die Folge $s_n = a_0 + a_1 + ... + a_n)_\nN$ \\
Bezeichnung: $\ds\sum_{k=1}^{\infty} a_k$\\
Wenn $S_n \to a$ für $n \to \infty$\\
Schreibe: $\ds\sum_{k = 0}^{\infty} a_k = a$
\Bsp{Geometrische Reihe}
$\ds|x| = 1 \Rarr \sum\limits_{k = 0}^{\infty} x^k = \frac{1}{1-x}$ für $x = 0$ setzte $0^0 = 1$\\[4pt]
Harmonische Reihe\\
$\displaystyle\sum\limits_{k = 1}^{\infty} \frac{1}{k}$ Konvergiert nicht.\\
\sS{Satz Rechenregeln für Reihen}
Seien $\sum\limits_{k = 0}^{\infty} a_k = a$ und $\sum\limits_{k = 0}^{\infty} b_k = b$ zwei konvergente Reihen. Dann:
\begin{enumerate}
\item{$\sum\limits_{k = 0}^{\infty} (a_k + b_k) = a + b$}
\item{Für $c \in \R{}$ ist $\sum\limits_{k = 0}^{\infty} c \cdot a_k = c \cdot a$}
\end{enumerate}
\bew
folgt aus 3.9.
\bem
Produkte von Reihen sind komplizierter.\\
\underline{Korrektur:}
Primzahlen-Vermutung: es gibt ∞ viele Primzahlen $p$ so dass $p + 2$ auch Prim ist.\\
Goldbach-Vermutung: Jede gerade natürliche Zahl ist die Summe von zwei Primzahlen.\\
\chapter{Konvergenzsätze}
Erinnerung: \R{} ist Dedekind-vollständig. Das heißt, jede nicht-leere nach oben beschränkte Teilmenge $M \subset R$ hat eine kleinste obere Schranke $sup(M)$\\
\phantom{XXX}\Rarr{} Existenz von Grenzwerten\\
\sS{Definition Monotone Folgen}
Eine Folge $(a_n)_{n \geq 0}$ heißt monoton wachsend, wenn $a{n + 1} \geq a_n$ für alle $n \in \N_0$\\
\phantom{Eine Folge $(a_n)_{n \geq 0}$ heißt}monoton fallend, wenn $a{n + 1} \leq a_n$ für alle $n \in \N{}_0$\\
\phantom{Eine Folge $(a_n)_{n \geq 0}$ heißt}streng monoton wachsend, wenn $a{n + 1} > a_n$ für alle $n \in \N{}_0$\\
\phantom{Eine Folge $(a_n)_{n \geq 0}$ heißt}streng monoton fallend, wenn $a{n + 1} < a_n$ für alle $n \in \N_0$
\bsp
$a_n = n$ ist streng monoton wachsend\\
$a_n = \frac{1}{n}$ ist streng monoton fallend\\
\sS{Satz}
\begin{enumerate}
\item{Jede nach oben beschränkte monoton wachsende Folge $(a_n)_{\nN}$ ist konvergent\\
% Bild?
Hier Fehlt was, das Bild, der Tafel, auf dem das stehen sollte ist nicht auffindbar, hast du da noch eine Mitschrift?
}
\item{Jede nach unten beschränkte monoton fallende Folge $(a_n)_\nN$ ist konvergent\\
% Bild?
}
\end{enumerate}
\bew
Sei $(a_n)$ nach oben beschränkt, monoton wachsend\\
Setze $a:= sup(\{a_n | n \in \N{}\})$\\
dann \begin{enumerate}
\item{$a_n \leq a$ für alle $n$}
\item{Für jedes $\epsilon > 0$ ist $a - \epsilon$ \underline{keine} obere Schranke, d.h. es gibt $N \in N$ so dass $a_N > a - \epsilon$
\\Für $n \geq N$ gilt\\
$a - \epsilon < a_N \leq a_n \leq a$\\
weil $(a_n)$ monoton wachsend\\
$\Rightarrow a - \epsilon < a_N \leq a_n \leq a \Rightarrow |a_n -a| < \epsilon$\\
Somit $a_n \to a$ für $n \to \infty$\phantom{XXX}$q.e.d.$\\
Monoton fallend: analog}
\end{enumerate}
\uS{Reihen mit nicht-negativen Gliedern}
\bem
Sei $\ds\sum\limits_{k=0}^{\infty} a_k$ Reihe reeller Zahlen\\
Die Folge der Partialsummen ist monoton wachsend $\Leftrightarrow a_n \geq 0$ für $n \geq 1$
\Satz
Eine Reihe $\ds\sum\limits_{k=0}^{\infty} a_k$ mit $a_k \geq 0$ für ale $k$ konvergiert, genau dann, wenn sie beschränkt ist (Das heißt die Folge der Partialsummen ist beschränkt)\qed
\Def
Sei $\displaystyle\sum\limits_{k=0}^{\infty} a_k$ eine Reihe mit $a_n \geq 0$ für alle $k$\\
Eine Reihe $\ds\sum\limits_{k=0}^{\infty} b_k$ heißt \underline{Majorante} von $\ds\sum a_k$ wenn $a_k \leq b_k$ für alle $k$
\sS{Satz Majorantenkriterium}
Wenn eine Reihe mit nicht-negativen Gliedern eine konvergente Majorante hat, dann konvergiert sie.
\bew
Sei $0 \leq a_k \leq b_k$ für alle $k \geq 0$\\
Es gilt $a_0 + ... + a_n \leq b_0 + ... + b_n$ \\
$\sum b$ konvergiert $\Rightarrow (b_0 + ... + b_n)_{n \geq 0}$  beschränkt\\
$\Rightarrow ((a_0 + ... + a_n)_{n \geq 0})$ beschränkt $\Rightarrow \ds\sum_{k= 0}^{\infty} a_k$ konvergiert.
\Bsp{4.6:}
$$\sum\limits_{k=1}^{\infty} \frac{1}{k^2} = \left( 1 + \frac{1}{4} + \frac{1}{9}+ \frac{1}{16} + … \right)$$
$$\sum\limits_{k=1}^{\infty} \frac{1}{k^2} = 1 + \sum\limits_{k=1}^{\infty} \frac{1}{(k + 1)^2}$$
$$\frac{1}{(k + 1)^2} \leq \frac{1}{k \cdot (k + 1)}$$
$$\Rarr \sum\limits_{k=1}^{\infty} \frac{1}{k \cdot (k + 1)}\text{ ist Majorante von }\sum\limits_{k=1}^{\infty} \frac{1}{k^2}$$
$$\sum\limits_{k=1}^{\infty} \frac{1}{k \cdot (k + 1)}\text{ konvergiert (bekannt)}$$
\sS{Satz Quotientenkriterium}
Sei $C \in \R{}, (a_n)$ eine Folge reeller Zahlen mit $a_n \geq 0$ für alle $n$ \ul{und} $a_{n + 1} \leq C \cdot a_n$ für fast alle nder$n$\\
$0 \leq C \leq 1$\\
Dann konvergiert die Reihe $\ds\sum\limits_{k=0}^{\infty} a_k$
\bew
Konvergenz ändert sich nicht, wenn endlich viele $a_n$ geändert werden.\\
Also kann man annehmen, dass $a_{n + 1} \leq C \cdot a_n$ für alle $n$ gilt.\\
Dann gilt $a_1 < C \cdot a_0$\\
$a_2 < C \cdot a_1 \leq C \cdot C \cdot a_0 = C^2 \cdot a_0$\\
$a_3 < C \cdot a_2 \leq C \cdot C \cdot a_1 = C^3 \cdot a_0$\\
etc. $\Rightarrow a_n \leq C^n \cdot a_0$\\
Somit ist $\displaystyle\sum\limits_{k=0}^{\infty} C^k \cdot a_0$ konvergente Majorante von $\ds\sum_{k=0}^{\infty} a_k$ (Geometrische Reihe)
\sS{Beispiel Die Exponentialreihe}
$$exp(x) := \sum\limits_{k=0}^{\infty} \frac{x^k}{k!} \text{ für } x \in \R{}, x \geq 0$$
Setze $a_k = \frac{x^k}{k!}$\\
$$a_n+1 = \frac{x^{n + 1}}{(n + 1)!} = \frac{x}{n+1} \cdot \frac{x^n}{n!} = \frac{x}{n+1} \cdot a_n \leq \frac{1}{2} a_n$$
\Rarr{} Quotientenregel ist erfüllt.\\
Reihe $exp(x)$ konvergiert.\\
Bezeichnung: $$exp(x) = \sum_{k=0}^{\infty} \frac{x^k}{k!} \eR$$
%
\uS{Bezeichnung:}
$$exp(1) = \sum\limits_{k=0}^{\infty} \frac{1}{k!} = e\text{ (Eulerische Zahl)}$$
\sS{Leibnitz-Kriterium}
Sei $(a_n)_{n \in \N{}_0}$ eine Monoton monoton fallende Nullfolge\footnote{$a_n → 0$ für $n→∞$} mit $a_n \geq 0$ für alle $n$\\
Dann konvergiert die alternierende Reihe\\
$$\sum_{k=0}^{\infty} (-1)^k · a_k$$
\bsp
$$a_k = \frac{1}{k + 1} \sum_{k=0}^{\infty} (-1)^k · a_k = 1 - \frac{1}{2} + \frac{1}{3} - \frac{1}{4} + \frac{1}{5}= log(2)$$
\bew
Sei $s_n = a_0 + ... + a_n$
\uS{Behauptung: }
$S_{2n + 1} \leq S_{2n + 3} \leq S_{2n + 2} \leq S_{2n}$ für jedes $n \in \N{}$
%
\ul{Rechne:}\\
$S_{2n + 2} - S_{2n} = - a_{2n + 1} + a_{2n + 2} \leq 0 \Rightarrow (3)$\\
$S_{2n + 3} - S_{2n + 1} = - a_{2n + 3} \leq 0 \Rightarrow (2)$\\
$S_{2n + 3} - S_{2n + 1} = - a_{2n + 2} - a_{2n + 3} \leq 0 \Rightarrow (1)$\\[8pt]
Die Folge $b_n = S_{2n}$\\
\phantom{Die Folge }$c_n = S_{2n + 1}$\\
sind beschränkt und monoton (fallend bzw. steigend)\\
$\Rightarrow b_n \text{ und } c_n$ konvergieren\\[4pt]
Sei $$b = \lim_{n \to \infty} b_n \qquad c = \lim_{n \to \infty} c_n$$
$$c - b = \lim_{n \to \infty} (c_n - b_n) = \lim_{n \to \infty} (a_{2n + 1}) = 0$$
weil $(a_n)$ Nullfolge
\sS{Behauptung:} % uS stande dort ich vermute so
$S_n \to b$ für $n \to \infty$\\[4pt]
Gegeben sei $\e > 0$. Es gibt $N \in \N{}$ so dass für $n \leq N$:\\
$|b_n - b| < \e, |c_n - c| < \epsilon$\\
Somit für $n \geq 2N+1 $\\
$|S_n - b| < \e$ also $S_n \to b$\qed\newpage
%%Kopfzeile links bzw. innen
\fancyhead[L]{\calligra {\Large Vorlesung Nr. 8}}
%Kopfzeile rechts bzw. außen
\fancyhead[R]{\calligra \Large{05.11.2012}}
% **************************************************
%
\Wdh{Konvergenzsätze}
\begin{itemize}
    \item{Eine monoton wachsende und beschränkte Folge konvergiert zwangsläufig.}
    \item{Eine Reihe $\sum_{k=0}^{∞} a_k$ mit $a_k \geq 0$ für alle $k$ konvergiert \equ die Folge der Partialsummen $(S_n = \sum_{k=0}^{n} a_k)_{n \in \N}$ ist beschränkt}
\end{itemize}
%
\bsp
    $\sum_{k=0}^{n} \frac{1}{k} = 1 + \frac{1}{2} + \frac{1}{3} …$ ist unbeschränkt\\
\bsp
    $\sum_{k=1}^{\infty} \frac{1}{k^2} = 1 + \frac{1}{4} + \frac{1}{9}+…$\\
\sss{Leibnitz:} 
    Sei $(a_n)$ monoton fallende Nullfolge.\\
    Dann konvergiert $\sum_{k=0}^{∞} (-1)^k \cdot a_k$\\
\bsp
    $(1 - \frac{1}{2} + \frac{1}{3} - \frac{1}{4})$ … konvergiert.
% satz 4.10 ***************
%\setcounter{chapter}{4}
%\setcounter{section}{9}
% *************************
\sS{Satz Verdichtungslemma von Candy}
Sei $(a_n)$ monoton fallende Nullfolge.\\
Die Reihe $\sum_{k=0}^{\infty} a_k$ konvergiert genau dann, wenn die verdichtete Reihe $\sum_{k=0}^{\infty} 2^k \cdot a_{2^k} = 1 · a_1 + 2 · a_2 + 4 \cdot a_4$ … konvergiert.\\
%
\bsp
    $a_k = \frac{1}{k}\qquad (k \geq 1)$\\
    $2^k \cdot a_{2^k} = 2^k \cdot \frac{1}{2^k} = 1$\\
%
\sS{Satz:} % sss orginal, ich vermute es ist so
$\sum_{k=0}^\infty \frac{1}{k}$ konvergent \equ $\sum_{k=0}^\infty 1$ konvergent (ist nicht der Fall.)\\
%underline{Beweis:}
\bew
Sei $b_n = \ds\sum_{k=2^n}^{2^{n+1}-1} a_k$\\
Für $\ds 2^n \leq k \leq 2^{n+1} - 1$ ist $\ds a_{2^n} \geq a_{k} \geq a_{2^{n+1} - 1} \geq a_{2^{n+1}}$ \Rarr $\ds 2^n · a_{2^n} \geq b_n \geq  2^n · a_{2^{n+1}}$\\
Wenn $\ds \sum_{k \geq 0} 2^k · a_2^k$ beschränkt \Rarr $\ds \sum_{k \geq 0} b_k$ beschränkt \Rarr $\ds \sum_{k \geq 0} a_k$ beschränkt\\
Hier immer beschränkt \equ konvergent\\
Wenn $\ds \sum_{k \geq 0} 2^k · a_k$ beschränkt \Rarr $\ds \sum_{k \geq 0} b_k$ beschränkt \Rarr $\ds \sum_{k \geq 0} 2^{k} · a_2^{k+1}$ beschränkt \equ $\ds \sum_{k \geq 0} 2^{k+1} · a_2^{k+1}$ beschränkt \equ $\ds \sum_{k \geq 0} 2^{k} · a_2^{k}$ beschränkt\\
Das zeigt den Satz.\\
\ssss{Anwendung:}
\ssss{Erinnerung:}
    Für $ x\geq 0$ und $a \in \R$ wird später $x^a \in R$ definiert\\
    Wenn $\ds a = \frac{n}{m}$ mit $m \geq 1$ d.h. $a \in \Q$ dann $\ds x^a = \sqrt[m]{x^n}$.\\
    Wenn $x > 1$ dann gilt: \\
    $x^a = \begin{cases} >1 \mbox{ wenn }a>0\\ =1 \mbox{ wenn }a=0\\ <1\mbox{ wenn }a<0 \end{cases}$
%
\sS{Satz}
Sei $a \in \R$. Die Reihe $\ds \sum_{k=1}^{\infty} \frac{1}{k^a}$ konvergiert genau dann, wenn $a > 1$\\
\bew
Wenn $a \leq 0$ dann $\frac{1}{k^a} \geq 1$ \Rarr Reihe divergiert.\\
Sei $a >0$, sei $a_n = \frac{1}{n^a}$ \\
$\ds n < n+1 \Rarr n^a < (n+1)^a \Rarr a_n > a_{n+1}$ Somit (a_n) monoton fallend.\\
$\ds \lim_{n\to \infty} n^a = \infty \Rarr \lim_{n \to \infty} \frac{1}{n^a} = 0$ \Rarr Verdichtungslemma ist anwendbar.\\
Bilde $\ds 2^n · a_{2^n} = 2^n · \frac{1}{(2^n)^a} = 2^n · 2^{-n · a} = 2^{n(1-a)} = (2^{1-a})^n = x^n$\\
mit $x:=2^{1-a}$\\
Erhalte:
$\ds \sum_{k=1}^\infty \frac{1}{k^a}$ konvergiert $\equ \ds \sum_{k=0}^\infty x^k$ konvergiert $\equ |x|<1 \equ x<1 \equ 2^{1-a} <1$\\
$\equ 1-a<0 \equ a>1$\qed\\[4pt]
Beziehung:
$\ds \sum_{k=1}^\infty\frac{1}{k^a}=\zeta (a)$ für $a>1$\\
Riemannsche Zetafunktion
Spezielle Werte:
$\zeta (2) = \sum_{k\geq1}\frac{1}{k^2}=\frac{\pi^2}{6}$\\
$\zeta (4) = \sum_{k\geq1}\frac{1}{k^2}=\frac{\pi^4}{90}$\\
$\zeta (6) = \sum_{k\geq1}\frac{1}{k^2}=\frac{\pi^6}{945}$\\
Frage: Für welche z ist $\zeta(z)=0$?
%
%\sss{Teilfolgen}
\uS{Teilfolgen}
\sS{Definition}
Sei $(a_n)$ eine Folge reeller Zahlen.\\
Eine Teilfolge von $(a_n)$ ist eine Folge der Form $(a_{n_k})_{k \geq 0}$ wobei $n_0, n_1, n_2,...$ streng monoton wachsende Folge in $\N_0$ ist.\\
\bsp
$(a_n) = (1, x, x^2, x^3 , x^4 ...)$\\
$(n_k) = (1, 4, 9, 16) \leadsto $ Teilfolge $(x, x^4, x^9, x^{16} ,…)$\\
\sS{Bemerkung}
Wenn $a_n \to a$ für alle $n \to \infty$ dann konvergiert jede Teilfolge von $(a_n)$ gegen $a$ (Präsenzübung Nr. 9)\\
%
%\sss{Schlüsselsatz:}
\uS{Schlüsselsatz}
\sS{Lemma}
Jede Folge reeller Zahlen $(a_n)_{n\geq 0}$ hat eine monotone Teilfolge.\\
%
\bew
Wir nennen $n\in \N_0$ \underline{extrem} wenn $a_n \geq a_m$ für alle $m \geq m$\\
Unterscheide zwei Fälle:\\
\begin{itemize}
    \item{Es gibt unendlich viele extreme $n \in \N$\\
Dies seien $n_0, < n_1, n_2...$\\
Dann $a_{n_0} \geq a_{n_1} \geq a_{n_2} …$\\
Weil n_0 extrem ... weil n_1 extrem.\\
→ monoton fallende Teilfolge gefunden}
    \item{Es gibt nur endlich viele extreme $n$\\
Wähle $n_0 \in \N$ s.d. gilt: $m \geq n \Rarr m$ nicht extrem.\\
$n_0$ nicht extrem $\Rarr$ es gibt $n_1 \geq n_0$ mit $a_{n_1} > a_{n_0}$ insbesondere $n_1 > n_0$\\
$n_1$ \phantom{nicht extrem }$\Rarr$ \phantom{es gibt }$n_2 \geq n_1$ mit $a_{n_2} > a_{n_1}$ insbesondere $n_2 > n_1$\\
$n_2$ \phantom{nicht extrem }$\Rarr$ \phantom{es gibt }$n_3 \geq n_2$ mit $a_{n_3} > a_{n_2}$ insbesondere $n_3 > n_2$\\
usw.\\
Erhalte $n_0 < n_1 < n_3 < …$ mit $a_{n_0} < a_{n_1} < a_{n_2} < …$ \\
$\to $ streng monotom wachsende Teilfolge gefunden.\qed
}
\end{itemize}
\sS{Satz Bolzano-Weierstraß}
Jede beschränkte Folge reeller Zahlen hat eine konvergernte Teilfolge.\\
\bew
Es gibt ein monotone Teilfolge (Lemma 4.14)\\
Diese ist beschränkt \Rarr konvergent.\qed
\sS{Definition}
Eine Folge reeller Zahlen $(a_n)_{n \geq 0}$ heißt Cauchyfolge wenn gilt:\\
Für jedes $\e > 0$ gibt es ein $N \in \N$ sodass für $m, n \geq N$ gilt: $|a_n - a_m| < \e$
\sS{Satz}
Eine Folge reeller Zahlen $(a_n)$ konvergiert genau dann, wenn sie eine Cauchyfolge ist.\\
\bew
$\Rarr$ Sei $a_n \to a$ für $n \to \infty$\\
Gegeben sei $\e > 0$. Es gilt $N \in \N$ so dass $|a_n - a| < \frac{\e}{2}$ für $n \geq N$\\
Für $n, m \geq N$ gilt:\\
$|a_n - a_m| = |a_n - a + a - a_m| \leq |a_n - a| + |a - a_m| < \frac{\e}{2} + \frac{\e}{2} = \e$\\
$\Rarr (a_n)$ ist eine Cauchyfolge\\
$Larr :$ Sei (a_n) eine Cauchyfolge\\
\sss{Behauptung:} $(a_n)$ ist beschränkt
\bew
Wähle $\e=1$ Es gibt $N\in\N$ mit $|a_n-a_m|<1$ für $m,n\geq N$\\
Sei $C=max\{|a_0|,|a_1|,|a_2| … |a_N|,|a_N|+1\}$\\
Dann $|a_n| \leq C$ für alle \N\\
$(n\geq N \Rarr |a_n-a_N| < 1 \Rarr |a_n|<|a_N|+1)$\\
Also ist $(a_n)$ beschränkt\\
%
\sS{Lemma}
\Rarr $(a_n)$ hat eine monotone Teilfolge $(a_{n_k})_{k\geq0}$ diese ist beschränkt \Rarr konvergent.\\
Sei $\ds \lim_{k→∞}(a_{n_k})$
\sss{Behauptung}
$a_n→a$ für $n→∞$\\
Sei $\e>0$ gegeben. Es gibt $n\in\N$ so dass
\begin{enumerate}
\item{$n,m\geq N \Rarr |a_n-a_m|< \frac{\e}{2}$}
\item{$k\geq N \Rarr |a_{n_k}-a|< \frac{\e}{2}$}
\end{enumerate}
%
Sei $k\geq N$\\
%
\bem
Für jedes $k\in\N$ ist $n_k\geq k$\\
$\ds |a_k-a|=|a_{n_k}+a_{n_k}-a| \leq |a_k-a_{n_k}|+|a_{n_k}-a|<\frac{\e}{2}+\frac{\e}{2}=\e$\\
Also $a_k→a$ für $n→∞\qed
%
\sss{Umformulierung für Reihen:}
\sS{Satz (Cauchy-Kriterium für Reihen)}
Eine reelle Reihe $\ds \sum_{k=0}^\infty a_k$ konvergiert genau dann, wenn gilt:\\
Für jedes $\e>0$ gibt es ein $N\in\N$ so dass für alle $\ds n,m\geq N, n\leq m |\sum_{k=n}^m|<\e$
%
\sss{Beweis: Partialsummen}
$\ds s_n=\sum_{k=0}^n a_k$\\
$\ds \sum_{k=n}^m=s_m-s_{n-1}$\\
Damit ist 4.19 äquivalent zu 4.18\newpage
%%Kopfzeile links bzw. innen
\fancyhead[L]{\calligra\Large Vorlesung Nr. 9}
%Kopfzeile rechts bzw. außen
\fancyhead[R]{\calligra\Large 08.11.2012}
% **************************************************
%
\wdh
Eine folge reeller Zahlen $(a_n)$ ist eine Cauchy-Folge wenn gilt:\\
Für jedes $\e>0$ gibt es ein $n\in\N$ so dass für $m,n\geq\N$ gilt $|a_n-a_m|<\e$\\
$(a_n)$ konvergiert \equ $(a_n)$ ist Cauchy-Folge\\
Für Reihen: $\ds \sum_{k=0}^{∞}a_k$ konvergiert \equ Für jedes $\e>0$ gibt es ein $N\in\N$ so dass für $m,n\geq\N$ mit $m\geq n$ ist $\ds \left|\sum_{k=n}^m a_n\right|<\e$

\uS{Absolute Konvergenz}

\sS{Definition}
Eine Reihe $\ds\sum_{k=0}^{∞} a_k$ mit $a_k\in\R$ heißt absolut konvergent wenn die Reihe $\ds\sum_{k=0}^{∞} |a_k|$ konvergiert

\sS{Satz}
Jede absolut konvergente Reihe konvergiert

\bew
Verwende Cauchy-Kriterium für Reihen\\
Sei $\ds\sum_{k=0}^{∞} a_k $ absolut von konvergent.\\
\Rarr Für jedes $\e>0$ gibt es $N\in\N$ mit:\\
Für $n\geq m\geq N$ gilt $\ds\sum_{k=m}^n |a_k| < \e \Rarr \left|\sum_{k=m}^n a_k\right| \underset{\overset{\uparrow}{Dreiecksungleichung}}{\leq} \sum_{k=m}^n |a_k| < \e \Rarr \sum_{k=m}^n a_k konvergiert$\qed

\bem
Umkehrung gilt nicht.
$\ds\sum_{k=1}^{∞} (-1)^k \frac{1}{k} = -1+\frac{1}{2}+\frac{1}{3}+\frac{1}{4}+...$\\
konvergiert (Leibnitz)\\
denn $\ds\sum_{k=1}^{∞} \left|(-1)^k \frac{1}{k}\right| = \sum_{k=1}^{∞} \frac{1}{k}$ divergiert

\sS{Definition}
Eine Reihe $\ds\sum_{k=0}^{∞} b_k$ heißt Majorante der Reihe $\ds\sum_{k=0}^{∞} a_k$, wenn $|a_k|\leq b_k$ für alle k\\
(schon gewesen wenn $a_k\geq 0$)

\sS{Satz (Majorantenkriterium)}
Wenn eine Reihe eine konvergente Majorante hat, dann konvergiert sie absolut.
\underline{Beweis} von Satz 4.5\qed

\uS{Umordnung von Reihen}
\sS{Definition}
Eine Umordnung einer Reihe $\ds\sum_{k=0}^{∞} a_k$ ist eine Reihe der Form $\ds\sum_{k=0}^{∞} a_{n_k}$ wobei $(n_0,n_1,n_2…)$ eine Folge natürlicher Zahlen ist, in der jedes $n\in\N_0$ genau einmal vorkommt.\\

\sS{Satz}
Jede Umordnung einer \underline{absolut} konvergenten Reihe ist wieder absolut konvergent und hat den gleichen Grenzwert.\\
Im Gegensatz dazu gilt:\\
\sS{Satz}
Sei $\ds\sum_{k=0}^{∞} a_k$ eine konvergente, nicht absolut konvergente, Reihe. Für jedes $c\in\R\cup\{-∞,∞\}$ hat $\sum a_k$ eine Umordnung, die gegen c konvergiert.

\bsp
Eine Reihe $\dfrac{1}{2}-\dfrac{1}{2}+\dfrac{1}{3}-\dfrac{1}{3}+\dfrac{1}{4}-\dfrac{1}{4}+\dfrac{1}{5}-\dfrac{1}{5}+…$
konvergiert gegen 0. Konvergiert aber nicht absolut:\\
Folge: $(\dfrac{1}{2},0,\dfrac{1}{3},0,\dfrac{1}{4},0,…→0)\quad\ds\sum_{k=1}^{∞} 2·1/k=∞$\\
Produziere Umordnung, die gegen ∞ konvergiert:\\
\[\dfrac{1}{2}-\dfrac{1}{2}+\underbrace{\dfrac{1}{3}+\dfrac{1}{4}}_{\geq\dfrac{1}{4}+\dfrac{1}{4}=\dfrac{1}{2}}-\dfrac{1}{3}+\underbrace{\dfrac{1}{5}+\dfrac{1}{6}+\dfrac{1}{7}+\dfrac{1}{8}}_{\geq\dfrac{1}{2}}-\dfrac{1}{4}+\underbrace{\dfrac{1}{5}+…+\dfrac{1}{16}}_{\geq\dfrac{1}{2}}-\dfrac{1}{5}+…\]\\
\[\leq\underbrace{\dfrac{1}{2}-\dfrac{1}{2}}_{\text{\large{0}}}+\underbrace{\dfrac{1}{2}-\dfrac{1}{3}}_{\dfrac{1}{6}}+\underbrace{\dfrac{1}{2}-\dfrac{1}{4}}_{<\qquad\dfrac{1}{4}\qquad<}+\underbrace{\dfrac{1}{2}-\dfrac{1}{5}}_{\dfrac{3}{10}}+…=∞\]\\
Beweise von 4.24, 4.25 eventuell später.

\uS{Produkte von Reihen}
Frage: was ist $\ds\left(\sum_{k=0}^{∞} a_k\right)·\left(\sum_{k=0}^{∞} b_k\right) ?$

\sS{Definition}
Das Cauchy-Produkt von zwei reihen $\ds\sum_{k=0}^{∞} a_k$ und $\ds\sum_{k=0}^{∞} b_k$ ist eine Reihe $\ds\sum_{k=0}^{∞} c_k$ mit $\ds c_n :=\sum_{k=0}^{∞} a_k·b_{n-k}=a_0·b_n+a_1·b_{n-1}+a_2·b_{n-2}+…+a_n·b_0$\\
2-dimensionale Anordnung der $a_k·b_l$ 
% Ed's heft
\sS{Satz}
Seien $\ds\sum_{k=0}^{∞} a_k$ und $\ds\sum_{k=0}^{∞} b_k$ konvergente Reihen, mindestens eine von ihnen absolut konvergent. Dann konvergiert ihr Cauchy-Produkt $\ds\sum_{k=0}^{∞} c_k$. Wenn $\ds\sum_{k=0}^{∞} a_k = a, \ds\sum_{k=0}^{∞} b_k = b$ $\ds\sum_{k=0}^{∞} c_k = a·b$



\Bew{von 4.27}
Sei $\sum a_k$ absolut konvergent, $\sum b_k$ konvergent, so zeige $\sum c_k\ $ konvergent, $\ds c_n :=\sum_{k=0}^{∞} a_k·b_{n-k}$
Schreibe:
$s_n=a_0+…+a_n$\\
$t_n=b_0+…+b_n$\\
$u_n=c_0+…+c_n$\\
$s_n→a$,$t_n→b$ (*)\\[8pt]
Zeige $u_n→a·b$\\[4pt]
(*)\Rarr $s_n·b→a·b$ Zeige $s_n·b-u_n→0$\\[4pt]
$u_n=a_0·b_0+(a_0·b_1+a_1·b_0)+(a_0·b_2+a_1·b_1+a_2·b_0)+…+a_n·b_0=a_1·t_{n-1}+a_2·t_{n-2}+…+a_n·t_0$\\[4pt]
$s_n·b=a_0·b+a_1·b+a_2·b+a_3·b+…+a_n·b$\\[4pt]
$s_n·b-u=a_0·(b-t_n)+a_1·(b-t_{n-1})+a_2·(b-t_{n-2})+a_3·(b-t_{n-3})+…+a_n·(b-t_0)\underset{?}{→}0$\\[8pt]
Sei $C\in\R$ mit $|b|\leq C$ und $|b-t_n|\leq C$ für alle n\\
Sei $\ds\sum_{k=0}^{∞} |a_n| = a^*.$\\
Gegeben sei $\ds\e>0$. Wähle $N\in\N$ so dass $C·(|a_N|+|a_{N+1}|+|a_{N+2}|+…)<\frac{\e}{2}$\\
(geht weil $\sum|a_k|$ konvergiert)\\
und $|b-t_n|<\frac{\e}{2a^*}^{(2)}$ für alle $n\geq N$\\
(geht weil $b-t_n→0$ für alle $m→∞$)
\bem
Wenn $a^*=0$ dann $a_n=0$ für alle k. Dann alles klar.
Für alle $n\geq 2N$ gilt:\\
$|a_0(b-t_n)+a_1(b-t_{n-1})+…+a_n(b-t_0)|\leq |a_0|·|(b-t_n)|+|a_1|·|(b-t_{n-1})|+…+|a_n|·|(b-t_0)|$\\[4pt]
$\leq(|a_0|+|a_1|+|a_2|+…|a_N|)·\underset{\overset{\uparrow}{wegen\left(2\right)}}{\frac{\e}{2a^*}} +(|a_{N+1}|+|a_{N+2}|+|a_{N+3}|+…|a_n|) · C \leq a^* ·\frac{\e}{2a^*}+\frac{\e}{2}=\frac{\e}{2}+\frac{\e}{2}=\e$\\
Also gilt: $s_n-u→0$ für $n→∞$\qed\\
\underline{Zusatz:} Wenn $\sum a_k$ und $\sum b_k$ beide absolut konvertieren, dann auch das Cauchy-Produkt $\sum c_k$\\
\bew
Sei $\sum a_k^*$ das Cauchy-Produkt von  $\sum |a_k|$ und  $\sum |b_k|$. Beide konvergieren \Rarr $\sum_n c_n^*$ konvergiert\\[8pt]
d.h. $c_n^*=|a_0·b_{n}|+|a_1·b_{n-1}|+…+|a_n·b_{0}|\geq|a_0·b_{n}+a_1·b_{n-1}+…+a_n·b_{0}|=|c_n|$\\[8pt]
Also $\sum_n c_n^*$ ist konvergente Majorante von $\sum_n c_n$ \Rarr $\sum_n c_n$ konvergent absolut\qed

\bsp
Die Reihe $\ds\sum_{k=0}^{∞} a_k = 1-+\dfrac{1}{\sqrt{2}}+\dfrac{1}{\sqrt{3}}-\dfrac{1}{\sqrt{4}}+\dfrac{1}{\sqrt{5}}-…\quad $ konvergiert (Leibnitz)\\
Das Cauchy-Produkt der Reihe von $\sum a_k$ und $\sum a_k$ konvergiert nicht.\\

\sS{Beispiel}
Für jedes $x\in\R$ ist die Exponentialreihe $\ds exp(x)=\sum_{k=0}^{∞} \frac{x^k}{k!}$ absolut konvergent.\\
Es gilt \fbox{$exp(x)-exp(y)=exp(x+y)$} Funktionalgleichung der Exponentialfunktion.\\
\bew
Betrag von $\ds \sum_{k=0}^{∞} |\frac{x^k}{k!}|=\sum_{k=0}^{∞} \frac{|x|^k}{k!}=exp(|x|)$ konvergiert (bekannt, Quotientenkriterium)\\
Berechne Cauchy-Produkt $\ds exp(x)·exp(y)=\sum_{k=0}^{∞} c_k$\\
$\ds c_k = \frac{x^0}{0!}·\frac{x^n}{n!}+\frac{x^1}{1!}·\frac{y^{n-1}}{(n-1)!}+…+\frac{x^n}{n!}·\frac{y^0}{0!}=\dfrac{1}{n!}·\left(\dfrac{n!}{0!·n!}·x^0y^n+\dfrac{n!}{1!·(n-1)!}·x^1y^{n-1}+…+\dfrac{n!}{n!·0!}·x^ny^0+\right)=\dfrac{1}{n!}\sum_{k=0}^{n}\dfrac{n!}{k!·(n-k)!}x^ky^{n-k} =\dfrac{1}{n!}\sum_{k=0}^{n}\binom{n}{k}x^ky^{n-k}\underset{binomische Formel}{=}\dfrac{1}{n!}(x+y)^n\Rarr\sum_{k=0}^{∞}c_k=exp(x+y)$\qed\newpage
%% Kopfzeile beim Kapitelanfang:
\fancypagestyle{plain}{
%Kopfzeile links bzw. innen
\fancyhead[L]{\calligra\Large Vorlesung Nr. 10}
%Kopfzeile rechts bzw. außen
\fancyhead[R]{\calligra\Large 12.11.2012}
}
%Kopfzeile links bzw. innen
\fancyhead[L]{\calligra {\Large Vorlesung Nr. 10}}
%Kopfzeile rechts bzw. außen
\fancyhead[R]{\calligra \Large{12.11.2012}}
% **************************************************
%
%\setcounter{chapter}{4}
\chapter{Abbildungen und Funktionen}
\sS{Definition Abbildung}
Seien $A, B$ Mengen. Eine Abbildung von $A$ nach $B$ ist eine Vorschrift, die jedem Element von $A$ ein Element von $B$ zuordnet.
\notat{$f: A \to B,\  a \mapsto f(a) \  a\in A$}
$A$ heißt Definitionsbereich von $f$\\*
$B$ heißt Wertebereich von $f$
%
\bsp
\begin{enumerate}
\item {Alle Personen in $L1 \mapsto \N$\\*
$P \mapsto$ Geburtsjahr von $P$}
%
\item{$f:\R → \R, \ f(x)=x^2$\\*
$g:\R→\R_{\geq 0}=\{x\in\R\mid x\geq 0\}, \ g(x)=x^2$\\*
$h: \R_{\geq 0} \to \R_{\geq 0} \ h(x) = x^2$}
\bem 
\item{
$f,g,h$ sind verschieden\\*
Sei $M$ Menge. Die Identität von $M$ ist die Abbildung $id_{M}:M→M, \ id_M(x)=x$}
\end{enumerate}

\sS{Definition In-/Sur-/Bijektivität}
Eine Abbildung $f: A \to B$ heißt:
\begin{enumerate}
\item{\ul{injektiv} wenn gilt: Für alle $a, a' \in A$ mit $f(a) = f(a')$ ist auch $a = a'$}
\item{\ul{surjektiv} wenn gilt: Für jede $b\in B$ gibt es ein $a\in A$ mit $f(a)=b$}
\item{\ul{bijektiv} wenn $f$ injektiv und surjektiv ist}
\end{enumerate}
%
% Tafel 2.2
% Bild zeichnen
%
% Tafel 3.1
% Beispielbild
%
\bem
$f$ ist $\left\{
\begin{array}{c}
\text{injektiv}\\*
\text{surjektiv}\\*
\text{bijektiv}
\end{array}
 \right\}$ genau dann wenn für jedes $b \in B$ $\left\{\begin{array}{c} \text{höchstens}\\* \text{mindestens}\\*
 \text{genau}
 \end{array} \right\}$ ein $a \in A$ mit $f(a) = b$
%
\bsp
$f,g,h$ wie oben
\begin{description}
\item[f]{
\hspace{5.5mm}nicht surjektiv: es gibt kein $a\in\R$ mit $f(a)=-1$\\*
nicht injektiv: $f(-2)=4=f(2), 2\neq -2.$}
%
\item[g]{
\hspace{5mm}ist surjektiv, denn für jedes $b \in \R_{\geq 0}$ gilt $f( \sqrt{b} ) = b$ also gibt es $b \in \R_a$\\*
 ist nicht injektiv (wie $f$)}
%
\item[h]{
\hspace{5mm}surjektiv wie g. $\left(\sqrt{b} \geq 0\right)$\\*
injektiv, denn: Wenn $a, a' \geq 0$ und $a^2 = (a')^2$ dann $a = a'$ also $h$ bijektiv.}
\end{description}

\sS{Definition Komposition}
Seien $f:A→B$, $g:B→C$ Abbildungen\\*
Die Komposition von $f$ und $g$ ist die Abbildung\\*
$g \circ  f: A→C$, $(g \circ f)(a):=g(f(a))$\\*
Sprich $\circ$: "nach", "verkettet"

\sS{Satz} 
Eine Abbildung $f: A \to B$ ist bijektiv \equ \ es gibt eine Abbildung $g: B \to A$ mit $f \circ g = id_B$\\*
(d.h. $f(g(b)) = b$ für alle $b \in B$ $g(f(a)) = a$ für alle $a \in A$)
%
\sss{Definition} % ohne Nummerierung!
Wenn $f:A→B$ bijektiv ist, heißt die eindeutige Abbildung $g:B→A$ wie oben die Umkehrabbildung (inverse Abbildung) von $f$
Bezeichnung: $g=f^{-1}$.
%
\bew
Angenommen, $g: B \to A$ gegeben mit $f \circ g = id_B, g \circ f = id_A$ \footnote{Dies gilt, weil $g$ als Umkehrfunktion von $f$ definiert ist.}\\*
$f$ surjektiv: Sei $b \in B$. $b = f(g(b)) = f(a)$ mit $a = g(b)$ \ok\\*
$f$ injektiv: Sei $a, a'$ mit $f(a) = f(a')$ zeige $a = a'$ \\*
$a = g(f(a)) = g(f(a')) = a' $\ok \\*[8pt]
%
Angenommen, $f$ ist bijektiv, zeige $g$ existiert.\\*
Gegeben sei $b \in B$ $f$ bijektiv $\Rightarrow$ es gibt genau ein $a \in A $ mit $f(a) = b$ 
Setze $g(b):=a$ Das definiert Abbildung $g:B→A$\\*
Zeige $g \circ f=id; f \circ g = id$\\*
$(f\circ g)(b)=f(g(b))=f(a)=b$ wobei $a$ wie eben\\*[8pt]
%
Zeige: $(g \circ f) (a) $ für alle $a \in A$\\*
$f$ injektiv: Reicht $f(g(f)a))) = f(a)$\\*
Das gilt weil $f \circ g = id_B$ \ok\\*[8pt]
Eindeutigkeit von $g$:\\*
Angenommen, $g^* : B \to A$ erfüllt $g^* \circ f = id_A$,
$f \circ g^* = id_B$ \\*
%
Dann gilt: $g=g\circ id_B=g\circ f\circ g^*=id_A\circ g^* = g^*$ \qed
\bsp
Bewiesen 5.12
\begin{itemize}
\item{$f: \R_{\geq 0} \to \R_{\geq 0}, f(x) = x^k$ bijektiv ($k \geq 1$)\\*
Die Umkehrabbildung $f^{-1}$ heißt k-te Wurzelabbildung $f^{-1}(x) = \sqrt[k]{x}$ }
%
\item{exp: $\R→\R_{>0}$ $exp(x) = \sum_{k=0}^{\infty}$ (absolut konvergente Reihe) ist bijektiv. Die Umkehrabbildung heißt Logarithmus. bew.
$log = exp()^{-1} \R_{\geq } \to \R_a$ }
\end{itemize}

\uS{Bild und Urbild}
\sS{Definition}
Sei $f:A→B$ Abbildung
\begin{enumerate}
\item{Für eine Teilmenge $X \subset A$ ist \\*
$f(x) := \{f(x) | x \in X\} \subseteq B$ \\*
das Bild von $X$ unter $f$}
\item{Für eine Teilmenge $Y \subseteq B$ ist $f^{-1}:=\{a\in A|f(a)\in Y\}\subseteq A$ das Urbild von $Y$ unter $f$}
\end{enumerate}
\ul{Vorsicht} nicht Urbild und Umkehrabbildung verwechseln.
\bsp
$$f:\R→\R,\ f(x)=x^2$$
$$f(\{1, 2, -2\}) = \{1, 4\}$$
$$f^{-1}(\{1,-2,4\})=\{1,-1,2,-2\}=f^{-1}(\{1,4\})$$
$$f^{-1}(\{9\})=\{3,-3\}\qquad f^{-1}(\{-5\})=\emptyset$$

\uS{Funktionen}
\sS{Definition Funktion}
Sei $D\subseteq\R$ Teilmenge. Eine reelle Funktion auf $D$ ist eine Abbildung $f:D→\R$\\*
%
Der \ul{Graph} von $f$ ist die Menge $\Gamma_f = \{(x, f(x)\mid x \in D \}$\\*
$ \Gamma_f \subseteq D \times \R$ 
%
\bem Oft ist $D$ ein Intervall
%
%\sS{Definition Intervalle}
  \tikz[scale=0.5,domain=-3.5:3.5, samples=200,prefix=plots/,smooth]{
      \draw[very thin, color=gray!50] (-3.5,-3.5) grid (3.5,8.9);
      \draw[->] (-3.5,0) -- (3.5,0) node[right] {$x$};
      \draw[->] (0,-3) -- (0,8.5) node[above] {$y$};
      \clip (-3.5,-3.0) rectangle (5.5,8.9);
      \draw[color=red] plot[id=x^2] function{x*x} node[below=3.5cm] {\footnotesize $f_1(x) =x^2$};
      \draw[color=blue] plot[id=abs,sharp plot] function{abs(x)} node {\footnotesize $f_2(x) = |x|$};
      \draw[color=cyan] plot[id=exp] function{exp(x)} node [below=13.5cm] {\footnotesize $f_2(x) = exp(x)$};
      \draw[color=green!60!black,const plot] plot[id=gaussklammer] function{floor(x)} node[below] {\footnotesize $f_5(x) = [x]$};
  }
seien $a, b \in \R$ \\*
$[a, b] = \{x \in \R| a \leq x \leq b\}$ (abgeschlossen)\\*
$(a, b] = \{x \in \R| a < x \leq b\}$ (halboffen)\\*
$[a, b) = \{x \in \R| a \leq x < b\}$ (halboffen)\\* %mit klammer zu überer zeile\\*
$(a, b) = \{x \in \R| a < x < b\}$ (offen)\\*
%
Uneigentliche Intervalle:
$$[a, \infty) = \{x \in \R | a \leq x\} = \R_{\geq a}$$
$$(a, \infty) = \{x \in \R | a < x\} = \R_{> a}$$
$$(- \infty, a] = \{x \in \R | x \leq a\} = \R_{\leq a}$$
$$(- \infty, a) = \{x \in \R | x < a\} = \R_{< a}$$
$$(- \infty, \infty) = \R$$
%
\Bsp{Funktionen}
\begin{enumerate}
\item{$f:[0,2]→\R, f(x)=x^2, \Gamma_f \leq [0,2] x\R$}
\item{Betragsfunktionen: $|\ |: \R→\R, x\mapsto|x|$
%noch mehr graphen aaaahahhahahah
}
An dieser Stelle fehlen noch Graphen.
\item{$g:\R\bs\{0\}→\R, g(x)=\dfrac{1}{x}$
%graph
Hier auch.
}
\item{$exp:\R→\R$.}
\item{[.] : $\R \to \R$ Gaußklammer\\*
$[x] := max\{n \in \Z | n \leq x \}$
\bsp
$[5] = 5$\\*
$[5,78] = 5$\\*
$[-1,2] = -2$}
\item{Sei $h:\R→\R$ definiert durch $h(x)=\begin{cases}0\ wenn\ x\in\Q\\* 1\ wenn\ x\notin\Q\end{cases}$\\*
$h(\sqrt{2}) = 1, h (\frac{3}{7}) = 0$}
\end{enumerate}

\sS{Definition (Rechnen mit Funktionen)}
Sei $D \subseteq \R , \ f,g: D→\R$ Funktionen auf D.\\*
Definiere
\begin{itemize}
\item{$f+g: D \to \R$ durch $(f + g)(x) := f(x) + g(x)$}
\item{$(f \cdot  g) (x) := f(x) \cdot g(x)$}
\item{Für $a\in\R$ setze $a·f: D→\R, (a·f)(x):=a·f(x)$}
\item{Angenommen, $f(x) \neq 0$ für alle $x \in D$
$$\frac{1}{f}: D \to \R, \frac{1}{f}(x) := \frac{1}{f(x)} = f(x)^{-1}$$
\ul{Vorsicht} nicht $\frac{1}{f}$ mit Umkehrbild oder Urbild verwechseln}
\end{itemize}

\sS{Definition Polynomfunktion}
\begin{itemize}
\item{Eine \ul{Polinomfunktion} ist eine Funktion der Form\\*
$f: \R → \R,\ f(x) = a_n x^n+a_{n-1}x^{n-1}+…+a_0=\ds\sum_{k=0}^n a_k x^k $\\*
wobei $a_0,…,a_n \in \R$ fest}
\item{Seien $f, g : \R \to \R $ Polynomfunktionen
Sei $D = \{x \in \R \mid g(x) \geq 0\}\leadsto \dfrac{f}{g} : D \to , Rx \mapsto \frac{f(x)}{g(x)}$
Solche Funktionen heißen rationale Funktionen.
\bsp
$f:\R\bs\{0,1\}→\R, \ f(x)=\dfrac{x^7+5x^2}{x(x-1)}$}
\end{itemize}
\sS{Definition}
Seien $f: C \to \R, g: D \to \R$ Funktionen, sodass $f(C) \subseteq D$
Eine Komposition von $ f $ und $ g $ ist 
%
$g \circ f : C \to \R$\\*
$(g \circ f) \ (x) = g(f(x))$
\end{document}

