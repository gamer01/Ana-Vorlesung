\documentclass[a4paper,10pt]{scrreprt}
\usepackage{amsmath}	% Mathebibliothek 2
\usepackage{amssymb}	% Mathebibliothek 2
\usepackage{breqn}
\usepackage{fontspec}
\usepackage{polyglossia}
\usepackage[math-style=TeX]{unicode-math}	% Unicode Unterstützung
\setmainfont{FreeSerif.otf}
\setmathfont{Asana-Math.otf}
\usepackage{calligra}	% Schrift
\usepackage{cancel}		% Kürzen und Durchstreichen
\usepackage{soulutf8}	% Für schöne Unterstreichungen
\usepackage{tikz} 		% Für Graphenzeichnerei
\usepackage{wrapfig}

% Language
\setdefaultlanguage{german}

% Definitionen fürs floaten

\fboxsep0pt
\fboxrule0pt

% set underline
\setul{.3mm}{.8pt}

% Beschreibaren Seitenbereich definieren
\usepackage[left=2.5cm,right=2.5cm,top=2cm,bottom=2.5cm]{geometry}

% Kopf- und Fußzeile einfügen
\usepackage{fancyhdr}
\pagestyle{fancy}
\setlength{\headheight}{25pt}

% Kopfzeile Linie oben
\renewcommand{\headrulewidth}{0.5pt}

% Fußzeile mittig
\fancyfoot[C]{\thepage}

% Linie unten
\renewcommand{\footrulewidth}{0.5pt}

% Kopfzeilen fürs Inhaltsverzeichnis:
% Kopfzeile links bzw. innen
\fancyhead[L]{\calligra\Large Inhaltsverzeichnis}
% Kopfzeile rechts bzw. außen
\fancyhead[R]{\calligra\Large{}}

% ***********************
% *** costum commands ***
% ***********************

% kürzerer command um neue mathemathische commands (ohne parameter) zu erstellen
\newcommand{\mcmd}[2]{\newcommand{ #1 }{\text{$#2$}}} 
% kürzerer command um neue temxt commands (ohne parameter) zu erstellen
\newcommand{\tcmd}[2]{\newcommand{ #1 }{\text{#2}}} 
% kürzerer command um neue commands (ohne parameter, ohne environment) zu erstellen
\newcommand{\cmd}{\newcommand}


% Mengen
\mcmd{\N}{\mathbb{N}}
\mcmd{\Z}{\mathbb{Z}}
\mcmd{\Q}{\mathbb{Q}}
\mcmd{\R}{\mathbb{R}}
\mcmd{\C}{\mathbb{C}}

\mcmd{\nN}{n\in\N}
\mcmd{\eN}{\in\N}
\mcmd{\eZ}{\in\Z}
\mcmd{\eQ}{\in\Q}
\mcmd{\eR}{\in\R}
\mcmd{\eC}{\in\C}

% griechische Buchstaben
\mcmd{\z}{\zeta}
\mcmd{\e}{\epsilon}	

% Pfeile
\mcmd{\equ}{\Leftrightarrow}
\mcmd{\Rarr}{\Rightarrow}
\mcmd{\Larr}{\Leftarrow}

% Symbole
\cmd{\qed}{\hfill\text{$\blacksquare$}}
\mcmd{\bs}{\backslash}
\mcmd{\ba}{\backslash}

% häufige mathemathische Ausdrücke
\mcmd{\nif}{n→∞}
\mcmd{\an}{(a_n)}

% Formelkürzel
\cmd{\bino}[2]{\text{$\ds\binom{#1}{#2} = \frac{#1!}{#2!\cdot(#1-#2)!}$}}

% Überschriften
\cmd{\sS}[1]{\section{#1}}
\cmd{\Def}{\sS{Definition:}}
\cmd{\Satz}{\sS{Satz:}}
\cmd{\uS}[1]{\section*{\ul{#1}}}
\cmd{\Wdh}[1]{\section*{Wiederholung #1}}
\cmd{\wdh}{\Wdh{}}
\cmd{\sss}[1]{\subsection*{\ul{#1}}}
\cmd{\bsp}{\sss{Beispiel:}}
\cmd{\Bsp}[1]{\subsection*{\ul{Beispiel:} #1}}
\cmd{\bem}{\sss{Bemerkung:}}
\cmd{\beh}{\sss{Behauptung:}}
\cmd{\bew}{\sss{Beweis:}}
\cmd{\Bew}[1]{\subsection*{\ul{Beweis} #1:}}
\cmd{\anm}{\sss{Anmerkung:}}
\cmd{\ssss}[1]{{\bf\ul{#1}}\\}

% Textkürzel
\cmd{\ok}{\hfill{\checkmark}}
\tcmd{\oE}{ohne Einschränkungen}

% Environments
\cmd{\ds}{\displaystyle}
%\cmd{\ul}[1]{\underline{#1}}
\cmd{\einruck}[2]{#1\vspace{-3.65ex}\begin{addmargin}{.05\textwidth}#2\end{addmargin}}
\cmd{\ind}[2]{\einruck{IA:}{#1\ok}\einruck{IS: }{$n\to n+1$#2\qed}}
\cmd{\indIV}[3]{\einruck{IA:}{#1\ok}\einruck{IV:}{#2}\\\einruck{IS: }{$n\to n+1$#3\qed}}
\cmd{\notat}[1]{{\em Notation:}\vspace{-2.8ex}\begin{addmargin}{.1\textwidth}{\em#1}\end{addmargin}}

% Author, title…
\title{Analysis Vorlesung}
\author{Stefan Heid, Christopher Jordan}
\date{\today}

% Less detailed TOC
\setcounter{tocdepth}{1}

\begin{document}
%\maketitle
%\tableofcontents\newpage
%%Kopfzeile links bzw. innen
\fancyhead[L]{\calligra\Large Vorlesung Nr. 1}
%Kopfzeile rechts bzw. außen
\fancyhead[R]{\calligra\Large 8.10.2012}
% **************************************************
\chapter{Mengen}
\Def
\begin{enumerate}
\item Eine Menge ist eine Ansammlung verschiedener Objekte
\item Die Objekte in einer Menge heißen \underline{Elemente}\\
%
\notat{
a $\in$ M heißt a ist Element der Menge M\\
a ${\not\in}$ M heißt a ist kein Element der Menge M}
%
\item Sei M eine Menge. Eine Menge U heißt Teilmenge von M, von der jedes Element von U auch Element von M ist\\
%
\notat{
U $\subseteq$ M heißt U ist Teilmenge von M\\
U ${\not\subseteq}$ M heißt U ist keine Teilmenge von M}
\end{enumerate}
%
\sS{Beispiele}
\begin{enumerate}
\item {\einruck{Sei}{M die Menge aller Studierenden in L1\\W  die Menge aller weiblichen Studierenden in L1\\F die Menge aller Frauen}
Dann gilt: W $\subseteq$ M, W $\subseteq$ F, M ${\not\subseteq}$ F, F ${\not\subseteq}$ M}
\item {Die Menge der natürlichen Zahlen
$\N = \{1,2,3,4 …\}$
G sei die Menge der geraden natürlichen Zahlen
$G := \{n \in \N | $n ist gerade$\} = \{2m | m \in \mathds{N}\} = \{2,4,6,8 …\}$
Es gilt G $\subseteq \N, \N \subseteq$ G}
\item {Die Menge der ganzen Zahlen
$\Z = \{0,1,-1,2,-2,3,-3, …\}$}
\item {Die Menge der rationalen Zahlen
$\mathds{Q} = \{a/b | a, b \in \mathds{Z}, b \neq 0\}$}
\item {Die Menge ohne Element heißt die leere Menge
Symbol: $\emptyset = \{\}$}
\end{enumerate}
%
\bem
\begin{enumerate}
\item Für jede Menge M gilt $\setminus \subseteq M$
\item $\N \subseteq \Z \subseteq \Q$
\end{enumerate}

\sS{Definition: Sei M eine Menge und U,V $\subseteq$ M Teilmengen}
\begin{enumerate}
\item Die Vereinbarung von U und V ist $U \cup V := \{x \in M \mid x \in U oder x \in V\}$
\item Der Durchschnitt von U und V ist $U \cap V := \{x \in M \mid x \in U oder x \in V\}$
U und V heißen disjunkt, wenn $U \cap V = \emptyset$
\item Die Differenzmenge von U und V ist $U \setminus V := \{x \in U \mid x \in V\}$
\item Das Komplement von U ist $U^C = M \setminus U = \{x \in M \mid x {\not\in} U\}$
%
\einruck{Bsp: }{Sei M = N \\
$\{1,3\} \cup \{3,5\} = \{1,3,5\}$\\
$\{1,3\} \cap \{3,5\} = \{3\}$\\
$\{1,3\} \cap \{2,4,7\} = \emptyset \leftarrow$ disjunkt\\
$\{1,2,3\} \setminus \{3,4,5\} = \{1,2\}$\\
$\{1,3,5\}^C = \{2,4,6,7,8,…\}$}
\end{enumerate}
%
\sS{Satz (de Morjensche Regeln)}
Sei M eine Menge, U,V $\subseteq$ M Teilmengen\\
Dann:
\begin{enumerate}
\item $(U \cap V)^C = U^C \cup V^C$
\item $(U \cup V)^C = U^C \cap V^C$
\end{enumerate}
%
\bew
\begin{enumerate}
\item Sei x$ \in M\\$Es gilt: x $\in (U \cap V)^C \Leftrightarrow x {\not\in} U \cap V \Leftrightarrow x {\not\in} $U oder x$ {\not\in} $V$\Leftrightarrow x \in U^C$ oder x$\in V^C \Leftrightarrow x\in U^C \cup V^C$
\item Sei x$ \in M\\$Es gilt: x $\in (U \cup V)^C \Leftrightarrow x {\not\in} U \cup V \Leftrightarrow x {\not\in} $U und x$ {\not\in} $V$\Leftrightarrow x \in U^C$ und x$\in V^C \Leftrightarrow x\in U^C \cap V^C$
\end{enumerate}
%
\section{Prinzip der Vollständigen Induktion}
Für jedes $n \in \N$ sei eine Aussage A(n) gegeben\\
Ziel: Beweisen, Dass A(n) für jedes $n \in \N$ mehr ist dafür reicht es zu zeigen
\begin{enumerate}
\item Induktionsanfang (IA): A(1) ist wahr
\item Induktionsschrit (IS): Wenn für ein $n \in \N$ A(n) wahr ist, dann ist auch A(n+1) wahr
\end{enumerate}
%
\Satz
Für jede natürliche Zahl n gilt: $\ds 1+2+3+4+5+…+n=\frac{n(n+1)}{2}$\\
Probe:\\
\begin{tabular}{r|c|c|c|c}
n & 1 & 2 & 3 & 4\\ \hline\hline
1+2+3...+n & 1 & 3 & 6 & 10\\ \hline
$\ds \frac{n(n+1)}{2}$ & 1 & 3 & 6 & 10\\
\end{tabular}
\sss{Beweis des Satzes mit Induktion}
Abkürzung: $S(n) := 1+2+3+…+n$
Aussage: A(n): $\ds S(n) = \frac{n(n+1)}{2}$
\begin{enumerate}
\item {Induktionsanfang (IA): n=1 $S(1) = 1 = \dfrac{1·2}{2}$\marginpar{ok!}}
\item {Induktionsschritt (IS): $n → n+1$\\
Annahme: A(n) gilt: $\ds S(n) = \frac{n(n+1)}{2}$\\
Zu zeigen: A(n+1) gilt: $S(n+1)=\frac{(n+1)\cdot(n+2)}{2}$\\
$\ds S(n+1)=S(n)+n+1=\frac{n(n+1)}{2}+\frac{2(n+1)}{2}=\frac{(n+2)(n+1)}{2}$\\
Das beendet den Beweis} \qed
\end{enumerate}
Zur Vereinfachung der Notation:\\
Seien $a_1,a_2,a_3,...,a_n$ Zahlen $n \in \N$\\
Setze: $\sum_{k=1}^n a_k := a_1+a_2+a_3+…+a_n$\\
\begin{tabbing}
Allgemeiner: \=Sei $l,m \in \mathds{N}$, $l \le m \le n$\\
\>$\sum_{k=l}^m a_k = a_l+a_{l+1}+…+a_m$\\
\end{tabbing}
Aussage des Satzes:
\[ \sum_{k=1}^n k = \frac{n(n+1)}{2}\]
\hfill\underline{Kombinatorik} (mathematisches Zählen)

\Def
Seien A, B Mengen. Das kartesische Produkt von A und B ist definiert als $A × B := \{(a,b)|a\in A, b \in B\}$ Die Elemente von $A × B$ heißen geordnete Paare\\
Bsp.: $\{1,7\}\times \{2,3\}=\{(1,2),(1,3),(7,2),(7,3)\}$\\
Allgemeiner: Gegeben seien Mengen
$A_1,…,A_k$ mit $k \in \N$. Das kartesische Produkt von $A_1,…,A_k$ ist $A_1\times …\times A_k = \{(a_1,…,a_k)|a\in A, $für $i=1,…,k\}$\\
Elemente von $A_1 × … × A_k$ heißen k-Tupel\\
Falls $A_1=A_2=…=A_k=A$, schreibe $\underbrace{A\times…\times A}_{k-mal}=A^k$

\section{Definition}
Eine Menge A ist endlich, wenn A nur endlich viele Elemente hat. Dann bezeichnet
$\#A = \{|A|\}$ die Anzahl der Elemente von A und somit dessen Kardinalit\"at
oder M\"achtigkeit. Wenn A nicht endlich ist, so schreibe: $\# A= \infty$\\
Bsp.: $\#\emptyset = 0, \#\mathds{N}=\infty, \# \{1,3,5\} = 3$

\section{Bemerkung}
\begin{enumerate}
\item Sei A endliche Menge. $U,V\subseteq A$ disjunkte Teilmengen\\
Dann $\#(U\cup V)=\# U + \# V$ 
\item Seien $A_1,...,A_k$ endliche Mengen $k \in \mathds{N}$\\
Dann: $\#(A_1 \times ... \times A_k)=(\#A_1)(\#A_2)...(\#A_k)$
\end{enumerate}

\section{Definition}
\begin{enumerate}
\item Für $n\in \mathds{N}$ setze $n!=1\cdot 2\cdot 3\cdot ... \cdot n=\prod_{k=i}^n k$
Setze $0!=1$
\item Für $k,n\in \Z$ mit $0\le k \le n$ sei ${n \choose k} := \frac{n!}{k!\cdot(n-1)!}$ $\leftarrow$ Binomialkoeffizient\\
\begin{tabular}{r|c|c|c|c|c|c|c}
n & 0 & 1 & 2 & 3 & 4 & 5 & 6\\ \hline
n! & 1 & 1 & 2 & 6 & 24 & 120 & 720
\end{tabular}\\
\bsp
${5 \choose 2} := \frac{5!}{2!\cdot 3!} = \frac{5\cdot 4 \cdot\cancel{3 \cdot 2 \cdot 1}}{2\cdot 1\cdot \cancel{3\cdot 2}\cdot 1 } = \frac{20}{2}=10$\\
Bemerkung: ${ n \choose 0 }= 1 = {n \choose n}$
\end{enumerate}\newpage
%%Kopfzeile links bzw. innen
\fancyhead[L]{\calligra\Large Vorlesung Nr. 2}
%Kopfzeile rechts bzw. außen
\fancyhead[R]{\calligra\Large 11.10.2012}
% **************************************************
%
\wdh
Sei M Menge.\\
Wenn M endlich: $\#M=Anzahl$ $Elemente\in M$\\
Wenn M unendlich: $\#M=\infty$\\
Für $n\in \N:=\{1,2,3,\ldots\}$\\
$$n!=1 · 2 · 3 · 4 · … · n \qquad 0!=1$$
Binomialkoeffizient: Für $0\leq k\leq n$\\
$$\bino{n}{k} \qquad\qquad\qquad \bino{n}{0}=\bino{n}{n}=1$$
%
\subsection{Lemma}
Für $0<k< n$ gilt:
$$\binom{n}{k} = \binom{n -1}{1} + \binom{n-1}{k}$$\\
%
\bew 
$$\binom{n-1}{k-1}+ \binom{n-1}{k}=\frac{(n-1)!}{(k-1)!·(n-k)!} +\frac{(n-1)!}{(k-1)!·(n-1-k)!} = \frac{k(n-1)!+(n-k)\cdot(n-1)!}{k! (n-k)!}=\frac{n(n-1)!}{k!(n-k)!}$$
%
\sS{Geometrische Anordnung (Pascalsches Dreieck)}
\parbox{0.5\textwidth}{\centering
$\binom{0}{0}$\\
$\binom{1}{0} \binom{1}{1}$\\
$\binom{2}{0} \binom{2}{1} \binom{2}{2}$\\
$\binom{3}{0} \binom{3}{1} \binom{3}{2} \binom{3}{3}$\\}
\parbox{0.5\textwidth}{\centering
1\\
1 1\\
1 2 1\\
1 3 3 1\\}\\[5mm]
Folge $\binom{n}{k}\in \N$ für alle $0\leq k\leq<n$
%
%
\Satz
Sei A endliche Menge. $\#A=n$\\[4pt]
Sei $k\in\Z$ mit $0\leq k\leq n$\\[4pt]
$P_k(A):=\{U\subseteq A| \#U=k\}$ (Menge aller k-elementigen Teilmengen von A)\\[4pt]
Dann gilt $\#P_k(A)=\binom{n}{k}$\\
\bsp
$A=\{1,2,3,4\}$ $n=4$ $k=2$\\[4pt]
2-elementige Teilmengen von A:
$\{1,2\}, \{1,3\}, \{1,4\}, \{2,3\}, \{2,4\}, \{3,4\} \to 6\qquad \binom{4}{2}=6$ \ok
%
\bew
Vorüberlegung: Sei $k=0 \vee k=n$\\
$P_0(A)=1=\binom{n}{0}$ $\#P_n(A)=1=\binom{n}{n}$\ok\\
Jetzt: Induktionsbeweis nach n\\[4pt]
\ind{$n=0$ Dann $k=0$}{Sei $\#A=n+1 \Rarr 0 \leq k \leq (n+1)$
Falls $k = 0\vee k = n + 1$\\
Sei also: $o < k < n + 1$\\
Wähle $a\in A$\\
Sei $B=A\bs\{a\}$\\
Dann $A=B\cup\{a\}, \#B=n$\\
Man kann die Wahl einer k-elementigen Teilmenge von A so strukturieren
\begin{enumerate}
\item Entscheiden, ob $a\in U \vee a\notin U$
\item\begin{enumerate}
\item Wenn $a\notin U$: Wähle k Elemente aus B
\item Wenn $a\in U$: Wähle k-1 Elemente aus B
\end{enumerate}
\end{enumerate}
$$\Rarr\ \#P_k(A)=\#P_k(B)+\#P_{k-1} (B) \stackrel{IV}{=} \binom{n}{k} + \binom{e}{ -1} \stackrel{1.11}{=} \binom{n+1}{k}$$
}
%
\sS{Satz (Binomische Formel)}
Seien $a,b$ Zahlen, $n\in\N$\\
Dann $(a+b)^n=a^n+\binom{n}{1} a^{n-1} b+\binom{n}{2}a^{n-2}b^2+…+b^n$
%
\bsp
$(a+b)^4=a^4+4a^3b+6a^2b^2+4ab^3+b^4$\\
$(a+b)^2=a^2+2ab+b^2$
%
\bew
Schreibe $(a+b)^n=\underbrace{(a+b)(a+b)(a+b)(a+b)…(a+b)}_{n-Faktoren}$
%
\sss{Ausmultiplizieren}
Halte Terme der Form $a^{n-k}b^k$ mit $0\leq k\leq n$\\
Häufigkeit von $a^{n-k}b^k$ = Anzahl der Möglichkeiten aus n-Faktoren k mal b zu wählen.\\
Das ist $\binom{n}{k}$ (Satz 1.13)
%
\sss{Folgerung}
Setze $a=b=1\qquad a^{n-k}b^k=1$\\
$(a+b)^n=2^n=\binom{n}{0}+\binom{n}{1}+\binom{n}{2}+…+\binom{n}{n}$\\
%
\bsp
$1+4+6+4+1=16=2^4$
%
\sS{Definition}
Sei A endliche Menge\\
Eine Anordnung von A ist ein n-Tupel\\
$(a_1,a_2,a_3,a_4,…,a_n)$ mit $a\in A$ für alle i und $a_i\neq a_j$ wenn $i\neq j$\\

\bsp
Anordnung von $\{1,2,3\}=(1,2,3)(1,3,2)(2,1,3)(2,3,1)(3,1,2)(3,2,1)→6$
%
\sS{Satz}
Sei $A$ endliche Menge, $\#A=n\geq 1$\\
Dann ist die Anzahl der Anordnungen von $A$ gleich $n!$
\bew
Induktion nach n
\ind{n=1}{Sei $\#A=n+1$\\
Wahl einer Anordnung von $A$ kann man so unterteilen:\\
\begin{enumerate}
\item{Wähle 1 Element $a_1\in A$ (n+1 Möglichkeiten)}
\item{Wähle Anordnungen von $A\bs\{a_1\}$\\
$\#(A\bs\{a_1\})=n$ \Rarr $n!$ Möglichkeiten bei 2\\
Insgesamt $(n+1)·n!=(n+1)!$}
\end{enumerate}}
%
\bem
(Zusammenhang zwischen Anordnung und Teilmengen)\\
Sei $A$ endliche Menge, $\#A=n,\ 0\leq k\leq n$\\
Sei $(a_1,…,a_n)$ Anordnung von $A$\\
$\leadsto$ Teilmenge $U:=\{a_1,…,a_n\}$\\
Dann $U\subseteq A,\ \#U=k\qquad U\in P_k(A)$\\
Jedes $U\in P_k(A)$ entsteht so, aber mehrfach:\\
\[\underset{\overset{\uparrow}{Anordnungen\ von\ U}}{k!}·\underset{\overset{\uparrow}{Anordnungen\ von\ A\backslash U}}{(n-k)!}-mal\]
$\#$ Anordnungen von $A=n!=\#P_k(A)·k!(n-k)!\Rarr\#P_k(A)=\frac{n!}{k!·(n-k)!)}=\binom{n}{k}$\\
%
\chapter{Die reellen Zahlen}
Was sind die reellen Zahlen?\\
Präzise Konstruktion ist umfangreich, daher Axiomatischer Zugang\\
Beschreibung der reellen Zahlen durch ihre Eigenschaften (Axiome):\\
\begin{enumerate}
\item{Grundrechenarten → Körper}
\item{Ungleichungen → angeordneter Körper}
\item{Lückenlosigkeit → Vollständigkeit}
\end{enumerate}
%
\uS{Körper}
%2.1
\Def
Ein Körper ist eine Menge $K$ mit 2 Rechenoperationen:\\
Addition (+) und Multiplikation (·), so dass folgende 9 Eigenschaften erfüllt sind:\\[8pt]
\underline{Addition}\\[-15pt]
\begin{enumerate}
\item{$(a+b)+c=a+(b+c)$ für alle $a,b,c\in K$ (Assotiativgesetz)}
\item{$a+b=b+a$ für alle $a,b\in K$ (Kommutativgesetz)}
\item{Es gibt ein $0\in K$ so dass $0+a=a$}
\item{Für jedes $a\in K$ gibt es ein $b\in K$ mit $a+b=0$}
\bem
$0\in K$ ist eindeutig
\bew
Wenn $0'\in K$ mit $0'+a=a$, dann $0=0'+0=0+0'=0'$\qed
\bem
Das $b$ in 4. ist auch eindeutig.\\
\notat{$b=-a$ (Negatives von $a$)}
\bew
Angenommen $b'+a=0$\\
$b=b+0=b+(a+b')=(b+a)+b'=0+b'=b'$\qed
\end{enumerate}
\underline{Multiplikation}\\[-15pt]
\begin{enumerate}
\setcounter{enumi}{4}
\item{$a(b·c)=(a·b)c\qquad ∀a,b,c\in K$}
\item{$a·b=b·a\qquad ∀a,b\in K$}
\item{Es gibt ein $1\in K$ mit $1\neq 0$, so dass $1·a=a\qquad ∀a\in K$}
\item{Für alle $a\in K,\ a\neq 0$, gibt es ein $b\in K$ mit $a·b=1$}
\bem
$1\in K$ ist eindeutig, $b$ in 8. ist eindeutig\\
Beziehung $b=a^{-1}$
\bew
Wie eben\qed
\item{$a(a+c)=a·b+a·c\qquad ∀a,b,c\in K$ (Distributivgesetz)}
\end{enumerate}
Weitere Bezeichnungen:\\
$a-b:=a+(-b),\ \frac{a}{b}=a·b^{-1}$, wenn $b≠0$
\bem
Die üblichen Rechenregeln folgen aus diesen Axiomen 1.-9.
\bsp
$$-(-a)=a,\ a(b-c)=a·b+a·c,\ a(-b)=-(a·b)$$
%
\sS{Beispiele}
\Q ist ein Körper\\
\Z ist kein Körper (8. nicht erfüllt)
%
\sS{Beispiel}
$\mathbb{F}_z=\{0,1\}$\\
\underline{Definitionen von + und · :}\\[8pt]
\parbox{.2\textwidth}{\begin{tabular}{c|cc}
+&0&1\\[2pt]\hline
0&0&1\\[2pt]
1&1&0
\end{tabular}}
\parbox{.2\textwidth}{\begin{tabular}{c|cc}
·&0&1\\[2pt]\hline
0&0&0\\[2pt]
1&0&1
\end{tabular}}
$1+1=0$\\[4pt]
\fbox{\underline{Übung:} Prüfe alle Körperaxiome}\\
\bem
Sei $K$ \underline{endlicher} Körper\\
Dann gilt $\#K=p^r$ wobei $p$ Primzahl, $r\in\N$\\
Für jede solche Zahl $q=p^r$ gibt es genau einen Körper

\newpage
%%Kopfzeile links bzw. innen
\fancyhead[L]{\calligra {\Large Vorlesung Nr. 3}}
%Kopfzeile rechts bzw. außen
\fancyhead[R]{\calligra \Large{15.10.2012}}
% **************************************************
%
\wdh
Ein Körper K ist eine Menge mit $+$ und $\cdot$, sodass gewisse Eigenschaften erfüllt sind:
\bsp
$\ds\Q = \left\{\frac{a}{b} | a \in \Z, b \neq 0\right\}$\\
$F_1 = \{0, 1\} \qquad 1 + 1 = 0$\\
\notat{Setze $a^n = \underbrace{ a · a · a · a· … · a}_{n-Faktoren}$\\
$\left.\begin{array}{lcc}
a^0 &=& 1\\
a^{-n} &=& (a^{-1})^n
\end{array}\right\}$ wenn $a \neq 0$}
Daraus folgt $a^n$ ist definiert, wenn $a \neq 0$ und $n \in \Z$\\
Regeln der Potenzgleichung:\\
$a^{n+m} = a^n \cdot a^m$\\
$a^{n \cdot m} = (a^{n})^m$\\
\bew
Übung
%
\sS{Definition}
    Ein angeordneter Körper ist ein Körper K für dessen Elemente eine "Kleiner als Beziehung" $<$ definiert ist, so dass folgende Eigenschaften erfüllt sind:\\
    \begin{enumerate}
    \item{Für alle $a, b \in K$ gilt genau eine von drei Notationen:\\
    $a < b oder a = b oder a > b$}
    \item{Für alle $a, b, c \in K$ gilt wenn $a < b$ und $b < c$ dann $a < c$\\ (Transitivität)}
    \item{Für alle $a, b, c \in K$ gilt wenn $a < b$ dann $a + c < b + c$}
    \item{für $a, b, c \in K$ gilt, wenn $a < b$ und $c \neq 0$ dann $a \cdot c < b \cdot c$}
    \end{enumerate}
	Weitere Beziehungen:\\
	$a > b$ heißt $b < a$\\
	\begin{enumerate}
	\item{Wenn $a < 0$ dann $-a > 0$:\\
	$a < 0 \Rarr a + (-a) > 0 + (-a) \Rarr 0 > -a$}
	\item{Für jedes $a \in K $ gilt wenn $a \neq 0$, dann $a^2 > 0$\\
	\begin{itemize}
	\item[(a)]{$\begin{array}{ccc}
	a &>& 0\\
	a · a &>& 0 · a\\
	a^2 &>& 0
	\end{array}$\qed}
	\item[(b)]{$\begin{array}{ccc}
		a &<& 0\\
		-a &>& 0 · a\\
		a^2 &=& (-a)^2 > 0
		\end{array}$\qed}
	\end{itemize}}
	\item{$1 > 0 $ denn $1 = 1^2$}
	\end{enumerate}
	Sei $K$ ein Angeordneter Körper:\\
	$0 < 1 \Rarr 1 < 1 + 1 \Rarr 1 + 1 < 1 + 1 + 1 $ etc.\\
	$0 < 1 < 1 + 1 < 1 + 1 +1 $ etc.
	Für $n \in \N$ setze $underbrace{n:= 1 + 1 + 1 + ... + 1}$
	
	\phantom{Für $n \in \N$ setze XXX} n-Faktoren\\
	Dann $0 < 1 < 2 < 3 ... $ in $K$
	\underline{Folge:} Verschiedene natürliche Zahlen bleiben in $K$ verschieden.\\
	%Falscher Pfeil, verbesserung kommt noch
	Fasse $\N$ als Teilmenge von $K$ auf.\\
	Dann $$\Z = \left\lbrace a - b | a, b \in N \right\rbrace \subseteq K$$\\
	\phantom{Dann }$$Q = \left\lbrace \frac{a}{b} | a, b \in N\right\rbrace \subseteq K$$\\ \\
	Insbesondere ist $K$ unendlich.\\
	\\
	z.B. hat $F_z$ keine Anordnung.\\
	\\
\sS{Definition}
	Sei $K$ ein angeordneter Körper mit $a \in K$\\
	Der Absolutbetrag von $a$ ist definiert als \\
	$|a| = \left\lbrace \begin{array}{lll}
	\text{$a$ wenn a > 0}\\
	\text{$-a$ wenn a < 0}
	\end{array}\right.$
\sS{Satz}
	Sei $K$ ein angeordneter Körper $a, b, c \in K$\\
	Dann gilt:
	\begin{enumerate}
	\item{$a = 0$ wenn $|a| = 0$}
	\item{$-|a| \leq a \leq |a|$}
	\item{Dreiecksungleichung\\
	$|a + b| \leq |a| + |b|$}
	\item{untere Dreiecksungleichung
	$|a - b| \geq |a| - |b|$} % Nochmal prüfen...
	\end{enumerate}
\bew
	\begin{enumerate}
	\item{klar.}
	\item{wenn $a \geq 0$:\\
	$|a| \geq 0$\\
	$\Rarr -|a| \leq 0 \leq a \leq |a|$\\
	wenn $a \leq 0$:
	$-|a| \leq a \leq 0 \leq |a|$}
	\item{Es gilt: $-|a| \leq a \leq |a|$, $-|b| \leq b \leq |b|$\\
	wenn $a + b \geq 0$\\
	$|a + b| = a + b \leq |a| + b \leq |a| + |b|$
	wenn $a + b < 0$\\
	$|a + b| = -(a + b) = (-a) + (-b) \leq |a| + |b|$}
	\item{$(a - b) + b = a$\\
	$\Rarr |a| = |(a - b) + b| \leq |a - b| + b$\\
	$|a - b| \leq |a - b|$} % ??? Noch mal gegenlesen, mathematisch schwierig.
	\end{enumerate}
\sS{Satz Bernoulli'sche Ungleichungen}
	Sei $K$ ein angeordneter Körper $a, b \in K, a > -1$ und $n \in \N \{0, 1, 2, 3, 4, ...\}$.\\
	Dann gilt:\\
	$(1 + a)^n \geq 1 + n \cdot a$\\
	\\
	Beweis durch vollständige Induktion:\\
	\ind{
		$$n = 0\\
		(1 + a)^0 = 1 = 1 + 0 \cdot a$$
	}{
		Annahme:\\
		$$(1 + a)^{n + 1} = (1 + a)(1 + a)^n \geq (1 + a)(1 + n \cdot a)$$\\
		weil $1 + a > 0$\\
		$= 1 + a + n \cdot a + n \cdot a^2$\\
		$= 1 + (n + 1) \cdot a + n \cdot a^2$\\
		weil $a^2 \geq 0 \Rarr n \cdot a^2 \geq 0$
	}
\sS{Definition}
	Sei $K$ ein angeordneter Körper, $M \subseteq K$ eine Teilmenge, $a \in K$.\\
	\begin{enumerate}
	\item{$M \leq a$ bedeutet: $x \leq a$ für jedes $x \in M$}
	\item{$a$ heißt ",obere Schranke"' von $M$, wenn $M \leq a$.\\
		$a$ heißt ",untere Schranke"' wenn $M \geq a$}
	\item{$M$ heißt nach oben beschränkt wenn $M$ eine obere Schranke hat.\\
	Analog: nach unten beschränkt wenn $M$ eine untere Schranke hat.}
	\item{$a$ heißt Maximum von $M$, wenn $M \leq a$ \underline{und} $a \in M$. $a = max(M)$\\
		$a$ heißt Minimum von $M$, wenn $M \geq a$ \underline{und} $a \in M$. $a = min{M}$}
	\end{enumerate}
\bew
	Sei $a, b \in M$\\
	$M \leq a, M \leq b$\\
	Dann $b \leq a$ und $b \leq a \Rarr a = b$ \qed\\
\bsp
	$K = \Q$
	\begin{enumerate}
	\item{$M = \N$\\
	Sei $a \in \Q$\\
	$a \leq N \equ a \leq n$ für alle $n \in N$\\
	\phantom{$a \leq \N $} $\equ a \leq 1$\\
	Wenn $N$ nach unten beschränkt $1 = min(N)$}
	\item{$M = \left\lbrace -\frac{1}{n} | n \in \N \right\rbrace \qquad 0 \notin M$\\
	\begin{tabbing}
	$-1 = min(M)$ \= $\Rarr M$ ist nach unten beschränkt.\\
	$M \leq 0$	\> $\Rarr M$ ist nach oben beschränkt.\\
	\end{tabbing}
	$M$ hat kein Maximum.\\
	Sei $a \in M$ dann $a = -\frac{1}{n}, n \in N, -\frac{1}{n + 1} \in M$\\
	$n + 1 > n \Rarr \frac{1}{n + 1} < \frac{1}{n} \Rarr -\frac{1}{n + 1} > -\frac{1}{n}$\\
	$M \nleq -\frac{1}{n}$ $a$ ist keine obere Schranke.}
	\item{$M = \left\lbrace -\frac{1}{n} | n \in \N\right\rbrace \cup \{ 0 \}$\\
	$min(M) = -1$\\
	$max(M) = 0$}
	\item{$M = \emptyset$ hat weder ein $min(M)$ noch ein $max(M)$\\
	Jedes $a \in \Q$ erfüllt $a \leq M$ und $M \leq a$}
	\end{enumerate}
\sS{Satz}
	\begin{enumerate}
	\item{Sei $K$ ein angeordneter Körper.\\
	Wenn $M$ endlich und nicht leer, dann hat $M$ auch ein $max$ und ein $min$}
	\item{Wohlordnungsprinzip\\
	Jede nicht leere Teilmenge $M \in \N$ hat ein Minimum.}
	\end{enumerate}
\bew
	\begin{enumerate}
	\item{klar.}
	\item{$M$ ist nicht leer, wähle $n \in M$\\
	$\{1, 2, 3, 4, 5, ... n\}$, endlich aber nicht leer.\\
	Dann $min(\{1, 2, 3, 4, 5, ... n\} \cap M) = min(M)$ \qed}
	\end{enumerate}
\sS{Definition: Vollständigkeit}
	Sei $K$ ein angeordneter Körper und $M \subseteq K$, $a \in K$\\
	$a \ heißt kleinste obere Schranke von M oder Supremum.$
	\begin{enumerate}
	\item{$M \leq a$}\\[6pt]
	und
	\item{kein $b \in K$ mit $b < a$ erfüllt $M \leq b$}
	\end{enumerate}
	$a$ ist größte untere Schranke oder Infimum vom $M$, wenn
	\begin{enumerate}
	\item{$a \leq M$}\\[6pt]
	und
	\item{Kein $b \in M$ mit $a < b$ erfüllt $b \leq M$}
	\end{enumerate}
\notat{
	$a = sup(M)$\\
	$a = inf(M)$}
\bem
	Wenn $a = max(M) \Rarr a = sup(M)$ 
\bew
	Sei $a, b \in M$ und $a \nleq b$\\
	$\Rarr M \nleq b \Rarr a$ ist Supremum
\bem
	Wenn ein Supremum existiert, ist es eindeutig.
\bew
	$a, b$ sind Supremum von M\\
	$M \leq a$, $M \leq b \Rarr a \leq b $ und $b \leq a \Rarr a = b$ \qed
\bsp
	$sup(\{ -\frac{1}{n} | n \in \N \})$\newpage
%\wdh
Angeordneter Körper:\\
Menge $K$ mit $+, ·, <$\\
so dass gewisse Eigenschaften erfüllt sind
\bsp
\Q{} sind ein angeordneter Körper\\
Sei $K$ angeordneter Körper, $M\subseteq K$ Teilmenge $a\in K$ ist obere Schranke von $M$, wenn $U\subseteq a$, d.h.: $x\leq a\qquad ?x\in M$\\
$a\in K$ ist kleinste obere Schranke, wenn\\
\begin{enumerate}
\item{$M\leq a$}
\item{Wenn $b < a$, dann \underline{nicht} $M\leq b$}
\end{enumerate}
\vspace*{-9.5ex}\hspace*{15.5em}
$\left.
\begin{array}{l}
{}\vspace*{2ex}\\{}
\end{array}
\right\}$
\vspace*{-5ex}Bezeichnung $a=sup(M)$
\vspace*{5ex}
%
\bsp
$K=\Q\qquad M=\{-\frac{1}{n}|n\in\N\}=\{-1,-\frac{1}{2},-\frac{1}{3},?\}$\\
\sss{Behauptung}
$sup(M)=0$
\bew
\begin{enumerate}
\item {Zeige: $M \leq 0$, d.h.: $\frac{1}{n}<0$ für alle $n\in\N$\ok}
\item {Wenn $b=\Q,\ b<0$, dann nicht $M\leq b$}
\end{enumerate}
Schreibe $b=\frac{m}{n},\ m\in\Z, n\in\N$\\[1ex]
$b<0$ heißt $m<0,\ m\leq -1$\\[1ex]
$b=\frac{m}{n} \leq \frac{-1}{n} \leq \frac{-1}{n+1}\in M$\\[1ex]
\Rarr{} $M\not\leq b$ (nicht $M\leq b$)\qed
%
\uS{Vollständigkeit}
\Def
Ein angeordneter Körper $K$ heißt Dedekind-vollständig, wenn jede nach oben beschränkte Teilmenge von $K$ eine kleinste obere Schranke hat.
\Satz
Es gibt genau einen Dedekind-vollständigen, angeordneten Körper $K$\\
Dieser heißt Körper der reellen Zahlen\\
\underline{Bezeichnung:} \R\\
(Beweis ausgelassen)
%
\Satz
Die Teilmenge \N{} von \R{} ist unbeschränkt
\bew
(verwende nur die Axiome)\\
Indirekter Beweis: Angenommen, \N{} ist beschränkt\\
\einruck{Vollständigkeit:}{\vspace{3.65ex}\N{} hat eine kleinste obere Schranke $a\in\R$\\
Es gilt $a-1<a \Rarr{} a-1$ ist kleinste obere Schranke von $\N\ n\leq a\qquad ∀ n\in\N\\
\Rarr{} n+1\leq a\qquad ∀ n\in\N\\
\Rarr{} n\leq a-1\qquad∀n\in\N$ Widerspruch!\\
Also Annahme falsch, d.h. \N{} ist unbeschränkt\qed}
\begin{tabular}{lcl}
beschränkt &=& nach oben beschränkt und nach unten beschränkt\\
unbeschränkt &=& nicht nach oben beschränkt oder nicht nach unten beschränkt
\end{tabular}
%
\sS{Folgerung (Prinzip des Archimedes)}
Seien $x,y\in\R,\ x>0$, Dann gibt es $n\in\N$ mit $n·x>y$\\
SKIZZE % SKIZZE
%
\bew
$nx>y \equ n>\frac{y}{x}$ (weil $x>0$)\\
\N{} unbeschränkt und nicht nach oben beschränkt \Rarr{} $\frac{y}{x}$ ist keine obere Schranke von \N\\
\Rarr{} es gibt $n\in\N$ mit $n>\frac{y}{x}$\qed
%
\sS{Folgerung}
Sei $x\in\R,\ x>0$ Dann gibt es $n\in\N$ mit $\frac{1}{n}<x$\\
SKIZZE % SKIZZE
\bew
$\frac{1}{n}<x \equ 1<n·x \equ \frac{1}{x}<n$ (weil $x$ positiv)\\
$\frac{1}{x}$ keine obere Schranke von \N{} \Rarr{} es gibt \nN{} mit $\frac{1}{x}<n$\qed
%
\Satz
Seien $x,y\eR$ mit $x<y$\\
Dann gibt es $a\eQ$ mit $x<a<y$, man sagt \Q{} liegen dicht in \R\\
SKIZZE % SKIZZE
\bew
$y-x>0$ Wähle \nN{} mit $\frac{1}{n}<y-x$\\
Ansatz: $a=\frac{m}{n}$ mit $m\eZ$\\
Sei $M:=\{m\eZ|x<\frac{m}{n}\}=\{m\eZ|nx<m\}$\\
$M$ ist nach unten beschränkt und nicht leer (wegen Archimedes)\\
$M$ hat Minimum\\
Sei $m=min(M)$\\
$m\in M \Rarr x<\frac{m}{n}$\\
$m-1\not\in M \Rarr x\geq\frac{m-1}{n}$\\
$y-\frac{m}{n} =y-x+x-\frac{m}{n}>\frac{1}{n}+x-\frac{m}{n}=x-\frac{m-1}{n}\geq0$\\
$y>\frac{m}{n}$\qed
%
\uS{Wurzeln}
\Satz
Es gibt kein $a\eQ$ mit $a^2=2$\\
\bew
Angenommen $a\frac{m}{n}\eQ,\ a^2=2,\ m,\nN$\\
Kürze den Bruch $\Rarr \frac{m}{n}$ teilerfremd\\
$$a^2=2\Rarr \frac{m^2}{n^2}=2 \Rarr m^2=2n^2 \Rarr m^2 \text{ gerade } \Rarr m \text{ gerade } \Rarr m=2q,\ q\eN$$
$$(2q)^2=2n^2 \Rarr 4q^2=2n^2 \Rarr 2q^2=n^2 \Rarr  n^2 \text{ gerade } \Rarr n \text{ gerade }$$
Widerspruch zur Annahme $m,n$ teilerfremd\qed\\
SKIZZE WURZEL 2 \Rarr $\sqrt{2}$ sollte existieren % SKIZZE
\bem
Wenn \nN, keine Quadratzahl, dann gibt es kein $a\eQ$ mit $a^2=n$ (ähnlicher Beweis)
%
\Satz
Sei $x\eR, x\geq 0, \nN$\\
Dann gibt es \underline{genau ein} $y\eR, x\geq 0$ mit $y^n=x$\\
Bezeichnung: $x=\sqrt[n]{x}$
\bew
später\\
\underline{Ansatz:} $sup\{a\eQ|a^n\leq x\}=:y$ (sup existiert weil \R{} Dedekind-vollständig)
%
\Def
Sei $x\eR,\ x>0\qquad \frac{m}{n}\eQ$\\
\nN, $m\eZ\qquad x^{\frac{m}{n}}=\sqrt[n]{x^m}\qquad x^{\frac{1}{n}}=\sqrt[n]{x}$\\
\sss{Potenzrechnung:} $$x^{(a+b)}=x^a·x^b,\ x^{a·b}=(x^a)^b$$\hfill für $x\eR,\ x>0,\ a,b\eQ$\\
\bem
Später wir definiert: $x^a$ für $x\eR,\ x>0,\ a\eR$
\chapter{Folgen und Reihen reeller Zahlen}

\newpage
%% Kopfzeile beim Kapitelanfang:
\fancypagestyle{plain}{
%Kopfzeile links bzw. innen
\fancyhead[L]{\calligra\Large Vorlesung Nr. 5}
%Kopfzeile rechts bzw. außen
\fancyhead[R]{\calligra\Large 22.10.2012}
}
%Kopfzeile links bzw. innen
\fancyhead[L]{\calligra\Large Vorlesung Nr. 5}
%Kopfzeile rechts bzw. außen
\fancyhead[R]{\calligra\Large 22.10.2012}
% *****************************************
%
\wdh
Eine Folge $(a_n)_{n\eN_0}$ reeller Zahlen konvergiert genen $a\eR$ wenn gilt:\\
Für jedes $\e>0$ gibt es ein $N\eN$ so dass $|a_1·a|<\e$ für alle $n\geq\N$.\\
%bez oder so was, nicht ganz klar zu lesen
\ul{Bez"uglich} $a_n→a$ für $n→∞$ oder $\ds\lim_{n→∞}(a_n)=a$\\
\bsp
$\frac{1}{n}→0$ für $n→∞\qquad(-1)^n$ divergiert\\
$(a_n)$ ist divergent, wenn sie gegen kein $a\eR$ konvergiert
\bsp
$(1,0,\frac{1}{2},0,\frac{1}{3},0,\frac{1}{4},0,…)$ konvergiert gegen $0$
%
\sS{Satz: (Eindeutigkeit der Grenzwerte)}
Sei $(a_n)$ Folge reeller Zahlen und $a,b\eR$ mit $a_n→a$ und $a_n→b$ für $n→∞$. Dann ist $a·b$ %mut zur lücke, würde ich behaupten
\bem
Dann ist bez %wie oben, whatever it means
$a=\ds\lim_{n→∞}(a_n)$ sinnvoll
\bew
Angenommen $a\neq b$\\
Sei $\e:=\dfrac{|a-b|}{2}$ SKIZZE\\ % SKIZZE
Konvergenz: es gibt $N_1\eN$ mit $|a_n-a|<\e$, $N_2\eN$ mit $|a_n-b|<\e$ für $n\geq N_2$\\
Sei $n=max(N_1,N_2)$\\
$|a-b|=|a-a_n+a_n-b|\leq|a-a_n|+|a_n-b|<\e+\e=|a-b|$\\
\Rarr $|a+b|<|a-b|$ Widerspruch\\
\Rarr nicht $a\neq b$, d.h. $a=b$\qed
%
\sS{Definition}
Eine Folge $(a_n)$ reeller Zahlen heißt $\left\{\begin{array}{c}\text{nach oben beschränkt}\\\text{nach unten beschränkt}\\\text{beschränkt}
\end{array}\right\}$ wenn die menge $\{a_n|n\eN_0\}$ dieselbe Eigenschaft hat.
%
\sS{Satz:}
Jede konvergente Folge reeller Zahlen ist beschränkt.
\bew
Angenommen $a_n→a$ für \nif\\
Wähle $\e=1$, Es gibt $N\eN$ so dass $|a_n-a|<1$ für $n\geq N$\\
Sei $C:=max\{|a_0|,|a_1|,…,|a_{n-1}|,|a|+1\}$\\
Dann $|a_n|\leq C$ für $n\leq N-1$\\
Für $n\geq N$ gilt:
$$|a_n|=|a_n-a+a|\leq|a_n-a|+|a|<1+|a|\leq C$$
Somit $|a_n|\leq C$ für alle $n$\\
$-C\leq a_n\leq C$ für alle $n$\\
\Rarr Folge $(a_n)$ ist beschränkt.
\bem
Nicht jede beschränkte Folge konvergiert.\\
z.B. $((-1)^n)_{n\eN_0}$ ist beschränkt, aber konvergiert nicht.
%
\sS{Definition}
Eine Folge reeller Zahlen $(a_n)$ konvergiert uneigentlich gegen ∞ wenn gilt:\\
Für jedes $C\eR$ gibt es $N\eR$ mit $a_m>C$ für alle $n\geq N$ SKIZZE
\bem
Alternative Terminologie:\\
"konvergiert uneigentlich"="divergiert bestimmt"
\bsp
\begin{enumerate}
\item{$a_n=n.\ a_n→∞$}
\item{$a_n=(-1)^n.\ (0,-1,2,-3,4,-5,…)$ konvergiert \ul{nicht} uneigentlich gegen ∞}
\end{enumerate}
\notat{$a_n→∞$ für \nif{} $\ds\lim_\nif\an=∞$}
%
\sS{Satz (Potenzwachstum)}
Sei $x\eR$ betrachtete % sehr unsicher, steht nur "Betr." dort
Folge $(x^n)_n\geq 0$
\begin{enumerate}
\item{wenn $|x|>1$ dann ist $(x^n)$ divergent}
\item{wenn $x>1$ dann $x^n→∞$ für \nif}
\item{wenn $|x|<1$ dann ist $x^n→0$  für \nif}
\end{enumerate}
%
% folgendes ist von der einrückunt und unterordnung sehr unsicher bitte überprüfen
%
\Bew{2)}
Sei $x>1$\\[4pt]
Schreibe $x=1+a$. Dann $a>0$ Gegeben $C\eR$\\
$\underset{\text{Satz 2.9}}{\Rarr} x^n=(1+a)^n\geq 1+n·a$\\
Archimedes: $∃ N\eN$ mit $N·a>C$\ok
\Bew{1}
Sei $|x|>1$ Dann $|x^n|=|x|^n,\ |x|>1\underset{\text{2)}}{\Rarr}|x^n|$ ist nicht beschränkt für \nN{} \Rarr{} $(x^n)$ divergiert
\Bew{3}
Sei $|x|<1$ Wenn $x=0 \Rarr x^n=0$ für alle $n$\ok\\
Sei $0<|x|<1$\\
Dann $\frac{1}{|x|}>1$\\
Gegeben sei $\e>0$\\
Setze $C=\frac{1}{\e}$\\
$\underset{\text{2)}}{\Rarr}$ es gibt $N\eN$ mit $\frac{1}{|x|^n}>C$ für $n\geq N$ \Rarr $|x|^n<\e$ für $n\geq N$\qed 
%
\sS{Satz (Rechenregeln)}
Seien $(a_n)_{\nN_0},\ (b_n)_{\nN_0}$ zwei konvergente Folgen reeller Zahlen\\
Sei $a_n→a$ für \nif\\
\phantom{Sei }$b_n→a$ für \nif\\ % EINRÜCKUNG
Dann gilt:
\begin{enumerate}
\item{$(a_n+b_n)→a+b$ für \nif}
\item{$(a_n·b_n)→a·b$ für \nif}
\item{Angenommen $b\neq 0$\\
Dann ist $b\neq 0$ für fast alle \nN{} und $\frac{1}{b_n}→\frac{1}{b}$ für \nif}
\end{enumerate}
%
\sss{Definition}
"fast alle"="alle bis auf endlich viele".
%
\bew
\begin{enumerate}
\item{Gegeben sei $\e>0$\\
Es gibt $N_1\eN$ mit $|a_n-a|<\frac{\e}{2}$ für $n\geq N_1$\\
Es gibt $N_2\eN$ mit $|b_n-b|<\frac{\e}{2}$ für $n\geq N_2$\\
Sei $N=max(N_1,N_2)$ für $n\geq N$ gilt:\\
$$|a_n+b_n-(a+b)|=|(a_n-a)+(b_n-b)|\leq |a_n-a|+|b_n-b|<\frac{\e}{2}+\frac{\e}{2}=\e \Rarr \text{1)}$$}
\item{$(a_n)$ konvergiert \Rarr{} ist beschränkt.\\
Es gibt $C\eR$ mit $|a_n|<C$ für alle $\nN_0$\\
ohne Einschränkungen sei $C>|b|$\\
Rechne:
$$|a_n·b_n-a·b|=|a_n·b_n-a_n·b+a_n·b-a·b|=|a_n(b_n-b)+b(a_n-a)|\geq |a_n|·|b_n-b|+|b|·|a_n-a|$$
Es gibt $N\eN$ mit $\left.\begin{array}{l}|a_n-a|<\frac{1}{2C}·\e\\|b_n-b|<\frac{1}{2C}·\e\end{array} \right\}$ für $n\geq N$\\
Für $n\geq N$ gilt:
$$|a_n·b_n-a·b|<|a_n|\frac{1}{2C}\e+|b|\frac{1}{2C}\e\leq C·\frac{1}{2C}\e+C·\frac{1}{2C}\e=\e \Rarr \text{ 2) gilt}$$}
\item{Sei $b\neq 0$\\
Wähle $\e=\frac{1}{2}|b|>0$ SKIZZE\\
Es gibt $N\eN$ mit $|b_n-b|<\frac{1}{2}|b|$ für $n\geq N$\\
Dann gilt für $n\geq N$:
$$|b_n|=|b_n-b+b|=|b-b+b_n|=|b-(b-b_n)|\geq |b|-|b-b_n|>|b|-\frac{1}{2}|b|=\frac{1}{2}|b|$$
Insbesondere $|b_n|\neq 0$ für $n\geq N$\\
Rechne:
$$\left|\frac{1}{b}-\frac{1}{b_n}\right|=\left|\frac{b_n-b}{b·b_n}\right|=\frac{1}{|b|·|b_n|}·|b_n-b|<\frac{2}{|b|^2}·|b_n-b| \text{ für $n\geq N$}$$\footnote{NR: $|b_n|>\frac{1}{2}|b| \Rarr \frac{1}{|b_n|}<\frac{2}{|b_n|}$}
Gegeben sei $\e>0$\\
Es gibt $N_1\eN$ mit $|b_n-b|<\frac{|b|^2}{2}\e$ für $n\geq N_1$\Rarr{} für $n\geq max(N_1,N_2)$ gilt:\\
$$|\frac{1}{b_n}-\frac{1}{b}<\frac{2}{|b|^2}·\frac{|b|^2}{2}\e=\e \Rarr \text{ 3) gilt}$$\qed}
\end{enumerate}
\ul\{Zusatz:\} Wenn \$a\_n→a\$ und \$b\_n→b\$ für \verb+\+nif\{\} dann gilt:
\begin{enumerate}
\setcounter{enumi}{3}
\item{Für $C\eR$ ist $C·a_n→C·a$ für \nif }% C groß oder klein?
\item{$(a_n-b_n)→a-b$ für \nif}
\item{Wenn $b\neq 0$ dann $\frac{a_n}{b_n}→\frac{a}{b}$ für \nif}
\end{enumerate}
\bew
Übung\newpage
%% Kopfzeile beim Kapitelanfang:
\fancypagestyle{plain}{
%Kopfzeile links bzw. innen
\fancyhead[L]{\calligra\Large Vorlesung Nr. 6}
%Kopfzeile rechts bzw. außen
\fancyhead[R]{\calligra\Large 25.10.2012}
}
%Kopfzeile links bzw. innen
\fancyhead[L]{\calligra\Large Vorlesung Nr. 6}
%Kopfzeile rechts bzw. außen
\fancyhead[R]{\calligra\Large 25.10.2012}
% *****************************************
%
%\setcounter{chapter}{3}
%\setcounter{section}{9}
%
\wdh
Eine Folge reeller Zahlen $(a_n)$ konvergiert uneigentlich gegen ∞ wenn gilt:\\
Für jedes $C\eR$ gibt es ein \nN{} mit $a_n > C$ für jedes \nN\\[4pt]
$(a_n)$ konvergiert uneigentlich gegen $- ∞$ wenn $(-a_n)$ gegen $∞$ konvergiert.\\
%
\notat{
$a_n \to ∞ \qquad \text{ für } n \to ∞$\\
$a_n \to - ∞ \qquad \text{ für } n \to ∞$
}
%
\bsp
$a_n = n^2 \to ∞$\\
$a_n = -n^2 \to -∞$\\
$a_n = (-1)^n · n^2$\\
$(0, -1, 4, -9)$ konvergiert weder gegen $∞$ noch gegen $ - ∞$
%
\sss{Rechenregeln:}
Angenommen $(a_n), (b_n)$ sind konvergente Folgen.\\
\begin{enumerate}
\item{$(a_n + b_n) \to a + b$}
\item{$(a_n · b_n) \to ab$}
\item{$\ds\frac{1}{b_n} \to \frac{1}{b}$}
\item{$c · a_n \to c · a$}
\item{$a_n - b_n \to a - b$}
\item{$\ds\frac{a_n}{b_n} \to \frac{a}{b}$}
\end{enumerate}
%
\Bew{6}
3) $\Rightarrow\ds\frac{1}{b_n}→\frac{1}{b}$\\
$\ds\frac{a_n}{b_n} = a_n ·\frac{1}{b}$\\
2) $\ds\Rightarrow a_n · \frac{1}{b_n} \to a ·\frac{1}{b} = \frac{a}{b}$\qed
%
\bsp
\begin{tabular}{l|c|c|c|c|c|r}
$n$   & 0 & 1 & 2 & 3 & 10 & 100\\\hline
$a_n$ & 0 & 0 & $\frac{2}{9}$ & $\frac{6}{19}$ & $\frac{90}{201}$ & $\frac{9900}{20001}$ \\
\end{tabular}
\vspace{5mm}
Vermutung: $a_n \to \ds\frac{1}{2}$ für $n \to ∞$\\
Rechenregel 6 anwenden:\\
\begin{itemize}
\item[1.]{Versuch:\\
$a_n = \frac{b_n}{c_n}$\\[4pt]
$b_n = n^2 -n; c_n = 2n^2 + 1$\\
$(b_n) und (c_n)$ sind divergend. Schlecht.}
\item[2.]{Versuch:\\
$\ds\frac{n^2 - n}{2n^2 + 1} = \frac{n^2(1 - \frac{1}{n})}{n^2(2 + \frac{1}{n^2}}$ für $n \geq 1$\\[4pt]
$= \ds\frac{1-\frac{1}{n}}{2 +\frac{1}{n^2}}=\frac{b_n}{c_n}$ mit $b_n:=1-\frac{1}{n},\ c_n = 2 + \frac{1}{n^2}$\\[4pt]
$\ds\frac{1}{n} \to 0$ für $n \to ∞ $\\[4pt]
$\ds\Rightarrow 1 - \frac{1}{n} \to 1 - 0 = 1$ für $n \to ∞$\\[4pt]
$\Rightarrow 2 + \ds\frac{1}{n^2} \to 2 + 0 = 2$ für $n \to ∞$}
\end{itemize}
%
$\Rightarrow a_n \to \frac{1}{2}$ für $n \to ∞$
%
\sS{Satz}
Seien $a_n \to a$, $b_n \to b$ zwei konvergente Folgen reeler Zahlen.\\
wenn $a_n \leq b_n$ für unendlich viele $n \in \N{}$ dann ist $a \leq b$.
\bew
Angenommen: $a > b$\\

Wähle $\e := \ds\frac{a - b}{2} > 0$\\
Es gibt $N \in \N{}$ so dass:
$
\left.
\begin{array}{ll}
\mid a_n - a \mid  < \e \\
\mid b_n - b \mid  < \e
\end{array} \right\rbrace$ für $n \geq N$\\
$\Rightarrow a_n > a - \e$\\ \\
$= a - \frac{a - b}{2} = \ds\frac{a + b}{2} = b + \ds\frac{a - b}{2}\\
\\
= b + \e > b_n \Rightarrow a_n > b_n$ für $n \geq \N{}$\\
Widerspruch zur Annahme.\\
$a_n \leq b_n$ für unendlich viele $n \in \N$\qed

\sS{Definition: Reihen}
Sei $(a_n)_{n \geq 0}$ eine Folge reeler Zahlen.\\
Bilde eine Folge:
\begin{align*}
s_0 &= a_0\\
s_1 &= a_0 + a_1\\
s_2 &= a_0 + a_1 + a_2\\
&\vdots\\
s_n &= a_0 + a_1 + a_n = \sum\limits_{k = 0}^{n} a_k
\end{align*}
Die Folge $(s_n)_{n \geq 0}$ heißt Reihe mit den Gliedern $a_n$.\\
$s_n$ heißen die \underline{Partialsummen} der Reihe.\\
Bezeichnung:\\
$\sum\limits_{k = 0}^{∞} a_k$ oder $a_0 + a_1 + a_2 + a_3 + …$\\ \\
Wenn $s_n \to s \in \R{}$ für $n \to ∞$ dann schreiben wir:\\
$\sum\limits_{k = 0}^{∞} a_k = s$\\
Summe der Reihe.\\[4pt]
\ul{Achtung:} Symbol $\ds\sum_{k = 0}^{∞} a_k$ hat \ul{zwei} Bedeutungen:
\begin{enumerate}
\item{die Folge $(s_n)$}\\[8pt]
oder 
\item{deren Grenzwert}
\end{enumerate}
\bsp
\begin{enumerate}
\item{$\sum\limits_{k = 1}^{∞} 1 = 1+1+1+…$\\
ist die Folge $(1, 2, 3, 4,…) = (n + 1)_{n \in \N{}_{0}}$}
\item{$\sum\limits_{k = 1}^{∞} k = 0 + 1 + 2 + 3+ …$ \\
ist die Folge $(1, 3, 6, 10,…) = (\ds\frac{n(n - 1)}{2})_{n \in \N{}}$ }
\item{$\sum\limits_{k = 1}^{∞} \ds\frac{1}{k(k+1)} = \ds\frac{1}{2} + \ds\frac{1}{6} + \ds\frac{1}{12} + …$\\
ist die Folge $(\ds\frac{1}{2}, \ds\frac{2}{3}, \ds\frac{3}{4})$}
\end{enumerate}
\vspace{5mm}
Vorüberlegung:\\ %bullshit
$\ds\frac{1}{k(k+1)} = \ds\frac{(k+1) - k}{k(k+1)} = \frac{1}{k} - \ds\frac{1}{k + 1}$\\ \\
$s_n := \sum\limits_{k = 1}^{∞} \ds\frac{1}{k(k+1)}
= (\ds\frac{1}{1} - \ds\frac{1}{2}) + (\ds\frac{1}{2} - \ds\frac{1}{3}) + … + (\ds\frac{1}{n} - \ds\frac{1}{n + 1})\\ \\
= 1 - \ds\frac{1}{n + 1}\\ $ Teleskopsumme \\ \\
$\ds\frac{1}{n + 1} \to 0$ für $n \to ∞$\\ \\
Summe der Reihe:\\ \\
$\sum\limits_{k = 1}^{∞} \ds\frac{1}{k(k+1)} = \lim_{n \to ∞}(1 - \ds\frac{1}{n + 1}) = 1$\qed\\ 
\\
\bem Jede Folge kann man auch als Reihe Schreiben. (Differenzen bilden)\\
z.B.: die Folge der Primzahlen:\\
$(2, 3, 5, 7, 11, 13, 17, 19)$\\
ist die Reihe:\\
$(2 + 1 + 2+ 4+2+4+2+…)$\\
Goldbachsche Vermutung: in dieser Reihe kommt die Zahl 2 unendlich oft vor.\\
\sS{Satz, Die geometrische Reihe}
Sei $x \in \R{}$\\
a) $ \sum\limits_{k = 0}^{∞} x^k = 1 + x^1 + x^2 + x^3 + … = \frac{1}{1-x} \text{ wenn } \mid x \mid < 1$\\
b) $ \sum\limits_{k = 0}^{∞} x^k \text{ divergiert wenn } \mid x \mid \geq 1$\\
\begin{itemize}
\item[a] {wenn $|x| < 1$\\
dann folgt $\sum{k=0}{∞} a_k = \ds\lim_{n \to ∞}(\frac{1}{1 - x} - \frac{x}{1-x} · x^n) = \frac{1}{1 - x}$}
\item[b]{wenn $|x| > 1$\\
dann $(x^n)$ divergent $\Rightarrow (\frac{x}{1-x} · x^n)$ divergent\\
denn $\frac{x}{1-x} \neq 0$\\
$\Rightarrow (\frac{?}{?})$}
\end{itemize}
\bew
$x = 1 \phantom{xxx} \sum_{k = 0}^{∞} x^k = (1 + 1 + 1 +…)\text{ divergiert, ok}\\
\text{Sei nun }x \neq\\
\text{Bekannt aus der Übung: } \ds\sum_{k = 0}^{∞} x^k = 1 + x + x^2 +x^3 … +x^n = \ds\frac{1 -x^{n+1}}{1 - x} = \ds\frac{1}{1 - x} - \ds\frac{x}{1 - x} · x^n \\ $
Potenzenwachstum\\
$x^n \to 0$ für $ n \to ∞$ \underline{wenn} $|x| < 1$\\
$(x^n)$ divergiert, wenn $(|x| \geq 1 \text{ und } x \neq 1)$\\
%
\sS{Satz}
Wenn die Reihe $\ds\sum\limits_{k=0}^{∞} a_k $ kovergiert, dann ist $(a_n)_{n \in \N{}}$ eine Nullfolge.\\
\\
\bew Gegeben sei $\e > 0$\\
Sei $a = \ds\sum\limits_{k = 0}^{∞} a_k = $ $\ds\lim_{n \to ∞}(s_n)$ mit $s_n = a_0 + … + a_n$\\
Es gibt $ N \ in \N{}$ mit $|s_n - a| < \ds\frac{\e}{2}$ für $n \geq N$\\
$|a_n| = |s_n - s_{n-1}|$\\
\phantom{$|a_n| $} = $|s_n - a + a - s_{n-1}|$\\
\phantom{$|a_n| $} $\leq |s_n - a| + |a - s_{n-1}| < \ds\frac{\e}{2} + \frac{\e}{2} = \e$\\
für $n \geq N + 1$\\
$\Rightarrow a_n \to 0$ für $n \to ∞$\\
%
\sS{Satz, die harmonische Reihe}
$$\ds\sum\limits_{k = 1}^{∞} \frac{1}{k}= 1 + \frac{1}{2} + \frac{1}{3} + …\text{ divergiert}$$
\sss{Beweisidee:}
$\ds\phantom{= }1 + \frac{1}{2} + \frac{1}{3} +\frac{1}{4} +\frac{1}{5} +\frac{1}{6} +\frac{1}{7} + \frac{1}{8} +\frac{1}{9} +…$\\
$\phantom{\geq }1 + \ds\frac{1}{2} +\frac{1}{4}+\frac{1}{4}+\frac{1}{8}+\frac{1}{8} + \frac{1}{8} + \ds\frac{1}{8} + \ds\frac{1}{16} + …$\\
$\phantom{= }1 + \ds\frac{1}{2} + \ds\frac{2}{4} + \ds\frac{4}{8} + \ds\frac{8}{16} + …$\\
$\phantom{= }1 + \ds\frac{1}{2} + \ds\frac{1}{2} + \ds\frac{1}{2} + \ds\frac{1}{2} + … = ∞$\newpage
%% Kopfzeile beim Kapitelanfang:
\fancypagestyle{plain}{
%Kopfzeile links bzw. innen
\fancyhead[L]{\calligra\Large Vorlesung Nr. 7}
%Kopfzeile rechts bzw. außen
\fancyhead[R]{\calligra\Large 29.10.2012}
}
%Kopfzeile links bzw. innen
\fancyhead[L]{\calligra\Large Vorlesung Nr. 7}
%Kopfzeile rechts bzw. außen
\fancyhead[R]{\calligra\Large 29.10.2012}
%
% set chapters end sections
%\setcounter{chapter}3
%
Sei $(n_n)$ eine Folge reeler Zahlen.\\
Die Reihe mit den Gliedern $a_n$ ist die Folge $s_n = a_0 + a_1 + ... + a_n)_\nN$ \\
Bezeichnung: $\ds\sum_{k=1}^{\infty} a_k$\\
Wenn $S_n \to a$ für $n \to \infty$\\
Schreibe: $\ds\sum_{k = 0}^{\infty} a_k = a$
\Bsp{Geometrische Reihe}
$\ds|x| = 1 \Rarr \sum\limits_{k = 0}^{\infty} x^k = \frac{1}{1-x}$ für $x = 0$ setzte $0^0 = 1$\\[4pt]
Harmonische Reihe\\
$\displaystyle\sum\limits_{k = 1}^{\infty} \frac{1}{k}$ Konvergiert nicht.\\
\sS{Satz Rechenregeln für Reihen}
Seien $\sum\limits_{k = 0}^{\infty} a_k = a$ und $\sum\limits_{k = 0}^{\infty} b_k = b$ zwei konvergente Reihen. Dann:
\begin{enumerate}
\item{$\sum\limits_{k = 0}^{\infty} (a_k + b_k) = a + b$}
\item{Für $c \in \R{}$ ist $\sum\limits_{k = 0}^{\infty} c \cdot a_k = c \cdot a$}
\end{enumerate}
\bew
folgt aus 3.9.
\bem
Produkte von Reihen sind komplizierter.\\
\underline{Korrektur:}
Primzahlen-Vermutung: es gibt ∞ viele Primzahlen $p$ so dass $p + 2$ auch Prim ist.\\
Goldbach-Vermutung: Jede gerade natürliche Zahl ist die Summe von zwei Primzahlen.\\
\chapter{Konvergenzsätze}
Erinnerung: \R{} ist Dedekind-vollständig. Das heißt, jede nicht-leere nach oben beschränkte Teilmenge $M \subset R$ hat eine kleinste obere Schranke $sup(M)$\\
\phantom{XXX}\Rarr{} Existenz von Grenzwerten\\
\sS{Definition Monotone Folgen}
Eine Folge $(a_n)_{n \geq 0}$ heißt monoton wachsend, wenn $a{n + 1} \geq a_n$ für alle $n \in \N_0$\\
\phantom{Eine Folge $(a_n)_{n \geq 0}$ heißt}monoton fallend, wenn $a{n + 1} \leq a_n$ für alle $n \in \N{}_0$\\
\phantom{Eine Folge $(a_n)_{n \geq 0}$ heißt}streng monoton wachsend, wenn $a{n + 1} > a_n$ für alle $n \in \N{}_0$\\
\phantom{Eine Folge $(a_n)_{n \geq 0}$ heißt}streng monoton fallend, wenn $a{n + 1} < a_n$ für alle $n \in \N_0$
\bsp
$a_n = n$ ist streng monoton wachsend\\
$a_n = \frac{1}{n}$ ist streng monoton fallend\\
\sS{Satz}
\begin{enumerate}
\item{Jede nach oben beschränkte monoton wachsende Folge $(a_n)_{\nN}$ ist konvergent\\
% Bild?
Hier Fehlt was, das Bild, der Tafel, auf dem das stehen sollte ist nicht auffindbar, hast du da noch eine Mitschrift?
}
\item{Jede nach unten beschränkte monoton fallende Folge $(a_n)_\nN$ ist konvergent\\
% Bild?
}
\end{enumerate}
\bew
Sei $(a_n)$ nach oben beschränkt, monoton wachsend\\
Setze $a:= sup(\{a_n | n \in \N{}\})$\\
dann \begin{enumerate}
\item{$a_n \leq a$ für alle $n$}
\item{Für jedes $\epsilon > 0$ ist $a - \epsilon$ \underline{keine} obere Schranke, d.h. es gibt $N \in N$ so dass $a_N > a - \epsilon$
\\Für $n \geq N$ gilt\\
$a - \epsilon < a_N \leq a_n \leq a$\\
weil $(a_n)$ monoton wachsend\\
$\Rightarrow a - \epsilon < a_N \leq a_n \leq a \Rightarrow |a_n -a| < \epsilon$\\
Somit $a_n \to a$ für $n \to \infty$\phantom{XXX}$q.e.d.$\\
Monoton fallend: analog}
\end{enumerate}
\uS{Reihen mit nicht-negativen Gliedern}
\bem
Sei $\ds\sum\limits_{k=0}^{\infty} a_k$ Reihe reeller Zahlen\\
Die Folge der Partialsummen ist monoton wachsend $\Leftrightarrow a_n \geq 0$ für $n \geq 1$
\Satz
Eine Reihe $\ds\sum\limits_{k=0}^{\infty} a_k$ mit $a_k \geq 0$ für ale $k$ konvergiert, genau dann, wenn sie beschränkt ist (Das heißt die Folge der Partialsummen ist beschränkt)\qed
\Def
Sei $\displaystyle\sum\limits_{k=0}^{\infty} a_k$ eine Reihe mit $a_n \geq 0$ für alle $k$\\
Eine Reihe $\ds\sum\limits_{k=0}^{\infty} b_k$ heißt \underline{Majorante} von $\ds\sum a_k$ wenn $a_k \leq b_k$ für alle $k$
\sS{Satz Majorantenkriterium}
Wenn eine Reihe mit nicht-negativen Gliedern eine konvergente Majorante hat, dann konvergiert sie.
\bew
Sei $0 \leq a_k \leq b_k$ für alle $k \geq 0$\\
Es gilt $a_0 + ... + a_n \leq b_0 + ... + b_n$ \\
$\sum b$ konvergiert $\Rightarrow (b_0 + ... + b_n)_{n \geq 0}$  beschränkt\\
$\Rightarrow ((a_0 + ... + a_n)_{n \geq 0})$ beschränkt $\Rightarrow \ds\sum_{k= 0}^{\infty} a_k$ konvergiert.
\Bsp{4.6:}
$$\sum\limits_{k=1}^{\infty} \frac{1}{k^2} = \left( 1 + \frac{1}{4} + \frac{1}{9}+ \frac{1}{16} + … \right)$$
$$\sum\limits_{k=1}^{\infty} \frac{1}{k^2} = 1 + \sum\limits_{k=1}^{\infty} \frac{1}{(k + 1)^2}$$
$$\frac{1}{(k + 1)^2} \leq \frac{1}{k \cdot (k + 1)}$$
$$\Rarr \sum\limits_{k=1}^{\infty} \frac{1}{k \cdot (k + 1)}\text{ ist Majorante von }\sum\limits_{k=1}^{\infty} \frac{1}{k^2}$$
$$\sum\limits_{k=1}^{\infty} \frac{1}{k \cdot (k + 1)}\text{ konvergiert (bekannt)}$$
\sS{Satz Quotientenkriterium}
Sei $C \in \R{}, (a_n)$ eine Folge reeller Zahlen mit $a_n \geq 0$ für alle $n$ \ul{und} $a_{n + 1} \leq C \cdot a_n$ für fast alle nder$n$\\
$0 \leq C \leq 1$\\
Dann konvergiert die Reihe $\ds\sum\limits_{k=0}^{\infty} a_k$
\bew
Konvergenz ändert sich nicht, wenn endlich viele $a_n$ geändert werden.\\
Also kann man annehmen, dass $a_{n + 1} \leq C \cdot a_n$ für alle $n$ gilt.\\
Dann gilt $a_1 < C \cdot a_0$\\
$a_2 < C \cdot a_1 \leq C \cdot C \cdot a_0 = C^2 \cdot a_0$\\
$a_3 < C \cdot a_2 \leq C \cdot C \cdot a_1 = C^3 \cdot a_0$\\
etc. $\Rightarrow a_n \leq C^n \cdot a_0$\\
Somit ist $\displaystyle\sum\limits_{k=0}^{\infty} C^k \cdot a_0$ konvergente Majorante von $\ds\sum_{k=0}^{\infty} a_k$ (Geometrische Reihe)
\sS{Beispiel Die Exponentialreihe}
$$exp(x) := \sum\limits_{k=0}^{\infty} \frac{x^k}{k!} \text{ für } x \in \R{}, x \geq 0$$
Setze $a_k = \frac{x^k}{k!}$\\
$$a_n+1 = \frac{x^{n + 1}}{(n + 1)!} = \frac{x}{n+1} \cdot \frac{x^n}{n!} = \frac{x}{n+1} \cdot a_n \leq \frac{1}{2} a_n$$
\Rarr{} Quotientenregel ist erfüllt.\\
Reihe $exp(x)$ konvergiert.\\
Bezeichnung: $$exp(x) = \sum_{k=0}^{\infty} \frac{x^k}{k!} \eR$$
%
\uS{Bezeichnung:}
$$exp(1) = \sum\limits_{k=0}^{\infty} \frac{1}{k!} = e\text{ (Eulerische Zahl)}$$
\sS{Leibnitz-Kriterium}
Sei $(a_n)_{n \in \N{}_0}$ eine Monoton monoton fallende Nullfolge\footnote{$a_n → 0$ für $n→∞$} mit $a_n \geq 0$ für alle $n$\\
Dann konvergiert die alternierende Reihe\\
$$\sum_{k=0}^{\infty} (-1)^k · a_k$$
\bsp
$$a_k = \frac{1}{k + 1} \sum_{k=0}^{\infty} (-1)^k · a_k = 1 - \frac{1}{2} + \frac{1}{3} - \frac{1}{4} + \frac{1}{5}= log(2)$$
\bew
Sei $s_n = a_0 + ... + a_n$
\uS{Behauptung: }
$S_{2n + 1} \leq S_{2n + 3} \leq S_{2n + 2} \leq S_{2n}$ für jedes $n \in \N{}$
%
\ul{Rechne:}\\
$S_{2n + 2} - S_{2n} = - a_{2n + 1} + a_{2n + 2} \leq 0 \Rightarrow (3)$\\
$S_{2n + 3} - S_{2n + 1} = - a_{2n + 3} \leq 0 \Rightarrow (2)$\\
$S_{2n + 3} - S_{2n + 1} = - a_{2n + 2} - a_{2n + 3} \leq 0 \Rightarrow (1)$\\[8pt]
Die Folge $b_n = S_{2n}$\\
\phantom{Die Folge }$c_n = S_{2n + 1}$\\
sind beschränkt und monoton (fallend bzw. steigend)\\
$\Rightarrow b_n \text{ und } c_n$ konvergieren\\[4pt]
Sei $$b = \lim_{n \to \infty} b_n \qquad c = \lim_{n \to \infty} c_n$$
$$c - b = \lim_{n \to \infty} (c_n - b_n) = \lim_{n \to \infty} (a_{2n + 1}) = 0$$
weil $(a_n)$ Nullfolge
\sS{Behauptung:} % uS stande dort ich vermute so
$S_n \to b$ für $n \to \infty$\\[4pt]
Gegeben sei $\e > 0$. Es gibt $N \in \N{}$ so dass für $n \leq N$:\\
$|b_n - b| < \e, |c_n - c| < \epsilon$\\
Somit für $n \geq 2N+1 $\\
$|S_n - b| < \e$ also $S_n \to b$\qed\newpage
%%Kopfzeile links bzw. innen
\fancyhead[L]{\calligra {\Large Vorlesung Nr. 8}}
%Kopfzeile rechts bzw. außen
\fancyhead[R]{\calligra \Large{05.11.2012}}
% **************************************************
%
\Wdh{Konvergenzsätze}
\begin{itemize}
    \item{Eine monoton wachsende und beschränkte Folge konvergiert zwangsläufig.}
    \item{Eine Reihe $\sum_{k=0}^{∞} a_k$ mit $a_k \geq 0$ für alle $k$ konvergiert \equ die Folge der Partialsummen $(S_n = \sum_{k=0}^{n} a_k)_{n \in \N}$ ist beschränkt}
\end{itemize}
%
\bsp
    $\sum_{k=0}^{n} \frac{1}{k} = 1 + \frac{1}{2} + \frac{1}{3} …$ ist unbeschränkt\\
\bsp
    $\sum_{k=1}^{\infty} \frac{1}{k^2} = 1 + \frac{1}{4} + \frac{1}{9}+…$\\
\sss{Leibnitz:} 
    Sei $(a_n)$ monoton fallende Nullfolge.\\
    Dann konvergiert $\sum_{k=0}^{∞} (-1)^k \cdot a_k$\\
\bsp
    $(1 - \frac{1}{2} + \frac{1}{3} - \frac{1}{4})$ … konvergiert.
% satz 4.10 ***************
%\setcounter{chapter}{4}
%\setcounter{section}{9}
% *************************
\sS{Satz Verdichtungslemma von Candy}
Sei $(a_n)$ monoton fallende Nullfolge.\\
Die Reihe $\sum_{k=0}^{\infty} a_k$ konvergiert genau dann, wenn die verdichtete Reihe $\sum_{k=0}^{\infty} 2^k \cdot a_{2^k} = 1 · a_1 + 2 · a_2 + 4 \cdot a_4$ … konvergiert.\\
%
\bsp
    $a_k = \frac{1}{k}\qquad (k \geq 1)$\\
    $2^k \cdot a_{2^k} = 2^k \cdot \frac{1}{2^k} = 1$\\
%
\sS{Satz:} % sss orginal, ich vermute es ist so
$\sum_{k=0}^\infty \frac{1}{k}$ konvergent \equ $\sum_{k=0}^\infty 1$ konvergent (ist nicht der Fall.)\\
%underline{Beweis:}
\bew
Sei $b_n = \ds\sum_{k=2^n}^{2^{n+1}-1} a_k$\\
Für $\ds 2^n \leq k \leq 2^{n+1} - 1$ ist $\ds a_{2^n} \geq a_{k} \geq a_{2^{n+1} - 1} \geq a_{2^{n+1}}$ \Rarr $\ds 2^n · a_{2^n} \geq b_n \geq  2^n · a_{2^{n+1}}$\\
Wenn $\ds \sum_{k \geq 0} 2^k · a_2^k$ beschränkt \Rarr $\ds \sum_{k \geq 0} b_k$ beschränkt \Rarr $\ds \sum_{k \geq 0} a_k$ beschränkt\\
Hier immer beschränkt \equ konvergent\\
Wenn $\ds \sum_{k \geq 0} 2^k · a_k$ beschränkt \Rarr $\ds \sum_{k \geq 0} b_k$ beschränkt \Rarr $\ds \sum_{k \geq 0} 2^{k} · a_2^{k+1}$ beschränkt \equ $\ds \sum_{k \geq 0} 2^{k+1} · a_2^{k+1}$ beschränkt \equ $\ds \sum_{k \geq 0} 2^{k} · a_2^{k}$ beschränkt\\
Das zeigt den Satz.\\
\ssss{Anwendung:}
\ssss{Erinnerung:}
    Für $ x\geq 0$ und $a \in \R$ wird später $x^a \in R$ definiert\\
    Wenn $\ds a = \frac{n}{m}$ mit $m \geq 1$ d.h. $a \in \Q$ dann $\ds x^a = \sqrt[m]{x^n}$.\\
    Wenn $x > 1$ dann gilt: \\
    $x^a = \begin{cases} >1 \mbox{ wenn }a>0\\ =1 \mbox{ wenn }a=0\\ <1\mbox{ wenn }a<0 \end{cases}$
%
\sS{Satz}
Sei $a \in \R$. Die Reihe $\ds \sum_{k=1}^{\infty} \frac{1}{k^a}$ konvergiert genau dann, wenn $a > 1$\\
\bew
Wenn $a \leq 0$ dann $\frac{1}{k^a} \geq 1$ \Rarr Reihe divergiert.\\
Sei $a >0$, sei $a_n = \frac{1}{n^a}$ \\
$\ds n < n+1 \Rarr n^a < (n+1)^a \Rarr a_n > a_{n+1}$ Somit (a_n) monoton fallend.\\
$\ds \lim_{n\to \infty} n^a = \infty \Rarr \lim_{n \to \infty} \frac{1}{n^a} = 0$ \Rarr Verdichtungslemma ist anwendbar.\\
Bilde $\ds 2^n · a_{2^n} = 2^n · \frac{1}{(2^n)^a} = 2^n · 2^{-n · a} = 2^{n(1-a)} = (2^{1-a})^n = x^n$\\
mit $x:=2^{1-a}$\\
Erhalte:
$\ds \sum_{k=1}^\infty \frac{1}{k^a}$ konvergiert $\equ \ds \sum_{k=0}^\infty x^k$ konvergiert $\equ |x|<1 \equ x<1 \equ 2^{1-a} <1$\\
$\equ 1-a<0 \equ a>1$\qed\\[4pt]
Beziehung:
$\ds \sum_{k=1}^\infty\frac{1}{k^a}=\zeta (a)$ für $a>1$\\
Riemannsche Zetafunktion
Spezielle Werte:
$\zeta (2) = \sum_{k\geq1}\frac{1}{k^2}=\frac{\pi^2}{6}$\\
$\zeta (4) = \sum_{k\geq1}\frac{1}{k^2}=\frac{\pi^4}{90}$\\
$\zeta (6) = \sum_{k\geq1}\frac{1}{k^2}=\frac{\pi^6}{945}$\\
Frage: Für welche z ist $\zeta(z)=0$?
%
%\sss{Teilfolgen}
\uS{Teilfolgen}
\sS{Definition}
Sei $(a_n)$ eine Folge reeller Zahlen.\\
Eine Teilfolge von $(a_n)$ ist eine Folge der Form $(a_{n_k})_{k \geq 0}$ wobei $n_0, n_1, n_2,...$ streng monoton wachsende Folge in $\N_0$ ist.\\
\bsp
$(a_n) = (1, x, x^2, x^3 , x^4 ...)$\\
$(n_k) = (1, 4, 9, 16) \leadsto $ Teilfolge $(x, x^4, x^9, x^{16} ,…)$\\
\sS{Bemerkung}
Wenn $a_n \to a$ für alle $n \to \infty$ dann konvergiert jede Teilfolge von $(a_n)$ gegen $a$ (Präsenzübung Nr. 9)\\
%
%\sss{Schlüsselsatz:}
\uS{Schlüsselsatz}
\sS{Lemma}
Jede Folge reeller Zahlen $(a_n)_{n\geq 0}$ hat eine monotone Teilfolge.\\
%
\bew
Wir nennen $n\in \N_0$ \underline{extrem} wenn $a_n \geq a_m$ für alle $m \geq m$\\
Unterscheide zwei Fälle:\\
\begin{itemize}
    \item{Es gibt unendlich viele extreme $n \in \N$\\
Dies seien $n_0, < n_1, n_2...$\\
Dann $a_{n_0} \geq a_{n_1} \geq a_{n_2} …$\\
Weil n_0 extrem ... weil n_1 extrem.\\
→ monoton fallende Teilfolge gefunden}
    \item{Es gibt nur endlich viele extreme $n$\\
Wähle $n_0 \in \N$ s.d. gilt: $m \geq n \Rarr m$ nicht extrem.\\
$n_0$ nicht extrem $\Rarr$ es gibt $n_1 \geq n_0$ mit $a_{n_1} > a_{n_0}$ insbesondere $n_1 > n_0$\\
$n_1$ \phantom{nicht extrem }$\Rarr$ \phantom{es gibt }$n_2 \geq n_1$ mit $a_{n_2} > a_{n_1}$ insbesondere $n_2 > n_1$\\
$n_2$ \phantom{nicht extrem }$\Rarr$ \phantom{es gibt }$n_3 \geq n_2$ mit $a_{n_3} > a_{n_2}$ insbesondere $n_3 > n_2$\\
usw.\\
Erhalte $n_0 < n_1 < n_3 < …$ mit $a_{n_0} < a_{n_1} < a_{n_2} < …$ \\
$\to $ streng monotom wachsende Teilfolge gefunden.\qed
}
\end{itemize}
\sS{Satz Bolzano-Weierstraß}
Jede beschränkte Folge reeller Zahlen hat eine konvergernte Teilfolge.\\
\bew
Es gibt ein monotone Teilfolge (Lemma 4.14)\\
Diese ist beschränkt \Rarr konvergent.\qed
\sS{Definition}
Eine Folge reeller Zahlen $(a_n)_{n \geq 0}$ heißt Cauchyfolge wenn gilt:\\
Für jedes $\e > 0$ gibt es ein $N \in \N$ sodass für $m, n \geq N$ gilt: $|a_n - a_m| < \e$
\sS{Satz}
Eine Folge reeller Zahlen $(a_n)$ konvergiert genau dann, wenn sie eine Cauchyfolge ist.\\
\bew
$\Rarr$ Sei $a_n \to a$ für $n \to \infty$\\
Gegeben sei $\e > 0$. Es gilt $N \in \N$ so dass $|a_n - a| < \frac{\e}{2}$ für $n \geq N$\\
Für $n, m \geq N$ gilt:\\
$|a_n - a_m| = |a_n - a + a - a_m| \leq |a_n - a| + |a - a_m| < \frac{\e}{2} + \frac{\e}{2} = \e$\\
$\Rarr (a_n)$ ist eine Cauchyfolge\\
$Larr :$ Sei (a_n) eine Cauchyfolge\\
\sss{Behauptung:} $(a_n)$ ist beschränkt
\bew
Wähle $\e=1$ Es gibt $N\in\N$ mit $|a_n-a_m|<1$ für $m,n\geq N$\\
Sei $C=max\{|a_0|,|a_1|,|a_2| … |a_N|,|a_N|+1\}$\\
Dann $|a_n| \leq C$ für alle \N\\
$(n\geq N \Rarr |a_n-a_N| < 1 \Rarr |a_n|<|a_N|+1)$\\
Also ist $(a_n)$ beschränkt\\
%
\sS{Lemma}
\Rarr $(a_n)$ hat eine monotone Teilfolge $(a_{n_k})_{k\geq0}$ diese ist beschränkt \Rarr konvergent.\\
Sei $\ds \lim_{k→∞}(a_{n_k})$
\sss{Behauptung}
$a_n→a$ für $n→∞$\\
Sei $\e>0$ gegeben. Es gibt $n\in\N$ so dass
\begin{enumerate}
\item{$n,m\geq N \Rarr |a_n-a_m|< \frac{\e}{2}$}
\item{$k\geq N \Rarr |a_{n_k}-a|< \frac{\e}{2}$}
\end{enumerate}
%
Sei $k\geq N$\\
%
\bem
Für jedes $k\in\N$ ist $n_k\geq k$\\
$\ds |a_k-a|=|a_{n_k}+a_{n_k}-a| \leq |a_k-a_{n_k}|+|a_{n_k}-a|<\frac{\e}{2}+\frac{\e}{2}=\e$\\
Also $a_k→a$ für $n→∞\qed
%
\sss{Umformulierung für Reihen:}
\sS{Satz (Cauchy-Kriterium für Reihen)}
Eine reelle Reihe $\ds \sum_{k=0}^\infty a_k$ konvergiert genau dann, wenn gilt:\\
Für jedes $\e>0$ gibt es ein $N\in\N$ so dass für alle $\ds n,m\geq N, n\leq m |\sum_{k=n}^m|<\e$
%
\sss{Beweis: Partialsummen}
$\ds s_n=\sum_{k=0}^n a_k$\\
$\ds \sum_{k=n}^m=s_m-s_{n-1}$\\
Damit ist 4.19 äquivalent zu 4.18\newpage
%%Kopfzeile links bzw. innen
\fancyhead[L]{\calligra\Large Vorlesung Nr. 9}
%Kopfzeile rechts bzw. außen
\fancyhead[R]{\calligra\Large 08.11.2012}
% **************************************************
%
\wdh
Eine folge reeller Zahlen $(a_n)$ ist eine Cauchy-Folge wenn gilt:\\
Für jedes $\e>0$ gibt es ein $n\in\N$ so dass für $m,n\geq\N$ gilt $|a_n-a_m|<\e$\\
$(a_n)$ konvergiert \equ $(a_n)$ ist Cauchy-Folge\\
Für Reihen: $\ds \sum_{k=0}^{∞}a_k$ konvergiert \equ Für jedes $\e>0$ gibt es ein $N\in\N$ so dass für $m,n\geq\N$ mit $m\geq n$ ist $\ds \left|\sum_{k=n}^m a_n\right|<\e$

\uS{Absolute Konvergenz}

\sS{Definition}
Eine Reihe $\ds\sum_{k=0}^{∞} a_k$ mit $a_k\in\R$ heißt absolut konvergent wenn die Reihe $\ds\sum_{k=0}^{∞} |a_k|$ konvergiert

\sS{Satz}
Jede absolut konvergente Reihe konvergiert

\bew
Verwende Cauchy-Kriterium für Reihen\\
Sei $\ds\sum_{k=0}^{∞} a_k $ absolut von konvergent.\\
\Rarr Für jedes $\e>0$ gibt es $N\in\N$ mit:\\
Für $n\geq m\geq N$ gilt $\ds\sum_{k=m}^n |a_k| < \e \Rarr \left|\sum_{k=m}^n a_k\right| \underset{\overset{\uparrow}{Dreiecksungleichung}}{\leq} \sum_{k=m}^n |a_k| < \e \Rarr \sum_{k=m}^n a_k konvergiert$\qed

\bem
Umkehrung gilt nicht.
$\ds\sum_{k=1}^{∞} (-1)^k \frac{1}{k} = -1+\frac{1}{2}+\frac{1}{3}+\frac{1}{4}+...$\\
konvergiert (Leibnitz)\\
denn $\ds\sum_{k=1}^{∞} \left|(-1)^k \frac{1}{k}\right| = \sum_{k=1}^{∞} \frac{1}{k}$ divergiert

\sS{Definition}
Eine Reihe $\ds\sum_{k=0}^{∞} b_k$ heißt Majorante der Reihe $\ds\sum_{k=0}^{∞} a_k$, wenn $|a_k|\leq b_k$ für alle k\\
(schon gewesen wenn $a_k\geq 0$)

\sS{Satz (Majorantenkriterium)}
Wenn eine Reihe eine konvergente Majorante hat, dann konvergiert sie absolut.
\underline{Beweis} von Satz 4.5\qed

\uS{Umordnung von Reihen}
\sS{Definition}
Eine Umordnung einer Reihe $\ds\sum_{k=0}^{∞} a_k$ ist eine Reihe der Form $\ds\sum_{k=0}^{∞} a_{n_k}$ wobei $(n_0,n_1,n_2…)$ eine Folge natürlicher Zahlen ist, in der jedes $n\in\N_0$ genau einmal vorkommt.\\

\sS{Satz}
Jede Umordnung einer \underline{absolut} konvergenten Reihe ist wieder absolut konvergent und hat den gleichen Grenzwert.\\
Im Gegensatz dazu gilt:\\
\sS{Satz}
Sei $\ds\sum_{k=0}^{∞} a_k$ eine konvergente, nicht absolut konvergente, Reihe. Für jedes $c\in\R\cup\{-∞,∞\}$ hat $\sum a_k$ eine Umordnung, die gegen c konvergiert.

\bsp
Eine Reihe $\dfrac{1}{2}-\dfrac{1}{2}+\dfrac{1}{3}-\dfrac{1}{3}+\dfrac{1}{4}-\dfrac{1}{4}+\dfrac{1}{5}-\dfrac{1}{5}+…$
konvergiert gegen 0. Konvergiert aber nicht absolut:\\
Folge: $(\dfrac{1}{2},0,\dfrac{1}{3},0,\dfrac{1}{4},0,…→0)\quad\ds\sum_{k=1}^{∞} 2·1/k=∞$\\
Produziere Umordnung, die gegen ∞ konvergiert:\\
\[\dfrac{1}{2}-\dfrac{1}{2}+\underbrace{\dfrac{1}{3}+\dfrac{1}{4}}_{\geq\dfrac{1}{4}+\dfrac{1}{4}=\dfrac{1}{2}}-\dfrac{1}{3}+\underbrace{\dfrac{1}{5}+\dfrac{1}{6}+\dfrac{1}{7}+\dfrac{1}{8}}_{\geq\dfrac{1}{2}}-\dfrac{1}{4}+\underbrace{\dfrac{1}{5}+…+\dfrac{1}{16}}_{\geq\dfrac{1}{2}}-\dfrac{1}{5}+…\]\\
\[\leq\underbrace{\dfrac{1}{2}-\dfrac{1}{2}}_{\text{\large{0}}}+\underbrace{\dfrac{1}{2}-\dfrac{1}{3}}_{\dfrac{1}{6}}+\underbrace{\dfrac{1}{2}-\dfrac{1}{4}}_{<\qquad\dfrac{1}{4}\qquad<}+\underbrace{\dfrac{1}{2}-\dfrac{1}{5}}_{\dfrac{3}{10}}+…=∞\]\\
Beweise von 4.24, 4.25 eventuell später.

\uS{Produkte von Reihen}
Frage: was ist $\ds\left(\sum_{k=0}^{∞} a_k\right)·\left(\sum_{k=0}^{∞} b_k\right) ?$

\sS{Definition}
Das Cauchy-Produkt von zwei reihen $\ds\sum_{k=0}^{∞} a_k$ und $\ds\sum_{k=0}^{∞} b_k$ ist eine Reihe $\ds\sum_{k=0}^{∞} c_k$ mit $\ds c_n :=\sum_{k=0}^{∞} a_k·b_{n-k}=a_0·b_n+a_1·b_{n-1}+a_2·b_{n-2}+…+a_n·b_0$\\
2-dimensionale Anordnung der $a_k·b_l$ 
% Ed's heft
\sS{Satz}
Seien $\ds\sum_{k=0}^{∞} a_k$ und $\ds\sum_{k=0}^{∞} b_k$ konvergente Reihen, mindestens eine von ihnen absolut konvergent. Dann konvergiert ihr Cauchy-Produkt $\ds\sum_{k=0}^{∞} c_k$. Wenn $\ds\sum_{k=0}^{∞} a_k = a, \ds\sum_{k=0}^{∞} b_k = b$ $\ds\sum_{k=0}^{∞} c_k = a·b$



\Bew{von 4.27}
Sei $\sum a_k$ absolut konvergent, $\sum b_k$ konvergent, so zeige $\sum c_k\ $ konvergent, $\ds c_n :=\sum_{k=0}^{∞} a_k·b_{n-k}$
Schreibe:
$s_n=a_0+…+a_n$\\
$t_n=b_0+…+b_n$\\
$u_n=c_0+…+c_n$\\
$s_n→a$,$t_n→b$ (*)\\[8pt]
Zeige $u_n→a·b$\\[4pt]
(*)\Rarr $s_n·b→a·b$ Zeige $s_n·b-u_n→0$\\[4pt]
$u_n=a_0·b_0+(a_0·b_1+a_1·b_0)+(a_0·b_2+a_1·b_1+a_2·b_0)+…+a_n·b_0=a_1·t_{n-1}+a_2·t_{n-2}+…+a_n·t_0$\\[4pt]
$s_n·b=a_0·b+a_1·b+a_2·b+a_3·b+…+a_n·b$\\[4pt]
$s_n·b-u=a_0·(b-t_n)+a_1·(b-t_{n-1})+a_2·(b-t_{n-2})+a_3·(b-t_{n-3})+…+a_n·(b-t_0)\underset{?}{→}0$\\[8pt]
Sei $C\in\R$ mit $|b|\leq C$ und $|b-t_n|\leq C$ für alle n\\
Sei $\ds\sum_{k=0}^{∞} |a_n| = a^*.$\\
Gegeben sei $\ds\e>0$. Wähle $N\in\N$ so dass $C·(|a_N|+|a_{N+1}|+|a_{N+2}|+…)<\frac{\e}{2}$\\
(geht weil $\sum|a_k|$ konvergiert)\\
und $|b-t_n|<\frac{\e}{2a^*}^{(2)}$ für alle $n\geq N$\\
(geht weil $b-t_n→0$ für alle $m→∞$)
\bem
Wenn $a^*=0$ dann $a_n=0$ für alle k. Dann alles klar.
Für alle $n\geq 2N$ gilt:\\
$|a_0(b-t_n)+a_1(b-t_{n-1})+…+a_n(b-t_0)|\leq |a_0|·|(b-t_n)|+|a_1|·|(b-t_{n-1})|+…+|a_n|·|(b-t_0)|$\\[4pt]
$\leq(|a_0|+|a_1|+|a_2|+…|a_N|)·\underset{\overset{\uparrow}{wegen\left(2\right)}}{\frac{\e}{2a^*}} +(|a_{N+1}|+|a_{N+2}|+|a_{N+3}|+…|a_n|) · C \leq a^* ·\frac{\e}{2a^*}+\frac{\e}{2}=\frac{\e}{2}+\frac{\e}{2}=\e$\\
Also gilt: $s_n-u→0$ für $n→∞$\qed\\
\underline{Zusatz:} Wenn $\sum a_k$ und $\sum b_k$ beide absolut konvertieren, dann auch das Cauchy-Produkt $\sum c_k$\\
\bew
Sei $\sum a_k^*$ das Cauchy-Produkt von  $\sum |a_k|$ und  $\sum |b_k|$. Beide konvergieren \Rarr $\sum_n c_n^*$ konvergiert\\[8pt]
d.h. $c_n^*=|a_0·b_{n}|+|a_1·b_{n-1}|+…+|a_n·b_{0}|\geq|a_0·b_{n}+a_1·b_{n-1}+…+a_n·b_{0}|=|c_n|$\\[8pt]
Also $\sum_n c_n^*$ ist konvergente Majorante von $\sum_n c_n$ \Rarr $\sum_n c_n$ konvergent absolut\qed

\bsp
Die Reihe $\ds\sum_{k=0}^{∞} a_k = 1-+\dfrac{1}{\sqrt{2}}+\dfrac{1}{\sqrt{3}}-\dfrac{1}{\sqrt{4}}+\dfrac{1}{\sqrt{5}}-…\quad $ konvergiert (Leibnitz)\\
Das Cauchy-Produkt der Reihe von $\sum a_k$ und $\sum a_k$ konvergiert nicht.\\

\sS{Beispiel}
Für jedes $x\in\R$ ist die Exponentialreihe $\ds exp(x)=\sum_{k=0}^{∞} \frac{x^k}{k!}$ absolut konvergent.\\
Es gilt \fbox{$exp(x)-exp(y)=exp(x+y)$} Funktionalgleichung der Exponentialfunktion.\\
\bew
Betrag von $\ds \sum_{k=0}^{∞} |\frac{x^k}{k!}|=\sum_{k=0}^{∞} \frac{|x|^k}{k!}=exp(|x|)$ konvergiert (bekannt, Quotientenkriterium)\\
Berechne Cauchy-Produkt $\ds exp(x)·exp(y)=\sum_{k=0}^{∞} c_k$\\
$\ds c_k = \frac{x^0}{0!}·\frac{x^n}{n!}+\frac{x^1}{1!}·\frac{y^{n-1}}{(n-1)!}+…+\frac{x^n}{n!}·\frac{y^0}{0!}=\dfrac{1}{n!}·\left(\dfrac{n!}{0!·n!}·x^0y^n+\dfrac{n!}{1!·(n-1)!}·x^1y^{n-1}+…+\dfrac{n!}{n!·0!}·x^ny^0+\right)=\dfrac{1}{n!}\sum_{k=0}^{n}\dfrac{n!}{k!·(n-k)!}x^ky^{n-k} =\dfrac{1}{n!}\sum_{k=0}^{n}\binom{n}{k}x^ky^{n-k}\underset{binomische Formel}{=}\dfrac{1}{n!}(x+y)^n\Rarr\sum_{k=0}^{∞}c_k=exp(x+y)$\qed\newpage
%% Kopfzeile beim Kapitelanfang:
\fancypagestyle{plain}{
%Kopfzeile links bzw. innen
\fancyhead[L]{\calligra\Large Vorlesung Nr. 10}
%Kopfzeile rechts bzw. außen
\fancyhead[R]{\calligra\Large 12.11.2012}
}
%Kopfzeile links bzw. innen
\fancyhead[L]{\calligra {\Large Vorlesung Nr. 10}}
%Kopfzeile rechts bzw. außen
\fancyhead[R]{\calligra \Large{12.11.2012}}
% **************************************************
%
%\setcounter{chapter}{4}
\chapter{Abbildungen und Funktionen}
\sS{Definition Abbildung}
Seien $A, B$ Mengen. Eine Abbildung von $A$ nach $B$ ist eine Vorschrift, die jedem Element von $A$ ein Element von $B$ zuordnet.
\notat{$f: A \to B,\  a \mapsto f(a) \  a\in A$}
$A$ heißt Definitionsbereich von $f$\\*
$B$ heißt Wertebereich von $f$
%
\bsp
\begin{enumerate}
\item {Alle Personen in $L1 \mapsto \N$\\*
$P \mapsto$ Geburtsjahr von $P$}
%
\item{$f:\R → \R, \ f(x)=x^2$\\*
$g:\R→\R_{\geq 0}=\{x\in\R\mid x\geq 0\}, \ g(x)=x^2$\\*
$h: \R_{\geq 0} \to \R_{\geq 0} \ h(x) = x^2$}
\bem 
\item{
$f,g,h$ sind verschieden\\*
Sei $M$ Menge. Die Identität von $M$ ist die Abbildung $id_{M}:M→M, \ id_M(x)=x$}
\end{enumerate}

\sS{Definition In-/Sur-/Bijektivität}
Eine Abbildung $f: A \to B$ heißt:
\begin{enumerate}
\item{\ul{injektiv} wenn gilt: Für alle $a, a' \in A$ mit $f(a) = f(a')$ ist auch $a = a'$}
\item{\ul{surjektiv} wenn gilt: Für jede $b\in B$ gibt es ein $a\in A$ mit $f(a)=b$}
\item{\ul{bijektiv} wenn $f$ injektiv und surjektiv ist}
\end{enumerate}
%
% Tafel 2.2
% Bild zeichnen
%
% Tafel 3.1
% Beispielbild
%
\bem
$f$ ist $\left\{
\begin{array}{c}
\text{injektiv}\\*
\text{surjektiv}\\*
\text{bijektiv}
\end{array}
 \right\}$ genau dann wenn für jedes $b \in B$ $\left\{\begin{array}{c} \text{höchstens}\\* \text{mindestens}\\*
 \text{genau}
 \end{array} \right\}$ ein $a \in A$ mit $f(a) = b$
%
\bsp
$f,g,h$ wie oben
\begin{description}
\item[f]{
\hspace{5.5mm}nicht surjektiv: es gibt kein $a\in\R$ mit $f(a)=-1$\\*
nicht injektiv: $f(-2)=4=f(2), 2\neq -2.$}
%
\item[g]{
\hspace{5mm}ist surjektiv, denn für jedes $b \in \R_{\geq 0}$ gilt $f( \sqrt{b} ) = b$ also gibt es $b \in \R_a$\\*
 ist nicht injektiv (wie $f$)}
%
\item[h]{
\hspace{5mm}surjektiv wie g. $\left(\sqrt{b} \geq 0\right)$\\*
injektiv, denn: Wenn $a, a' \geq 0$ und $a^2 = (a')^2$ dann $a = a'$ also $h$ bijektiv.}
\end{description}

\sS{Definition Komposition}
Seien $f:A→B$, $g:B→C$ Abbildungen\\*
Die Komposition von $f$ und $g$ ist die Abbildung\\*
$g \circ  f: A→C$, $(g \circ f)(a):=g(f(a))$\\*
Sprich $\circ$: "nach", "verkettet"

\sS{Satz} 
Eine Abbildung $f: A \to B$ ist bijektiv \equ \ es gibt eine Abbildung $g: B \to A$ mit $f \circ g = id_B$\\*
(d.h. $f(g(b)) = b$ für alle $b \in B$ $g(f(a)) = a$ für alle $a \in A$)
%
\sss{Definition} % ohne Nummerierung!
Wenn $f:A→B$ bijektiv ist, heißt die eindeutige Abbildung $g:B→A$ wie oben die Umkehrabbildung (inverse Abbildung) von $f$
Bezeichnung: $g=f^{-1}$.
%
\bew
Angenommen, $g: B \to A$ gegeben mit $f \circ g = id_B, g \circ f = id_A$ \footnote{Dies gilt, weil $g$ als Umkehrfunktion von $f$ definiert ist.}\\*
$f$ surjektiv: Sei $b \in B$. $b = f(g(b)) = f(a)$ mit $a = g(b)$ \ok\\*
$f$ injektiv: Sei $a, a'$ mit $f(a) = f(a')$ zeige $a = a'$ \\*
$a = g(f(a)) = g(f(a')) = a' $\ok \\*[8pt]
%
Angenommen, $f$ ist bijektiv, zeige $g$ existiert.\\*
Gegeben sei $b \in B$ $f$ bijektiv $\Rightarrow$ es gibt genau ein $a \in A $ mit $f(a) = b$ 
Setze $g(b):=a$ Das definiert Abbildung $g:B→A$\\*
Zeige $g \circ f=id; f \circ g = id$\\*
$(f\circ g)(b)=f(g(b))=f(a)=b$ wobei $a$ wie eben\\*[8pt]
%
Zeige: $(g \circ f) (a) $ für alle $a \in A$\\*
$f$ injektiv: Reicht $f(g(f)a))) = f(a)$\\*
Das gilt weil $f \circ g = id_B$ \ok\\*[8pt]
Eindeutigkeit von $g$:\\*
Angenommen, $g^* : B \to A$ erfüllt $g^* \circ f = id_A$,
$f \circ g^* = id_B$ \\*
%
Dann gilt: $g=g\circ id_B=g\circ f\circ g^*=id_A\circ g^* = g^*$ \qed
\bsp
Bewiesen 5.12
\begin{itemize}
\item{$f: \R_{\geq 0} \to \R_{\geq 0}, f(x) = x^k$ bijektiv ($k \geq 1$)\\*
Die Umkehrabbildung $f^{-1}$ heißt k-te Wurzelabbildung $f^{-1}(x) = \sqrt[k]{x}$ }
%
\item{exp: $\R→\R_{>0}$ $exp(x) = \sum_{k=0}^{\infty}$ (absolut konvergente Reihe) ist bijektiv. Die Umkehrabbildung heißt Logarithmus. bew.
$log = exp()^{-1} \R_{\geq } \to \R_a$ }
\end{itemize}

\uS{Bild und Urbild}
\sS{Definition}
Sei $f:A→B$ Abbildung
\begin{enumerate}
\item{Für eine Teilmenge $X \subset A$ ist \\*
$f(x) := \{f(x) | x \in X\} \subseteq B$ \\*
das Bild von $X$ unter $f$}
\item{Für eine Teilmenge $Y \subseteq B$ ist $f^{-1}:=\{a\in A|f(a)\in Y\}\subseteq A$ das Urbild von $Y$ unter $f$}
\end{enumerate}
\ul{Vorsicht} nicht Urbild und Umkehrabbildung verwechseln.
\bsp
$$f:\R→\R,\ f(x)=x^2$$
$$f(\{1, 2, -2\}) = \{1, 4\}$$
$$f^{-1}(\{1,-2,4\})=\{1,-1,2,-2\}=f^{-1}(\{1,4\})$$
$$f^{-1}(\{9\})=\{3,-3\}\qquad f^{-1}(\{-5\})=\emptyset$$

\uS{Funktionen}
\sS{Definition Funktion}
Sei $D\subseteq\R$ Teilmenge. Eine reelle Funktion auf $D$ ist eine Abbildung $f:D→\R$\\*
%
Der \ul{Graph} von $f$ ist die Menge $\Gamma_f = \{(x, f(x)\mid x \in D \}$\\*
$ \Gamma_f \subseteq D \times \R$ 
%
\bem Oft ist $D$ ein Intervall
%
%\sS{Definition Intervalle}
  \tikz[scale=0.5,domain=-3.5:3.5, samples=200,prefix=plots/,smooth]{
      \draw[very thin, color=gray!50] (-3.5,-3.5) grid (3.5,8.9);
      \draw[->] (-3.5,0) -- (3.5,0) node[right] {$x$};
      \draw[->] (0,-3) -- (0,8.5) node[above] {$y$};
      \clip (-3.5,-3.0) rectangle (5.5,8.9);
      \draw[color=red] plot[id=x^2] function{x*x} node[below=3.5cm] {\footnotesize $f_1(x) =x^2$};
      \draw[color=blue] plot[id=abs,sharp plot] function{abs(x)} node {\footnotesize $f_2(x) = |x|$};
      \draw[color=cyan] plot[id=exp] function{exp(x)} node [below=13.5cm] {\footnotesize $f_2(x) = exp(x)$};
      \draw[color=green!60!black,const plot] plot[id=gaussklammer] function{floor(x)} node[below] {\footnotesize $f_5(x) = [x]$};
  }
seien $a, b \in \R$ \\*
$[a, b] = \{x \in \R| a \leq x \leq b\}$ (abgeschlossen)\\*
$(a, b] = \{x \in \R| a < x \leq b\}$ (halboffen)\\*
$[a, b) = \{x \in \R| a \leq x < b\}$ (halboffen)\\* %mit klammer zu überer zeile\\*
$(a, b) = \{x \in \R| a < x < b\}$ (offen)\\*
%
Uneigentliche Intervalle:
$$[a, \infty) = \{x \in \R | a \leq x\} = \R_{\geq a}$$
$$(a, \infty) = \{x \in \R | a < x\} = \R_{> a}$$
$$(- \infty, a] = \{x \in \R | x \leq a\} = \R_{\leq a}$$
$$(- \infty, a) = \{x \in \R | x < a\} = \R_{< a}$$
$$(- \infty, \infty) = \R$$
%
\Bsp{Funktionen}
\begin{enumerate}
\item{$f:[0,2]→\R, f(x)=x^2, \Gamma_f \leq [0,2] x\R$}
\item{Betragsfunktionen: $|\ |: \R→\R, x\mapsto|x|$
%noch mehr graphen aaaahahhahahah
}
An dieser Stelle fehlen noch Graphen.
\item{$g:\R\bs\{0\}→\R, g(x)=\dfrac{1}{x}$
%graph
Hier auch.
}
\item{$exp:\R→\R$.}
\item{[.] : $\R \to \R$ Gaußklammer\\*
$[x] := max\{n \in \Z | n \leq x \}$
\bsp
$[5] = 5$\\*
$[5,78] = 5$\\*
$[-1,2] = -2$}
\item{Sei $h:\R→\R$ definiert durch $h(x)=\begin{cases}0\ wenn\ x\in\Q\\* 1\ wenn\ x\notin\Q\end{cases}$\\*
$h(\sqrt{2}) = 1, h (\frac{3}{7}) = 0$}
\end{enumerate}

\sS{Definition (Rechnen mit Funktionen)}
Sei $D \subseteq \R , \ f,g: D→\R$ Funktionen auf D.\\*
Definiere
\begin{itemize}
\item{$f+g: D \to \R$ durch $(f + g)(x) := f(x) + g(x)$}
\item{$(f \cdot  g) (x) := f(x) \cdot g(x)$}
\item{Für $a\in\R$ setze $a·f: D→\R, (a·f)(x):=a·f(x)$}
\item{Angenommen, $f(x) \neq 0$ für alle $x \in D$
$$\frac{1}{f}: D \to \R, \frac{1}{f}(x) := \frac{1}{f(x)} = f(x)^{-1}$$
\ul{Vorsicht} nicht $\frac{1}{f}$ mit Umkehrbild oder Urbild verwechseln}
\end{itemize}

\sS{Definition Polynomfunktion}
\begin{itemize}
\item{Eine \ul{Polinomfunktion} ist eine Funktion der Form\\*
$f: \R → \R,\ f(x) = a_n x^n+a_{n-1}x^{n-1}+…+a_0=\ds\sum_{k=0}^n a_k x^k $\\*
wobei $a_0,…,a_n \in \R$ fest}
\item{Seien $f, g : \R \to \R $ Polynomfunktionen
Sei $D = \{x \in \R \mid g(x) \geq 0\}\leadsto \dfrac{f}{g} : D \to , Rx \mapsto \frac{f(x)}{g(x)}$
Solche Funktionen heißen rationale Funktionen.
\bsp
$f:\R\bs\{0,1\}→\R, \ f(x)=\dfrac{x^7+5x^2}{x(x-1)}$}
\end{itemize}
\sS{Definition}
Seien $f: C \to \R, g: D \to \R$ Funktionen, sodass $f(C) \subseteq D$
Eine Komposition von $ f $ und $ g $ ist 
%
$g \circ f : C \to \R$\\*
$(g \circ f) \ (x) = g(f(x))$\newpage
%% Kopfzeile beim Kapitelanfang:
\fancypagestyle{plain}{
%Kopfzeile links bzw. innen
\fancyhead[L]{\calligra\Large Vorlesung Nr. 11}
%Kopfzeile rechts bzw. außen
\fancyhead[R]{\calligra\Large 15.11.2012}
}
%Kopfzeile links bzw. innen
\fancyhead[L]{\calligra\Large Vorlesung Nr. 11}
%Kopfzeile rechts bzw. außen
\fancyhead[R]{\calligra\Large 15.11.2012}
% **************************************************
%
\wdh
Eine Abbildung $f:x→y$\\*
\begin{itemize}
\item{ist \ul{injektiv} wenn gilt:\\*
für alle $a,b\in X$ mit $f(a)=f(b)$ ist $a=b$}
\item{ist \ul{surjektiv} wenn für jedes $y\in Y$ ein $a\in X$ existiert mit $f(a)=y$}
\end{itemize}
Sei $D\subseteq\R$ Teilmenge. Eine Funktion auf $D$ ist eine Abbildung $f:D→\R$
%
\uS{Monotone Funktionen}
\bem
Eine Funktion $(a_n)_{n\geq 0}$ reeller Zahlen ist eine Abbildung $a:\N_0→\R$ d.h. eine Funktion auf $\N_0$
%
\sS{Definition}
Sei $D\subseteq\R$. Eine Funktion  $f:D→\R$ heißt:
\begin{enumerate}
\item{\ul{monoton wachsend} wenn gilt:\\*
Für alle $a,b\in D$ mit $a<b$ ist immer $f(a)\leq f(b)$}
\item{\ul{streng monoton wachsend}: $a<b\Rarr f(a)<f(b)$}
\item{\ul{monoton fallend}: $a<b\Rarr f(a)\geq f(b)$}
\item{\ul{streng monoton fallend}: $a<b\Rarr f(a)> f(b)$}
\end{enumerate}
%
\bem
Jede streng monotone Funktion $f$ ist injektiv
%
\bew
Zeige: $a\neq b\Rarr f(a)\neq f(b)$\\*
Wenn $a\neq b$ dann $a< b$ oder $b<a$\\*
Wenn $f$ streng monoton wachsend: Folgt $f(a)< f(b)$ oder $f(b)< f(a)$ also $f(a)\neq f(b)$\\*
Wenn $f$ streng monoton fällt: es folgt $f(a)> f(b)$ oder $f(b)> f(a)$ also $f(a)\neq f(b)$\qed
%
\sS{Beispiel}
\begin{enumerate}
\item{$f:\R_{\geq 0}→\R,\ x\mapsto x^k =:f(x)$ mit $k\geq 1$\\*%umgekehrtes define
$f$ ist streng monoton wachsend/steigend 
%bild tafel 2.2
\item{$h: \R \to \R, h(x) = [x]$}\\*
% Stufenfunktion Diagramm
$h$ ist monoton wachsend, aber nicht streng monoton.\\*
Monoton wachsend: $x < y \Rarr [x] < [y]$\\*
$x < y \not\Rarr [x] < [y]$\\*
z. B.: $1,2 < 1,3 , [1,2] = 1 = [1,3] $}
\item{Exponentialfunktion\\*
$exp: \R \to \R,\ exp(x)= \ds\sum_{k=0}^{\infty} \frac{x^k}{k!}$\\*
Ist streng monoton wachsend.\\*
\bew
\begin{enumerate}
\item{$exp(0) = 1 + \frac{0}{1!} + \frac{0}{2!} + ... = 1$}
\item{Sei $a > 0$\\*
$exp(a) = = 1 + \frac{a}{1!} + \frac{a}{2!} + ... > 1$}
\item{Sei $a > 0 exp(-a) \cdot exp(a) = exp(-a + a) = exp(0) = 1$\\*
$\Rarr exp(-a) = \frac{1}{exp(a)} \Rarr 0 < exp(a) < 1$\\*
%das kann ich nicht lesen... (iPhone Bild)
$exp(b) > 0$ für alle $b \in \R$}
\item{Sei $a > b$\\*
$exp(a) = exp(a - b + b) = \overbrace{exp(a - b) \cdot exp(b)}^{>0}$
> exp(b) $\Rarr$ exp streng monoton wachsend \qed }
\end{enumerate}
}
\end{enumerate}
%
\chapter{Stetigkeit}
\ul{Idee:} Eine Funktion ohne Sprünge heißt \ul{stetig}\\*
\sS{Definition}
Sei $D\subseteq \R,\ f:D→\R$ eine Funktion\\*
\begin{enumerate}
\item{$f$ heißt stetig in $x\in D$ wenn gilt:\\*
Für jedes $\e>0$ gibt es ein $\delta>0$ so dass für jedes $y\in D$ mit $|x-y|<\delta$ gilt $|f(x)-f(y)|<\e$ %graph 6.2
}
\item{$f$ heißt stetig wenn f in jedem $x\in D$ stetig ist}
\end{enumerate}
\sS{Beispiel}
\begin{enumerate}
\item{Die Funktion $id:\R→\R,\ x\mapsto x$ ist stetig}
\item{Die Funktion $f:\R→\R,\ f(x)=x^2$ ist stetig. %graph klein
\bew
Sei $x,y\in\R\quad y=x+h$.\\*
$$f(y)-f(x)=(x+h)^2-x^2=x^2+2xh+h^2-x^2=2xh+h^2$$\\*
Wähle jedenfalls $\delta\leq 1$. Wenn $|h|=|x-y|<\delta$ dann $|h|<1$\\*
$$|f(y)-f(x)|=|2xh+h^2|\leq|2x|·|h|+|h|^2<|2x|·|h|+|h|=(|2x|+1)·|h|$$\\*
Gegeben sei $\e>0$\\*
Wähle $\delta=min\left\{1,\dfrac{\e}{|2x|+1}\right\}$\\*
Wenn $|x-y|<\delta$ dann $$|f(x)-f(y)|<(2|x|+1)·|h|<(2|x|+1)·\dfrac{\e}{2|x|+1}=\e$$\\*
Also $f$ stetig in $x$}
\item{$g:=\R \to \R,\ g(x):=\{x\}$\\*
%graph
g ist stetig an $x$ \equ $x\notin\Z$
\Bew{$g$ nicht stetig an $x\in\Z$:}
Zeige: es gibt ein $\e>0$ so dass kein $\delta>0$ existiert mit: $|x-y|>\delta\Rarr|g(x)-g(y)|<\e$\\*
z.B. $\e=1$ Sei $\delta>0.\ y=x-\dfrac{\delta}{2}\quad |x-y|=\dfrac{\delta}{2}<\delta$\\*
aber $g(y)=\{x-\dfrac{\delta}{2}\}=x-1$ (weil $x\in\Z$)\\*
$|g(x)-g(y)|=|x-(x-1)|=1 \not<\e\qed$}
\end{enumerate}
%
\sS{Satz}
Die Exponentialfunktion $exp: \R \to \R$ ist stetig.\\*
\bew
Verwende nur:
\begin{itemize}
\item{Funktionalgleichung: $exp(x + y) = exp(x) \cdot exp(y)$}
\item{exp ist streng monoton wachsend}
\item{exp(0) = 1}
\end{itemize}
\subsection*{Behauptung}
Für jedes $\e > 0$ gibt es ein $n \in \N$ mit $exp(\frac{1}{n}) < 1 + \e$\\*
%Graf
Angenommen, $exp(\frac{1}{n}) \geq 1 + \e$\\*
Dann $exp(1) = \frac{1}{n} + ... \frac{1}{n}$ % underbrace unter den Brüchen {n}
\phantom{Dann $exp(1) $} $= exp(\frac{1}{n}) + ... + exp(\frac{1}{n}) = exp(\frac{1}{n})^n$ \\*
\phantom{Dann $exp(1) $} $ \geq (1 + \e)^n \geq 1 + n \e $ %(Pfeil auf letztes geq Bernoulli)
\\*
$exp(1) \geq 1 + n \e$\\*
$n \leq \frac{exp(1) - 1}{\e}$\\*
Das gilt nur für endliche viele $n \in \N$\\*
$\Rarr$ Beh.\\*
\ul{Zeige:} exp ist stetig an 0. Gegeben sei $\e > 0, OE ? \e < 1$\\*
Wähle $n \in \N$ mit $exp(\frac{1}{n}) < 1 + \e$\\*
$$\Rarr exp(-\frac{1}{n}) = exp(\frac{1}{n})^{-1} < \frac{1}{1 + \e} = \frac{1 - \e}{(1+\e)(1-\e)} = \frac{1-\e}{1 - \e^2} > 1 - \e$$\\*
Sei $\delta \frac{1}{n}$\\*
Sei $y \in \R, |0 - y| < \delta = \frac{1}{n}$\\*
$|y| < \frac{1}{n}$ d.h.\\*
$-\frac{1}{n} < y < \frac{1}{n}$\\* \\*
exp streng monoton wachsend.\\*
$$\Rarr 1 - \e < exp(-\frac{1}{n}) < \exp(y) < exp(\frac{1}{n}) < 1 + \e$$\\*
$\Rarr |exp(y) - exp(0)| < \e$ Also exp stetig in 0\\*
\ul{Zeige:} exp ist eine stetig in $x \in \R$. Gegeben sei $\e > $\\*
Sei $y = x + h$, $|h| < \delta$ ($\delta$ noch zu wählen)
$$|exp(y) - exp(x)| = |exp(x + h) - exp(x)| = |exp(x) \cdot exp(h) - exp(x)| = exp(x) \cdot exo(h) -1$$\\*
$|exp(y) -exp(x) | < \e$\\*
$$\Leftrightarrow exp(x) \cdot |exp(h) - 1| < \e \Leftrightarrow exp(h) - 1 < \frac{\e}{exp(x)} = \e '$$\\**
Weil exp stetig in 0 ist gibt es ein $\delta > 0$ mit $|h| < \delta \Rarr |exp(h) -1| < \frac{\e}{exp(x)}$\\*
$\Rarr$ exp ist stetig in x \qed\\*
%
\sS{Satz (Folgenstetigkeit)}
Sei $D\subseteq\R,\ x\in D,\ f:D→\R$ Funktion $f$ ist genau dann stetig in $x$ wenn gilt:\\*
\begin{itemize}
\item{Für jede Folge $(x_n)_{n\geq 0}$ mit $x_n\in D,\ x_n→x$ für $n→∞$ gilt auch $f(x_n)→f(x)$ für $n→∞$}
\end{itemize}
%
\sS{Satz}
Sei $D\subseteq\R,\ f,g:D→\R$ in $x\in D$\\*
Dann gilt:\\*
\begin{itemize}
\item{$f+g:D→\R$ stetig in $x$}
\item{$f·g:D→\R$ stetig in $x$}
\item{Wenn $g(x)\neq 0$ für alle $x'\in D$}
\end{itemize}
Dann ist $\dfrac{1}{f}:D→\R$ stetig in x.
\Bew{mit Folgenstetigkeit}
Sei $x_n \to x$ für $n \to \infty$\\*
mit $x_n \in D$\\*
$f, g$ stetig $\Rarr f(x_n) \to f(x)$ \\*
\phantom{$f, g$ stetig \Rarr}$g(x_n) \to g(x)$\\*
$\Rarr f(x_n) + g(x_n) \to f(x) + g(x)$
$\phantom{\Rarr} f(x_n) \cdot g(x_n) \to f(x) \cdot g(x)$\\*
Wenn also $f(x) \neq 0$\\*
$f(x_n)^{-1} \to f(x)^{-1}$\\*
$\Rarr f + g, f) \cdot g, \frac{1}{f}$ stetig in x \qed\newpage
%%Kopfzeile links bzw. innen
\fancyhead[L]{\calligra {\Large Vorlesung Nr. 12}}
%Kopfzeile rechts bzw. außen
\fancyhead[R]{\calligra \Large{19.11.2012}}
% **************************************************
\wdh
Sei $D \subseteq \R$. eine Funktion $f: D \to \R$ ist stetig in $x \in D$ wenn gilt:\\
\begin{array}{ll}
\text{Für jedes $\epsilon > 0$ gibt es ein $\delta > 0$ so dass gilt:}\\
\text{wenn $y \in D$ mit $|x - y| < \d$ dann $|f(x) - f(y)| < \e$}
\end{array}
\bsp
$exp:R→\R$ ist stetig (d.h. stetig an jedem $x\eR$)\\
$[·]:R→\R$ ist stetig an $x \equ x\not\in\Z$\\
%Diagramm, Gausklammern
\sS{Satz Folgenstetigkeit}
Eine Funktion $f: D \to \R$ ist stetig in $x \in D$ $\equ$ Für jede Folge $(x_n)$ mit $x_n \in D$ für alle $n$ und $x_n \to x$ gilt auch $f(x_n) \to f(x)$.\\
(d.h. f erhält Konvergenz)\\
\bew
"\Rarr" Angenommen $f$ ist stetig in $x$, $x_n→x$ mit $x_n\in D$. Zeige $f(x_n)→f(x)$\\
Gegeben $\e>0$. Stetigkeit \Rarr{} es gibt $\delta>0$ mit:\\
Wenn $y\in D$ mit $|x-y|<\delta$ dann $|f(x)-f(y)|<\e$\\
Wähle $N\eN$ so dass gilt:\\
Für $n\geq N$ ist $|x-x_n|<\delta \Rarr |f(x)-f(x_n)|<\e$\\
Also gilt $f(x_n)→f(x)$\\
"\Larr" Angenommen $f$ ist nicht stetig.\\
Zeige: Es gibt eine Folge $(x_n)$ mit $x_n\in D$ und $x_n→x$ aber nicht $f(x_n)→f(x)$ für nicht stetig in $x$ \Rarr{} es gibt ein $\e>0$ so dass für jedes $\delta>0$ ein $y\in D$ existiert mit $|x-y|<\delta$ und $|f(x)-f(y)|\geq \e$\\
Wähle für $\delta=\frac{1}{n}$ ein $x_n\in D$ mit $|x_n-x|<\delta,\ |f(x_n)-f(x)|\geq\e$\\
Dann gilt $x_n→x$ aber \underline{nicht} $f(x_1)→f(x)\qed$
%
\wdh
\sS{Satz 6.5}
Wenn $f, g: D \to \R$ stetig in $x \in D$ dann auch $f + g,\ f \cdot g$ und $\frac{1}{g}$ (falls $g(y) \neq 0$ für alle $y \in D$) \\
Und $a \cdot f$ für $a \in \R$
\sS{Korollar} 
Polynomfunktionen und rationale Funktionen sind stetig.
\bew
\begin{enumerate}
\item{$id_{\R}:\R→\R,\ id_{\R}(x)=x$ ist stetig}
\item{6.5 und Induktion \Rarr{} für $\R→\R,\ f(x)=x^n$ stetig für jedes $n\quad (x^n=x·x^{n-1}$}
\item{6.5 \Rarr{} Jede Funktion $f(x)=a_n·x^{n}+a_{n-1}·x^{n-1}+…+a_0,\ f:\R→\R$ ist stetig}
\item{6.5 \Rarr{} wenn $f,g:\R→\R$ Polynomfunktionen, dann $\frac{f}{g}:D→\R$ stetig, wobei $D=\{x\eR|g(x)\neq 0\}$ (denn $(\frac{f}{g}=f·\frac{1}{g})$\qed}
\end{{enumerate}
\sS{Satz Stetigkeit der Komposition}
Sei $f: C \to \R$, $g: D \to \R$ Funktionen mit $f(C) \subseteq D$. Wenn $f$ stetig in $x \in D$ und g stetig in $f(x)$ dann ist:\\
$g \cird f: C \to \R$ stetig in x.\\
\bew mit Folgenstetigkeit:
Sei $x_n \to x$ mit $x_n \in C$\\
$f$ stetig in $x \Rarr \ f(x_n) \to f(x)$\\
$g$ stetig in $f(x) \Rarr g(f(x_n)) \to g(f(x))$\\
d.h. $(g \circ f)(x_n) \to (g \circ f)(x)$\\
also ist $g \circ f: C \to \R$ stetig in $x$ \qed
\sS{Definition (Konvergenz bei Funktionen)}
Sei $D\subseteq\R$ und $f:D→\R$ Funktion\\
\begin{enumerate}
\item{Ein $a\eR\cup\{-∞,∞\}$ heißt Berührpunkt von $D$ wenn es eine Folge $(x_n)$ mit $x_n\in D$ und $x_n→a$ gibt
\bem
Jedes $a\in D$ ist Berührpunkt von $D$ (wähle konstante Folge $x_n=a$)}
\item{Angenommen, $a$ ist ein Berührpunkt von $D$\\
Schreibe $\ds\lim_{x→a}f(x)=y$ wenn gilt:\\
Für jede Folge $(x_n)$ mit $x_n→a$ und $x_n\in D$ gibt $f(x_n)→y$}
\item{Angenommen, $a\neq ∞$ ist eine Berührpunkt von $D\cap(a,∞)$ SKIZZE \\% SKIZZE
Schreibe $\ds\lim_{x\searrow a}f(x)=y$ wenn gilt:\\
Für jede Folge $(x_n)$ mit $x_n→a$ und $x_n\in D$ und $x_n>a$ gilt $f(x_n)→y$}
\item{Angenommen, $a\neq -∞$ ist eine Berührpunkt von $D\cap(-∞,a)$ Schreibe $\ds\lim_{x\nearrow a}f(x)=y$ wenn für jede Folge $(x_n)$ mit $x_n→a$ und $x_n\in D$ und $x_n<a$ gilt $f(x_n)→y$
\bsp
$D=\R\bsc\{0\}\ f:D→\R,\ f(x)=\frac{|x|}{x}$ SKIZZE\\
\underline{Bemerke} f(x)=$\left\{\begin{array}{lcl}1 & \text{wenn} & x>0\\-1 & \text{wenn} & x<0\end{array}\right.$\\
$\ds\lim_{x\nearrow a}f(x)=-1,\ \lim_{x\searrow a}f(x)=1,\ \lim_{x→0}f(x)existiert nicht$\\
\ul{Vorher} $a = 0$ ist Berührpunkt vpn $D$ und $D \cap (0, \infty)$ und $D \cap (- \infty , 0)$, denn $\frac{1}{n} \to 0$ und $-\frac{1}{n} \to 0$}
\end{enumerate}
Sei $g: \R \to \R$, $g(x) = x^3$\\
$\infty , - \infty$ sind Berührpunkte von $D = \R$\\
$\ds\lim_{x \to \infty} g(x) = \infty$\\
$\ds\lim_{x \to -\infty} g(x) = -\infty$\\ 
\bem{Umformulierung von Satz 6.4}
Eine Funktion $f: D \to \R$ ist stetig in $a \in D \\ $
$\equ \quad \ds\lim_{x \to a} f(x) = f(a)$
\uS{Sätze über stetige Funktionen}
\Def
Eine Funktion $f:D→\Re$ heißt nach oben \ul{beschränkt} wenn die Menge $f(D)$ nach oben beschränkt ist, d.h. es gibt $C\eR$ mit $f(x)\leq C$ für alle $x\in D$\\
$f$ heißt nach unten beschränkt, wenn $f(D)$ nach unten beschränkt\\
$f$ heißt \ul{beschränkt}, wenn $f)$ nach oben und nach unten beschränkt\\
\sS{Definition}
Sei $M \in \R$ eine nicht-leere Teilmenge wenn $M$ nach oben unbeschränkt, schreibe $sub(M) = \infty$ (Sprich: uneigentliches Supremum)\\
\Satz
Sei $a,b\eR$ mit $a\leq b$ und $f:[a,b]→\R$ eine \ul{stetige} Funktion, dann ist $f$ beschränkt und nimmt ihr Maximum und Minimum an, d.h. es gibt $x_{min},x_{max}\in[a,b]$ mit: $f(x_{min}\leq f(x) \leq (x_{max})$ für jedes $x\eR$
\bsp
\begin{enumerate}
\item{$f: (0, 1) \to \R, \quad f(x) = \frac{1}{x}$ % Graph
$\ds\lim_{x \to 0} f(x) = \infty$\\
Somit $f$ nicht nach oben beschränkt}
\item{$g(0,1) \to \R, \ g(x) = x$\\
$g$ beschränkt. $g((0,1)) = (0,1)$\\
$sup \{\g(x) | x \in (0, 1) \} = 1$\\
Aber $g(x) < 1$ für alle $x \in (0, 1)$. Also nimmt g nicht ihr Maximum an.}
\end{enumerate}
\sss{Beweis von 6.11}
Sei $y:=sup\{f(x)|x\in D}\eR \cup \{∞\}$\\
Wähle eine Folge $(x_n)$ mit $x_n\in D$ und $f(x_n)→y$ für $n→∞$\\
Bolzano-Weierstraß \Rarr{} Es gibt eine konvergente Teilfolge $(x_{n_k})_{k\geq 0}$ von $(x_{n})_{k\geq 0}$ \marginpar{Die Folge $(x_n)$ ist beschränkt $D=[a,b]$}\\
Sei $x_{n_k}→x$ für $k→∞$ $a\leq x_n\leq b$ für alle $n$ \Rarr $a\leq x\leq b,\ x\in D=[a,b]$\\
\einruck{Und dann gilt:}{$f()x_{n_k})→y$ für $k→∞$ (Teilfolge einer konvergenten Folge)\\$f()x_{n_k})→f(x)$ für $k→∞$ weil f stetig}
Also $y = f(x)$\\
Insbesondere $y \neq \infty$ also $f$ beschränkt\\
Für alle $x' \in D$ gilt $f(x') \leq sup \{f(D)\} = y = f(x)$\\
Setze $x_{max} := x$. Dann $f(x') \leq f(x_{max})$ für alle $x' \in D$\\
Anfang findet man $x_{min}$ \qed
\sS{Satz (Zwischenwertsatz)}
Sei $a\leq b$ und $f:[a,b]→\R$ stetig.\\
Wenn $y\eR$ zwischen $f(a)$ und $f(b)$  liegt, d.h. $f(a)\leq y \leq f(b)$ oder $f(a)\geq y \geq f(b)$\\
Dann gibt es ein $x\in[a,b]$ mit $f(x)=y$ GRAPH\\\newpage
%% Kopfzeile beim Kapitelanfang:
\fancypagestyle{plain}{
%Kopfzeile links bzw. innen
\fancyhead[L]{\calligra\Large Vorlesung Nr. 13}
%Kopfzeile rechts bzw. außen
\fancyhead[R]{\calligra\Large 22.11.2012}
}
%Kopfzeile links bzw. innen
\fancyhead[L]{\calligra\Large Vorlesung Nr. 13}
%Kopfzeile rechts bzw. außen
\fancyhead[R]{\calligra\Large 22.11.2012}
% **************************************************
%
\wdh
Zwischenwertsatz\\*
Sei $a\leq b,\ f:[a,b]→\R$ stetig\\*
Sei $y\eR$ zwischen $f(a)$ und $f(b)$ d.h. $f(a)\leq y\leq f(b)$ oder $f(a)\geq y\geq f(b)$\\*
Dann gibt es ein $x\in[a,b]$ mit $f(x)=y$\\*
SKIZZE
\Bew {Intervallschachtelung}
Starte mit $[a_0,b_0]=[a,b]$\\*
Definiere unendliche Kette von Intervallen\\*
$[a_0,b_0]\supseteq [a_1,b_1]\supseteq [a_2,b_2]\supseteq …$\\*
So dass $[b_n-a_n]=2^{-n}[b_0,a_0]$ und $y$ zwischen $f(a_n)$ und $f(b_n)$ liegt.\\*
Annahme: $f(a)\leq y\leq f(b)$ (Anderer Fall $f(a)\geq y\geq f(b)$ analog)\\*
Angenommen, $[a_n,b_n]$ ist konstruiert so dass $[b_n-a_n]=2^{-n}(b_0,a_0)$ und $f(a_n)\leq y\leq f(b_n)$\\*
Betrachte $m:=\frac{a_n+b_n}{2}$, Wenn $f(m)\geq y$ dann setze $[a_{n+1},b_{n+1}]:=[a_n,m]$\\*
Wenn $f(m)<y$ dann setze $[a_{n+1},b_{n+1}]:=[m,b_n]$\\*
Dann gilt in beiden Fällen:
$$b_{n+1}-a_{n+1}=\frac{1}{2}(b_n-a_n)=2^{-1}·2^{-n}(b_0-a_0)=2^{-n-1}(b_0-a_0) \text{ und }f(a_{n+1})\leq y\leq f(b_{n+1})$$
%
\sss{Idee}
Folge von Intervallen $[a_n, b_n]$ "konvergent" gegen gesuchtes $x$.\\*
Die Folge $(a_n)_{n \geq 0}$ ist monoton wachsend und beschränkt, ($b$ ist obere Schranke)
\Rarr\\ $(a_n)$ konvergiert, sei $x:=\lim_{\nif} a_n$\\*
Die Folge $(b_n)_{n \geq 0}$ ist monoton fallen und beschränkt \Rarr konvergent nach 4.2\\*
Sei $x' = \lim_{n \to \infty} b_n$\\
$x' - x = \lim_{n \to \infty} (b_n - a_n)$\\*
$= \lim_{n \to \infty} (2^{-n} \cdot (b_0 - a_0)) = 0$\\*
also $x = x'$ $f(x) = ?$\\
$f$ stetig $\Rarr f(x) = \lim_{n \to \infty} f(a_n) = \lim_{n \to \infty} f(b_n)$\\
Wegen $f(a_n) \leq y \leq f(b_n)$ für alle $n$ gilt $f(x) = \lim_{n \to \infty} f(a_n) \leq y \leq \lim_{n \to \infty} f(b_n)$\\*
	$\Rarr f(x) = y$ 
%
\bem
Weil $a\leq a_n\leq b$ gilt $a\leq x\leq b$ d.h. $x\in[a,b]$\qed
\uS{Anwendung}
\sS{Satz Sichere Nullstellen}
Sei \nN{} \ul{ungerade}, $f:\R→\R$
$$f(x)=x^n+a_{n-1}·x^{n-1}+…+a_0$$
Dann hat $f$ eine Nullstelle, d.h. es gibt $x\eR$ mit $f(x)=0$
\bew
Für $x\neq 0$ betrachte
$$g(x)=\frac{1}{x^n} \cdot f(x)=1+\frac{a_{n-1}}{x}+\frac{a_{n-2}}{x^2}+…+\frac{a_0}{x^n}$$
Für $x→∞$ ist $g(x)→1$\\*
Für $x→-∞$ ist $g(x)→1$\\*
D.h. es gibt $a\eR$ mit $a>0$ und
$$x\geq a \Rarr g(x)>0$$
$$x\geq -a \Rarr g(x)>0$$
%
\ul{Also} $x \geq a \Rarr f(x) = x^n \cdot g(x) > 0$\\*
	$x \geq a \Rarr f(x) = \underset{\overset{\uparrow}{x^n < 0}}{x^n} \cdot \underset{\overset{\uparrow}{>0}}{g(x)} < 0$\\*
	$f(-a) < 0 < f(a)$\\*
	Zwischenwertsatz \Rarr ergibt $x \in [-a, a]$ mit $f(x) = 0$

\sS{Satz Ergänzung Zwischenwertsatz} % 6.14
	Sei $f: D \to \R$ \ul{stetig} und $D \subseteq \R$ ein nicht-leeres Intervall. Dann ist $f(D) = \{f(x) | x \in D\}$ auch ein Intervall (d.h. hat keine Lücken.)
\bem
	Hier sind auch uneigentliche Intervalle zugelassen. (z.B. $(0, \infty)$)
%
\bsp
$f(x)=x^3-x+20$ GRAPH
\bsp
Bedingung "$n$ ungerade" ist wesentlich, denn $f(x)=x^2+1$ hat keine Nullstelle
% Satz + bem
\bsp
$$f:(0,1)→\R,\ f(x)=\frac{1}{x}$$
$$f(D)=(1,∞)$$ GRAPH
\bsp
$$f:(-1,1)→\R,\ f(x)=x^2$$
$$f(D)=[0,1)$$ GRAPH
\bew
$f:D→\R$ stetig\\*
Sei $a:=inf(f(D))\eR\cup\{-∞\}$\\*
\phantom{Sei }$a:=inf(f(D))\eR\cup\{-∞\}$\\*
Angenommen $y\eR$ mit $a<y<b$ d.h. $x\in(a,b)$\\*
Es gibt $x_1,x_2\in D$ mit $a<f(x_1)<y<f(x_2)<b$\\*
Zwischenwertsatz \Rarr{} es gibt $x$ zwischen $x_1,x_2$\\*
(\Rarr $x\in D$ weil $D$ Intervall) mit $f(x)=y$\\*
Also $(a,b) \subseteq f(D)$\\*
Dann ist $f(D)$ eines der Intervalle $(a,b),[a,b),(a,b],\underset{\overset{\uparrow}{\text{nur wenn }a\neq -∞, b\neq ∞}}{[a,b]}$\qed
% 
\sS{Satz Umkehrfunktion}
	Sei $D$ ein Intervall, $f: D \to \R$, stetig, streng monoton wachsend oder fallend. Sei $D' = f(D)$ (Intervall nach 6.14)\\*
	Dann ist die Abbildung $f: D \to D'$ bijektiv und die Umkehrabbildung $f^{-1}: D' \to D$ ist stetig und streng monoton wachsend, bzw. fallen.
%
\bew
Die Abbildung $f:D→D'$ ist
\begin{itemize}
\item{surjektiv nach Definition von $D'$}
\item{streng monoton \Rarr{} injektiv}
\item{also bijektiv. Somit existiert $f^{-1}:D'→D$}
\end{itemize}
\sss{Annahme}
$f$ streng monoton wachsend (fallend analog)
%
\beh
$f^{-1}$ ist streng monoton wachsend, d.h. gegeben $x_1,x_2\in D'$ mit $x_1<x_2$ zeige:
$$f^{-1}(x_1)<f^{-1}(x_2)$$
Angenommen $f^{-1}(x_1)\geq f^{-1}(x_2)$ \Rarr{} $f$ monoton wachsend \Rarr{} $x_1=f(f^{-1}(x_1))\geq f(f^{-1}(x_2))=x_2$ \Rarr{} Widerspruch\\*
also $f^{-1}(x_1)<f^{-1}(x_2)$ \Rarr{} $f^{-1}$ streng monoton wachsend
%
\beh
	$f^{-1}$ ist stetig. Gegeben $x \in D$\\*
	Annahme $x$ ist kein Randpunkt des Intervalls $D'$\\*
	Gegeben sei $\e > 0$ Suche $\delta$ mit (Stetigkeitsdefinition)\\*
	$y := f^{-1}(x) \in D$ ist kein Randpunkt (weil $f,\ f^{-1}$ bijektiv und streng monoton.)\\*
	%Graph
	Wähle $\e'\leq \e$ mit $\e > 0$ sodass $[y-\e', y+\e'] \subseteq D'$\\*
	$f(y - \e') < f(y) = x \Larr y - \e' < y$\\*
	also $f(y - \e') = x - \delta_1 \qquad \delta_1 > 0$\\*
	$genauso f(y + \e') = x + \delta_2 \qquad \delta_2 > 0$ \\*
	%Graph
	Sei $\delta = min(\delta_1, \delta_2)$
\beh
	Wenn $z \in D'$ mit $|z - x| < \delta$ dann $|f^{-1}(z) - f^{-1}(x)| < \e$
%
\bew
$$x+δ_1\leq x-δ<z<x+δ\leq x+δ-2$$
$f^{-1}$ streng monoton wachsend \Rarr
$$f^{-1}(x)-ε'=f^{-1}(x-δ_1)<f^{-1}(z)<f^{-1}(x+δ-2)=y+ε'=f^{-1}(x)+ε'$$
$$\Rarr |f^{-1}(z)-f^{-1}(x)|<ε'\leq ε \text{\Rarr{} Behauptung}$$
Somit $f^{-1}$ stetig in $x$\\*
Falls $x$ Randpunkt: Betrachte $[x,x+δ]$ bzw. $[x-δ,x]$ wieder analog\qed
%
\uS{2 Anwendungen}
\sS{Beispiel}
	Sei $k \eN$\\*
	Die Abbildung $f: \R_{\geq 0} \to \R_{\geq 0} f(x) = x^k$\\*
	%Graph
	\ul{Bekannt} $f$ ist stetig streng monoton wachsend.\\*
	$f(0) = 0, \qquad \lim_{x \to \infty} x^k = \infty$\\*
	$D:= [0, \infty) = \R_{\geq 0}$\\*
	$f(D) = D' = [0, \infty)$\\*
	6.15 \Rarr\ $f$ hat \ul{stetige} und streng monoton wachsende Umkehrfunktionen\\*
	$f^{-1} : [0, \infty) \to [0, \infty)$\\*
	Bezeichnung: $f^{-1}(x) = \sqrt[k]{x}$
%
\uS{Logarithmus und allgemeine Potenzen}
\sS{Satz}
Die Exponentialfunktion $exp:\R→\R_{>0}=(0,∞)$ ist stetig, streng monoton wachsend und $exp(\R)=\R_{>0}$
\bew
Bekannt: $exp$ stetig, streng monoton wachsend.
Für $x>0$ ist $$exp(x)=1+x+\frac{x^2}{2}+…\geq 1+x$$
also gilt $$\lim_{x→∞} exp(x)=∞$$
$$exp(-x)=\frac{1}{exp(x)}\Rarr \lim_{x→-∞}exp(x)=\lim_{x→∞}\frac{1}{exp(x)}$$
Somit $exp(\R)=(0,∞)$\qed\\*
Folge mittels 6.15: $exp:\R→\R_{>0}$ ist bijektiv und die Umkehrfunktion $exp^{-1}:=log:\R_{>0}→\R$ ist stetig, streng monoton wachsend, bijektiv \footnote{$exp^{-1}=log$ heißt Logarithmusfunktion} konkret: $log(x)=y \equ x=exp(y)$ GRAPH
\newpage
% heading und wdh
%
\sS{Satz (Eigenschaften des Logarithmus)}
\begin{enumerate}
\item{$log(1)=0$}
\item{$log(x·y)=log(x)+log(y)$}
\item{$\lim\limits_{x→0}log(x)=-∞$}
\item{$\lim\limits_{x→∞}log(x)=∞$}
\end{enumerate}
\bew
Folgt aus Eigenschaften von $exp$, Details: Übung
\sss{Erinnerung:}
also $a>0,\ n\eZ,\ m\eN$ ist $a^{\frac{n}{m}}:=\sqrt[m]{a^n}$
%Lemma 6.19
%Chritopher
\sS{Definition}
Sei $a>0,\ x\eR$ setze $a^x:=exp(x·log(a))$\\*
\sS{Bemerkung}
Die Regeln der Potenzrechnung gelten:
$$a^{x+y}=a^x·a^y,\qquad a^{x·y}=(a^x)^y$$
\bew
$$a^{x+y}=exp((x+y)·log(a))=exp(x·log(a))·exp(y·log(a))=a^x·a^y$$
$$a^{x·y}=exp(x·y·log(a))=(a^x)^y=exp(y·log(exp(x·log(a))))\footnote{$log\circ exp= id$}=exp(y·x·log(a))$$\qed
\bem
Eulersche Zahl
$$e:=exp(1)=2{,}7…\footnote{$log(e)=1$}\qquad e^x=exp(x·log(e))=exp(x)$$
%Definition und Gleichmäßige Stetigkeit
\sS{Definition:}
Eine Funktion $f:D→\R$ heißt gleichmäßig stetig wenn gilt:\\*
Für jedes $ε>0$ gibt es ein $δ>0$ so dass für alle $x,y\in D$ mit $|x-y|<δ$ gilt:
$$|f(x)-f(y)|<ε$$
\sss{Wesentlich:}
$δ$ hängt nur von $ε$, nicht von $x$ ab.
\bsp
$D=(0,1)\qquad f:D→\R,\ f(x)=\frac{1}{x}$\\*
GRAPH $f$ stetig, aber nicht gleichmäßig stetig
\bew
Wähle $ε=1$. Angenommen es gibt $δ>0$ mit $|x-y|<δ\Rarr |f(x)-f(y)<1$
Wähle: $x=\frac{1}{n},\ y\frac{1}{n+1}$ so dass $\frac{1}{n·(n+1)}<δ$
$$|x-y|=|\frac{1}{n}-\frac{1}{n+1}|=|\frac{n+1-n}{n·(n+1)}|=\frac{1}{n·(n+1)}$$
Dann $$|f(x)-f(y)|=|n-(n+1)|=1$$
Das Zeigt: $δ$ existiert nicht.
%Satz und beweis
%
% christopher
%
\chapter{Komplexe Zahlen und Trigonometrie}
Der Körper \C{} der komplexen Zahlen\\*
Mangel von \R{}: Die Gleichung $x^2=-1$ hat keine Lösung\\*
\sS{Definition}
Es sei $\C=\R^2=\{(x,y)|x,y\eR\}$ mit folgender Addition und Multiplikation:
$$(x,y)+(x',y'):=(x+x',y+y')$$
$$(x,y)·(x',y'):=(x·x'-y·y',x·y'-y·x')$$
Addition gleich der Vektoraddition in \R^2 GRAPH
\sS{Satz}
\C{} ist ein Körper mit Null (0,0) und Eins (1,0)
\bew
Überprüfe Körperaxiome
\ssss{Exemplarisch:}
\begin{enumerate}
\item{Assotiativgesetz der Multiplikation\\*
Gegeben $(x,y),\ (x',y'),\ (x'',y'')\eC$
$$((x,y)·(x',y'))·(x'',y'')=(x·x'-y·y',x·y'-x·y'+y·x')(x'',y'')=(x·x'·x''-y·y'·x''-x·y'·y''-y·x'·y'',x·x'·y''-y·y'·y''+x·y'·x''+y·x'·x'')$$
$$(x,y)((x',y')(x'',y'')=…\text{ erhalte gleiches Ergebnis}$$}
%christopher \item
\end{enumerate}
\sss{Definition:}
$i:=(0,1)$ (imaginäre Einheit)\\*
Dann ist $$i^2=(0,1)·(0,1)=(-1,0) \Rarr i^2+1=0$$
\bem
Für $x,x'\eR$ gilt:
$$(x,0)+(x',0)=(x+x',0)$$
$$(x,0)·(x',0)=(x·x',0)$$
Die Abbildung $\R→\C,\ x\mapsto(x,0)$ ist injektiv und verträglich mit $+,·$\\*
$\leadsto$ Fasse \R{} mittels diese Abbildung als Teilmenge von \C{} auf, einschließlich der Körperstruktur\\
Dann $i^2=-1$.\\*
Für
$$(x,y)\eC$ gilt $(x,y)=(x,0)+(0,y)=(x,0)+(0,1)·(y,0)=x+i·y$$
Jede komplexe Zahl $z$ hat eine eindeutige Darstellung $z=x+i·y$ mit $x,y\eR$.\\
\sss{Idee}
\C{} entsteht aus \R{} durch Hinzunahme einer Zahl $i$ mit $i^2=-1$\\
Interpretation der Multiplikation in \C:\\*
$$(x+i·y)(x'+i·y')=(x·x'+x·i·y'+i·y·x'+i·y·i·y')=(x·x'-y·y')+i(x·y'+y·x')$$
%Definition*
\bem
\begin{enumerate}
\item{$$z·\={z}=(x+i·y)(x-i·y)=x^2+y^2+i(-x·y+y·x)=x^2+y^2=|z|^2$$}
\item{Insbesondere gilt $z·\={z}\eR$ und $z·\={z}\geq 0$}
\item{$|z|=\sqrt{x^2+y^2=}=\sqrt{z·\={z}}$ (sinnvoll wegen 2)}
\end{enumerate}
\newpage
\end{document}
