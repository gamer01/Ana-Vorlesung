\documentclass[a4paper,10pt]{scrreprt}
\usepackage{amsmath}	% Mathebibliothek 2
\usepackage{amssymb}	% Mathebibliothek 2
\usepackage{breqn}
\usepackage{fontspec}
\usepackage{polyglossia}
\usepackage[math-style=TeX]{unicode-math}	% Unicode Unterstützung
\setmainfont{FreeSerif.otf}
%\setmathfont{Asana-Math.otf}
\setmathfont{xits-math.otf}
\usepackage{calligra}	% Schrift
\usepackage{cancel}		% Kürzen und Durchstreichen
\usepackage{soulutf8}	% Für schöne Unterstreichungen
\usepackage{tikz} 		% Für Graphenzeichnerei
\usepackage{wrapfig}
\usepackage{pgf}
\usepackage{pbox}
\usepackage[hyperindex,backref,hidelinks]{hyperref}
\usetikzlibrary{calc}

% Language
\setdefaultlanguage{german}

% Definitionen fürs floaten
\fboxsep .5pt
\fboxrule .5pt

% set underline
\setul{.3mm}{.8pt}

% Schusterjungen und Hurenkinder
\clubpenalty = 10000
\widowpenalty = 10000
\displaywidowpenalty = 1000

% Beschreibaren Seitenbereich definieren
\usepackage[left=2.5cm,right=2.5cm,top=2cm,bottom=2.5cm]{geometry}

% Kopf- und Fußzeile einfügen
\usepackage{fancyhdr}
\pagestyle{fancy}
\setlength{\headheight}{25pt}

% Kopfzeile Linie oben
\renewcommand{\headrulewidth}{0.5pt}

% Fußzeile mittig
\fancyfoot[C]{\thepage}

% Linie unten
\renewcommand{\footrulewidth}{0.5pt}

% Kopfzeilen fürs Inhaltsverzeichnis:
% Kopfzeile links bzw. innen
\fancyhead[L]{\calligra\Large Inhaltsverzeichnis}
% Kopfzeile rechts bzw. außen
\fancyhead[R]{\calligra\Large{}}

% ***********************
% *** costum commands ***
% ***********************

% kürzerer command um neue mathemathische commands (ohne parameter) zu erstellen
\newcommand{\mcmd}[2]{\newcommand{ #1 }{\text{$#2$}}} 
% kürzerer command um neue temxt commands (ohne parameter) zu erstellen
\newcommand{\tcmd}[2]{\newcommand{ #1 }{\text{#2}}} 
% kürzerer command um neue commands (ohne parameter, ohne environment) zu erstellen
\newcommand{\cmd}{\newcommand}


% Mengen
\mcmd{\N}{\mathbb{N}}
\mcmd{\Z}{\mathbb{Z}}
\mcmd{\Q}{\mathbb{Q}}
\mcmd{\R}{\mathbb{R}}
\renewcommand{\C}{\text{$\mathbb{C}$}}

\mcmd{\nN}{n\in\N}
\mcmd{\eN}{\in\N}
\mcmd{\eZ}{\in\Z}
\mcmd{\eQ}{\in\Q}
\mcmd{\eR}{\in\R}
\mcmd{\eC}{\in\C}

% griechische Buchstaben
\mcmd{\z}{\zeta}
\mcmd{\e}{\epsilon}	
\mcmd{\de}{\delta}	

% Pfeile
\mcmd{\equ}{\Leftrightarrow}
\mcmd{\Rarr}{\Rightarrow}
\mcmd{\Larr}{\Leftarrow}

% Symbole
\cmd{\qed}{\hfill\text{$\blacksquare$}}
\mcmd{\bs}{\backslash}
\mcmd{\ba}{\bs}

% häufige mathemathische Ausdrücke
\mcmd{\nif}{n→∞}
\mcmd{\an}{(a_n)}
\cmd{\til}[1]{\widetilde{#1}}

% Formelkürzel
\cmd{\bino}[2]{\text{$\ds\binom{#1}{#2} = \frac{#1!}{#2!\cdot(#1-#2)!}$}}

% Überschriften
\cmd{\sS}[1]{\section{#1}}
\cmd{\Def}{\sS{Definition:}}
\cmd{\Satz}{\sS{Satz:}}
\cmd{\uS}[1]{\section*{\ul{#1}}}
\cmd{\Wdh}[1]{\section*{Wiederholung #1}}
\cmd{\wdh}{\Wdh{}}
\cmd{\sss}[1]{\subsection*{\ul{#1}}}
\cmd{\bsp}{\sss{Beispiel:}}
\cmd{\Bsp}[1]{\subsection*{\ul{Beispiel:} #1}}
\cmd{\bem}{\sss{Bemerkung:}}
\cmd{\Bem}[1]{\subsection*{\ul{Bemerkung} #1:}}
\cmd{\beh}{\sss{Behauptung:}}
\cmd{\bew}{\sss{Beweis:}}
\cmd{\Bew}[1]{\subsection*{\ul{Beweis} #1:}}
\cmd{\anm}{\sss{Anmerkung:}}
\cmd{\ssss}[1]{{\bf\ul{#1}}}

% Textkürzel
\cmd{\ok}{\hfill{\checkmark}}

% Environments
\cmd{\ds}{\displaystyle}
\cmd{\Sum}{\ds\sum}
\cmd{\Lim}{\ds\lim}
\cmd{\Int}{\ds\int}

% Deprecated %
\cmd{\einruck}[2]{#1\vspace{-3.65ex}\begin{addmargin}{.05\textwidth}#2\end{addmargin}}

\cmd{\desc}[2]{\begin{description}\item[\textrm{#1}]{\hfill\\*#2}\end{description}}
\cmd{\ind}[2]{\desc{IA:}{$n=0$\\*#1\ok}\desc{IS:}{$n\to n+1$\\*#2\qed}}
\cmd{\indIV}[3]{\desc{IA:}{$n=0$\\*#1\ok}\desc{IV:}{#2}\desc{IS:}{$n\to n+1$\\*#3\qed}}
\cmd{\notat}[1]{\begin{description}\item[\textsf{\em Notation:}]{\hfill\\*\textsf{\em #1}}\end{description}}
\cmd{\ol}[1]{\text{$\overline{#1}$}}
\cmd{\ary}[1]{\begin{array}{c}#1\end{array}}
\cmd{\alg}[1]{\begin{align*}#1\end{align*}}
\cmd{\case}[1]{\begin{cases}#1\end{cases}}
\cmd{\enum}[1]{\begin{enumerate}#1\end{enumerate}}
\cmd{\itm}[1]{\begin{itemize}#1\end{itemize}}

% Author, title…
\title{Analysis Vorlesung}
\author{Stefan Heid, Christopher Jordan}
\date{\today}

% Less detailed TOC
\setcounter{tocdepth}{1}

\begin{document}
\maketitle
\tableofcontents\newpage
% Kopfzeile beim Kapitelanfang:
\fancypagestyle{plain}{
%Kopfzeile links bzw. innen
\fancyhead[L]{\calligra\Large Vorlesung Nr. 1}
%Kopfzeile rechts bzw. außen
\fancyhead[R]{\calligra\Large 8.10.2012}}
%Kopfzeile links bzw. innen
\fancyhead[L]{\calligra\Large Vorlesung Nr. 1}
%Kopfzeile rechts bzw. außen
\fancyhead[R]{\calligra\Large 8.10.2012}
% **************************************************

\chapter{Mengen}
\sS{Definition Mengen}
\begin{enumerate}
\item{Eine Menge ist eine Ansammlung verschiedener Objekte}
\item{Die Objekte in einer Menge heißen \ul{Elemente}\\*
%
\notat{
$a \in M$ heißt $a$ ist Element der Menge $M$\\*
$a\not\in M$ heißt $a$ ist kein Element der Menge $M$}}
%
\item{Sei $M$ eine Menge. Eine Menge $U$ heißt Teilmenge von $M$, von der jedes Element von $U$ auch Element von $M$ ist\\*
%
\notat{
$U\subseteq M$ heißt $U$ ist Teilmenge von $M$\\*
$U \not\subseteq M$ heißt $U$ ist keine Teilmenge von $M$}}
\end{enumerate}

\sS{Beispiele}
\begin{enumerate}
\item{\begin{description}
\item[Sei]{$M$ die Menge aller Studierenden in L1\\*
$W$  die Menge aller weiblichen Studierenden in L1\\*
$F$ die Menge aller Frauen}
\end{description}
Dann gilt: $W \subseteq M,\ W \subseteq F,\ M \not\subseteq F,\ F\not\subseteq M$}
\item {Die Menge der natürlichen Zahlen
$\N = \{1,2,3,4 …\}$
$G$ sei die Menge der geraden natürlichen Zahlen
$$G := \{\nN \mid n\text{ ist gerade}\} = \{2m \mid m \eN\} = \{2,4,6,8 …\}$$
Es gilt $G \subseteq \N,\ \N \subseteq G$}
\item {Die Menge der ganzen Zahlen
$$\Z = \{0,1,-1,2,-2,3,-3, …\}$$}
\item {Die Menge der rationalen Zahlen
$$\Q{} = \{a/b\mid a, b \in \Z{}, b \neq 0\}$$}
\item {Die Menge ohne Element heißt die leere Menge
Symbol: $\emptyset = \{\}$}
\end{enumerate}
%
\bem
\begin{enumerate}
\item Für jede Menge $M$ gilt $\setminus \subseteq M$
\item $\N \subseteq \Z \subseteq \Q$
\end{enumerate}

\sS{Definition Mengenoperatoren}
Sei $M$ eine Menge und $U,V \subseteq M$ Teilmengen
\begin{enumerate}
\item Die Vereinbarung von $U$ und $V$ ist $U \cup V := \{x \in M \mid x \in U$ oder $x \in V\}$
\item Der Durchschnitt von $U$ und $V$ ist $U \cap V := \{x \in M \mid x \in U$ oder $x \in V\}$\\*
$U$ und $V$ heißen disjunkt, wenn $U \cap V = \emptyset$
\item Die Differenzmenge von $U$ und $V$ ist $U \setminus V := \{x \in U \mid x \in V\}$
\item Das Komplement von $U$ ist $U^C = M \setminus U = \{x \in M \mid x \notin U\}$\\*
%
\bsp
Sei $M = N$
$$\{1,3\} \cup \{3,5\} = \{1,3,5\}$$
$$\{1,3\} \cap \{3,5\} = \{3\}$$
$$\{1,3\} \cap \{2,4,7\} = \emptyset \leftarrow\text{ disjunkt}$$
$$\{1,2,3\} \setminus \{3,4,5\} = \{1,2\}$$
$$\{1,3,5\}^C = \{2,4,6,7,8,…\}$$
\end{enumerate}

\sS{Satz (de Morgan'sche Regeln)}
Sei $M$ eine Menge, $U,V \subseteq M$ Teilmengen\\*
Dann:
\begin{enumerate}
\item $(U \cap V)^C = U^C \cup V^C$
\item $(U \cup V)^C = U^C \cap V^C$
\end{enumerate}
%
\bew
\begin{enumerate}
\item{Sei $x \in M$\\*
Es gilt: $x \in (U \cap V)^C \equ{} x \notin U \cap V \equ{} x \notin U$ oder $x \notin V\equ{} x \in U^C$ oder $x \in V^C \equ{} x\in U^C \cup V^C$}
\item{ Sei $x \in M$\\* Es gilt: $x \in (U \cup V)^C \equ{} x \notin U \cup V \equ{} x \notin U$ und $x \notin V\equ{} x \in U^C$ und $x \in V^C \equ{} x\in U^C \cap V^C$}
\end{enumerate}

\sS{Prinzip der Vollständigen Induktion}
Für jedes $n \in \N$ sei eine Aussage $A(n)$ gegeben\\*
Ziel: Beweisen, Dass $A(n)$ für jedes $n \in \N$ mehr ist dafür reicht es zu zeigen
\begin{enumerate}
\item{Induktionsanfang (IA): $A(1)$ ist wahr}
\item{Induktionsschrit (IS): Wenn für ein $n \in \N$ A(n) wahr ist, dann ist auch $A(n+1)$ wahr}
\end{enumerate}

\sS{Satz: Summe der Zahlen bis $n$}
Für jede natürliche Zahl $n$ gilt:
$$1+2+3+4+5+…+n=\frac{n(n+1)}{2}$$
Probe:\\*
\begin{tabular}{r|c|c|c|c}
n & 1 & 2 & 3 & 4\\* \hline\hline
1+2+3…+n & 1 & 3 & 6 & 10\\* \hline
$\ds \frac{n(n+1)}{2}$ & 1 & 3 & 6 & 10\\*
\end{tabular}
\sss{Beweis des Satzes mit Induktion}
Abkürzung: $S(n) := 1+2+3+…+n$\\*
Aussage A(n): $\ds S(n) = \frac{n(n+1)}{2}$
\begin{enumerate}
\item {Induktionsanfang (IA): $n=1\quad S(1) = 1 = \dfrac{1·2}{2}$\ok}
\item {Induktionsschritt (IS): $n → n+1$\\*
Annahme: $A(n)$ gilt: $$ S(n) = \frac{n(n+1)}{2}$$
Zu zeigen: A(n+1) gilt: $$S(n+1)=\frac{(n+1)·(n+2)}{2}$$
$$S(n+1)=S(n)+n+1=\frac{n(n+1)}{2}+\frac{2(n+1)}{2}=\frac{(n+2)(n+1)}{2}$$
Das beendet den Beweis / quod erat demonstrandum / q.e.d.} \qed
\end{enumerate}
%
Zur Vereinfachung der Notation:\\*
Seien $a_1,a_2,a_3,…,a_n$ Zahlen $n \in \N$\\*
Setze: $\sum_{k=1}^n a_k := a_1+a_2+a_3+…+a_n$\\*
\begin{tabbing}
Allgemeiner: \=Sei $l,m \in \N$, $l \le m \le n$\\*
\>$\sum_{k=l}^m a_k = a_l+a_{l+1}+…+a_m$\\*
\end{tabbing}
Aussage des Satzes:
\[ \sum_{k=1}^n k = \frac{n(n+1)}{2}\]\footnote{\ul{Kombinatorik} (mathematisches Zählen)}

\sS{Definition (Kartesisches Produkt)}
Seien $A, B$ Mengen. Das kartesische Produkt von $A$ und $B$ ist definiert als $A × B := \{(a,b)\mid |a\in A, b \in B\}$ Die Elemente von $A × B$ heißen geordnete Paare\\*
Bsp.: $\{1,7\}\times \{2,3\}=\{(1,2),(1,3),(7,2),(7,3)\}$\\*
Allgemeiner: Gegeben seien Mengen
$A_1,…,A_k$ mit $k \in \N$. Das kartesische Produkt von $A_1,…,A_k$ ist $A_1\times …\times A_k = \{(a_1,…,a_k)\mid a\in A, $für $i=1,…,k\}$\\*
Elemente von $A_1 × … × A_k$ heißen k-Tupel\\*
Falls $A_1=A_2=…=A_k=A$, schreibe $\underbrace{A×…×A}_{k-mal}=A^k$

\sS{Definition Mächtigkeit}
Eine Menge $A$ ist endlich, wenn $A$ nur endlich viele Elemente hat. Dann bezeichnet
$\#A = \{|A|\}$ die Anzahl der Elemente von $A$ und somit dessen Kardinalität
oder Mächtigkeit. Wenn $A$ nicht endlich ist, so schreibe: $\# A= \infty$\\*
Bsp.: $\#\emptyset = 0, \#\N=\infty, \# \{1,3,5\} = 3$

\sS{Bemerkung}
\begin{enumerate}
\item Sei $A$ endliche Menge. $U,V\subseteq A$ disjunkte Teilmengen\\*
Dann $\#(U\cup V)=\# U + \# V$ 
\item Seien $A_1,…,A_k$ endliche Mengen $k \in \N$\\*
Dann: $\#(A_1 \times … \times A_k)=(\#A_1)(\#A_2)…(\#A_k)$
\end{enumerate}

\sS{Definition Fakultät}
\begin{enumerate}
\item Für $n\in \N$ setze $n!=1·2·3· … · n=\prod_{k=i}^n k$
Setze $0!=1$
\item Für $k,n\in \Z$ mit $0\le k \le n$ sei $\binom{n}{k}:= \frac{n!}{k!·(n-1)!}$ \Rarr{} Binomialkoeffizient\\*
\begin{tabular}{r|c|c|c|c|c|c|c}
n & 0 & 1 & 2 & 3 & 4 & 5 & 6\\* \hline
n! & 1 & 1 & 2 & 6 & 24 & 120 & 720
\end{tabular}\\*
\bsp
$\binom{5}{2} := \frac{5!}{2!· 3!} = \frac{5· 4 ·\cancel{3 · 2 · 1}}{2· 1· \cancel{3· 2}· 1 } = \frac{20}{2}=10$\\*
Bemerkung: $\binom{n}{0}= 1 = \binom{n}{n}$
\end{enumerate}\newpage
% Kopfzeile beim Kapitelanfang:
\fancypagestyle{plain}{
%Kopfzeile links bzw. innen
\fancyhead[L]{\calligra\Large Vorlesung Nr. 2}
%Kopfzeile rechts bzw. außen
\fancyhead[R]{\calligra\Large 11.10.2012}
}
%Kopfzeile links bzw. innen
\fancyhead[L]{\calligra\Large Vorlesung Nr. 2}
%Kopfzeile rechts bzw. außen
\fancyhead[R]{\calligra\Large 11.10.2012}
% **************************************************
%
\wdh
Sei $M$ Menge.\\*
Wenn $M$ endlich: $\#M=Anzahl$ $Elemente\in M$\\*
Wenn $M$ unendlich: $\#M=\infty$\\*
Für $n\in \N:=\{1,2,3,\ldots\}$
$$n!=1 · 2 · 3 · 4 · … · n \qquad 0!=1$$
Binomialkoeffizient: Für $0\leq k\leq n$
$$\bino{n}{k} \qquad\qquad\qquad \bino{n}{0}=\bino{n}{n}=1$$

\sS{Lemma}
Für $0<k< n$ gilt:
$$\binom{n}{k} = \binom{n -1}{k-1} + \binom{n-1}{k}$$
%
\bew 
$$\binom{n-1}{k-1}+ \binom{n-1}{k}=\frac{(n-1)!}{(k-1)!·(n-k)!} +\frac{(n-1)!}{(k-1)!·(n-1-k)!} = \frac{k(n-1)!+(n-k)\cdot(n-1)!}{k! (n-k)!}=\frac{n(n-1)!}{k!(n-k)!}$$

\sS{Geometrische Anordnung (Pascalsches Dreieck)}
\parbox{0.4\textwidth}{\centering
$\binom{0}{0}$\\*
$\binom{1}{0} \binom{1}{1}$\\*
$\binom{2}{0} \binom{2}{1} \binom{2}{2}$\\*
$\binom{3}{0} \binom{3}{1} \binom{3}{2} \binom{3}{3}$}\hfill
\parbox{0.4\textwidth}{\centering
1\\*
1 1\\*
1 2 1\\*
1 3 3 1}\\*[5mm]
Folge $\binom{n}{k}\in \N$ für alle $0\leq k\leq n$

\sS{Satz: Anzahl von Teilmengen}
Sei $A$ endliche Menge. $\#A=n$\\*[4pt]
Sei $k\in\Z$ mit $0\leq k\leq n$\\*[4pt]
$P_k(A):=\{U\subseteq A\mid \#U=k\}$ (Menge aller $k$-elementigen Teilmengen von $A$)\\*[4pt]
Dann gilt $\#P_k(A)=\binom{n}{k}$
\bsp
$A=\{1,2,3,4\}$ $n=4$ $k=2$\\*[4pt]
2-elementige Teilmengen von $A$:
$\{1,2\}, \{1,3\}, \{1,4\}, \{2,3\}, \{2,4\}, \{3,4\} \to 6\qquad \binom{4}{2}=6$ \ok
%
\bew
Vorüberlegung: Sei $k=0 \vee k=n$\\*
$P_0(A)=1=\binom{n}{0}$ $\#P_n(A)=1=\binom{n}{n}$\ok\\*
Jetzt: Induktionsbeweis nach $n$
\ind{$n=0$ Dann $k=0$}{Sei $\#A=n+1 \Rarr 0 \leq k \leq (n+1)$
Falls $k = 0\vee k = n + 1$\\*
Sei also: $o < k < n + 1$\\*
Wähle $a\in A$\\*
Sei $B=A\bs\{a\}$\\*
Dann $A=B\cup\{a\}, \#B=n$\\*
Man kann die Wahl einer $k$-elementigen Teilmenge von $A$ so strukturieren:
\begin{enumerate}
\item{Entscheiden, ob $a\in U \vee a\notin U$}
\item{\begin{enumerate}
\item{Wenn $a\notin U$: Wähle $k$ Elemente aus $B$}
\item{Wenn $a\in U$: Wähle $k-1$ Elemente aus $B$}
\end{enumerate}}
\end{enumerate}
$$\Rarr\ \#P_k(A)=\#P_k(B)+\#P_{k-1} (B) \stackrel{IV}{=} \binom{n}{k} + \binom{e}{ -1} \stackrel{1.11}{=} \binom{n+1}{k}$$
}

\sS{Satz (Binomische Formel)}
Seien $a,b$ Zahlen, $n\in\N$\\*
Dann $(a+b)^n=a^n+\binom{n}{1} a^{n-1} b+\binom{n}{2}a^{n-2}b^2+…+b^n$
%
\bsp
$(a+b)^4=a^4+4a^3b+6a^2b^2+4ab^3+b^4$\\*
$(a+b)^2=a^2+2ab+b^2$
%
\bew
Schreibe $(a+b)^n=\underbrace{(a+b)(a+b)(a+b)(a+b)…(a+b)}_{n-Faktoren}$
%
\sss{Ausmultiplizieren}
Halte Terme der Form $a^{n-k}b^k$ mit $0\leq k\leq n$\\*
Häufigkeit von $a^{n-k}b^k$ = Anzahl der Möglichkeiten aus n-Faktoren $k$ mal $b$ zu wählen.\\*
Das ist $\binom{n}{k}$ (Satz 1.13)
%
\sss{Folgerung}
Setze $a=b=1\qquad a^{n-k}b^k=1$\\*
$(a+b)^n=2^n=\binom{n}{0}+\binom{n}{1}+\binom{n}{2}+…+\binom{n}{n}$
%
\bsp
$1+4+6+4+1=16=2^4$

\sS{Definition: Anordnung}
Sei $A$ endliche Menge\\*
Eine Anordnung von $A$ ist ein $n$-Tupel\\*
$(a_1,a_2,a_3,a_4,…,a_n)$ mit $a\in A$ für alle $i$ und $a_i\neq a_j$ wenn $i\neq j$
%
\bsp
Anordnung von $\{1,2,3\}=(1,2,3)(1,3,2)(2,1,3)(2,3,1)(3,1,2)(3,2,1)→6$

\sS{Satz: Anzahl von Anordnungen}
Sei $A$ endliche Menge, $\#A=n\geq 1$\\*
Dann ist die Anzahl der Anordnungen von $A$ gleich $n!$
\bew
Induktion nach $n$
\ind{n=1}{Sei $\#A=n+1$\\*
Wahl einer Anordnung von $A$ kann man so unterteilen:
\enum{
\item{Wähle 1 Element $a_1\in A$ ($n+1$ Möglichkeiten)}
\item{Wähle Anordnungen von $A\bs\{a_1\}$\\*
$\#(A\bs\{a_1\})=n$ \Rarr $n!$ Möglichkeiten bei 2\\*
Insgesamt $(n+1)·n!=(n+1)!$}}}
%
\bem
(Zusammenhang zwischen Anordnung und Teilmengen)\\*
Sei $A$ endliche Menge, $\#A=n,\ 0\leq k\leq n$\\*
Sei $(a_1,…,a_n)$ Anordnung von $A$\\*
$\leadsto$ Teilmenge $U:=\{a_1,…,a_n\}$\\*
Dann $U\subseteq A,\ \#U=k\qquad U\in P_k(A)$\\*
Jedes $U\in P_k(A)$ entsteht so, aber mehrfach:
$$\underset{\overset{\uparrow}{Anordnungen\ von\ U}}{k!}·\underset{\overset{\uparrow}{Anordnungen\ von\ A\backslash U}}{(n-k)!}-mal$$
$\#$ Anordnungen von $A=n!=\#P_k(A)·k!(n-k)!\Rarr\#P_k(A)=\frac{n!}{k!·(n-k)!)}=\binom{n}{k}$

\chapter{Die reellen Zahlen}
Was sind die reellen Zahlen?\\*
Präzise Konstruktion ist umfangreich, daher Axiomatischer Zugang\\*
Beschreibung der reellen Zahlen durch ihre Eigenschaften (Axiome):
\enum{
\item{Grundrechenarten → Körper}
\item{Ungleichungen → angeordneter Körper}
\item{Lückenlosigkeit → Vollständigkeit}}

\uS{Körper}
%2.1
\Def
Ein Körper ist eine Menge $K$ mit 2 Rechenoperationen:\\*
Addition (+) und Multiplikation (·), so dass folgende 9 Eigenschaften erfüllt sind:
\sss{Addition}
\enum{
	\item{$(a+b)+c=a+(b+c)$ für alle $a,b,c\in K$ (Assotiativgesetz)}
	\item{$a+b=b+a$ für alle $a,b\in K$ (Kommutativgesetz)}
	\item{Es gibt ein $0\in K$ so dass $0+a=a$}
	\item{Für jedes $a\in K$ gibt es ein $b\in K$ mit $a+b=0$}
	\bem
	$0\in K$ ist eindeutig
	\bew
	Wenn $0'\in K$ mit $0'+a=a$, dann $0=0'+0=0+0'=0'$\qed
	\bem
	Das $b$ in 4. ist auch eindeutig.
	\notat{$b=-a$ (Negatives von $a$)}
	\bew
	Angenommen $b'+a=0$\\*
	$b=b+0=b+(a+b')=(b+a)+b'=0+b'=b'$\qed
}
\sss{Multiplikation}
\enum{
	\setcounter{enumi}{4}
	\item{$a(b·c)=(a·b)c\qquad ∀a,b,c\in K$}
	\item{$a·b=b·a\qquad ∀a,b\in K$}
	\item{Es gibt ein $1\in K$ mit $1\neq 0$, so dass $1·a=a\qquad ∀a\in K$}
	\item{Für alle $a\in K,\ a\neq 0$, gibt es ein $b\in K$ mit $a·b=1$}
	\bem
	$1\in K$ ist eindeutig, $b$ in 8. ist eindeutig\\*
	Beziehung $b=a^{-1}$
	\bew
	Wie eben\qed
	\item{$a(a+c)=a·b+a·c\qquad ∀a,b,c\in K$ (Distributivgesetz)}
}
Weitere Bezeichnungen:\\*
$a-b:=a+(-b),\ \frac{a}{b}=a·b^{-1}$, wenn $b≠0$
\bem
Die üblichen Rechenregeln folgen aus diesen Axiomen 1.-9.
\bsp
$$-(-a)=a,\ a(b-c)=a·b-a·c,\ a(-b)=-(a·b)$$

\sS{Beispiele bekannter Körper}
\Q\ ist ein Körper\\*
\Z\ ist kein Körper (8. nicht erfüllt)

\sS{Beispiel für einen Körper}
$\mathbb{F}_z=\{0,1\}$
\sss{Definitionen von + und · }
\parbox{.2\textwidth}{\begin{tabular}{c|cc}
+&0&1\\*[2pt]\hline
0&0&1\\*[2pt]
1&1&0
\end{tabular}}
\parbox{.2\textwidth}{\begin{tabular}{c|cc}
·&0&1\\*[2pt]\hline
0&0&0\\*[2pt]
1&0&1
\end{tabular}}
$1+1=0$\\*[4pt]
\fbox{\ul{Übung} Prüfe alle Körperaxiome}
\bem
Sei $K$ \ul{endlicher} Körper\\*
Dann gilt $\#K=p^r$ wobei $p$ Primzahl, $r\in\N$\\*
Für jede solche Zahl $q=p^r$ gibt es genau einen Körper
\newpage
% Kopfzeile beim Kapitelanfang:
\fancypagestyle{plain}{
%Kopfzeile links bzw. innen
\fancyhead[L]{\calligra\Large Vorlesung Nr. 3}
%Kopfzeile rechts bzw. außen
\fancyhead[R]{\calligra\Large 15.10.2012}
}
%Kopfzeile links bzw. innen
\fancyhead[L]{\calligra {\Large Vorlesung Nr. 3}}
%Kopfzeile rechts bzw. außen
\fancyhead[R]{\calligra \Large{15.10.2012}}
% **************************************************
%
\wdh
Ein Körper $K$ ist eine Menge mit $+$ und $·$, sodass gewisse Eigenschaften erfüllt sind:
\bsp
$\ds\Q = \left\{\frac{a}{b} \mid a \in \Z, b \neq 0\right\}$\\*
$F_1 = \{0, 1\} \qquad 1 + 1 = 0$\\*
\notat{Setze $a^n = \underbrace{ a · a · a · a· … · a}_{n-Faktoren}$\\*
$\left.\begin{array}{lcc}
a^0 &=& 1\\*
a^{-n} &=& (a^{-1})^n
\end{array}\right\}$ wenn $a \neq 0$}
Daraus folgt $a^n$ ist definiert, wenn $a \neq 0$ und $n \in \Z$\\*
Regeln der Potenzgleichung:\\*
$a^{n+m} = a^n \cdot a^m$\\*
$a^{n \cdot m} = (a^{n})^m$\\*
\bew
Übung

\sS{Definition angeordneter Körper}
    Ein angeordneter Körper ist ein Körper $K$ für dessen Elemente eine "Kleiner als Beziehung" $<$ definiert ist, so dass folgende Eigenschaften erfüllt sind:\\*
    \begin{enumerate}
    \item{Für alle $a, b \in K$ gilt genau eine von drei Notationen:\\*
    $a < b$ oder $a = b$ oder $a > b$}
    \item{Für alle $a, b, c \in K$ gilt wenn $a < b$ und $b < c$ dann $a < c$\\* (Transitivität)}
    \item{Für alle $a, b, c \in K$ gilt wenn $a < b$ dann $a + c < b + c$}
    \item{für $a, b, c \in K$ gilt, wenn $a < b$ und $c \neq 0$ dann $a \cdot c < b \cdot c$}
    \end{enumerate}
	Weitere Beziehungen:\\*
	$a > b$ heißt $b < a$
	\begin{enumerate}
	\item{Wenn $a < 0$ dann $-a > 0$:\\*
	$a < 0 \Rarr a + (-a) > 0 + (-a) \Rarr 0 > -a$}
	\item{Für jedes $a \in K $ gilt wenn $a \neq 0$, dann $a^2 > 0$
	\begin{itemize}
	\item[(a)]{$\begin{array}{ccc}
	a &>& 0\\*
	a · a &>& 0 · a\\*
	a^2 &>& 0
	\end{array}$\qed}
	\item[(b)]{$\begin{array}{ccc}
		a &<& 0\\*
		-a &>& 0 · a\\*
		a^2 &=& (-a)^2 > 0
		\end{array}$\qed}
	\end{itemize}}
	\item{$1 > 0 $ denn $1 = 1^2$}
	\end{enumerate}
	Sei $K$ ein Angeordneter Körper:
	$$0 < 1 \Rarr 1 < 1 + 1 \Rarr 1 + 1 < 1 + 1 + 1 \text{ etc.}$$
	$$0 < 1 < 1 + 1 < 1 + 1 +1 \text{ etc.}$$
	Für $n \in \N$ setze $\underbrace{n:= 1 + 1 + 1 + … + 1}_{n-Faktoren}$\\*
	Dann $0 < 1 < 2 < 3 … $ in $K$\\*[4pt]
	\ul{Folge:} Verschiedene natürliche Zahlen bleiben in $K$ verschieden.\\*
	%Falscher Pfeil, verbesserung kommt noch
	Fasse $\N$ als Teilmenge von $K$ auf.\\*
	\desc{Dann}{$\ds \Z = \left\lbrace a - b \mid a, b \in N \right\rbrace \subseteq K$\\*[4pt]
	$\ds \Q = \left\lbrace \frac{a}{b} \mid a, b \in N\right\rbrace \subseteq K$}
	Insbesondere ist $K$ unendlich.\\*[8pt]
	z.B. hat $F_z$ keine Anordnung.

\sS{Definition Absolutbetrag}
	Sei $K$ ein angeordneter Körper mit $a \in K$\\*
	Der Absolutbetrag von $a$ ist definiert als \\*
	$|a| = \left\lbrace \begin{array}{rcr}
	a&\text{ wenn }&a > 0\\*
	-a&\text{ wenn }&a < 0
	\end{array}\right.$

\sS{Satz (Dreiecksungleichung)}
	Sei $K$ ein angeordneter Körper $a, b, c \in K$\\*
	Dann gilt:
	\begin{enumerate}
	\item{$a = 0$ wenn $|a| = 0$}
	\item{$-|a| \leq a \leq |a|$}
	\item{Dreiecksungleichung: $\ds |a + b| \leq |a| + |b|$}
	\item{untere Dreiecksungleichung: $\ds |a - b| \geq |a| - |b|$}
	\end{enumerate}
\bew
	\begin{enumerate}
	\item{klar.}
	\item{wenn $a \geq 0$:\\*
	$|a| \geq 0$\\*
	$\Rarr\ -|a| \leq 0 \leq a \leq |a|$\\*
	wenn $a \leq 0$:
	$-|a| \leq a \leq 0 \leq |a|$}
	\item{Es gilt: $-|a| \leq a \leq |a|$, $-|b| \leq b \leq |b|$\\*
	wenn $a + b \geq 0$\\*
	$|a + b| = a + b \leq |a| + b \leq |a| + |b|$
	wenn $a + b < 0$\\*
	$|a + b| = -(a + b) = (-a) + (-b) \leq |a| + |b|$}
	\item{$(a - b) + b = a$\\*
	$\Rarr\ |a| = |(a - b) + b| \leq |a - b| + b$\\*
	$|a - b| \leq |a - b|$} % ??? Noch mal gegenlesen, mathematisch schwierig.
	\end{enumerate}

\sS{Satz Bernoulli'sche Ungleichungen}
	Sei $K$ ein angeordneter Körper $a, b \in K, a > -1$ und $n \in \N \{0, 1, 2, 3, 4, …\}$.\\*
	Dann gilt:
	$$(1 + a)^n \geq 1 + n \cdot a$$
	Beweis durch vollständige Induktion:\\*
	\ind{
		$$n = 0\\*
		(1 + a)^0 = 1 = 1 + 0 \cdot a$$
	}{
		Annahme:\\*
		$$(1 + a)^{n + 1} = (1 + a)(1 + a)^n \geq (1 + a)(1 + n \cdot a)$$\\*
		weil $1 + a > 0$\\*
		$= 1 + a + n \cdot a + n \cdot a^2$\\*
		$= 1 + (n + 1) \cdot a + n \cdot a^2$\\*
		weil $a^2 \geq 0\ \Rarr\ n \cdot a^2 \geq 0$
	}

\sS{Definition}
	Sei $K$ ein angeordneter Körper, $M \subseteq K$ eine Teilmenge, $a \in K$.\\*
	\begin{enumerate}
	\item{$M \leq a$ bedeutet: $x \leq a$ für jedes $x \in M$}
	\item{$a$ heißt ",obere Schranke"' von $M$, wenn $M \leq a$.\\*
		$a$ heißt ",untere Schranke"' wenn $M \geq a$}
	\item{$M$ heißt nach oben beschränkt wenn $M$ eine obere Schranke hat.\\*
	Analog: nach unten beschränkt wenn $M$ eine untere Schranke hat.}
	\item{$a$ heißt Maximum von $M$, wenn $M \leq a$ \ul{und} $a \in M$. $a = max(M)$\\*
		$a$ heißt Minimum von $M$, wenn $M \geq a$ \ul{und} $a \in M$. $a = min{M}$}
	\end{enumerate}
\bew
	Sei $a, b \in M$\\*
	$M \leq a, M \leq b$\\*
	Dann $b \leq a$ und $b \leq a\ \Rarr\ a = b$ \qed
\bsp
	$K = \Q$
	\begin{enumerate}
	\item{$M = \N$\\*
	Sei $a \in \Q$\\*
	\desc{$a \leq N$}{$\equ a \leq n$ für alle $n \in N$\\*
	$\equ a \leq 1$}
	Wenn $N$ nach unten beschränkt $1 = min(N)$}
	\item{$M = \left\lbrace -\frac{1}{n} | n \in \N \right\rbrace \qquad 0 \notin M$
	\begin{tabbing}
	$-1 = min(M)$ \= $\Rarr\ M$ ist nach unten beschränkt.\\*
	$M \leq 0$	\> $\Rarr\ M$ ist nach oben beschränkt.\\*
	\end{tabbing}
	$M$ hat kein Maximum.\\*
	Sei $a \in M$ dann $a = -\frac{1}{n}, n \in N, -\frac{1}{n + 1} \in M$\\*
	$n + 1 > n\ \Rarr\ \frac{1}{n + 1} < \frac{1}{n}\ \Rarr\ -\frac{1}{n + 1} > -\frac{1}{n}$\\*
	$M \nleq -\frac{1}{n}$ $a$ ist keine obere Schranke.}
	\item{$M = \left\lbrace -\frac{1}{n} \mid n \in \N\right\rbrace \cup \{ 0 \}$\\*
	$min(M) = -1$\\*
	$max(M) = 0$}
	\item{$M = \emptyset$ hat weder ein $min(M)$ noch ein $max(M)$\\*
	Jedes $a \in \Q$ erfüllt $a \leq M$ und $M \leq a$}
	\end{enumerate}

\sS{Satz}
	\begin{enumerate}
	\item{Sei $K$ ein angeordneter Körper.\\*
	Wenn $M$ endlich und nicht leer, dann hat $M$ auch ein $max$ und ein $min$}
	\item{Wohlordnungsprinzip\\*
	Jede nicht leere Teilmenge $M \in \N$ hat ein Minimum.}
	\end{enumerate}
\bew
	\begin{enumerate}
	\item{klar.}
	\item{$M$ ist nicht leer, wähle $n \in M$\\*
	$\{1, 2, 3, 4, 5, … n\}$, endlich aber nicht leer.\\*
	Dann $min(\{1, 2, 3, 4, 5, … n\} \cap M) = min(M)$ \qed}
	\end{enumerate}

\sS{Definition Vollständigkeit}
	Sei $K$ ein angeordneter Körper und $M \subseteq K$, $a \in K$\\*
	$a \ heißt kleinste obere Schranke von M oder Supremum.$
	\begin{enumerate}
	\item{$M \leq a$}\\*[6pt]
	und
	\item{kein $b \in K$ mit $b < a$ erfüllt $M \leq b$}
	\end{enumerate}
	$a$ ist größte untere Schranke oder Infimum vom $M$, wenn
	\begin{enumerate}
	\item{$a \leq M$}\\*[6pt]
	und
	\item{Kein $b \in M$ mit $a < b$ erfüllt $b \leq M$}
	\end{enumerate}
\notat{$a = sup(M)$\\*$a = inf(M)$}
\bem
	Wenn $a = max(M)\ \Rarr\ a = sup(M)$ 
\bew
	Sei $a, b \in M$ und $a \nleq b$\\*
	$\Rarr\ M \nleq b\ \Rarr\ a$ ist Supremum
\bem
	Wenn ein Supremum existiert, ist es eindeutig.
\bew
	$a, b$ sind Supremum von M\\*
	$M \leq a$, $M \leq b\ \Rarr\ a \leq b $ und $b \leq a\ \Rarr\ a = b$ \qed
\bsp
	$sup(\{ -\frac{1}{n} \mid n \in \N \})$\newpage
% Kopfzeile beim Kapitelanfang:
\fancypagestyle{plain}{
%Kopfzeile links bzw. innen
\fancyhead[L]{\calligra\Large Vorlesung Nr. 4}
%Kopfzeile rechts bzw. außen
\fancyhead[R]{\calligra\Large 18.10.2012}
}
%Kopfzeile links bzw. innen
\fancyhead[L]{\calligra\Large Vorlesung Nr. 4}
%Kopfzeile rechts bzw. außen
\fancyhead[R]{\calligra\Large 18.10.2012}
% *****************************************
\wdh
Angeordneter Körper:\\*
Menge $K$ mit $+, ·, <$\\*
so dass gewisse Eigenschaften erfüllt sind
\bsp
\Q{} sind ein angeordneter Körper\\*
Sei $K$ angeordneter Körper, $M\subseteq K$ Teilmenge $a\in K$ ist obere Schranke von $M$, wenn $U\subseteq a$, d.h.: $x\leq a\qquad ∀x\in M$\\*
$a\in K$ ist kleinste obere Schranke, wenn\\*
\begin{enumerate}
\item{$M\leq a$}
\item{Wenn $b < a$, dann \ul{nicht} $M\leq b$}
\end{enumerate}
\vspace*{-9.5ex}\hspace*{15.5em}
$\left.
\begin{array}{l}
{}\vspace*{2ex}\\*{}
\end{array}
\right\}$
\vspace*{-5ex}Bezeichnung $a=sup(M)$
\vspace*{5ex}
%
\bsp
$K=\Q\qquad M=\{-\frac{1}{n}|n\in\N\}=\{-1,-\frac{1}{2},-\frac{1}{3},?\}$
\sss{Behauptung}
$sup(M)=0$
\bew
\begin{enumerate}
\item {Zeige: $M \leq 0$, d.h.: $\frac{1}{n}<0$ für alle $n\in\N$\ok}
\item {Wenn $b=\Q,\ b<0$, dann nicht $M\leq b$}
\end{enumerate}
Schreibe $b=\frac{m}{n},\ m\in\Z, n\in\N$\\*[1ex]
$b<0$ heißt $m<0,\ m\leq -1$\\*[1ex]
$b=\frac{m}{n} \leq \frac{-1}{n} \leq \frac{-1}{n+1}\in M$\\*[1ex]
\Rarr{} $M\not\leq b$ (nicht $M\leq b$)\qed

\uS{Vollständigkeit}
\sS{Definition Vollständigkeit}
Ein angeordneter Körper $K$ heißt Dedekind-vollständig, wenn jede nach oben beschränkte Teilmenge von $K$ eine kleinste obere Schranke hat (die Element $K$ ist).

\sS{Satz}
Es gibt genau einen Dedekind-vollständigen, angeordneten Körper $K$\\*
Dieser heißt Körper der reellen Zahlen\\*
\ul{Bezeichnung} \R\\*
(Beweis ausgelassen)

\sS{Satz}
Die Teilmenge \N{} von \R{} ist unbeschränkt
\bew
(verwende nur die Axiome)\\*
Indirekter Beweis: Angenommen, \N{} ist beschränkt\\*
\desc{Vollständigkeit}{\N{} hat eine kleinste obere Schranke $a\in\R$\\*
Es gilt $a-1<a \Rarr{} a-1$ ist kleinste obere Schranke von $\N\ n\leq a\qquad ∀ n\in\N\\*
\Rarr{} n+1\leq a\qquad ∀ n\in\N\\*
\Rarr{} n\leq a-1\qquad ∀n\in\N$ Widerspruch!\\*
Also Annahme falsch, d.h. \N{} ist unbeschränkt\qed}
\begin{tabular}{lcl}
beschränkt &=& nach oben beschränkt und nach unten beschränkt\\*
unbeschränkt &=& nicht nach oben beschränkt oder nicht nach unten beschränkt
\end{tabular}
%
\sS{Folgerung (Prinzip des Archimedes)}
Seien $x,y\in\R,\ x>0$, Dann gibt es $n\in\N$ mit $n·x>y$\\*
SKIZZE % SKIZZE
%
\bew
$n·x>y\ \equ\ n>\frac{y}{x}$ (weil $x>0$)\\*
\N{} unbeschränkt und nicht nach oben beschränkt \Rarr{} $\frac{y}{x}$ ist keine obere Schranke von \N\\*
\Rarr{} es gibt $n\in\N$ mit $n>\frac{y}{x}$\qed

\sS{Folgerung}
Sei $x\in\R,\ x>0$, dann gibt es $n\in\N$ mit $\frac{1}{n}<x$\\*
SKIZZE % SKIZZE
\bew
$\frac{1}{n}<x\ \equ\ 1<n·x\ \equ\ \frac{1}{x}<n$ (weil $x$ positiv)\\*
$\frac{1}{x}$ keine obere Schranke von \N{} \Rarr{} es gibt \nN{} mit $\frac{1}{x}<n$\qed

\sS{Satz}
Seien $x,y\eR$ mit $x<y$\\*
Dann gibt es $a\eQ$ mit $x<a<y$, man sagt \Q{} liegen dicht in \R\\*
SKIZZE % SKIZZE
\bew
$y-x>0$ Wähle \nN{} mit $\frac{1}{n}<y-x$\\*
Ansatz: $a=\frac{m}{n}$ mit $m\eZ$\\*
Sei $M:=\{m\eZ|x<\frac{m}{n}\}=\{m\eZ|nx<m\}$\\*
$M$ ist nach unten beschränkt und nicht leer (wegen Archimedes)\\*
$M$ hat Minimum\\*
Sei $m=min(M)$\\*
$m\in M \Rarr x<\frac{m}{n}$\\*
$m-1\not\in M \Rarr x\geq\frac{m-1}{n}$\\*
$y-\frac{m}{n} =y-x+x-\frac{m}{n}>\frac{1}{n}+x-\frac{m}{n}=x-\frac{m-1}{n}\geq0$\\*
$y>\frac{m}{n}$\qed

\uS{Wurzeln}
\sS{Satz}
Es gibt kein $a\eQ$ mit $a^2=2$\\*
\bew
Angenommen $a\frac{m}{n}\eQ,\ a^2=2,\ m,\nN$\\*
Kürze den Bruch $\Rarr \frac{m}{n}$ teilerfremd\\*
$$a^2=2\ \Rarr\ \frac{m^2}{n^2}=2 \ \Rarr\ m^2=2n^2 \ \Rarr\ m^2 \text{ gerade } \ \Rarr\ m \text{ gerade } \ \Rarr\ m=2q,\ q\eN$$
$$(2q)^2=2n^2 \ \Rarr\ 4q^2=2n^2 \ \Rarr\ 2q^2=n^2 \ \Rarr\  n^2 \text{ gerade } \Rarr n \text{ gerade }$$
Widerspruch zur Annahme $m,n$ teilerfremd\qed\\*
SKIZZE WURZEL 2 \Rarr\ $\sqrt{2}$ sollte existieren % SKIZZE
\bem
Wenn \nN, keine Quadratzahl, dann gibt es kein $a\eQ$ mit $a^2=n$ (ähnlicher Beweis)

\sS{Satz}
Sei $x\eR, x\geq 0, \nN$\\*
Dann gibt es \ul{genau ein} $y\eR, x\geq 0$ mit $y^n=x$\\*
Bezeichnung: $x=\sqrt[n]{y}$
\bew
später\\*
\ul{Ansatz} $sup\{a\eQ\mid a^n\leq x\}=:y$ (sup existiert weil \R{} Dedekind-vollständig)

\sS{Definition Potentzrechnung}
Sei $x\eR,\ x>0\qquad \frac{m}{n}\eQ$\\*
\nN, $m\eZ\qquad x^{\frac{m}{n}}=\sqrt[n]{x^m}\qquad x^{\frac{1}{n}}=\sqrt[n]{x}$
\sss{Potenzrechnung}
$\ds x^{(a+b)}=x^a·x^b,\ x^{a·b}=(x^a)^b$\hfill für $x\eR,\ x>0,\ a,b\eQ$
\bem
Später wir definiert: $x^a$ für $x\eR,\ x>0,\ a\eR$

\chapter{Folgen und Reihen reeller Zahlen}
Grundbegriff der Analysis: Konvergenz
\bsp
Wenn \nN{} immer größer wird, geht $\frac{1}{n}$ immer näher an Null.\\*
Sei $\N_0=\{0,1,2,3,4…\}$

\sS{Definition Folge}
Eine Folge reeller Zahlen ist eine Abbildung $\N_0→\R$ d.h. jeder natürliche Zahl $n\geq 0$ wird eine reelle Zahl $a_n$ zugeordnet.\\*
\notat{$(a_n)_{\nN_0}$ oder $(a_n)_{\N\geq 0}$ oder $(a_0,a_1,a_2,a_3,…)$}
\sss{Variante}
Folgen, die bei $k\in\Z$ anfangen: $(a_n)_{\N\geq 0}=(a_k,a_{k+1},a_{k+2},…)$
\bsp
\begin{enumerate}
\item{konstante Folge: $a_n=a,\ a\eR$ fest: $(a,a,a,a,a,…)$}
\item{$a_n=\frac{1}{n}$ für $n\geq 1$\\*
$(a_n)_{n\geq 1}=(1,\frac{1}{2},\frac{1}{3},…)$}
\item{$a_n=(-1)^n\qquad n>0$\\*
$(1,-1,1,-1,1,-1,1,-1,1,-1,…$}
\item{$(\frac{n}{n+1})_{n\geq 0}n=(0,\frac{1}{2},\frac{2}{3},\frac{3}{4},\frac{4}{5},…$}
\end{enumerate}

\sS{Definition Konvergenz}
Sei $(a_n)_{n\geq 0}$ eine Folge reeller Zahlen
\begin{enumerate}
\item{Eine Folge $(a_n)$ konvergiert gegen $a\eR$ wenn gilt: Für jedes $\e>0$ gibt es ein $N\eN$, so dass $|a_n-a|<\e$ für jedes \nN{} mit $n\geq N$\\*
Dann heißt $a$ Grenzwert der Folge $(a_n)$\\*
\notat{$\ds\lim_{n→∞}a_n=a$ oder $a_n→a$ für $n→∞$}}
\item{Die Folge $(a_n)$ heißt \ul{Nullfolge}, wenn $a_n→a$ für $n→∞$}
\item{Die Folge $(a_n)$ ist divergent, wenn sie keinen Grenzwert hat.}
\end{enumerate}
%
\bsp
\begin{enumerate}
\item{$a_n=\frac{1}{n}$ für $n\geq 1$
\sss{Behauptung} $a_n→a$ für $n→∞$ SKIZZE % SKIZZE
\bew
Sei $\e>0$ wähle $N=0$ für $n\geq N$ gilt $|a_k-a|=0<\e$\qed}
\item{$a_n=(-1)^n=(1,-1,1,-1,1,-1,…)$ SKIZZE % SKIZZE
\sss{Behauptung} $(a_n)$ ist divergent. 
\bew
Angenommen, $a\eR$ ist Grenzwert der Folge.\\*
Wähle $\e=1$. Es gibt $N\eN$ mit $|a_n-a|<1$ für alle $n\geq N$\\*
Wenn $n$ gerade: $a_n=1\qquad |1-a|<1$
Wenn $n$ ungerade: $a_n=-1\qquad |-1-a|<1 \Rarr |1+a|<1$\\*
$2=|2|=|1-a+1+a|\leq |1-a|+|1+a|<2 \Rarr 2<2$\\*
Widerspruch: Also ist $(a_n)$ divergent\qed}
\end{enumerate}\newpage
% Kopfzeile beim Kapitelanfang:
\fancypagestyle{plain}{
%Kopfzeile links bzw. innen
\fancyhead[L]{\calligra\Large Vorlesung Nr. 5}
%Kopfzeile rechts bzw. außen
\fancyhead[R]{\calligra\Large 22.10.2012}
}
%Kopfzeile links bzw. innen
\fancyhead[L]{\calligra\Large Vorlesung Nr. 5}
%Kopfzeile rechts bzw. außen
\fancyhead[R]{\calligra\Large 22.10.2012}
% *****************************************
%
\wdh
Eine Folge $(a_n)_{n\eN_0}$ reeller Zahlen konvergiert genen $a\eR$ wenn gilt:\\
Für jedes $\e>0$ gibt es ein $N\eN$ so dass $|a_1·a|<\e$ für alle $n\geq\N$.\\
%bez oder so was, nicht ganz klar zu lesen
\ul{Bez"uglich} $a_n→a$ für $n→∞$ oder $\ds\lim_{n→∞}(a_n)=a$\\
\bsp
$\frac{1}{n}→0$ für $n→∞\qquad(-1)^n$ divergiert\\
$(a_n)$ ist divergent, wenn sie gegen kein $a\eR$ konvergiert
\bsp
$(1,0,\frac{1}{2},0,\frac{1}{3},0,\frac{1}{4},0,…)$ konvergiert gegen $0$
%
\sS{Satz: (Eindeutigkeit der Grenzwerte)}
Sei $(a_n)$ Folge reeller Zahlen und $a,b\eR$ mit $a_n→a$ und $a_n→b$ für $n→∞$. Dann ist $a·b$ %mut zur lücke, würde ich behaupten
\bem
Dann ist bez %wie oben, whatever it means
$a=\ds\lim_{n→∞}(a_n)$ sinnvoll
\bew
Angenommen $a\neq b$\\
Sei $\e:=\dfrac{|a-b|}{2}$ SKIZZE\\ % SKIZZE
Konvergenz: es gibt $N_1\eN$ mit $|a_n-a|<\e$, $N_2\eN$ mit $|a_n-b|<\e$ für $n\geq N_2$\\
Sei $n=max(N_1,N_2)$\\
$|a-b|=|a-a_n+a_n-b|\leq|a-a_n|+|a_n-b|<\e+\e=|a-b|$\\
\Rarr $|a+b|<|a-b|$ Widerspruch\\
\Rarr nicht $a\neq b$, d.h. $a=b$\qed
%
\sS{Definition}
Eine Folge $(a_n)$ reeller Zahlen heißt $\left\{\begin{array}{c}\text{nach oben beschränkt}\\\text{nach unten beschränkt}\\\text{beschränkt}
\end{array}\right\}$ wenn die menge $\{a_n|n\eN_0\}$ dieselbe Eigenschaft hat.
%
\sS{Satz:}
Jede konvergente Folge reeller Zahlen ist beschränkt.
\bew
Angenommen $a_n→a$ für \nif\\
Wähle $\e=1$, Es gibt $N\eN$ so dass $|a_n-a|<1$ für $n\geq N$\\
Sei $C:=max\{|a_0|,|a_1|,…,|a_{n-1}|,|a|+1\}$\\
Dann $|a_n|\leq C$ für $n\leq N-1$\\
Für $n\geq N$ gilt:
$$|a_n|=|a_n-a+a|\leq|a_n-a|+|a|<1+|a|\leq C$$
Somit $|a_n|\leq C$ für alle $n$\\
$-C\leq a_n\leq C$ für alle $n$\\
\Rarr Folge $(a_n)$ ist beschränkt.
\bem
Nicht jede beschränkte Folge konvergiert.\\
z.B. $((-1)^n)_{n\eN_0}$ ist beschränkt, aber konvergiert nicht.
%
\sS{Definition}
Eine Folge reeller Zahlen $(a_n)$ konvergiert uneigentlich gegen ∞ wenn gilt:\\
Für jedes $C\eR$ gibt es $N\eR$ mit $a_m>C$ für alle $n\geq N$ SKIZZE
\bem
Alternative Terminologie:\\
"konvergiert uneigentlich"="divergiert bestimmt"
\bsp
\begin{enumerate}
\item{$a_n=n.\ a_n→∞$}
\item{$a_n=(-1)^n.\ (0,-1,2,-3,4,-5,…)$ konvergiert \ul{nicht} uneigentlich gegen ∞}
\end{enumerate}
\notat{$a_n→∞$ für \nif{} $\ds\lim_\nif\an=∞$}
%
\sS{Satz (Potenzwachstum)}
Sei $x\eR$ betrachtete % sehr unsicher, steht nur "Betr." dort
Folge $(x^n)_n\geq 0$
\begin{enumerate}
\item{wenn $|x|>1$ dann ist $(x^n)$ divergent}
\item{wenn $x>1$ dann $x^n→∞$ für \nif}
\item{wenn $|x|<1$ dann ist $x^n→0$  für \nif}
\end{enumerate}
%
% folgendes ist von der einrückunt und unterordnung sehr unsicher bitte überprüfen
%
\Bew{2)}
Sei $x>1$\\[4pt]
Schreibe $x=1+a$. Dann $a>0$ Gegeben $C\eR$\\
$\underset{\text{Satz 2.9}}{\Rarr} x^n=(1+a)^n\geq 1+n·a$\\
Archimedes: $∃ N\eN$ mit $N·a>C$\ok
\Bew{1}
Sei $|x|>1$ Dann $|x^n|=|x|^n,\ |x|>1\underset{\text{2)}}{\Rarr}|x^n|$ ist nicht beschränkt für \nN{} \Rarr{} $(x^n)$ divergiert
\Bew{3}
Sei $|x|<1$ Wenn $x=0 \Rarr x^n=0$ für alle $n$\ok\\
Sei $0<|x|<1$\\
Dann $\frac{1}{|x|}>1$\\
Gegeben sei $\e>0$\\
Setze $C=\frac{1}{\e}$\\
$\underset{\text{2)}}{\Rarr}$ es gibt $N\eN$ mit $\frac{1}{|x|^n}>C$ für $n\geq N$ \Rarr $|x|^n<\e$ für $n\geq N$\qed 
%
\sS{Satz (Rechenregeln)}
Seien $(a_n)_{\nN_0},\ (b_n)_{\nN_0}$ zwei konvergente Folgen reeller Zahlen\\
Sei $a_n→a$ für \nif\\
\phantom{Sei }$b_n→a$ für \nif\\ % EINRÜCKUNG
Dann gilt:
\begin{enumerate}
\item{$(a_n+b_n)→a+b$ für \nif}
\item{$(a_n·b_n)→a·b$ für \nif}
\item{Angenommen $b\neq 0$\\
Dann ist $b\neq 0$ für fast alle \nN{} und $\frac{1}{b_n}→\frac{1}{b}$ für \nif}
\end{enumerate}
%
\sss{Definition}
"fast alle"="alle bis auf endlich viele".
%
\bew
\begin{enumerate}
\item{Gegeben sei $\e>0$\\
Es gibt $N_1\eN$ mit $|a_n-a|<\frac{\e}{2}$ für $n\geq N_1$\\
Es gibt $N_2\eN$ mit $|b_n-b|<\frac{\e}{2}$ für $n\geq N_2$\\
Sei $N=max(N_1,N_2)$ für $n\geq N$ gilt:\\
$$|a_n+b_n-(a+b)|=|(a_n-a)+(b_n-b)|\leq |a_n-a|+|b_n-b|<\frac{\e}{2}+\frac{\e}{2}=\e \Rarr \text{1)}$$}
\item{$(a_n)$ konvergiert \Rarr{} ist beschränkt.\\
Es gibt $C\eR$ mit $|a_n|<C$ für alle $\nN_0$\\
ohne Einschränkungen sei $C>|b|$\\
Rechne:
$$|a_n·b_n-a·b|=|a_n·b_n-a_n·b+a_n·b-a·b|=|a_n(b_n-b)+b(a_n-a)|\geq |a_n|·|b_n-b|+|b|·|a_n-a|$$
Es gibt $N\eN$ mit $\left.\begin{array}{l}|a_n-a|<\frac{1}{2C}·\e\\|b_n-b|<\frac{1}{2C}·\e\end{array} \right\}$ für $n\geq N$\\
Für $n\geq N$ gilt:
$$|a_n·b_n-a·b|<|a_n|\frac{1}{2C}\e+|b|\frac{1}{2C}\e\leq C·\frac{1}{2C}\e+C·\frac{1}{2C}\e=\e \Rarr \text{ 2) gilt}$$}
\item{Sei $b\neq 0$\\
Wähle $\e=\frac{1}{2}|b|>0$ SKIZZE\\
Es gibt $N\eN$ mit $|b_n-b|<\frac{1}{2}|b|$ für $n\geq N$\\
Dann gilt für $n\geq N$:
$$|b_n|=|b_n-b+b|=|b-b+b_n|=|b-(b-b_n)|\geq |b|-|b-b_n|>|b|-\frac{1}{2}|b|=\frac{1}{2}|b|$$
Insbesondere $|b_n|\neq 0$ für $n\geq N$\\
Rechne:
$$\left|\frac{1}{b}-\frac{1}{b_n}\right|=\left|\frac{b_n-b}{b·b_n}\right|=\frac{1}{|b|·|b_n|}·|b_n-b|<\frac{2}{|b|^2}·|b_n-b| \text{ für $n\geq N$}$$\footnote{NR: $|b_n|>\frac{1}{2}|b| \Rarr \frac{1}{|b_n|}<\frac{2}{|b_n|}$}
Gegeben sei $\e>0$\\
Es gibt $N_1\eN$ mit $|b_n-b|<\frac{|b|^2}{2}\e$ für $n\geq N_1$\Rarr{} für $n\geq max(N_1,N_2)$ gilt:\\
$$|\frac{1}{b_n}-\frac{1}{b}<\frac{2}{|b|^2}·\frac{|b|^2}{2}\e=\e \Rarr \text{ 3) gilt}$$\qed}
\end{enumerate}
\ul\{Zusatz:\} Wenn \$a\_n→a\$ und \$b\_n→b\$ für \verb+\+nif\{\} dann gilt:
\begin{enumerate}
\setcounter{enumi}{3}
\item{Für $C\eR$ ist $C·a_n→C·a$ für \nif }% C groß oder klein?
\item{$(a_n-b_n)→a-b$ für \nif}
\item{Wenn $b\neq 0$ dann $\frac{a_n}{b_n}→\frac{a}{b}$ für \nif}
\end{enumerate}
\bew
Übung\newpage
% Kopfzeile beim Kapitelanfang:
\fancypagestyle{plain}{
%Kopfzeile links bzw. innen
\fancyhead[L]{\calligra\Large Vorlesung Nr. 6}
%Kopfzeile rechts bzw. außen
\fancyhead[R]{\calligra\Large 25.10.2012}
}
%Kopfzeile links bzw. innen
\fancyhead[L]{\calligra\Large Vorlesung Nr. 6}
%Kopfzeile rechts bzw. außen
\fancyhead[R]{\calligra\Large 25.10.2012}
% *****************************************
%
%\setcounter{chapter}{3}
%\setcounter{section}{9}
%
\wdh
Eine Folge reeller Zahlen $(a_n)$ konvergiert uneigentlich gegen ∞ wenn gilt:\\
Für jedes $C\eR$ gibt es ein \nN{} mit $a_n > C$ für jedes \nN\\[4pt]
$(a_n)$ konvergiert uneigentlich gegen $- ∞$ wenn $(-a_n)$ gegen $∞$ konvergiert.\\
%
\notat{
$a_n \to ∞ \qquad \text{ für } n \to ∞$\\
$a_n \to - ∞ \qquad \text{ für } n \to ∞$
}
%
\bsp
$a_n = n^2 \to ∞$\\
$a_n = -n^2 \to -∞$\\
$a_n = (-1)^n · n^2$\\
$(0, -1, 4, -9)$ konvergiert weder gegen $∞$ noch gegen $ - ∞$
%
\sss{Rechenregeln:}
Angenommen $(a_n), (b_n)$ sind konvergente Folgen.\\
\begin{enumerate}
\item{$(a_n + b_n) \to a + b$}
\item{$(a_n · b_n) \to ab$}
\item{$\ds\frac{1}{b_n} \to \frac{1}{b}$}
\item{$c · a_n \to c · a$}
\item{$a_n - b_n \to a - b$}
\item{$\ds\frac{a_n}{b_n} \to \frac{a}{b}$}
\end{enumerate}
%
\Bew{6}
3) $\Rightarrow\ds\frac{1}{b_n}→\frac{1}{b}$\\
$\ds\frac{a_n}{b_n} = a_n ·\frac{1}{b}$\\
2) $\ds\Rightarrow a_n · \frac{1}{b_n} \to a ·\frac{1}{b} = \frac{a}{b}$\qed
%
\bsp
\begin{tabular}{l|c|c|c|c|c|r}
$n$   & 0 & 1 & 2 & 3 & 10 & 100\\\hline
$a_n$ & 0 & 0 & $\frac{2}{9}$ & $\frac{6}{19}$ & $\frac{90}{201}$ & $\frac{9900}{20001}$ \\
\end{tabular}
\vspace{5mm}
Vermutung: $a_n \to \ds\frac{1}{2}$ für $n \to ∞$\\
Rechenregel 6 anwenden:\\
\begin{itemize}
\item[1.]{Versuch:\\
$a_n = \frac{b_n}{c_n}$\\[4pt]
$b_n = n^2 -n; c_n = 2n^2 + 1$\\
$(b_n) und (c_n)$ sind divergend. Schlecht.}
\item[2.]{Versuch:\\
$\ds\frac{n^2 - n}{2n^2 + 1} = \frac{n^2(1 - \frac{1}{n})}{n^2(2 + \frac{1}{n^2}}$ für $n \geq 1$\\[4pt]
$= \ds\frac{1-\frac{1}{n}}{2 +\frac{1}{n^2}}=\frac{b_n}{c_n}$ mit $b_n:=1-\frac{1}{n},\ c_n = 2 + \frac{1}{n^2}$\\[4pt]
$\ds\frac{1}{n} \to 0$ für $n \to ∞ $\\[4pt]
$\ds\Rightarrow 1 - \frac{1}{n} \to 1 - 0 = 1$ für $n \to ∞$\\[4pt]
$\Rightarrow 2 + \ds\frac{1}{n^2} \to 2 + 0 = 2$ für $n \to ∞$}
\end{itemize}
%
$\Rightarrow a_n \to \frac{1}{2}$ für $n \to ∞$
%
\sS{Satz}
Seien $a_n \to a$, $b_n \to b$ zwei konvergente Folgen reeler Zahlen.\\
wenn $a_n \leq b_n$ für unendlich viele $n \in \N{}$ dann ist $a \leq b$.
\bew
Angenommen: $a > b$\\

Wähle $\e := \ds\frac{a - b}{2} > 0$\\
Es gibt $N \in \N{}$ so dass:
$
\left.
\begin{array}{ll}
\mid a_n - a \mid  < \e \\
\mid b_n - b \mid  < \e
\end{array} \right\rbrace$ für $n \geq N$\\
$\Rightarrow a_n > a - \e$\\ \\
$= a - \frac{a - b}{2} = \ds\frac{a + b}{2} = b + \ds\frac{a - b}{2}\\
\\
= b + \e > b_n \Rightarrow a_n > b_n$ für $n \geq \N{}$\\
Widerspruch zur Annahme.\\
$a_n \leq b_n$ für unendlich viele $n \in \N$\qed

\sS{Definition: Reihen}
Sei $(a_n)_{n \geq 0}$ eine Folge reeler Zahlen.\\
Bilde eine Folge:
\begin{align*}
s_0 &= a_0\\
s_1 &= a_0 + a_1\\
s_2 &= a_0 + a_1 + a_2\\
&\vdots\\
s_n &= a_0 + a_1 + a_n = \sum\limits_{k = 0}^{n} a_k
\end{align*}
Die Folge $(s_n)_{n \geq 0}$ heißt Reihe mit den Gliedern $a_n$.\\
$s_n$ heißen die \underline{Partialsummen} der Reihe.\\
Bezeichnung:\\
$\sum\limits_{k = 0}^{∞} a_k$ oder $a_0 + a_1 + a_2 + a_3 + …$\\ \\
Wenn $s_n \to s \in \R{}$ für $n \to ∞$ dann schreiben wir:\\
$\sum\limits_{k = 0}^{∞} a_k = s$\\
Summe der Reihe.\\[4pt]
\ul{Achtung:} Symbol $\ds\sum_{k = 0}^{∞} a_k$ hat \ul{zwei} Bedeutungen:
\begin{enumerate}
\item{die Folge $(s_n)$}\\[8pt]
oder 
\item{deren Grenzwert}
\end{enumerate}
\bsp
\begin{enumerate}
\item{$\sum\limits_{k = 1}^{∞} 1 = 1+1+1+…$\\
ist die Folge $(1, 2, 3, 4,…) = (n + 1)_{n \in \N{}_{0}}$}
\item{$\sum\limits_{k = 1}^{∞} k = 0 + 1 + 2 + 3+ …$ \\
ist die Folge $(1, 3, 6, 10,…) = (\ds\frac{n(n - 1)}{2})_{n \in \N{}}$ }
\item{$\sum\limits_{k = 1}^{∞} \ds\frac{1}{k(k+1)} = \ds\frac{1}{2} + \ds\frac{1}{6} + \ds\frac{1}{12} + …$\\
ist die Folge $(\ds\frac{1}{2}, \ds\frac{2}{3}, \ds\frac{3}{4})$}
\end{enumerate}
\vspace{5mm}
Vorüberlegung:\\ %bullshit
$\ds\frac{1}{k(k+1)} = \ds\frac{(k+1) - k}{k(k+1)} = \frac{1}{k} - \ds\frac{1}{k + 1}$\\ \\
$s_n := \sum\limits_{k = 1}^{∞} \ds\frac{1}{k(k+1)}
= (\ds\frac{1}{1} - \ds\frac{1}{2}) + (\ds\frac{1}{2} - \ds\frac{1}{3}) + … + (\ds\frac{1}{n} - \ds\frac{1}{n + 1})\\ \\
= 1 - \ds\frac{1}{n + 1}\\ $ Teleskopsumme \\ \\
$\ds\frac{1}{n + 1} \to 0$ für $n \to ∞$\\ \\
Summe der Reihe:\\ \\
$\sum\limits_{k = 1}^{∞} \ds\frac{1}{k(k+1)} = \lim_{n \to ∞}(1 - \ds\frac{1}{n + 1}) = 1$\qed\\ 
\\
\bem Jede Folge kann man auch als Reihe Schreiben. (Differenzen bilden)\\
z.B.: die Folge der Primzahlen:\\
$(2, 3, 5, 7, 11, 13, 17, 19)$\\
ist die Reihe:\\
$(2 + 1 + 2+ 4+2+4+2+…)$\\
Goldbachsche Vermutung: in dieser Reihe kommt die Zahl 2 unendlich oft vor.\\
\sS{Satz, Die geometrische Reihe}
Sei $x \in \R{}$\\
a) $ \sum\limits_{k = 0}^{∞} x^k = 1 + x^1 + x^2 + x^3 + … = \frac{1}{1-x} \text{ wenn } \mid x \mid < 1$\\
b) $ \sum\limits_{k = 0}^{∞} x^k \text{ divergiert wenn } \mid x \mid \geq 1$\\
\begin{itemize}
\item[a] {wenn $|x| < 1$\\
dann folgt $\sum{k=0}{∞} a_k = \ds\lim_{n \to ∞}(\frac{1}{1 - x} - \frac{x}{1-x} · x^n) = \frac{1}{1 - x}$}
\item[b]{wenn $|x| > 1$\\
dann $(x^n)$ divergent $\Rightarrow (\frac{x}{1-x} · x^n)$ divergent\\
denn $\frac{x}{1-x} \neq 0$\\
$\Rightarrow (\frac{?}{?})$}
\end{itemize}
\bew
$x = 1 \phantom{xxx} \sum_{k = 0}^{∞} x^k = (1 + 1 + 1 +…)\text{ divergiert, ok}\\
\text{Sei nun }x \neq\\
\text{Bekannt aus der Übung: } \ds\sum_{k = 0}^{∞} x^k = 1 + x + x^2 +x^3 … +x^n = \ds\frac{1 -x^{n+1}}{1 - x} = \ds\frac{1}{1 - x} - \ds\frac{x}{1 - x} · x^n \\ $
Potenzenwachstum\\
$x^n \to 0$ für $ n \to ∞$ \underline{wenn} $|x| < 1$\\
$(x^n)$ divergiert, wenn $(|x| \geq 1 \text{ und } x \neq 1)$\\
%
\sS{Satz}
Wenn die Reihe $\ds\sum\limits_{k=0}^{∞} a_k $ kovergiert, dann ist $(a_n)_{n \in \N{}}$ eine Nullfolge.\\
\\
\bew Gegeben sei $\e > 0$\\
Sei $a = \ds\sum\limits_{k = 0}^{∞} a_k = $ $\ds\lim_{n \to ∞}(s_n)$ mit $s_n = a_0 + … + a_n$\\
Es gibt $ N \ in \N{}$ mit $|s_n - a| < \ds\frac{\e}{2}$ für $n \geq N$\\
$|a_n| = |s_n - s_{n-1}|$\\
\phantom{$|a_n| $} = $|s_n - a + a - s_{n-1}|$\\
\phantom{$|a_n| $} $\leq |s_n - a| + |a - s_{n-1}| < \ds\frac{\e}{2} + \frac{\e}{2} = \e$\\
für $n \geq N + 1$\\
$\Rightarrow a_n \to 0$ für $n \to ∞$\\
%
\sS{Satz, die harmonische Reihe}
$$\ds\sum\limits_{k = 1}^{∞} \frac{1}{k}= 1 + \frac{1}{2} + \frac{1}{3} + …\text{ divergiert}$$
\sss{Beweisidee:}
$\ds\phantom{= }1 + \frac{1}{2} + \frac{1}{3} +\frac{1}{4} +\frac{1}{5} +\frac{1}{6} +\frac{1}{7} + \frac{1}{8} +\frac{1}{9} +…$\\
$\phantom{\geq }1 + \ds\frac{1}{2} +\frac{1}{4}+\frac{1}{4}+\frac{1}{8}+\frac{1}{8} + \frac{1}{8} + \ds\frac{1}{8} + \ds\frac{1}{16} + …$\\
$\phantom{= }1 + \ds\frac{1}{2} + \ds\frac{2}{4} + \ds\frac{4}{8} + \ds\frac{8}{16} + …$\\
$\phantom{= }1 + \ds\frac{1}{2} + \ds\frac{1}{2} + \ds\frac{1}{2} + \ds\frac{1}{2} + … = ∞$\newpage
% Kopfzeile beim Kapitelanfang:
\fancypagestyle{plain}{
%Kopfzeile links bzw. innen
\fancyhead[L]{\calligra\Large Vorlesung Nr. 7}
%Kopfzeile rechts bzw. außen
\fancyhead[R]{\calligra\Large 29.10.2012}}
%Kopfzeile links bzw. innen
\fancyhead[L]{\calligra\Large Vorlesung Nr. 7}
%Kopfzeile rechts bzw. außen
\fancyhead[R]{\calligra\Large 29.10.2012}
%
% set chapters end sections
%\setcounter{chapter}3
\wdh
Sei $(n_n)$ eine Folge reeler Zahlen.\\*
Die Reihe mit den Gliedern $a_n$ ist die Folge $s_n = a_0 + a_1 + ... + a_n)_\nN$ \\*
Bezeichnung: $\ds\sum_{k=1}^{\infty} a_k$\\*
Wenn $S_n \to a$ für $n \to \infty$\\*
Schreibe: $\ds\sum_{k = 0}^{\infty} a_k = a$
\Bsp{Geometrische Reihe}
$\ds|x| = 1 \Rarr \sum\limits_{k = 0}^{\infty} x^k = \frac{1}{1-x}$ für $x = 0$ setzte $0^0 = 1$\\*[4pt]
Harmonische Reihe\\*
$\displaystyle\sum\limits_{k = 1}^{\infty} \frac{1}{k}$ Konvergiert nicht.\\*
\sS{Satz Rechenregeln für Reihen}
Seien $\sum\limits_{k = 0}^{\infty} a_k = a$ und $\sum\limits_{k = 0}^{\infty} b_k = b$ zwei konvergente Reihen. Dann:
\begin{enumerate}
\item{$\sum\limits_{k = 0}^{\infty} (a_k + b_k) = a + b$}
\item{Für $c \in \R{}$ ist $\sum\limits_{k = 0}^{\infty} c \cdot a_k = c \cdot a$}
\end{enumerate}
\bew
folgt aus 3.9.
\bem
Produkte von Reihen sind komplizierter.\\*
\ul{Korrektur:}
Primzahlen-Vermutung: es gibt ∞ viele Primzahlen $p$ so dass $p + 2$ auch Prim ist.\\*
Goldbach-Vermutung: Jede gerade natürliche Zahl ist die Summe von zwei Primzahlen.\\*
\chapter{Konvergenzsätze}
Erinnerung: \R{} ist Dedekind-vollständig. Das heißt, jede nicht-leere nach oben beschränkte Teilmenge $M \subset R$ hat eine kleinste obere Schranke $sup(M)$\ \Rarr{} Existenz von Grenzwerten

\sS{Definition Monotone Folgen}
\desc{Eine Folge $(a_n)_{n \geq 0}$ heißt}{monoton wachsend, wenn $a{n + 1} \geq a_n$ für alle $n \in \N_0$\\*
monoton fallend, wenn $a{n + 1} \leq a_n$ für alle $n \in \N{}_0$\\*
streng monoton wachsend, wenn $a{n + 1} > a_n$ für alle $n \in \N{}_0$\\*
streng monoton fallend, wenn $a{n + 1} < a_n$ für alle $n \in \N_0$}
\bsp
$a_n = n$ ist streng monoton wachsend\\*
$a_n = \frac{1}{n}$ ist streng monoton fallend\\*

\sS{Satz}
\begin{enumerate}
\item{Jede nach oben beschränkte monoton wachsende Folge $(a_n)_{\nN}$ ist konvergent\\*
% Bild?
Hier Fehlt was, das Bild, der Tafel, auf dem das stehen sollte ist nicht auffindbar, hast du da noch eine Mitschrift?
}
\item{Jede nach unten beschränkte monoton fallende Folge $(a_n)_\nN$ ist konvergent\\*
% Bild?
}
\end{enumerate}
\bew
Sei $(a_n)$ nach oben beschränkt, monoton wachsend\\*
Setze $a:= sup(\{a_n | n \in \N{}\})$\\*
dann \begin{enumerate}
\item{$a_n \leq a$ für alle $n$}
\item{Für jedes $\epsilon > 0$ ist $a - \epsilon$ \ul{keine} obere Schranke, d.h. es gibt $N \in N$ so dass $a_N > a - \epsilon$
\\*Für $n \geq N$ gilt\\*
$a - \epsilon < a_N \leq a_n \leq a$\\*
weil $(a_n)$ monoton wachsend\\*
$\Rightarrow a - \epsilon < a_N \leq a_n \leq a \Rightarrow |a_n -a| < \epsilon$\\*
Somit $a_n \to a$ für $n \to \infty$\phantom{XXX}$q.e.d.$\\*
Monoton fallend: analog}
\end{enumerate}

\uS{Reihen mit nicht-negativen Gliedern}
\bem
Sei $\ds\sum\limits_{k=0}^{\infty} a_k$ Reihe reeller Zahlen\\*
Die Folge der Partialsummen ist monoton wachsend $\Leftrightarrow a_n \geq 0$ für $n \geq 1$

\sS{Satz}
Eine Reihe $\ds\sum\limits_{k=0}^{\infty} a_k$ mit $a_k \geq 0$ für ale $k$ konvergiert, genau dann, wenn sie beschränkt ist (Das heißt die Folge der Partialsummen ist beschränkt)\qed

\sSDef
Sei $\displaystyle\sum\limits_{k=0}^{\infty} a_k$ eine Reihe mit $a_n \geq 0$ für alle $k$\\*
Eine Reihe $\ds\sum\limits_{k=0}^{\infty} b_k$ heißt \ul{Majorante} von $\ds\sum a_k$ wenn $a_k \leq b_k$ für alle $k$
\sS{Satz Majorantenkriterium}
Wenn eine Reihe mit nicht-negativen Gliedern eine konvergente Majorante hat, dann konvergiert sie.
\bew
Sei $0 \leq a_k \leq b_k$ für alle $k \geq 0$\\*
Es gilt $a_0 + ... + a_n \leq b_0 + ... + b_n$ \\*
$\sum b$ konvergiert $\Rightarrow (b_0 + ... + b_n)_{n \geq 0}$  beschränkt\\*
$\Rightarrow ((a_0 + ... + a_n)_{n \geq 0})$ beschränkt $\Rightarrow \ds\sum_{k= 0}^{\infty} a_k$ konvergiert.
\Bsp{4.6:}
$$\sum\limits_{k=1}^{\infty} \frac{1}{k^2} = \left( 1 + \frac{1}{4} + \frac{1}{9}+ \frac{1}{16} + … \right)$$
$$\sum\limits_{k=1}^{\infty} \frac{1}{k^2} = 1 + \sum\limits_{k=1}^{\infty} \frac{1}{(k + 1)^2}$$
$$\frac{1}{(k + 1)^2} \leq \frac{1}{k \cdot (k + 1)}$$
$$\Rarr \sum\limits_{k=1}^{\infty} \frac{1}{k \cdot (k + 1)}\text{ ist Majorante von }\sum\limits_{k=1}^{\infty} \frac{1}{k^2}$$
$$\sum\limits_{k=1}^{\infty} \frac{1}{k \cdot (k + 1)}\text{ konvergiert (bekannt)}$$
\sS{Satz Quotientenkriterium}
Sei $C \in \R{}, (a_n)$ eine Folge reeller Zahlen mit $a_n \geq 0$ für alle $n$ \ul{und} $a_{n + 1} \leq C \cdot a_n$ für fast alle nder$n$\\*
$0 \leq C \leq 1$\\*
Dann konvergiert die Reihe $\ds\sum\limits_{k=0}^{\infty} a_k$
\bew
Konvergenz ändert sich nicht, wenn endlich viele $a_n$ geändert werden.\\*
Also kann man annehmen, dass $a_{n + 1} \leq C \cdot a_n$ für alle $n$ gilt.\\*
Dann gilt $a_1 < C \cdot a_0$\\*
$a_2 < C \cdot a_1 \leq C \cdot C \cdot a_0 = C^2 \cdot a_0$\\*
$a_3 < C \cdot a_2 \leq C \cdot C \cdot a_1 = C^3 \cdot a_0$\\*
etc. $\Rightarrow a_n \leq C^n \cdot a_0$\\*
Somit ist $\displaystyle\sum\limits_{k=0}^{\infty} C^k \cdot a_0$ konvergente Majorante von $\ds\sum_{k=0}^{\infty} a_k$ (Geometrische Reihe)
\sS{Beispiel Die Exponentialreihe}
$$exp(x) := \sum\limits_{k=0}^{\infty} \frac{x^k}{k!} \text{ für } x \in \R{}, x \geq 0$$
Setze $a_k = \frac{x^k}{k!}$\\*
$$a_n+1 = \frac{x^{n + 1}}{(n + 1)!} = \frac{x}{n+1} \cdot \frac{x^n}{n!} = \frac{x}{n+1} \cdot a_n \leq \frac{1}{2} a_n$$
\Rarr{} Quotientenregel ist erfüllt.\\*
Reihe $exp(x)$ konvergiert.\\*
Bezeichnung: $$exp(x) = \sum_{k=0}^{\infty} \frac{x^k}{k!} \eR$$
%
\uS{Bezeichnung:}
$$exp(1) = \sum\limits_{k=0}^{\infty} \frac{1}{k!} = e\text{ (Eulerische Zahl)}$$
\sS{Leibnitz-Kriterium}
Sei $(a_n)_{n \in \N{}_0}$ eine Monoton monoton fallende Nullfolge\footnote{$a_n → 0$ für $n→∞$} mit $a_n \geq 0$ für alle $n$\\*
Dann konvergiert die alternierende Reihe\\*
$$\sum_{k=0}^{\infty} (-1)^k · a_k$$
\bsp
$$a_k = \frac{1}{k + 1} \sum_{k=0}^{\infty} (-1)^k · a_k = 1 - \frac{1}{2} + \frac{1}{3} - \frac{1}{4} + \frac{1}{5}= log(2)$$
\bew
Sei $s_n = a_0 + ... + a_n$
\uS{Behauptung: }
$S_{2n + 1} \leq S_{2n + 3} \leq S_{2n + 2} \leq S_{2n}$ für jedes $n \in \N{}$
%
\ul{Rechne:}\\*
$S_{2n + 2} - S_{2n} = - a_{2n + 1} + a_{2n + 2} \leq 0 \Rightarrow (3)$\\*
$S_{2n + 3} - S_{2n + 1} = - a_{2n + 3} \leq 0 \Rightarrow (2)$\\*
$S_{2n + 3} - S_{2n + 1} = - a_{2n + 2} - a_{2n + 3} \leq 0 \Rightarrow (1)$\\*[8pt]
Die Folge $b_n = S_{2n}$\\*
\phantom{Die Folge }$c_n = S_{2n + 1}$\\*
sind beschränkt und monoton (fallend bzw. steigend)\\*
$\Rightarrow b_n \text{ und } c_n$ konvergieren\\*[4pt]
Sei $$b = \lim_{n \to \infty} b_n \qquad c = \lim_{n \to \infty} c_n$$
$$c - b = \lim_{n \to \infty} (c_n - b_n) = \lim_{n \to \infty} (a_{2n + 1}) = 0$$
weil $(a_n)$ Nullfolge
\sS{Behauptung:} % uS stande dort ich vermute so
$S_n \to b$ für $n \to \infty$\\*[4pt]
Gegeben sei $\e > 0$. Es gibt $N \in \N{}$ so dass für $n \leq N$:\\*
$|b_n - b| < \e, |c_n - c| < \epsilon$\\*
Somit für $n \geq 2N+1 $\\*
$|S_n - b| < \e$ also $S_n \to b$\qed\newpage
% Kopfzeile beim Kapitelanfang:
\fancypagestyle{plain}{
%Kopfzeile links bzw. innen
\fancyhead[L]{\calligra\Large Vorlesung Nr. 8}
%Kopfzeile rechts bzw. außen
\fancyhead[R]{\calligra\Large 05.11.2012}
}
%Kopfzeile links bzw. innen
\fancyhead[L]{\calligra\Large Vorlesung Nr. 8}
%Kopfzeile rechts bzw. außen
\fancyhead[R]{\calligra\Large 05.11.2012}
% **************************************************
%
\Wdh{Konvergenzsätze}
\begin{itemize}
    \item{Eine monoton wachsende und beschränkte Folge konvergiert zwangsläufig.}
    \item{Eine Reihe $\sum_{k=0}^{∞} a_k$ mit $a_k \geq 0$ für alle $k$ konvergiert \equ die Folge der Partialsummen $(S_n = \sum_{k=0}^{n} a_k)_{n \in \N}$ ist beschränkt}
\end{itemize}
%
\bsp
    $\sum_{k=0}^{n} \frac{1}{k} = 1 + \frac{1}{2} + \frac{1}{3} …$ ist unbeschränkt
\bsp
    $\sum_{k=1}^{\infty} \frac{1}{k^2} = 1 + \frac{1}{4} + \frac{1}{9}+…$
\sss{Leibnitz} 
    Sei $(a_n)$ monoton fallende Nullfolge.\\
    Dann konvergiert $\sum_{k=0}^{∞} (-1)^k \cdot a_k$
\bsp
    $(1 - \frac{1}{2} + \frac{1}{3} - \frac{1}{4})$ … konvergiert.
% satz 4.10 ***************
%\setcounter{chapter}{4}
%\setcounter{section}{9}{Die Folge }
% *************************

\sS{Satz Verdichtungslemma von Cauchy}
Sei $(a_n)$ monoton fallende Nullfolge.\\*
Die Reihe $\Sum_{k=0}^{\infty} a_k$ konvergiert genau dann, wenn die verdichtete Reihe $\Sum_{k=0}^{\infty} 2^k \cdot a_{2^k} = 1 · a_1 + 2 · a_2 + 4 \cdot a_4$ … konvergiert.
%
\bsp
    $a_k = \frac{1}{k}\qquad (k \geq 1)$\\*
    $2^k \cdot a_{2^k} = 2^k \cdot \frac{1}{2^k} = 1$

\section*{Satz} % Keine Nummerierung!
$\Sum_{k=0}^\infty \frac{1}{k}$ konvergent \equ\ $\Sum_{k=0}^{∞} 1$ konvergent (ist nicht der Fall.)
\bew
Sei $b_n = \ds\sum_{k=2^n}^{2^{n+1}-1} a_k$\\*
Für $\ds 2^n \leq k \leq 2^{n+1} - 1$ ist $\ds a_{2^n} \geq a_{k} \geq a_{2^{n+1} - 1} \geq a_{2^{n+1}}$ \Rarr $\ds 2^n · a_{2^n} \geq b_n \geq  2^n · a_{2^{n+1}}$\\*
Wenn $\ds \sum_{k \geq 0} 2^k · a_2^k$ beschränkt \Rarr $\ds \sum_{k \geq 0} b_k$ beschränkt \Rarr $\ds \sum_{k \geq 0} a_k$ beschränkt\\*
Hier immer beschränkt \equ konvergent\\*
Wenn $\ds \sum_{k \geq 0} 2^k · a_k$ beschränkt \Rarr $\ds \sum_{k \geq 0} b_k$ beschränkt \Rarr $\ds \sum_{k \geq 0} 2^{k} · a_2^{k+1}$ beschränkt \equ $\ds \sum_{k \geq 0} 2^{k+1} · a_2^{k+1}$ beschränkt \equ $\ds \sum_{k \geq 0} 2^{k} · a_2^{k}$ beschränkt\\*
Das zeigt den Satz.
\sss{Anwendung}
\sss{Erinnerung}
    Für $ x\geq 0$ und $a \in \R$ wird später $x^a \in R$ definiert\\*
    Wenn $\ds a = \frac{n}{m}$ mit $m \geq 1$ d.h. $a \in \Q$ dann $\ds x^a = \sqrt[m]{x^n}$.\\*
    Wenn $x > 1$ dann gilt: \\*
    $x^a = \begin{cases} >1 \text{ wenn }a>0\\* =1 \text{ wenn }a=0\\* <1\text{ wenn }a<0 \end{cases}$

\sS{Satz}
Sei $a \in \R$. Die Reihe $\ds \sum_{k=1}^{∞} \frac{1}{k^a}$ konvergiert genau dann, wenn $a > 1$
\bew
Wenn $a \leq 0$ dann $\frac{1}{k^a} \geq 1$ \Rarr Reihe divergiert.\\*
Sei $a >0$, sei $a_n = \frac{1}{n^a}$ \\*
$\ds n < n+1 \Rarr n^a < (n+1)^a \Rarr a_n > a_{n+1}$ Somit $(a_n)$ monoton fallend.\\*
$\ds \lim_{n\to \infty} n^a = \infty \Rarr \lim_{n \to \infty} \frac{1}{n^a} = 0$ \Rarr Verdichtungslemma ist anwendbar.\\*
Bilde $\ds 2^n · a_{2^n} = 2^n · \frac{1}{(2^n)^a} = 2^n · 2^{-n · a} = 2^{n(1-a)} = (2^{1-a})^n = x^n$\\*
mit $x:=2^{1-a}$\\*
Erhalte:
$\ds \sum_{k=1}^\infty \frac{1}{k^a}$ konvergiert $\equ\ \ds \sum_{k=0}^\infty x^k$ konvergiert $\equ |x|<1 \equ x<1 \equ 2^{1-a} <1$\\*
$\equ 1-a<0 \equ a>1$\qed\\*[4pt]
Beziehung:
$\ds \sum_{k=1}^\infty\frac{1}{k^a}=\zeta (a)$ für $a>1$\\*
Riemannsche Zetafunktion
Spezielle Werte:\\*
$\zeta (2) = \sum_{k\geq1}\frac{1}{k^2}=\frac{\pi^2}{6}$\\*
$\zeta (4) = \sum_{k\geq1}\frac{1}{k^4}=\frac{\pi^4}{90}$\\*
$\zeta (6) = \sum_{k\geq1}\frac{1}{k^6}=\frac{\pi^6}{945}$\\*
Frage: Für welche z ist $\zeta(z)=0$?

\uS{Teilfolgen}
\sS{Definition Teilfolge}
Sei $(a_n)$ eine Folge reeller Zahlen.\\*
Eine Teilfolge von $(a_n)$ ist eine Folge der Form $(a_{n_k})_{k \geq 0}$ wobei $n_0, n_1, n_2,…$ streng monoton wachsende Folge in $\N_0$ ist.
\bsp
$(a_n) = (1, x, x^2, x^3 , x^4 …)$\\*
$(n_k) = (1, 4, 9, 16) \leadsto $ Teilfolge $(x, x^4, x^9, x^{16} ,…)$

\sS{Bemerkung}
Wenn $a_n \to a$ für alle $n \to \infty$ dann konvergiert jede Teilfolge von $(a_n)$ gegen $a$ (Präsenzübung Nr. 9)

\uS{Schlüsselsatz}
\sS{Lemma}
Jede Folge reeller Zahlen $(a_n)_{n\geq 0}$ hat eine monotone Teilfolge.
\bew
Wir nennen $n\in \N_0$ \ul{extrem} wenn $a_n \geq a_m$ für alle $m \geq n$\\*
Unterscheide zwei Fälle:
\begin{itemize}
    \item{Es gibt unendlich viele extreme $n \in \N$\\*
Dies seien $n_0, < n_1, n_2…$\\*
Dann $a_{n_0} \geq a_{n_1} \geq a_{n_2} …$\\*
Weil $n_0$ extrem … weil $n_1$ extrem.\\*
→ monoton fallende Teilfolge gefunden}
    \item{Es gibt nur endlich viele extreme $n$\\*
Wähle $n_0 \in \N$ s.d. gilt: $m \geq n_0\ \Rarr\ m$ nicht extrem.\\*
$n_0$ nicht extrem $\Rarr$ es gibt $n_1 \geq n_0$ mit $a_{n_1} > a_{n_0}$ insbesondere $n_1 > n_0$\\*
$n_1$ \phantom{nicht extrem }$\Rarr$ \phantom{es gibt }$n_2 \geq n_1$ mit $a_{n_2} > a_{n_1}$ insbesondere $n_2 > n_1$\\*
$n_2$ \phantom{nicht extrem }$\Rarr$ \phantom{es gibt }$n_3 \geq n_2$ mit $a_{n_3} > a_{n_2}$ insbesondere $n_3 > n_2$\\*
usw.\\*
Erhalte $n_0 < n_1 < n_3 < …$ mit $a_{n_0} < a_{n_1} < a_{n_2} < …$ \\*
$\to $ streng monoton wachsende Teilfolge gefunden.\qed
}
\end{itemize}

\sS{Satz Bolzano-Weierstraß}
Jede beschränkte Folge reeller Zahlen hat eine konvergernte Teilfolge.
\bew
Es gibt ein monotone Teilfolge (Lemma 4.14)\\*
Diese ist beschränkt \Rarr konvergent.

\sS{Definition Cauchyfolge}
Eine Folge reeller Zahlen $(a_n)_{n \geq 0}$ heißt Cauchyfolge wenn gilt:\\*
Für jedes $\e > 0$ gibt es ein $N \in \N$ sodass für $m, n \geq N$ gilt: $|a_n - a_m| < \e$

\sS{Satz Cauchykriterium}
Eine Folge reeller Zahlen $(a_n)$ konvergiert genau dann, wenn sie eine Cauchyfolge ist.
\bew
\begin{description}
\item["\Rarr"]{Sei $a_n \to a$ für $n \to \infty$\\*
Gegeben sei $\e > 0$. Es gilt $N \in \N$ so dass $|a_n - a| < \frac{\e}{2}$ für $n \geq N$\\*
Für $n, m \geq N$ gilt:\\*
$|a_n - a_m| = |a_n - a + a - a_m| \leq |a_n - a| + |a - a_m| < \frac{\e}{2} + \frac{\e}{2} = \e$\\*
$\Rarr (a_n)$ ist eine Cauchyfolge}
\item["\Larr"]{Sei $(a_n)$ eine Cauchyfolge
\sss{Behauptung} $(a_n)$ ist beschränkt
\bew
Wähle $\e=1$ Es gibt $N\in\N$ mit $|a_n-a_m|<1$ für $m,n\geq N$\\*
Sei $C=max\{|a_0|,|a_1|,|a_2| … |a_N|,|a_N|+1\}$\\*
Dann $|a_n| \leq C$ für alle $\N$\\*
$(n\geq N \Rarr |a_n-a_N| < 1 \Rarr |a_n|<|a_N|+1)$\\*
Also ist $(a_n)$ beschränkt\\*
$\underset{Lemma}{\Rarr}\ (a_n)$ hat eine monotone Teilfolge $(a_{n_k})_{k\geq0}$ diese ist beschränkt \Rarr konvergent.\\*
Sei $\ds \lim_{k→∞}(a_{n_k})$
\sss{Behauptung}
$a_n→a$ für $n→∞$\\*
Sei $\e>0$ gegeben. Es gibt $n\in\N$ so dass
\begin{enumerate}
\item{$n,m\geq N \Rarr |a_n-a_m|< \frac{\e}{2}$}
\item{$k\geq N \Rarr |a_{n_k}-a|< \frac{\e}{2}$}
\end{enumerate}
Sei $k\geq N$
\bem
Für jedes $k\in\N$ ist $n_k\geq k$\\*
$\ds |a_k-a|=|a_{n_k}+a_{n_k}-a| \leq |a_k-a_{n_k}|+|a_{n_k}-a|<\frac{\e}{2}+\frac{\e}{2}=\e$\\*
Also $a_k→a$ für $n→∞$\qed}
\end{description}

\uS{Umformulierung für Reihen}
\sS{Satz (Cauchy-Kriterium für Reihen)}
Eine reelle Reihe $\ds \sum_{k=0}^\infty a_k$ konvergiert genau dann, wenn gilt:\\*
Für jedes $\e>0$ gibt es ein $N\in\N$ so dass für alle $\ds n,m\geq N,\ n\leq m$ $$\left|\sum_{k=n}^m\right|<\e$$
%
\sss{Beweis: Partialsummen}
$\ds s_n=\sum_{k=0}^n a_k$\\*
$\ds \sum_{k=n}^m=s_m-s_{n-1}$\\*
Damit ist 4.19 äquivalent zu 4.18\newpage
% Kopfzeile beim Kapitelanfang:
\fancypagestyle{plain}{
%Kopfzeile links bzw. innen
\fancyhead[L]{\calligra\Large Vorlesung Nr. 9}
%Kopfzeile rechts bzw. außen
\fancyhead[R]{\calligra\Large 08.11.2012}
}
%Kopfzeile links bzw. innen
\fancyhead[L]{\calligra\Large Vorlesung Nr. 9}
%Kopfzeile rechts bzw. außen
\fancyhead[R]{\calligra\Large 08.11.2012}
% **************************************************
%
\wdh
Eine Folge reeller Zahlen $(a_n)$ ist eine Cauchyfolge wenn gilt:\\*
Für jedes $\e>0$ gibt es ein $n\in\N$ so dass für $m,n\geq\N$ gilt $|a_n-a_m|<\e$\\*
$(a_n)$ konvergiert \equ\ $(a_n)$ ist Cauchyfolge\\*
Für Reihen: $\ds \sum_{k=0}^{∞}a_k$ konvergiert \equ\ Für jedes $\e>0$ gibt es ein $N\in\N$ so dass für $m,n\geq\N$ mit $m\geq n$ ist $\ds \left|\sum_{k=n}^m a_n\right|<\e$

\uS{Absolute Konvergenz}
\sS{Definition Absolute Konvergenz}
Eine Reihe $\ds\sum_{k=0}^{∞} a_k$ mit $a_k\in\R$ heißt absolut konvergent wenn die Reihe $\ds\sum_{k=0}^{∞} |a_k|$ konvergiert

\sS{Satz}
Jede absolut konvergente Reihe konvergiert
%
\bew
Verwende Cauchy-Kriterium für Reihen\\*
Sei $\ds\sum_{k=0}^{∞} a_k $ absolut von konvergent.\\*
\Rarr Für jedes $\e>0$ gibt es $N\in\N$ mit:\\*
Für $n\geq m\geq N$ gilt $\ds\sum_{k=m}^n |a_k| < \e\ \Rarr\ \left|\sum_{k=m}^n a_k\right| \underset{\overset{\uparrow}{Dreiecksungleichung}}{\leq} \sum_{k=m}^n |a_k| < \e\ \Rarr\ \sum_{k=m}^n a_k$ konvergiert\qed
\bem
Umkehrung gilt nicht.
$\ds\sum_{k=1}^{∞} (-1)^k \frac{1}{k} = -1+\frac{1}{2}+\frac{1}{3}+\frac{1}{4}+…$\\*
konvergiert (Leibnitz)\\*
denn $\ds\sum_{k=1}^{∞} \left|(-1)^k \frac{1}{k}\right| = \sum_{k=1}^{∞} \frac{1}{k}$ divergiert

\sS{Definition Majorante}
Eine Reihe $\ds\sum_{k=0}^{∞} b_k$ heißt Majorante der Reihe $\ds\sum_{k=0}^{∞} a_k$, wenn $|a_k|\leq b_k$ für alle k\\*
(schon gewesen wenn $a_k\geq 0$)

\sS{Satz (Majorantenkriterium)}
Wenn eine Reihe eine konvergente Majorante hat, dann konvergiert sie absolut.
\ul{Beweis} von Satz 4.5\qed

\uS{Umordnung von Reihen}
\sS{Definition Umordnung von Reihen}
Eine Umordnung einer Reihe $\ds\sum_{k=0}^{∞} a_k$ ist eine Reihe der Form $\ds\sum_{k=0}^{∞} a_{n_k}$ wobei $(n_0,n_1,n_2…)$ eine Folge natürlicher Zahlen ist, in der jedes $n\in\N_0$ genau einmal vorkommt.

\sS{Satz}
Jede Umordnung einer \ul{absolut} konvergenten Reihe ist wieder absolut konvergent und hat den gleichen Grenzwert.\\*
Im Gegensatz dazu gilt:

\sS{Satz}
Sei $\ds\sum_{k=0}^{∞} a_k$ eine konvergente, nicht absolut konvergente, Reihe. Für jedes $c\in\R\cup\{-∞,∞\}$ hat $\sum a_k$ eine Umordnung, die gegen $c$ konvergiert.
\bsp
Eine Reihe $\dfrac{1}{2}-\dfrac{1}{2}+\dfrac{1}{3}-\dfrac{1}{3}+\dfrac{1}{4}-\dfrac{1}{4}+\dfrac{1}{5}-\dfrac{1}{5}+…$
konvergiert gegen 0. Konvergiert aber nicht absolut:\\*
Folge: $$\left(\dfrac{1}{2},0,\dfrac{1}{3},0,\dfrac{1}{4},0,…→0\right)\qquad \sum_{k=1}^{∞} 2·\frac{1}{k}=∞$$
Produziere Umordnung, die gegen ∞ konvergiert:
$$\dfrac{1}{2}-\dfrac{1}{2}+\underbrace{\dfrac{1}{3}+\dfrac{1}{4}}_{\geq\dfrac{1}{4}+\dfrac{1}{4}=\dfrac{1}{2}}-\dfrac{1}{3}+\underbrace{\dfrac{1}{5}+\dfrac{1}{6}+\dfrac{1}{7}+\dfrac{1}{8}}_{\geq\dfrac{1}{2}}-\dfrac{1}{4}+\underbrace{\dfrac{1}{5}+…+\dfrac{1}{16}}_{\geq\dfrac{1}{2}}-\dfrac{1}{5}+…$$
$$\leq\underbrace{\dfrac{1}{2}-\dfrac{1}{2}}_{\text{\large{0}}}+\underbrace{\dfrac{1}{2}-\dfrac{1}{3}}_{\dfrac{1}{6}}+\underbrace{\dfrac{1}{2}-\dfrac{1}{4}}_{<\qquad\dfrac{1}{4}\qquad<}+\underbrace{\dfrac{1}{2}-\dfrac{1}{5}}_{\dfrac{3}{10}}+…=∞$$
Beweise von 4.24, 4.25 eventuell später.

\uS{Produkte von Reihen}
Frage: was ist $$\ds\left(\sum_{k=0}^{∞} a_k\right)·\left(\sum_{k=0}^{∞} b_k\right) ?$$

\sS{Definition Produkt von Reihen}
Das Cauchy-Produkt von zwei Reihen $\Sum_{k=0}^{∞} a_k$ und $\Sum_{k=0}^{∞} b_k$ ist eine Reihe $\ds\sum_{k=0}^{∞} c_k$ mit
$$c_n :=\sum_{k=0}^{∞} a_k·b_{n-k}=a_0·b_n+a_1·b_{n-1}+a_2·b_{n-2}+…+a_n·b_0$$
2-dimensionale Anordnung der $a_k·b_l$ 
SKIZZE% Ed's heft

\sS{Satz}
Seien $\ds\sum_{k=0}^{∞} a_k$ und $\ds\sum_{k=0}^{∞} b_k$ konvergente Reihen, mindestens eine von ihnen absolut konvergent. Dann konvergiert ihr Cauchy-Produkt $\ds\sum_{k=0}^{∞} c_k$. Wenn $\ds\sum_{k=0}^{∞} a_k = a, \ds\sum_{k=0}^{∞} b_k = b$ $\ds\sum_{k=0}^{∞} c_k = a·b$
\Bew{von 4.27}
Sei $\sum a_k$ absolut konvergent, $\sum b_k$ konvergent, so zeige $\sum c_k\ $ konvergent, $\ds c_n :=\sum_{k=0}^{∞} a_k·b_{n-k}$\\*[4pt]
Schreibe:
$s_n=a_0+…+a_n$\\*
$t_n=b_0+…+b_n$\\*
$u_n=c_0+…+c_n$\\*
$s_n→a$,$t_n→b$ (*)\\*[8pt]
Zeige $u_n→a·b$\\*[4pt]
(*)\Rarr $s_n·b→a·b$ Zeige $s_n·b-u_n→0$\\*[4pt]
$u_n=a_0·b_0+(a_0·b_1+a_1·b_0)+(a_0·b_2+a_1·b_1+a_2·b_0)+…+a_n·b_0=a_1·t_{n-1}+a_2·t_{n-2}+…+a_n·t_0$\\*[4pt]
$s_n·b=a_0·b+a_1·b+a_2·b+a_3·b+…+a_n·b$\\*[4pt]
$s_n·b-u=a_0·(b-t_n)+a_1·(b-t_{n-1})+a_2·(b-t_{n-2})+a_3·(b-t_{n-3})+…+a_n·(b-t_0)\underset{?}{→}0$\\*[8pt]
Sei $C\in\R$ mit $|b|\leq C$ und $|b-t_n|\leq C$ für alle n\\*
Sei $\ds\sum_{k=0}^{∞} |a_n| = a^*.$\\*
Gegeben sei $\ds\e>0$. Wähle $N\in\N$ so dass $C·(|a_N|+|a_{N+1}|+|a_{N+2}|+…)<\frac{\e}{2}$\\*
(geht weil $\sum|a_k|$ konvergiert)\\*
und $|b-t_n|<\frac{\e}{2a^*}^{(2)}$ für alle $n\geq N$\\*
(geht weil $b-t_n→0$ für alle $m→∞$)
\bem
Wenn $a^*=0$ dann $a_n=0$ für alle k. Dann alles klar. Für alle $n\geq 2N$ gilt:
$$|a_0(b-t_n)+a_1(b-t_{n-1})+…+a_n(b-t_0)|\leq |a_0|·|(b-t_n)|+|a_1|·|(b-t_{n-1})|+…+|a_n|·|(b-t_0)|$$
$$\leq(|a_0|+|a_1|+|a_2|+…|a_N|)·\underset{\overset{\uparrow}{wegen\left(2\right)}}{\frac{\e}{2a^*}} +(|a_{N+1}|+|a_{N+2}|+|a_{N+3}|+…|a_n|) · C \leq a^* ·\frac{\e}{2a^*}+\frac{\e}{2}=\frac{\e}{2}+\frac{\e}{2}=\e$$
Also gilt: $s_n-u→0$ für $n→∞$\qed\\[4pt]
\ul{Zusatz} Wenn $\sum a_k$ und $\sum b_k$ beide absolut konvertieren, dann auch das Cauchy-Produkt $\sum c_k$
\bew
Sei $\sum a_k^*$ das Cauchy-Produkt von  $\sum |a_k|$ und  $\sum |b_k|$. Beide konvergieren \Rarr $\sum_n c_n^*$ konvergiert\\*[8pt]
d.h. $c_n^*=|a_0·b_{n}|+|a_1·b_{n-1}|+…+|a_n·b_{0}|\geq|a_0·b_{n}+a_1·b_{n-1}+…+a_n·b_{0}|=|c_n|$\\*[8pt]
Also $\sum_n c_n^*$ ist konvergente Majorante von $\sum_n c_n$ \Rarr $\sum_n c_n$ konvergent absolut\qed
\bsp
Die Reihe $\ds\sum_{k=0}^{∞} a_k = 1-+\dfrac{1}{\sqrt{2}}+\dfrac{1}{\sqrt{3}}-\dfrac{1}{\sqrt{4}}+\dfrac{1}{\sqrt{5}}-…\quad $ konvergiert (Leibnitz)\\*
Das Cauchy-Produkt der Reihe von $\sum a_k$ und $\sum a_k$ konvergiert nicht.

\sS{Beispiel}
Für jedes $x\in\R$ ist die Exponentialreihe $\ds exp(x)=\sum_{k=0}^{∞} \frac{x^k}{k!}$ absolut konvergent.\\*
Es gilt \fbox{$exp(x)·exp(y)=exp(x+y)$} Funktionalgleichung der Exponentialfunktion.
%
\bew
Betrag von $\ds \sum_{k=0}^{∞} \left|\frac{x^k}{k!}\right|=\sum_{k=0}^{∞} \frac{|x|^k}{k!}=exp(|x|)$ konvergiert (bekannt, Quotientenkriterium)\\*
Berechne Cauchy-Produkt $\ds exp(x)·exp(y)=\sum_{k=0}^{∞} c_k$
$$c_k = \frac{x^0}{0!}·\frac{x^n}{n!}+\frac{x^1}{1!}·\frac{y^{n-1}}{(n-1)!}+…+\frac{x^n}{n!}·\frac{y^0}{0!}=\dfrac{1}{n!}·\left(\dfrac{n!}{0!·n!}·x^0y^n+\dfrac{n!}{1!·(n-1)!}·x^1y^{n-1}+…+\dfrac{n!}{n!·0!}·x^ny^0+\right)$$
$$=\dfrac{1}{n!}\sum_{k=0}^{n}\dfrac{n!}{k!·(n-k)!}x^ky^{n-k}=\dfrac{1}{n!}\sum_{k=0}^{n}\binom{n}{k}x^ky^{n-k}\underset{\overset{ }{binomische Formel}}{=}\dfrac{1}{n!}(x+y)^n\Rarr\sum_{k=0}^{∞}c_k=exp(x+y)$$\qed\newpage
% Kopfzeile beim Kapitelanfang:
\fancypagestyle{plain}{
%Kopfzeile links bzw. innen
\fancyhead[L]{\calligra\Large Vorlesung Nr. 10}
%Kopfzeile rechts bzw. außen
\fancyhead[R]{\calligra\Large 12.11.2012}
}
%Kopfzeile links bzw. innen
\fancyhead[L]{\calligra {\Large Vorlesung Nr. 10}}
%Kopfzeile rechts bzw. außen
\fancyhead[R]{\calligra \Large{12.11.2012}}
% **************************************************
%
%\setcounter{chapter}{4}
\chapter{Abbildungen und Funktionen}
\sS{Definition Abbildung}
Seien $A, B$ Mengen. Eine Abbildung von $A$ nach $B$ ist eine Vorschrift, die jedem Element von $A$ ein Element von $B$ zuordnet.
\notat{$f: A \to B,\  a \mapsto f(a) \  a\in A$}
$A$ heißt Definitionsbereich von $f$\\*
$B$ heißt Wertebereich von $f$
%
\bsp
\begin{enumerate}
\item {Alle Personen in $L1 \mapsto \N$\\*
$P \mapsto$ Geburtsjahr von $P$}
%
\item{$f:\R → \R, \ f(x)=x^2$\\*
$g:\R→\R_{\geq 0}=\{x\in\R\mid x\geq 0\}, \ g(x)=x^2$\\*
$h: \R_{\geq 0} \to \R_{\geq 0} \ h(x) = x^2$}
\bem 
\item{
$f,g,h$ sind verschieden\\*
Sei $M$ Menge. Die Identität von $M$ ist die Abbildung $id_{M}:M→M, \ id_M(x)=x$}
\end{enumerate}

\sS{Definition In-/Sur-/Bijektivität}
Eine Abbildung $f: A \to B$ heißt:
\begin{enumerate}
\item{\ul{injektiv} wenn gilt: Für alle $a, a' \in A$ mit $f(a) = f(a')$ ist auch $a = a'$}
\item{\ul{surjektiv} wenn gilt: Für jede $b\in B$ gibt es ein $a\in A$ mit $f(a)=b$}
\item{\ul{bijektiv} wenn $f$ injektiv und surjektiv ist}
\end{enumerate}
%
% Tafel 2.2
% Bild zeichnen
%
% Tafel 3.1
% Beispielbild
%
\bem
$f$ ist $\left\{
\begin{array}{c}
\text{injektiv}\\*
\text{surjektiv}\\*
\text{bijektiv}
\end{array}
 \right\}$ genau dann wenn für jedes $b \in B$ $\left\{\begin{array}{c} \text{höchstens}\\* \text{mindestens}\\*
 \text{genau}
 \end{array} \right\}$ ein $a \in A$ mit $f(a) = b$
%
\bsp
$f,g,h$ wie oben
\begin{description}
\item[f]{
\hspace{5.5mm}nicht surjektiv: es gibt kein $a\in\R$ mit $f(a)=-1$\\*
nicht injektiv: $f(-2)=4=f(2), 2\neq -2.$}
%
\item[g]{
\hspace{5mm}ist surjektiv, denn für jedes $b \in \R_{\geq 0}$ gilt $f( \sqrt{b} ) = b$ also gibt es $b \in \R_a$\\*
 ist nicht injektiv (wie $f$)}
%
\item[h]{
\hspace{5mm}surjektiv wie g. $\left(\sqrt{b} \geq 0\right)$\\*
injektiv, denn: Wenn $a, a' \geq 0$ und $a^2 = (a')^2$ dann $a = a'$ also $h$ bijektiv.}
\end{description}

\sS{Definition Komposition}
Seien $f:A→B$, $g:B→C$ Abbildungen\\*
Die Komposition von $f$ und $g$ ist die Abbildung\\*
$g \circ  f: A→C$, $(g \circ f)(a):=g(f(a))$\\*
Sprich $\circ$: "nach", "verkettet"

\sS{Satz} 
Eine Abbildung $f: A \to B$ ist bijektiv \equ \ es gibt eine Abbildung $g: B \to A$ mit $f \circ g = id_B$\\*
(d.h. $f(g(b)) = b$ für alle $b \in B$ $g(f(a)) = a$ für alle $a \in A$)
%
\sss{Definition} % ohne Nummerierung!
Wenn $f:A→B$ bijektiv ist, heißt die eindeutige Abbildung $g:B→A$ wie oben die Umkehrabbildung (inverse Abbildung) von $f$
Bezeichnung: $g=f^{-1}$.
%
\bew
Angenommen, $g: B \to A$ gegeben mit $f \circ g = id_B, g \circ f = id_A$ \footnote{Dies gilt, weil $g$ als Umkehrfunktion von $f$ definiert ist.}\\*
$f$ surjektiv: Sei $b \in B$. $b = f(g(b)) = f(a)$ mit $a = g(b)$ \ok\\*
$f$ injektiv: Sei $a, a'$ mit $f(a) = f(a')$ zeige $a = a'$ \\*
$a = g(f(a)) = g(f(a')) = a' $\ok \\*[8pt]
%
Angenommen, $f$ ist bijektiv, zeige $g$ existiert.\\*
Gegeben sei $b \in B$ $f$ bijektiv $\Rightarrow$ es gibt genau ein $a \in A $ mit $f(a) = b$ 
Setze $g(b):=a$ Das definiert Abbildung $g:B→A$\\*
Zeige $g \circ f=id; f \circ g = id$\\*
$(f\circ g)(b)=f(g(b))=f(a)=b$ wobei $a$ wie eben\\*[8pt]
%
Zeige: $(g \circ f) (a) $ für alle $a \in A$\\*
$f$ injektiv: Reicht $f(g(f)a))) = f(a)$\\*
Das gilt weil $f \circ g = id_B$ \ok\\*[8pt]
Eindeutigkeit von $g$:\\*
Angenommen, $g^* : B \to A$ erfüllt $g^* \circ f = id_A$,
$f \circ g^* = id_B$ \\*
%
Dann gilt: $g=g\circ id_B=g\circ f\circ g^*=id_A\circ g^* = g^*$ \qed
\bsp
Bewiesen 5.12
\begin{itemize}
\item{$f: \R_{\geq 0} \to \R_{\geq 0}, f(x) = x^k$ bijektiv ($k \geq 1$)\\*
Die Umkehrabbildung $f^{-1}$ heißt k-te Wurzelabbildung $f^{-1}(x) = \sqrt[k]{x}$ }
%
\item{exp: $\R→\R_{>0}$ $exp(x) = \sum_{k=0}^{\infty}$ (absolut konvergente Reihe) ist bijektiv. Die Umkehrabbildung heißt Logarithmus. bew.
$log = exp()^{-1} \R_{\geq } \to \R_a$ }
\end{itemize}

\uS{Bild und Urbild}
\sS{Definition}
Sei $f:A→B$ Abbildung
\begin{enumerate}
\item{Für eine Teilmenge $X \subset A$ ist \\*
$f(x) := \{f(x) | x \in X\} \subseteq B$ \\*
das Bild von $X$ unter $f$}
\item{Für eine Teilmenge $Y \subseteq B$ ist $f^{-1}:=\{a\in A|f(a)\in Y\}\subseteq A$ das Urbild von $Y$ unter $f$}
\end{enumerate}
\ul{Vorsicht} nicht Urbild und Umkehrabbildung verwechseln.
\bsp
$$f:\R→\R,\ f(x)=x^2$$
$$f(\{1, 2, -2\}) = \{1, 4\}$$
$$f^{-1}(\{1,-2,4\})=\{1,-1,2,-2\}=f^{-1}(\{1,4\})$$
$$f^{-1}(\{9\})=\{3,-3\}\qquad f^{-1}(\{-5\})=\emptyset$$

\uS{Funktionen}
\sS{Definition Funktion}
Sei $D\subseteq\R$ Teilmenge. Eine reelle Funktion auf $D$ ist eine Abbildung $f:D→\R$\\*
%
Der \ul{Graph} von $f$ ist die Menge $\Gamma_f = \{(x, f(x)\mid x \in D \}$\\*
$ \Gamma_f \subseteq D \times \R$ 
%
\bem Oft ist $D$ ein Intervall
%
%\sS{Definition Intervalle}
  \tikz[scale=0.5,domain=-3.5:3.5, samples=200,prefix=plots/,smooth]{
      \draw[very thin, color=gray!50] (-3.5,-3.5) grid (3.5,8.9);
      \draw[->] (-3.5,0) -- (3.5,0) node[right] {$x$};
      \draw[->] (0,-3) -- (0,8.5) node[above] {$y$};
      \clip (-3.5,-3.0) rectangle (5.5,8.9);
      \draw[color=red] plot[id=x^2] function{x*x} node[below=3.5cm] {\footnotesize $f_1(x) =x^2$};
      \draw[color=blue] plot[id=abs,sharp plot] function{abs(x)} node {\footnotesize $f_2(x) = |x|$};
      \draw[color=cyan] plot[id=exp] function{exp(x)} node [below=13.5cm] {\footnotesize $f_2(x) = exp(x)$};
      \draw[color=green!60!black,const plot] plot[id=gaussklammer] function{floor(x)} node[below] {\footnotesize $f_5(x) = [x]$};
  }
seien $a, b \in \R$ \\*
$[a, b] = \{x \in \R| a \leq x \leq b\}$ (abgeschlossen)\\*
$(a, b] = \{x \in \R| a < x \leq b\}$ (halboffen)\\*
$[a, b) = \{x \in \R| a \leq x < b\}$ (halboffen)\\* %mit klammer zu überer zeile\\*
$(a, b) = \{x \in \R| a < x < b\}$ (offen)\\*
%
Uneigentliche Intervalle:
$$[a, \infty) = \{x \in \R | a \leq x\} = \R_{\geq a}$$
$$(a, \infty) = \{x \in \R | a < x\} = \R_{> a}$$
$$(- \infty, a] = \{x \in \R | x \leq a\} = \R_{\leq a}$$
$$(- \infty, a) = \{x \in \R | x < a\} = \R_{< a}$$
$$(- \infty, \infty) = \R$$
%
\Bsp{Funktionen}
\begin{enumerate}
\item{$f:[0,2]→\R, f(x)=x^2, \Gamma_f \leq [0,2] x\R$}
\item{Betragsfunktionen: $|\ |: \R→\R, x\mapsto|x|$
%noch mehr graphen aaaahahhahahah
}
An dieser Stelle fehlen noch Graphen.
\item{$g:\R\bs\{0\}→\R, g(x)=\dfrac{1}{x}$
%graph
Hier auch.
}
\item{$exp:\R→\R$.}
\item{[.] : $\R \to \R$ Gaußklammer\\*
$[x] := max\{n \in \Z | n \leq x \}$
\bsp
$[5] = 5$\\*
$[5,78] = 5$\\*
$[-1,2] = -2$}
\item{Sei $h:\R→\R$ definiert durch $h(x)=\begin{cases}0\ wenn\ x\in\Q\\* 1\ wenn\ x\notin\Q\end{cases}$\\*
$h(\sqrt{2}) = 1, h (\frac{3}{7}) = 0$}
\end{enumerate}

\sS{Definition (Rechnen mit Funktionen)}
Sei $D \subseteq \R , \ f,g: D→\R$ Funktionen auf D.\\*
Definiere
\begin{itemize}
\item{$f+g: D \to \R$ durch $(f + g)(x) := f(x) + g(x)$}
\item{$(f \cdot  g) (x) := f(x) \cdot g(x)$}
\item{Für $a\in\R$ setze $a·f: D→\R, (a·f)(x):=a·f(x)$}
\item{Angenommen, $f(x) \neq 0$ für alle $x \in D$
$$\frac{1}{f}: D \to \R, \frac{1}{f}(x) := \frac{1}{f(x)} = f(x)^{-1}$$
\ul{Vorsicht} nicht $\frac{1}{f}$ mit Umkehrbild oder Urbild verwechseln}
\end{itemize}

\sS{Definition Polynomfunktion}
\begin{itemize}
\item{Eine \ul{Polinomfunktion} ist eine Funktion der Form\\*
$f: \R → \R,\ f(x) = a_n x^n+a_{n-1}x^{n-1}+…+a_0=\ds\sum_{k=0}^n a_k x^k $\\*
wobei $a_0,…,a_n \in \R$ fest}
\item{Seien $f, g : \R \to \R $ Polynomfunktionen
Sei $D = \{x \in \R \mid g(x) \geq 0\}\leadsto \dfrac{f}{g} : D \to , Rx \mapsto \frac{f(x)}{g(x)}$
Solche Funktionen heißen rationale Funktionen.
\bsp
$f:\R\bs\{0,1\}→\R, \ f(x)=\dfrac{x^7+5x^2}{x(x-1)}$}
\end{itemize}
\sS{Definition}
Seien $f: C \to \R, g: D \to \R$ Funktionen, sodass $f(C) \subseteq D$
Eine Komposition von $ f $ und $ g $ ist 
%
$g \circ f : C \to \R$\\*
$(g \circ f) \ (x) = g(f(x))$\newpage
% Kopfzeile beim Kapitelanfang:
\fancypagestyle{plain}{
%Kopfzeile links bzw. innen
\fancyhead[L]{\calligra\Large Vorlesung Nr. 11}
%Kopfzeile rechts bzw. außen
\fancyhead[R]{\calligra\Large 15.11.2012}
}
%Kopfzeile links bzw. innen
\fancyhead[L]{\calligra\Large Vorlesung Nr. 11}
%Kopfzeile rechts bzw. außen
\fancyhead[R]{\calligra\Large 15.11.2012}
% **************************************************
%
\wdh
Eine Abbildung $f:x→y$\\*
\begin{itemize}
\item{ist \ul{injektiv} wenn gilt:\\*
für alle $a,b\in X$ mit $f(a)=f(b)$ ist $a=b$}
\item{ist \ul{surjektiv} wenn für jedes $y\in Y$ ein $a\in X$ existiert mit $f(a)=y$}
\end{itemize}
Sei $D\subseteq\R$ Teilmenge. Eine Funktion auf $D$ ist eine Abbildung $f:D→\R$
%
\uS{Monotone Funktionen}
\bem
Eine Funktion $(a_n)_{n\geq 0}$ reeller Zahlen ist eine Abbildung $a:\N_0→\R$ d.h. eine Funktion auf $\N_0$
%
\sS{Definition}
Sei $D\subseteq\R$. Eine Funktion  $f:D→\R$ heißt:
\begin{enumerate}
\item{\ul{monoton wachsend} wenn gilt:\\*
Für alle $a,b\in D$ mit $a<b$ ist immer $f(a)\leq f(b)$}
\item{\ul{streng monoton wachsend}: $a<b\Rarr f(a)<f(b)$}
\item{\ul{monoton fallend}: $a<b\Rarr f(a)\geq f(b)$}
\item{\ul{streng monoton fallend}: $a<b\Rarr f(a)> f(b)$}
\end{enumerate}
%
\bem
Jede streng monotone Funktion $f$ ist injektiv
%
\bew
Zeige: $a\neq b\Rarr f(a)\neq f(b)$\\*
Wenn $a\neq b$ dann $a< b$ oder $b<a$\\*
Wenn $f$ streng monoton wachsend: Folgt $f(a)< f(b)$ oder $f(b)< f(a)$ also $f(a)\neq f(b)$\\*
Wenn $f$ streng monoton fällt: es folgt $f(a)> f(b)$ oder $f(b)> f(a)$ also $f(a)\neq f(b)$\qed
%
\sS{Beispiel}
\begin{enumerate}
\item{$f:\R_{\geq 0}→\R,\ x\mapsto x^k =:f(x)$ mit $k\geq 1$\\*%umgekehrtes define
$f$ ist streng monoton wachsend/steigend 
%bild tafel 2.2
\item{$h: \R \to \R, h(x) = [x]$}\\*
% Stufenfunktion Diagramm
$h$ ist monoton wachsend, aber nicht streng monoton.\\*
Monoton wachsend: $x < y \Rarr [x] < [y]$\\*
$x < y \not\Rarr [x] < [y]$\\*
z. B.: $1,2 < 1,3 , [1,2] = 1 = [1,3] $}
\item{Exponentialfunktion\\*
$exp: \R \to \R,\ exp(x)= \ds\sum_{k=0}^{\infty} \frac{x^k}{k!}$\\*
Ist streng monoton wachsend.\\*
\bew
\begin{enumerate}
\item{$exp(0) = 1 + \frac{0}{1!} + \frac{0}{2!} + ... = 1$}
\item{Sei $a > 0$\\*
$exp(a) = = 1 + \frac{a}{1!} + \frac{a}{2!} + ... > 1$}
\item{Sei $a > 0 exp(-a) \cdot exp(a) = exp(-a + a) = exp(0) = 1$\\*
$\Rarr exp(-a) = \frac{1}{exp(a)} \Rarr 0 < exp(a) < 1$\\*
%das kann ich nicht lesen... (iPhone Bild)
$exp(b) > 0$ für alle $b \in \R$}
\item{Sei $a > b$\\*
$exp(a) = exp(a - b + b) = \overbrace{exp(a - b) \cdot exp(b)}^{>0}$
> exp(b) $\Rarr$ exp streng monoton wachsend \qed }
\end{enumerate}
}
\end{enumerate}
%
\chapter{Stetigkeit}
\ul{Idee:} Eine Funktion ohne Sprünge heißt \ul{stetig}\\*
\sS{Definition}
Sei $D\subseteq \R,\ f:D→\R$ eine Funktion\\*
\begin{enumerate}
\item{$f$ heißt stetig in $x\in D$ wenn gilt:\\*
Für jedes $\e>0$ gibt es ein $\delta>0$ so dass für jedes $y\in D$ mit $|x-y|<\delta$ gilt $|f(x)-f(y)|<\e$ %graph 6.2
}
\item{$f$ heißt stetig wenn f in jedem $x\in D$ stetig ist}
\end{enumerate}
\sS{Beispiel}
\begin{enumerate}
\item{Die Funktion $id:\R→\R,\ x\mapsto x$ ist stetig}
\item{Die Funktion $f:\R→\R,\ f(x)=x^2$ ist stetig. %graph klein
\bew
Sei $x,y\in\R\quad y=x+h$.\\*
$$f(y)-f(x)=(x+h)^2-x^2=x^2+2xh+h^2-x^2=2xh+h^2$$\\*
Wähle jedenfalls $\delta\leq 1$. Wenn $|h|=|x-y|<\delta$ dann $|h|<1$\\*
$$|f(y)-f(x)|=|2xh+h^2|\leq|2x|·|h|+|h|^2<|2x|·|h|+|h|=(|2x|+1)·|h|$$\\*
Gegeben sei $\e>0$\\*
Wähle $\delta=min\left\{1,\dfrac{\e}{|2x|+1}\right\}$\\*
Wenn $|x-y|<\delta$ dann $$|f(x)-f(y)|<(2|x|+1)·|h|<(2|x|+1)·\dfrac{\e}{2|x|+1}=\e$$\\*
Also $f$ stetig in $x$}
\item{$g:=\R \to \R,\ g(x):=\{x\}$\\*
%graph
g ist stetig an $x$ \equ $x\notin\Z$
\Bew{$g$ nicht stetig an $x\in\Z$:}
Zeige: es gibt ein $\e>0$ so dass kein $\delta>0$ existiert mit: $|x-y|>\delta\Rarr|g(x)-g(y)|<\e$\\*
z.B. $\e=1$ Sei $\delta>0.\ y=x-\dfrac{\delta}{2}\quad |x-y|=\dfrac{\delta}{2}<\delta$\\*
aber $g(y)=\{x-\dfrac{\delta}{2}\}=x-1$ (weil $x\in\Z$)\\*
$|g(x)-g(y)|=|x-(x-1)|=1 \not<\e\qed$}
\end{enumerate}
%
\sS{Satz}
Die Exponentialfunktion $exp: \R \to \R$ ist stetig.\\*
\bew
Verwende nur:
\begin{itemize}
\item{Funktionalgleichung: $exp(x + y) = exp(x) \cdot exp(y)$}
\item{exp ist streng monoton wachsend}
\item{exp(0) = 1}
\end{itemize}
\subsection*{Behauptung}
Für jedes $\e > 0$ gibt es ein $n \in \N$ mit $exp(\frac{1}{n}) < 1 + \e$\\*
%Graf
Angenommen, $exp(\frac{1}{n}) \geq 1 + \e$\\*
Dann $exp(1) = \frac{1}{n} + ... \frac{1}{n}$ % underbrace unter den Brüchen {n}
\phantom{Dann $exp(1) $} $= exp(\frac{1}{n}) + ... + exp(\frac{1}{n}) = exp(\frac{1}{n})^n$ \\*
\phantom{Dann $exp(1) $} $ \geq (1 + \e)^n \geq 1 + n \e $ %(Pfeil auf letztes geq Bernoulli)
\\*
$exp(1) \geq 1 + n \e$\\*
$n \leq \frac{exp(1) - 1}{\e}$\\*
Das gilt nur für endliche viele $n \in \N$\\*
$\Rarr$ Beh.\\*
\ul{Zeige:} exp ist stetig an 0. Gegeben sei $\e > 0, OE ? \e < 1$\\*
Wähle $n \in \N$ mit $exp(\frac{1}{n}) < 1 + \e$\\*
$$\Rarr exp(-\frac{1}{n}) = exp(\frac{1}{n})^{-1} < \frac{1}{1 + \e} = \frac{1 - \e}{(1+\e)(1-\e)} = \frac{1-\e}{1 - \e^2} > 1 - \e$$\\*
Sei $\delta \frac{1}{n}$\\*
Sei $y \in \R, |0 - y| < \delta = \frac{1}{n}$\\*
$|y| < \frac{1}{n}$ d.h.\\*
$-\frac{1}{n} < y < \frac{1}{n}$\\* \\*
exp streng monoton wachsend.\\*
$$\Rarr 1 - \e < exp(-\frac{1}{n}) < \exp(y) < exp(\frac{1}{n}) < 1 + \e$$\\*
$\Rarr |exp(y) - exp(0)| < \e$ Also exp stetig in 0\\*
\ul{Zeige:} exp ist eine stetig in $x \in \R$. Gegeben sei $\e > $\\*
Sei $y = x + h$, $|h| < \delta$ ($\delta$ noch zu wählen)
$$|exp(y) - exp(x)| = |exp(x + h) - exp(x)| = |exp(x) \cdot exp(h) - exp(x)| = exp(x) \cdot exo(h) -1$$\\*
$|exp(y) -exp(x) | < \e$\\*
$$\Leftrightarrow exp(x) \cdot |exp(h) - 1| < \e \Leftrightarrow exp(h) - 1 < \frac{\e}{exp(x)} = \e '$$\\**
Weil exp stetig in 0 ist gibt es ein $\delta > 0$ mit $|h| < \delta \Rarr |exp(h) -1| < \frac{\e}{exp(x)}$\\*
$\Rarr$ exp ist stetig in x \qed\\*
%
\sS{Satz (Folgenstetigkeit)}
Sei $D\subseteq\R,\ x\in D,\ f:D→\R$ Funktion $f$ ist genau dann stetig in $x$ wenn gilt:\\*
\begin{itemize}
\item{Für jede Folge $(x_n)_{n\geq 0}$ mit $x_n\in D,\ x_n→x$ für $n→∞$ gilt auch $f(x_n)→f(x)$ für $n→∞$}
\end{itemize}
%
\sS{Satz}
Sei $D\subseteq\R,\ f,g:D→\R$ in $x\in D$\\*
Dann gilt:\\*
\begin{itemize}
\item{$f+g:D→\R$ stetig in $x$}
\item{$f·g:D→\R$ stetig in $x$}
\item{Wenn $g(x)\neq 0$ für alle $x'\in D$}
\end{itemize}
Dann ist $\dfrac{1}{f}:D→\R$ stetig in x.
\Bew{mit Folgenstetigkeit}
Sei $x_n \to x$ für $n \to \infty$\\*
mit $x_n \in D$\\*
$f, g$ stetig $\Rarr f(x_n) \to f(x)$ \\*
\phantom{$f, g$ stetig \Rarr}$g(x_n) \to g(x)$\\*
$\Rarr f(x_n) + g(x_n) \to f(x) + g(x)$
$\phantom{\Rarr} f(x_n) \cdot g(x_n) \to f(x) \cdot g(x)$\\*
Wenn also $f(x) \neq 0$\\*
$f(x_n)^{-1} \to f(x)^{-1}$\\*
$\Rarr f + g, f) \cdot g, \frac{1}{f}$ stetig in x \qed\newpage
%Kopfzeile links bzw. innen
\fancyhead[L]{\calligra {\Large Vorlesung Nr. 12}}
%Kopfzeile rechts bzw. außen
\fancyhead[R]{\calligra \Large{19.11.2012}}
% **************************************************
\wdh
Sei $D \subseteq \R$. eine Funktion $f: D \to \R$ ist stetig in $x \in D$ wenn gilt:\\
\begin{array}{ll}
\text{Für jedes $\epsilon > 0$ gibt es ein $\delta > 0$ so dass gilt:}\\
\text{wenn $y \in D$ mit $|x - y| < \d$ dann $|f(x) - f(y)| < \e$}
\end{array}
\bsp
$exp:R→\R$ ist stetig (d.h. stetig an jedem $x\eR$)\\
$[·]:R→\R$ ist stetig an $x \equ x\not\in\Z$\\
%Diagramm, Gausklammern
\sS{Satz Folgenstetigkeit}
Eine Funktion $f: D \to \R$ ist stetig in $x \in D$ $\equ$ Für jede Folge $(x_n)$ mit $x_n \in D$ für alle $n$ und $x_n \to x$ gilt auch $f(x_n) \to f(x)$.\\
(d.h. f erhält Konvergenz)\\
\bew
"\Rarr" Angenommen $f$ ist stetig in $x$, $x_n→x$ mit $x_n\in D$. Zeige $f(x_n)→f(x)$\\
Gegeben $\e>0$. Stetigkeit \Rarr{} es gibt $\delta>0$ mit:\\
Wenn $y\in D$ mit $|x-y|<\delta$ dann $|f(x)-f(y)|<\e$\\
Wähle $N\eN$ so dass gilt:\\
Für $n\geq N$ ist $|x-x_n|<\delta \Rarr |f(x)-f(x_n)|<\e$\\
Also gilt $f(x_n)→f(x)$\\
"\Larr" Angenommen $f$ ist nicht stetig.\\
Zeige: Es gibt eine Folge $(x_n)$ mit $x_n\in D$ und $x_n→x$ aber nicht $f(x_n)→f(x)$ für nicht stetig in $x$ \Rarr{} es gibt ein $\e>0$ so dass für jedes $\delta>0$ ein $y\in D$ existiert mit $|x-y|<\delta$ und $|f(x)-f(y)|\geq \e$\\
Wähle für $\delta=\frac{1}{n}$ ein $x_n\in D$ mit $|x_n-x|<\delta,\ |f(x_n)-f(x)|\geq\e$\\
Dann gilt $x_n→x$ aber \underline{nicht} $f(x_1)→f(x)\qed$
%
\wdh
\sS{Satz 6.5}
Wenn $f, g: D \to \R$ stetig in $x \in D$ dann auch $f + g,\ f \cdot g$ und $\frac{1}{g}$ (falls $g(y) \neq 0$ für alle $y \in D$) \\
Und $a \cdot f$ für $a \in \R$
\sS{Korollar} 
Polynomfunktionen und rationale Funktionen sind stetig.
\bew
\begin{enumerate}
\item{$id_{\R}:\R→\R,\ id_{\R}(x)=x$ ist stetig}
\item{6.5 und Induktion \Rarr{} für $\R→\R,\ f(x)=x^n$ stetig für jedes $n\quad (x^n=x·x^{n-1}$}
\item{6.5 \Rarr{} Jede Funktion $f(x)=a_n·x^{n}+a_{n-1}·x^{n-1}+…+a_0,\ f:\R→\R$ ist stetig}
\item{6.5 \Rarr{} wenn $f,g:\R→\R$ Polynomfunktionen, dann $\frac{f}{g}:D→\R$ stetig, wobei $D=\{x\eR|g(x)\neq 0\}$ (denn $(\frac{f}{g}=f·\frac{1}{g})$\qed}
\end{{enumerate}
\sS{Satz Stetigkeit der Komposition}
Sei $f: C \to \R$, $g: D \to \R$ Funktionen mit $f(C) \subseteq D$. Wenn $f$ stetig in $x \in D$ und g stetig in $f(x)$ dann ist:\\
$g \cird f: C \to \R$ stetig in x.\\
\bew mit Folgenstetigkeit:
Sei $x_n \to x$ mit $x_n \in C$\\
$f$ stetig in $x \Rarr \ f(x_n) \to f(x)$\\
$g$ stetig in $f(x) \Rarr g(f(x_n)) \to g(f(x))$\\
d.h. $(g \circ f)(x_n) \to (g \circ f)(x)$\\
also ist $g \circ f: C \to \R$ stetig in $x$ \qed
\sS{Definition (Konvergenz bei Funktionen)}
Sei $D\subseteq\R$ und $f:D→\R$ Funktion\\
\begin{enumerate}
\item{Ein $a\eR\cup\{-∞,∞\}$ heißt Berührpunkt von $D$ wenn es eine Folge $(x_n)$ mit $x_n\in D$ und $x_n→a$ gibt
\bem
Jedes $a\in D$ ist Berührpunkt von $D$ (wähle konstante Folge $x_n=a$)}
\item{Angenommen, $a$ ist ein Berührpunkt von $D$\\
Schreibe $\ds\lim_{x→a}f(x)=y$ wenn gilt:\\
Für jede Folge $(x_n)$ mit $x_n→a$ und $x_n\in D$ gibt $f(x_n)→y$}
\item{Angenommen, $a\neq ∞$ ist eine Berührpunkt von $D\cap(a,∞)$ SKIZZE \\% SKIZZE
Schreibe $\ds\lim_{x\searrow a}f(x)=y$ wenn gilt:\\
Für jede Folge $(x_n)$ mit $x_n→a$ und $x_n\in D$ und $x_n>a$ gilt $f(x_n)→y$}
\item{Angenommen, $a\neq -∞$ ist eine Berührpunkt von $D\cap(-∞,a)$ Schreibe $\ds\lim_{x\nearrow a}f(x)=y$ wenn für jede Folge $(x_n)$ mit $x_n→a$ und $x_n\in D$ und $x_n<a$ gilt $f(x_n)→y$
\bsp
$D=\R\bsc\{0\}\ f:D→\R,\ f(x)=\frac{|x|}{x}$ SKIZZE\\
\underline{Bemerke} f(x)=$\left\{\begin{array}{lcl}1 & \text{wenn} & x>0\\-1 & \text{wenn} & x<0\end{array}\right.$\\
$\ds\lim_{x\nearrow a}f(x)=-1,\ \lim_{x\searrow a}f(x)=1,\ \lim_{x→0}f(x)existiert nicht$\\
\ul{Vorher} $a = 0$ ist Berührpunkt vpn $D$ und $D \cap (0, \infty)$ und $D \cap (- \infty , 0)$, denn $\frac{1}{n} \to 0$ und $-\frac{1}{n} \to 0$}
\end{enumerate}
Sei $g: \R \to \R$, $g(x) = x^3$\\
$\infty , - \infty$ sind Berührpunkte von $D = \R$\\
$\ds\lim_{x \to \infty} g(x) = \infty$\\
$\ds\lim_{x \to -\infty} g(x) = -\infty$\\ 
\bem{Umformulierung von Satz 6.4}
Eine Funktion $f: D \to \R$ ist stetig in $a \in D \\ $
$\equ \quad \ds\lim_{x \to a} f(x) = f(a)$
\uS{Sätze über stetige Funktionen}
\Def
Eine Funktion $f:D→\Re$ heißt nach oben \ul{beschränkt} wenn die Menge $f(D)$ nach oben beschränkt ist, d.h. es gibt $C\eR$ mit $f(x)\leq C$ für alle $x\in D$\\
$f$ heißt nach unten beschränkt, wenn $f(D)$ nach unten beschränkt\\
$f$ heißt \ul{beschränkt}, wenn $f)$ nach oben und nach unten beschränkt\\
\sS{Definition}
Sei $M \in \R$ eine nicht-leere Teilmenge wenn $M$ nach oben unbeschränkt, schreibe $sub(M) = \infty$ (Sprich: uneigentliches Supremum)\\
\Satz
Sei $a,b\eR$ mit $a\leq b$ und $f:[a,b]→\R$ eine \ul{stetige} Funktion, dann ist $f$ beschränkt und nimmt ihr Maximum und Minimum an, d.h. es gibt $x_{min},x_{max}\in[a,b]$ mit: $f(x_{min}\leq f(x) \leq (x_{max})$ für jedes $x\eR$
\bsp
\begin{enumerate}
\item{$f: (0, 1) \to \R, \quad f(x) = \frac{1}{x}$ % Graph
$\ds\lim_{x \to 0} f(x) = \infty$\\
Somit $f$ nicht nach oben beschränkt}
\item{$g(0,1) \to \R, \ g(x) = x$\\
$g$ beschränkt. $g((0,1)) = (0,1)$\\
$sup \{\g(x) | x \in (0, 1) \} = 1$\\
Aber $g(x) < 1$ für alle $x \in (0, 1)$. Also nimmt g nicht ihr Maximum an.}
\end{enumerate}
\sss{Beweis von 6.11}
Sei $y:=sup\{f(x)|x\in D}\eR \cup \{∞\}$\\
Wähle eine Folge $(x_n)$ mit $x_n\in D$ und $f(x_n)→y$ für $n→∞$\\
Bolzano-Weierstraß \Rarr{} Es gibt eine konvergente Teilfolge $(x_{n_k})_{k\geq 0}$ von $(x_{n})_{k\geq 0}$ \marginpar{Die Folge $(x_n)$ ist beschränkt $D=[a,b]$}\\
Sei $x_{n_k}→x$ für $k→∞$ $a\leq x_n\leq b$ für alle $n$ \Rarr $a\leq x\leq b,\ x\in D=[a,b]$\\
\einruck{Und dann gilt:}{$f()x_{n_k})→y$ für $k→∞$ (Teilfolge einer konvergenten Folge)\\$f()x_{n_k})→f(x)$ für $k→∞$ weil f stetig}
Also $y = f(x)$\\
Insbesondere $y \neq \infty$ also $f$ beschränkt\\
Für alle $x' \in D$ gilt $f(x') \leq sup \{f(D)\} = y = f(x)$\\
Setze $x_{max} := x$. Dann $f(x') \leq f(x_{max})$ für alle $x' \in D$\\
Anfang findet man $x_{min}$ \qed
\sS{Satz (Zwischenwertsatz)}
Sei $a\leq b$ und $f:[a,b]→\R$ stetig.\\
Wenn $y\eR$ zwischen $f(a)$ und $f(b)$  liegt, d.h. $f(a)\leq y \leq f(b)$ oder $f(a)\geq y \geq f(b)$\\
Dann gibt es ein $x\in[a,b]$ mit $f(x)=y$ GRAPH\\\newpage
% Kopfzeile beim Kapitelanfang:
\fancypagestyle{plain}{
%Kopfzeile links bzw. innen
\fancyhead[L]{\calligra\Large Vorlesung Nr. 13}
%Kopfzeile rechts bzw. außen
\fancyhead[R]{\calligra\Large 22.11.2012}
}
%Kopfzeile links bzw. innen
\fancyhead[L]{\calligra\Large Vorlesung Nr. 13}
%Kopfzeile rechts bzw. außen
\fancyhead[R]{\calligra\Large 22.11.2012}
% **************************************************
%
\wdh
Zwischenwertsatz\\*
Sei $a\leq b,\ f:[a,b]→\R$ stetig\\*
Sei $y\eR$ zwischen $f(a)$ und $f(b)$ d.h. $f(a)\leq y\leq f(b)$ oder $f(a)\geq y\geq f(b)$\\*
Dann gibt es ein $x\in[a,b]$ mit $f(x)=y$\\*
SKIZZE
\Bew {Intervallschachtelung}
Starte mit $[a_0,b_0]=[a,b]$\\*
Definiere unendliche Kette von Intervallen\\*
$[a_0,b_0]\supseteq [a_1,b_1]\supseteq [a_2,b_2]\supseteq …$\\*
So dass $[b_n-a_n]=2^{-n}[b_0,a_0]$ und $y$ zwischen $f(a_n)$ und $f(b_n)$ liegt.\\*
Annahme: $f(a)\leq y\leq f(b)$ (Anderer Fall $f(a)\geq y\geq f(b)$ analog)\\*
Angenommen, $[a_n,b_n]$ ist konstruiert so dass $[b_n-a_n]=2^{-n}(b_0,a_0)$ und $f(a_n)\leq y\leq f(b_n)$\\*
Betrachte $m:=\frac{a_n+b_n}{2}$, Wenn $f(m)\geq y$ dann setze $[a_{n+1},b_{n+1}]:=[a_n,m]$\\*
Wenn $f(m)<y$ dann setze $[a_{n+1},b_{n+1}]:=[m,b_n]$\\*
Dann gilt in beiden Fällen:
$$b_{n+1}-a_{n+1}=\frac{1}{2}(b_n-a_n)=2^{-1}·2^{-n}(b_0-a_0)=2^{-n-1}(b_0-a_0) \text{ und }f(a_{n+1})\leq y\leq f(b_{n+1})$$
%
\sss{Idee}
Folge von Intervallen $[a_n, b_n]$ "konvergent" gegen gesuchtes $x$.\\*
Die Folge $(a_n)_{n \geq 0}$ ist monoton wachsend und beschränkt, ($b$ ist obere Schranke)
\Rarr\\ $(a_n)$ konvergiert, sei $x:=\lim_{\nif} a_n$\\*
Die Folge $(b_n)_{n \geq 0}$ ist monoton fallen und beschränkt \Rarr konvergent nach 4.2\\*
Sei $x' = \lim_{n \to \infty} b_n$\\
$x' - x = \lim_{n \to \infty} (b_n - a_n)$\\*
$= \lim_{n \to \infty} (2^{-n} \cdot (b_0 - a_0)) = 0$\\*
also $x = x'$ $f(x) = ?$\\
$f$ stetig $\Rarr f(x) = \lim_{n \to \infty} f(a_n) = \lim_{n \to \infty} f(b_n)$\\
Wegen $f(a_n) \leq y \leq f(b_n)$ für alle $n$ gilt $f(x) = \lim_{n \to \infty} f(a_n) \leq y \leq \lim_{n \to \infty} f(b_n)$\\*
	$\Rarr f(x) = y$ 
%
\bem
Weil $a\leq a_n\leq b$ gilt $a\leq x\leq b$ d.h. $x\in[a,b]$\qed
\uS{Anwendung}
\sS{Satz Sichere Nullstellen}
Sei \nN{} \ul{ungerade}, $f:\R→\R$
$$f(x)=x^n+a_{n-1}·x^{n-1}+…+a_0$$
Dann hat $f$ eine Nullstelle, d.h. es gibt $x\eR$ mit $f(x)=0$
\bew
Für $x\neq 0$ betrachte
$$g(x)=\frac{1}{x^n} \cdot f(x)=1+\frac{a_{n-1}}{x}+\frac{a_{n-2}}{x^2}+…+\frac{a_0}{x^n}$$
Für $x→∞$ ist $g(x)→1$\\*
Für $x→-∞$ ist $g(x)→1$\\*
D.h. es gibt $a\eR$ mit $a>0$ und
$$x\geq a \Rarr g(x)>0$$
$$x\geq -a \Rarr g(x)>0$$
%
\ul{Also} $x \geq a \Rarr f(x) = x^n \cdot g(x) > 0$\\*
	$x \geq a \Rarr f(x) = \underset{\overset{\uparrow}{x^n < 0}}{x^n} \cdot \underset{\overset{\uparrow}{>0}}{g(x)} < 0$\\*
	$f(-a) < 0 < f(a)$\\*
	Zwischenwertsatz \Rarr ergibt $x \in [-a, a]$ mit $f(x) = 0$

\sS{Satz Ergänzung Zwischenwertsatz} % 6.14
	Sei $f: D \to \R$ \ul{stetig} und $D \subseteq \R$ ein nicht-leeres Intervall. Dann ist $f(D) = \{f(x) | x \in D\}$ auch ein Intervall (d.h. hat keine Lücken.)
\bem
	Hier sind auch uneigentliche Intervalle zugelassen. (z.B. $(0, \infty)$)
%
\bsp
$f(x)=x^3-x+20$ GRAPH
\bsp
Bedingung "$n$ ungerade" ist wesentlich, denn $f(x)=x^2+1$ hat keine Nullstelle
% Satz + bem
\bsp
$$f:(0,1)→\R,\ f(x)=\frac{1}{x}$$
$$f(D)=(1,∞)$$ GRAPH
\bsp
$$f:(-1,1)→\R,\ f(x)=x^2$$
$$f(D)=[0,1)$$ GRAPH
\bew
$f:D→\R$ stetig\\*
Sei $a:=inf(f(D))\eR\cup\{-∞\}$\\*
\phantom{Sei }$a:=inf(f(D))\eR\cup\{-∞\}$\\*
Angenommen $y\eR$ mit $a<y<b$ d.h. $x\in(a,b)$\\*
Es gibt $x_1,x_2\in D$ mit $a<f(x_1)<y<f(x_2)<b$\\*
Zwischenwertsatz \Rarr{} es gibt $x$ zwischen $x_1,x_2$\\*
(\Rarr $x\in D$ weil $D$ Intervall) mit $f(x)=y$\\*
Also $(a,b) \subseteq f(D)$\\*
Dann ist $f(D)$ eines der Intervalle $(a,b),[a,b),(a,b],\underset{\overset{\uparrow}{\text{nur wenn }a\neq -∞, b\neq ∞}}{[a,b]}$\qed
% 
\sS{Satz Umkehrfunktion}
	Sei $D$ ein Intervall, $f: D \to \R$, stetig, streng monoton wachsend oder fallend. Sei $D' = f(D)$ (Intervall nach 6.14)\\*
	Dann ist die Abbildung $f: D \to D'$ bijektiv und die Umkehrabbildung $f^{-1}: D' \to D$ ist stetig und streng monoton wachsend, bzw. fallen.
%
\bew
Die Abbildung $f:D→D'$ ist
\begin{itemize}
\item{surjektiv nach Definition von $D'$}
\item{streng monoton \Rarr{} injektiv}
\item{also bijektiv. Somit existiert $f^{-1}:D'→D$}
\end{itemize}
\sss{Annahme}
$f$ streng monoton wachsend (fallend analog)
%
\beh
$f^{-1}$ ist streng monoton wachsend, d.h. gegeben $x_1,x_2\in D'$ mit $x_1<x_2$ zeige:
$$f^{-1}(x_1)<f^{-1}(x_2)$$
Angenommen $f^{-1}(x_1)\geq f^{-1}(x_2)$ \Rarr{} $f$ monoton wachsend \Rarr{} $x_1=f(f^{-1}(x_1))\geq f(f^{-1}(x_2))=x_2$ \Rarr{} Widerspruch\\*
also $f^{-1}(x_1)<f^{-1}(x_2)$ \Rarr{} $f^{-1}$ streng monoton wachsend
%
\beh
	$f^{-1}$ ist stetig. Gegeben $x \in D$\\*
	Annahme $x$ ist kein Randpunkt des Intervalls $D'$\\*
	Gegeben sei $\e > 0$ Suche $\delta$ mit (Stetigkeitsdefinition)\\*
	$y := f^{-1}(x) \in D$ ist kein Randpunkt (weil $f,\ f^{-1}$ bijektiv und streng monoton.)\\*
	%Graph
	Wähle $\e'\leq \e$ mit $\e > 0$ sodass $[y-\e', y+\e'] \subseteq D'$\\*
	$f(y - \e') < f(y) = x \Larr y - \e' < y$\\*
	also $f(y - \e') = x - \delta_1 \qquad \delta_1 > 0$\\*
	$genauso f(y + \e') = x + \delta_2 \qquad \delta_2 > 0$ \\*
	%Graph
	Sei $\delta = min(\delta_1, \delta_2)$
\beh
	Wenn $z \in D'$ mit $|z - x| < \delta$ dann $|f^{-1}(z) - f^{-1}(x)| < \e$
%
\bew
$$x+δ_1\leq x-δ<z<x+δ\leq x+δ-2$$
$f^{-1}$ streng monoton wachsend \Rarr
$$f^{-1}(x)-ε'=f^{-1}(x-δ_1)<f^{-1}(z)<f^{-1}(x+δ-2)=y+ε'=f^{-1}(x)+ε'$$
$$\Rarr |f^{-1}(z)-f^{-1}(x)|<ε'\leq ε \text{\Rarr{} Behauptung}$$
Somit $f^{-1}$ stetig in $x$\\*
Falls $x$ Randpunkt: Betrachte $[x,x+δ]$ bzw. $[x-δ,x]$ wieder analog\qed
%
\uS{2 Anwendungen}
\sS{Beispiel}
	Sei $k \eN$\\*
	Die Abbildung $f: \R_{\geq 0} \to \R_{\geq 0} f(x) = x^k$\\*
	%Graph
	\ul{Bekannt} $f$ ist stetig streng monoton wachsend.\\*
	$f(0) = 0, \qquad \lim_{x \to \infty} x^k = \infty$\\*
	$D:= [0, \infty) = \R_{\geq 0}$\\*
	$f(D) = D' = [0, \infty)$\\*
	6.15 \Rarr\ $f$ hat \ul{stetige} und streng monoton wachsende Umkehrfunktionen\\*
	$f^{-1} : [0, \infty) \to [0, \infty)$\\*
	Bezeichnung: $f^{-1}(x) = \sqrt[k]{x}$
%
\uS{Logarithmus und allgemeine Potenzen}
\sS{Satz}
Die Exponentialfunktion $exp:\R→\R_{>0}=(0,∞)$ ist stetig, streng monoton wachsend und $exp(\R)=\R_{>0}$
\bew
Bekannt: $exp$ stetig, streng monoton wachsend.
Für $x>0$ ist $$exp(x)=1+x+\frac{x^2}{2}+…\geq 1+x$$
also gilt $$\lim_{x→∞} exp(x)=∞$$
$$exp(-x)=\frac{1}{exp(x)}\Rarr \lim_{x→-∞}exp(x)=\lim_{x→∞}\frac{1}{exp(x)}$$
Somit $exp(\R)=(0,∞)$\qed\\*
Folge mittels 6.15: $exp:\R→\R_{>0}$ ist bijektiv und die Umkehrfunktion $exp^{-1}:=log:\R_{>0}→\R$ ist stetig, streng monoton wachsend, bijektiv \footnote{$exp^{-1}=log$ heißt Logarithmusfunktion} konkret: $log(x)=y \equ x=exp(y)$ GRAPH
\newpage
% Kopfzeile beim Kapitelanfang:
\fancypagestyle{plain}{
%Kopfzeile links bzw. innen
\fancyhead[L]{\calligra\Large Vorlesung Nr. 14}
%Kopfzeile rechts bzw. außen
\fancyhead[R]{\calligra\Large 26.11.2012}
}
%Kopfzeile links bzw. innen
\fancyhead[L]{\calligra\Large Vorlesung Nr. 14}
%Kopfzeile rechts bzw. außen
\fancyhead[R]{\calligra\Large 26.11.2012}
% **************************************************
\wdh{Logarithmus und allgemeine Potenzen}
Die Funktion $exp: \R \to \R_{\geq 0}$ ist stetig, streng monoton wachsend, bijektiv.\\*
Die Umkehrfunktion heißt \ul{Logarithmus},\\*
$$log = exp^{-1}: \R_{\geq 0} \to \R$$
explizit definiert durch $log(x) = y\ \equ\ x = exp(y)$\\*
$\underset{\text{\scriptsize{nach satz 6.5}}}{\Rarr}$ log ist stetig, streng monoton wachsend, bijektiv.
% GRAPH exp(x) | GRAPH log(x)
\sS{Satz (Eigenschaften des Logarithmus)}
\begin{enumerate}
\item{$log(1)=0$}
\item{$log(x·y)=log(x)+log(y)$}
\item{$\lim\limits_{x→0}log(x)=-∞$}
\item{$\lim\limits_{x→∞}log(x)=∞$}
\end{enumerate}
\bew
Folgt aus Eigenschaften von $exp$, Details: Übung
\sss{Erinnerung:}
also $a>0,\ n\eZ,\ m\eN$ ist $a^{\frac{n}{m}}:=\sqrt[m]{a^n}$
\sS{Lemma}
Es gibt $a^{\frac{n}{m}} =  exp(\frac{n}{m} \cdot log(a))$\\*
\bew
	\begin{enumerate}
	\item{Sei $n \geq 0$:\\*
	$$exp(n \cdot log(a)) = exp(\overbrace{log(a) + log(a) + …+ log(a)}^{n}) = exp(log(a)) \cdot … \cdot exp(log(a))$$}
	\item{Sei $n < 0$\\*
	 $$exp(n \cdot log(a)) = exp(\underbrace{-n}_{-n > 0} \cdot log (a)) = (a^{-1})^{-1} = a^n$$}
	 \item{Rechne: $$exp(\frac{n}{m} log(a))^m = exp(m \cdot \frac{n}{m} \cdot log(a)) = exp(n \log(a)) = a^n$$}
	\end{enumerate}	 
	$\sqrt[m]{\ } \Rarr exp(\frac{n}{m} log(a)) = \sqrt[m]{a^n} = a^{\frac{n}{m}}$
\sS{Definition}
Sei $a>0,\ x\eR$ setze $a^x:=exp(x·log(a))$\\*
\sS{Bemerkung}
Die Regeln der Potenzrechnung gelten:
$$a^{x+y}=a^x·a^y,\qquad a^{x·y}=(a^x)^y$$
\bew
$$a^{x+y}=exp((x+y)·log(a))=exp(x·log(a))·exp(y·log(a))=a^x·a^y$$
$$a^{x·y}=exp(x·y·log(a))=(a^x)^y=exp(y·log(exp(x·log(a))))\footnote{$log\circ exp= id$}=exp(y·x·log(a))$$\qed
\bem
Eulersche Zahl
$$e:=\footnote{$log(e)=1$}exp(1)=2{,}7…\qquad e^x=exp(x·log(e))=exp(x)$$

\sS{Definition Logarithmusbasis}
Sei $a > 1 \qquad x \in \R$ 
$$log_a (x) = \frac{log(x)}{log(a)}$$
\bem
$a \neq 0 \Rarr log\left(a\right) \neq 0$\\*
Dann:
$$a^{log_a \left(x\right)} = exp\left(log_a \left(x\right) \cdot log\left(a\right)\right) = exp\left(\frac{log\left(x\right)}{log\left(a\right)} \cdot log(a)\right) = exp(log(x)) = x$$

\uS{Gleichmäßige Stetigkeit}
\wdh
$f: D \to \R$ stetig an $x \in D$ wenn gilt:\\*
Für jedes $\e > 0$ gilt $\delta > 0$ mit wenn $y \in D$ mit $|x - y| < \delta$ dann $|f(x) - f(y)| < \e$\\*
Hier hängt $\delta$ im allgemeinen von $\e$ und $x$ ab!

\sS{Definition gleichmäßige Stetigkeit:}
Eine Funktion $f:D→\R$ heißt gleichmäßig stetig wenn gilt:\\*
Für jedes $ε>0$ gibt es ein $δ>0$ so dass für alle $x,y\in D$ mit $|x-y|<δ$ gilt:
$$|f(x)-f(y)|<ε$$
\sss{Wesentlich:}
$δ$ hängt nur von $ε$, nicht von $x$ ab.
\bsp
$D=(0,1)\qquad f:D→\R,\ f(x)=\frac{1}{x}$\\*
GRAPH $f$ stetig, aber nicht gleichmäßig stetig
\bew
Wähle $ε=1$. Angenommen es gibt $δ>0$ mit $|x-y|<δ\Rarr |f(x)-f(y)<1$
Wähle: $x=\frac{1}{n},\ y\frac{1}{n+1}$ so dass $\frac{1}{n·(n+1)}<δ$
$$|x-y|=\left|\frac{1}{n}-\frac{1}{n+1}\right|=\left|\frac{n+1-n}{n·(n+1)}\right|=\frac{1}{n·(n+1)}$$
Dann $$|f(x)-f(y)|=|n-(n+1)|=1$$
Das zeigt: $δ$ existiert nicht.

\sS{Satz}
Seien $a \leq b$ reelle Zahlen\\*
Jede stetige Funktion $f: [a, b] \to \R$ ist gleichmäßig stetig.\\*
\bew
	Angenommen, $f$ ist nicht gleichmäßig stetig:
	Es gibt ein $\e > 0$ sodass für jedes $\delta > 0$ zwei Zahlen $x, y \in D$ existieren, mit $|x - y| < \delta$ und $|f(x) - f(y)| \geq \e$\\*[4pt]
	$f(y)| \geq \e$ (*)\\*[4pt]
Bolzano-Weierstraß \Rarr\ Es gibt eine Teilfolge $({x_n}_k)_{k\geq 0}$, die konvergiert. Sei $x = \ds\lim_{k\to \infty}({x_n}_k)_{k\geq 0} \in  [a, b] = D$\\*
Dann $$\ds\lim_{k \to \infty} ({y_n}_k)_{k\geq 0} = \ds\lim_{k \to \infty} (({y_n}_k)_{k\geq 0} - ({x_n}_k)_{k\geq 0}) = 0 + x = x$$
$${x_n}_k)_{k\geq 0} \to x, \ {y_n}_k)_{k\geq 0} \to x \ f: k \to \infty $$
$f$ stetig $$\Rarr\ f({x_n}_k)_{k} \to f(x), f({y_n}_k)_{k} \to f(x) \ f \ k \to \infty $$
$$\Rarr\ (f({x_n}_k)_{k}) - f({y_n}_k)_{k})  \to f(x) - f(x) = 0$$
Widerspruch zu (*) Also ist $f$ gleichmäßig stetig. \qed

\chapter{Komplexe Zahlen und Trigonometrie}
Der Körper \C{} der komplexen Zahlen\\*
Mangel von \R{}: Die Gleichung $x^2=-1$ hat keine Lösung\\*
\sS{Definition Komplexe Zahlen}
Es sei $\C=\R^2=\{(x,y)|x,y\eR\}$ mit folgender Addition und Multiplikation:
$$(x,y)+(x',y'):=(x+x',y+y')$$
$$(x,y)·(x',y'):=(x·x'-y·y',x·y'-y·x')$$
Addition gleich der Vektoraddition in $\R^2$ GRAPH
\sS{Satz: \C\ ist Körper}
\C{} ist ein Körper mit Null (0,0) und Eins (1,0)
\bew
Überprüfe Körperaxiome \ul{exemplarisch:}
\begin{enumerate}
\item{Assotiativgesetz der Multiplikation\\*
Gegeben $(x,y),\ (x',y'),\ (x'',y'')\eC$
\begin{align*}
((x,y)·(x',y'))&·(x'',y'')=(x·x'-y·y',x·y'-x·y'+y·x')(x'',y'')\\
&=(x·x'·x''-y·y'·x''-x·y'·y''-y·x'·y'',x·x'·y''-y·y'·y''+x·y'·x''+y·x'·x'')\\ \\
(x,y)((x',y')(x'',y''))&=…\text{ erhalte gleiches Ergebnis}
\end{align*}}
\item{Existenz vom Inversen:\\*
Sei $z = (x, y) \in \C, \ x, y \in \R, \ z \neq 0$ \\*[4pt]
Zeige es gibt ein $z^{-1} \in \C$ mit $z \cdot z^{-1} = (1,\ 0)$\\*[4pt]
$z \neq 0 \Rarr x \neq 0$ oder $y \neq 0 \equ x^2 + y^2 > 0$
$$w:= \left(\frac{x}{x^2 + y^2}, \frac{-y}{x^2 + y^2}\right)$$
Rechne $z \cdot w = (x, y) \cdot \left(\frac{x}{x^2 + y^2}, \frac{-y}{x^2 + y^2}\right)$
$=\left(\frac{x^2}{x^2 + y^2} - \frac{-y^2}{x^2 + y^2}, \frac{-yx}{x^2 + y^2} + \frac{yx}{x^2 + y^2}\right) = (1, 0)$
Also $w = z^{-1}$ \qed \\*
Weitere Axiome ähnlich.}
\end{enumerate}
\sss{Definition: imaginäre Einheit}
$i:=(0,1)$ (imaginäre Einheit)\\*
Dann ist $$i^2=(0,1)·(0,1)=(-1,0)\ \Rarr\ i^2+1=0$$
\bem
Für $x,x'\eR$ gilt:
$$(x,0)+(x',0)=(x+x',0)$$
$$(x,0)·(x',0)=(x·x',0)$$
Die Abbildung $\R→\C,\ x\mapsto(x,0)$ ist injektiv und verträglich mit $+,\ ·$\\*
$\leadsto$ Fasse \R{} mittels diese Abbildung als Teilmenge von \C{} auf, einschließlich der Körperstruktur\\
Dann $i^2=-1$.\\*
Für $(x,y)\eC$ gilt $(x,y)=(x,0)+(0,y)=(x,0)+(0,1)·(y,0)=x+i·y$
Jede komplexe Zahl $z$ hat eine eindeutige Darstellung $z=x+i·y$ mit $x,y\eR$.
\sss{Idee}
\C{} entsteht aus \R{} durch Hinzunahme einer Zahl $i$ mit $i^2=-1$\\
Interpretation der Multiplikation in \C:\\*
$$(x+i·y)(x'+i·y')=(x·x'+x·i·y'+i·y·x'+i·y·i·y')=(x·x'-y·y')+i(x·y'+y·x')$$
\sss{Definition}
Sei $z = x + iy \in \C$\\*
Realteil: Re(z) := x \\*
Imagnärteil: Im(z) := y\\*
Komplex konjugierte Zahl $\ol{z} = x - iy$\\*
Komplex konjugation = Spiegelung an der x-Achse.\\*
\ul{Definition:} Der Betrag von $z = x + iy \in \C$ ist $|z| = \sqrt{x^2 + y^2}$ Abstand von $0 = (0, 0)$ zu $z$\\*
\bem
\begin{enumerate}
\item{$\ds z·\ol{z}=(x+i·y)(x-i·y)=x^2+y^2+i(-x·y+y·x)=x^2+y^2=|z|^2$}
\item{Insbesondere gilt $z·\ol{z}\eR$ und $z·\ol{z}\geq 0$}
\item{$|z|=\sqrt{x^2+y^2}=\sqrt{z·\ol{z}}$ (sinnvoll wegen 2)}
\end{enumerate}\newpage
% Kopfzeile beim Kapitelanfang:
\fancypagestyle{plain}{
%Kopfzeile links bzw. innen
\fancyhead[L]{\calligra\Large Vorlesung Nr. 15}
%Kopfzeile rechts bzw. außen
\fancyhead[R]{\calligra\Large 29.11.2012}
}
%Kopfzeile links bzw. innen
\fancyhead[L]{\calligra\Large Vorlesung Nr. 15}
%Kopfzeile rechts bzw. außen
\fancyhead[R]{\calligra\Large 29.11.2012}
% **************************************************
%
\wdh
$\C=\R^2$ mit 
$$(x,y)+(x',y')=(x+x',y+y')$$
$$(x,y)·(x',y')=(x·x'-y·y',x·y'-y·x')$$
$$(x,y)=x+i·y\qquad i=(0,1)\qquad i^2=(-1)$$
\begin{tikzpicture}[domain=-1:2.5]
    \draw[very thin,color=gray] (-1,-0.5) grid (2.5,1.5);
    \draw[->] (-1,0) -- (2.5,0) node[right] {$x$};
    \draw[->] (0,-0.5) -- (0,1.5) node[above] {$i$};
    \draw node at(2,1) {$2+i$};
    \draw node at (0.8, 0.5) {$|2i|$};
    \draw[->] (0,0) -- (2,1);
\end{tikzpicture}
$$z=x+i·y \eC\ \Rarr\ Re(z)=x \quad Im(z)=y$$
$\overline{z}=x-i·y$
$$z·\overline{z}=(x+i·y)(x-i·y)=x^2-(i·y)^2=x^2-i^2·y^2=x^2·y^2$$
Abstand von 0 nach $z$ ist:
$$|z|=\sqrt{x^2+y^2}=\sqrt{z·\overline{z}}\footnote{\ul{Betrag} von $z$}$$
$$|2+i|=\sqrt{2^2+i^2}=\sqrt{5}$$%muss Wurzel 3 rauskommen

Berechnung von $z^{-1}$\\*
$$\frac{1}{z} = \frac{z}{\overline{z} \cdot z}$$ % Pfeil auf z {Reelle Zahl}
$$\frac{1}{2+i} = \frac{2 - i}{(2+i)(2-i)} = \frac{2-1}{5} = \frac{2}{5} = \frac{i}{5}$$
\sS{Lemma}
Sei $z,w\eC$ dann gilt:
\begin{enumerate}
\item{$$\ol{zw}=\ol{z}·\ol{w},\ \ol{z+w}=\ol{z}+\ol{w}$$}
\item{$z+\ol{z}=2 Re(z),\ z-\ol{z}=2i·Im(z)$\\*
$Re(z)=\frac{z+\ol{z}}{z}\qquad Re(z)=\frac{z-\ol{z}}{2i}$}
\end{enumerate}
%
\bew
\begin{enumerate}
\setcounter{enumi}{1}
\item{$$z=x+i·y$$
$$z+\ol{z}=(x+i·y)+(x-i·y)=2x=2Re(z)$$
$$z-\ol{z}=(x+i·y)-(x-i·y)=2iy=2i·Im(z)$$}
\setcounter{enumi}{0}
\item{$z = x+iy,\ w = a + ib$\\*
$$\ol{z+w} = \ol{x+iy + a + ib} = \ol{x+a + i(y+b)} = x + a - i(y + b) = (x - iy)+(a + ib) = \ol{z} + \ol{w}$$
$$\ol{zw} = \ol{(x+iy)(a + ib) = \ol{(ax - by) + i(ay + bx)} = (ax - by) - i(ay + bx) = |x|}$$
$$\ol{z} \cdot \ol{w} = \ol{(x+iy)} \cdot \ol{a+ib} = (x-iy) \cdot (a+ib) = ax - (-y)(-b) + i(a \cdot (-y) + (-b) \cdot x)$$
$$= ax - by -i(ay + bx) =  |x|$$}
\end{enumerate}

\bem
Aus 3) folgt: \\*
$$z = x+iy \Rarr |z| \leq |iy| = |x| + |y|$$
Erinnerung:\\*
$$|z| \leq |x|,\ |z| \leq |y| \text{ (aus der NR)}$$
\uS{Folgen und Reihen komplexer Zahlen}
\sS{Definition}
Sei $(c_n)_{n\geq 0}$ eine Folge komplexer Zahlen, $c \in \C$\\*
Die Folge $c_n$ konvergiert gegen $c$ wenn gilt:\\*
Für jedes $\e > 0$ gibt es ein $N \in \N$ so dass für jedes $n \geq \N$ gilt $|c - c_n| < \e$
\notat{$\ds\lim_{n \to \infty} c_n = c$ \\
		oder $c_n \to c$ für $\nif$}
\sS{Satz}
Sei $z,w\eC$
\begin{enumerate}
\item{$|z|\geq 0$, und $|z|=0$ \equiv{} $z=0$ (klar)}
\item{$|z·w|=|z|·|w|,\ |\ol{z}|=|z|$}
\item{$|z+w|\leq|z|+|w|$ (Dreiecksungleichung)}
\begin{tikzpicture}[domain=-1:2.5]
    \draw[very thin,color=gray] (-1,-0.5) grid (2.5,1.5);
    \draw[->] (-1,0) -- (2.5,0) node[right] {$x$};
    \draw[->] (0,-1) -- (0,1.5) node[above] {$i$};
    \draw node at (0.4,1) {$w$};
    \draw node at (1.2, -0.5) {$z$};
    \draw node at (1.6, 0.5) {$z+w$};
    \draw[->] (0,0) -- (0.4,1);
    \draw[->] (0,0) -- (1.2,-0.5);
    \draw[->] (0.4,1) -- (1.2,-0.5);
\end{tikzpicture}
\end{enumerate}
\bew
\begin{enumerate}
\setcounter{enumi}{1}
\item{$$|z·w|=\sqrt{zw·\ol{zw}}\underset{7.3}{=}\sqrt{z·w·\ol{z}·\ol{w}}=\sqrt{z·\ol{z}·w·\ol{w}}=\sqrt{z·\ol{z}}\sqrt{w·\ol{w}}=|z|·|w|$$
beide reell $\geq 0$
Nachtrag rechts unten 3. tafel}
\item{$$|z+w|=(z+w)·(\ol{z+w})\underset{7.3}{=}(z+w)·(\ol{z}+\ol{w})=z·\ol{z}+z·\ol{w}+w·\ol{z}+w·\ol{w}=\footnote{$\ol{z·\ol{w}}=\ol{z}·\ol{\ol{w}}=\ol{z}·w=w·\ol{z}$}z·\ol{z}+\underbrace{z·\ol{w}+\ol{z·\ol{w}}}_{2·Re(z·\ol{w})}+w·\ol{w}$$
$$=z·\ol{z}+2·Re(z·\ol{w})+w·\ol{w}\leq |z|^2+2·|z|·|w|+|w|^2=(|z|+|w|)^2\footnote{Für jedes $u=c+d·i\eC$ gilt $|u|=\sqrt{c^2+d^2}\geq \sqrt{c^2}=|c|=|Re(u)|$}$$
$$\sqrt{}\ \Rarr\ |z+w|=|z|+|w|$$\qed}
\end{enumerate}
\bem
\begin{enumerate}
\item{Es gilt $c_n \to c \equ \ol{c_n} \to \ol{c}$}
\item{Wenn $c_n \to c$ dann $|c_n| \to |c|$}
\end{enumerate}
\bew
\begin{enumerate}
%% Er war einsam aber schneller.
\item{$|\ol{c_n} - \ol{c}| = |\ol{c_n - c}| = |c_n - c| \Rarr beh$}
\item{Übung}
\end{enumerate}


\sS{Satz:}
Sei $c_n=a_n+i*b_n,\quad c=a+ib$\\*
Es gilt $c_n→c\ \equ \ a_n→a$ und $b_n→b$
\bew
\begin{description}
\item["\Rarr"]{Es gilt:
$$|a_n-a|=|Re(c_n-c)|\leq |c_n-c|$$
$$|b_n-b|=|Im(c_n-c)|\leq |c_n-c|$$
also gilt: $|c_n-c|<ε\ \Rarr \ |a_n-a|<ε$ und  $|b_n-b|<ε$ Somit gilt die "\Rarr"}

\item[$",\Larr"'$]{Verwende $|c_n - c| \leq |a_n - a| + |b_n - b| (*)$\\*
Gegeben $\e > 0$\\*
Es gibt $N \in \N$ so dass für jedes $n \geq N$:\\
$$|a_n - a| < \frac{\e}{2}, \ |b_n - b| < \frac{\e}{2}$$
Dann gilt für $n \geq N$:\\*
$|c_n - c| \leq \frac{\e}{2} + \frac{\e}{2} = \e$\\*}
\end{description}
\sS{Definition}
Eine Folge komplexer Zahlen $(c_n)_{n\geq 0}$ heißt Cauchy-Folge, wenn für jedes $ε>0$ ein $N\eN$ existiert, so dass gilt:\\*
Für alle $n,m\geq N$ gilt $|c_n-c_m|<ε$
\sS{Satz}
Sei $c_n=a_n+ib_n$\\*
$(c_n)$ ist Cauchy-Folge \equ{} $(a_n)$ und $(b_n)$ sind Cauchy-Folgen
\bew
Genau wie Beweis von 7.6 verwende:
$$|a_n-a_m|\leq |c_n-c_m|$$
$$|b_n-b_m|\leq |c_n-c_m|$$
$$|c_n-c_m|\leq |a_n-a_m|+|b_n-b_m|$$\qed
%Satz 7.9
\sS{Satz}
Wenn $c_n→c, c'_n→c'$ konvergente Folgen komplexer Zahlen sind, dann gilt:
\begin{enumerate}
\item{$c_n+c_m→c+c'$}
\item{$c_n·c_m→c·c'$}
\item{Wenn $c\neq 0$ dann $c_n\neq 0$ für fast alle $n$ und $\frac{1}{c_n}→\frac{1}{c}$}
\end{enumerate}
\bew
Analog zum Fall reeller Folgen\qed
\sS{Definition}
Eine Reihe komplexer Zahlen\\*
$\ds\sum_{n=0}^{∞}c_n$ heißt \ul{absolut konvergent, wenn die Reihe $\ds\sum_{n=0}^{∞}|c_n|$ konvergent ist}
\sS{Satz}
Eine Folge komplexer Zahlen $(c_n)$ konvergiert \equ{} $(c_n)$ ist Cauchy-Folge
\bew
	Sei $c_n = a_n + i \cdot b_n$\\*
	$(c_n)$ konvergiert $\underset{7.6}{\equ} \ (a_n$ und $(b_n)$ konvergieren $\equ \ (a_n)$ und $(b_n)$ sind Couchy-Folgen $\underset{7.8}{\equ}$ $(c_n)$ ist Cauchy-Folge \qed
\sS{Satz}
Sei $c_n\eC$ für \nN\\*
\begin{enumerate}
\item{Majorantenkriterium:\\*
Wenn reelle Zahlen $a_n$ existieren, so dass $|c_n|\leq |a_n|$ und $\sum a_n$ konvergiert, dann konvergiert auch $\sum c_n$ absolut}
\item{Quotientenkriterium:\\*
Wenn eine reelle zahl $b\eR$ existiert mit $0\leq b<1$, so dass $|c_{n+1}\leq b·|c_n|$ für fast alle \nN\\*
Dann konvergiert $\ds\sum_{n\geq 0} c_n$ absolut}
\end{enumerate}
\sS{Satz}
Seien $\sum c_n$ und $\sum d_n$ zwei konvergente Reihen komplexer Zahlen, $\sum c_n=c,\ \sum d_n=d$\\*
Wenn eine der Reihen absolut konvergiert, konvergiert auch das Cauchy-Produkt mit Grenzwert $c·d$

\sS{Satz}
Wenn die Reihe $\sum_{n = 0}^{\infty} c_n$ absolut konvergent ist, dann ist die konvergent.
\bew
Sei $c_n = a_n + i \cdot b_n$ $$|c_n| \geq |a_n|\ \ |c_n| \geq |b_n|$$
$\sum |c_n|$ konvergent \Rarr $\sum |a|,\ \sum |b_n|$ konvergent. (Majorantenkriterium)\\*
d.h.: $\sum a_n,\ \sum |b_n|$ konvergiert absolut\\*
$\Rarr \sum a_n,\ \sum b_n$ konvergiert.\\*
$\overset{7.7}{\Rarr} \sum c_n$ konvergent \qed

\ul{Zusatz}\\*
Angenommen $\sum c_n$ konvergiert absolut, dann: 
$$\left| \sum_{n \geq 0} c_n \right| = \sum_{n \geq 0} c_n$$
(Dreiecksungleichung für $\infty$ viele Summanden)
\bew
Gewöhnliche Dreiecksungleichung \Rarr
$$|c_0 + c_1 + ... + c_n| \leq |c_0| + |c_1 + ... + c_n|$$
$$\text{(Partialsummen)} \leq ... \leq |c_0| + |c_1| + ... + |c_n| (*)$$
Wenn $c = \sum c_n$, dann $c = \ds\lim_{n \to \infty} (c_0 + c_1 + ... + c_n)$ (Definition des Grenzwertes einer Reihe)\\*
$\Rarr |c| = \ds\lim_{n \to \infty} |c_0 + c_1 + ... + c_n| = \left| \sum_{n = 0}^{\infty} c_n \right| \leq \lim_{n \to \infty} |c_0| + |c_1| + ... + |c_n| = \sum_{n = 0}^{\infty} |c_n|$\newpage
% Kopfzeile beim Kapitelanfang:
\fancypagestyle{plain}{
%Kopfzeile links bzw. innen
\fancyhead[L]{\calligra\Large Vorlesung Nr. 16}
%Kopfzeile rechts bzw. außen
\fancyhead[R]{\calligra\Large 03.12.2012}
}
%Kopfzeile links bzw. innen
\fancyhead[L]{\calligra\Large Vorlesung Nr. 16}
%Kopfzeile rechts bzw. außen
\fancyhead[R]{\calligra\Large 03.12.2012}
% **************************************************
\wdh
	Eine Folge komplexer Zahlen $(c_n)n\geq 0$ konvergiert gegen $c \in \C$ wenn gilt:\\
	Für jedes $\e > 0$ gibt es ein $N \in \N$ so dass $n \geq N \Rarr |c_n - c| < \e$\\
	Eine Reihe $\ds\sum_{n=0}^{\infty} c_n$ mit $c_n \in \C$ heißt \ul{absolut} konvergent, wenn die reelle Reihe $\ds\sum_{n=0}^{\infty} |c_n|$ konvergiert.\\
	Absolute Konvergenz \Rarr{} Konvergenz
	Nach 7.15:\\
	Seien $\ds\sum_{n\geq0} c_n = c$ und $\ds\sum_{n\geq0} c'_n = c'$ konvergente komplexe Reihen, mindestens eine absolut konvergent. Dann konvergiert hier Cauchy-Produkt $\ds\sum_{n\geq0} d_n$ mit dem Grenzwert $c \cdot c'$\\
	\ul{Erinnerung:}\\
	$d_n = \ds\sum_{n\geq0} c_k \cdot c_{n - k}$\\
\bew
	Wörtlich wie bei reellen Reihen. \qed

\uS{Die komplexe Exponentialfunktion}
\sS{Satz}
Für $z\eC$ konvergiert die "Exponentialreihe"
$$\sum_{n=0}^{∞}\frac{z^n}{n!}=1+\frac{z}{1}+\frac{z^2}{2}+\frac{z^3}{6}+…$$absolut (Somit konvergiert sie)
\bew
$$\sum_{n=0}^{∞}\left|\frac{z^n}{n!}\right|=\sum_{n=0}^{∞}\frac{|z^n|}{n!}=exp(|z|)$$
Bekannt: $exp(|z|)$ konvergiert\qed

\sS{Definition komplexe Exponentialfunktion}
Die komplexe Exponentialfunktion ist die Abbildung $\ds exp:\C →\C\quad exp(z)=\sum_{n=0}^{∞}\frac{z^n}{n!}$

\uS{Eigenschaften}
\sS{Satz}
Seien $z,w\eC$
\begin{enumerate}
\item{$exp(0)=1$ (klar)}
\item{$exp(z+w)=exp(z)+exp(w)$}
\item{$exp(z)≠0,\ exp(z)^{-1}=exp(-z)$}
\item{$exp(\ol{z})=\ol{exp(z)}$ (Komplexe Konjugation)}
\item{Für $x\eR$ ist $|exp(ix)|=1$}
  \begin{tikzpicture}[domain=-1.6:1.6,samples=200,prefix=plots/,smooth]
    \draw[very thin,color=gray] (-1.49,-1.49) grid (1.49,1.49);
    \draw[->] (-1.6,0) -- (1.6,0) node[right] {$x$};
    \draw[->] (0,-1.6) -- (0,1.6) node[above] {$i$};
	\draw (0,0) circle (1cm);
	\node (0.5, 0.5) (expo) {$exp(i \cdot x)$};
\end{tikzpicture}
\end{enumerate}
\bew
\begin{enumerate}
\setcounter{enumi}{1}
\item{Wie bei der reellen Exponentialfunktion:\\*
Die Reihe $exp(z+w)$ ist das Cauchy-Produkt der Reihen $exp(z)$ und $exp(w)$, dann folgt $(z)$ aus 7.15}
\item{$exp(z)·exp(-z)\underset{2)}{=}exp(z-z)=exp(0)\underset{1)}{=}1$}
\item{Sei $\ds s_n=\sum_{k=0}^{n}\frac{z^k}{k!}$ somit nach Definition $exp(z)=\lim\limits_{n→∞}s_n$\\*[4pt]
Sei $\ds s_n'=\sum_{k=0}^{n}\frac{\ol{z}^k}{k!}$ somit $exp(\ol{z})=\lim\limits_{n→∞}s_n'$\\*[4pt]
Es gilt $$s_n'=\ol{\sum_{k=0}^{n}\frac{z^k}{k!}}=\sum_{k=0}^{n}\ol{(\frac{z^k}{k!})}=\sum_{k=0}^{n}(\frac{\ol{z}^k}{k!})=s_n'$$
Somit $\ol{exp(z)}=\lim\limits_{\nif}(\ol{s_n})=\lim\limits_{\nif}s_n'=exp(\ol{z})$}
\item{$$|exp(ix)|^2=exp(ix)·\ol{exp(ix)}\underset{4)}{=}exp(ix)·exp(\ol{ix})=exp(ix)·exp(-ix)\underset{2)}{=}exp(ix-ix)=exp(0)\underset{1)}{=}1\ \overset{\sqrt{}}{\Rarr}\ |exp(ix)|=1$$}
\end{enumerate}

\uS{Trigonometrische Funktionen}
\sS{Definition}
Sei $x \in \R$\\*
$sin(x) = Im(exp(i \cdot x))$ (Sinus)\\*
$cos(x) = Re(exp(i \cdot x))$ (Cosinus)\\*
\begin{tikzpicture}[domain=-1.6:1.6, scale=2,samples=200,prefix=plots/,smooth]
	\clip (-1.3,-0.3) rectangle (1.3,1.3);
    \draw[very thin,color=gray] (-1.2,-0.49) grid (1.2,1.2);
    \draw[->] (-1.25,0) -- (1.25,0) node[right] {$x$};
    \draw[->] (0,-1.25) -- (0,1.25) node[above] {$i$};
	\draw (0,0) circle (1);
	\node (0.5, 0.5) (expo) {$exp(i \cdot x)$};
\end{tikzpicture}
\bem
Für jede komplexe Zahl $z$ gilt $z = Re(z) + i \cdot Im(z)$\\*
\Rarr{} $exp(i \cdot x) = cos(x) + i \cdot sin(x)$ (Eulersche Formel)

\sS{Satz}
\begin{enumerate}
\item{$cos(0)=1,\ sin(0)=0$
\begin{tikzpicture}[domain=-4:4,samples=200,prefix=plots/,smooth]
    \draw[very thin,color=gray] (-4,-1.25) grid (4,1.25);
    \draw[->] (-4,0) -- (4,0) node[right] {$x$};
    \draw[->] (0,-1.25) -- (0,1.25) node[above] {$i$};
	\draw[color=red] plot[id=cos] function{cos(x)} node[right] {\footnotesize $f(x) = cos(x)$};
\end{tikzpicture}
}
\item{$cos(-x)=cos(x),\ sin(-x)=-sin(x)$}
\item{$sin(x)^2+cos(x)^2=1$
  \begin{tikzpicture}[domain=-4:4,samples=200,prefix=plots/,smooth]
    \draw[very thin,color=gray] (-4,-1.25) grid (4,1.25);
    \draw[->] (-4,0) -- (4,0) node[right] {$x$};
    \draw[->] (0,-1.25) -- (0,1.25) node[above] {$i$};
	\draw[color=red] plot[id=sin] function{sin(x)} node[right] {\footnotesize $f_1(x) =sin(x)$};
\end{tikzpicture}
}
\item{\desc{Additionstheoreme:}{$sin(x+y)=sin(x)·cos(y)+cos(x)·sin(y)$\\*$cos(x+y)=cos(x)·cos(y)-sin(x)·sin(y)$}}
\end{enumerate}
\bew
\begin{enumerate}
\item{$exp(0i)=1=1+0i\ \Rarr\ cos(0)=1, sin(0)=0$}
\item{$exp(-ix)=exp(\ol{ix})=\ol{exp(ix)}=cos(x)-i·sin(x)$\\*
$exp(-ix)=cos(-x)+i·sin{-x}\ \Rarr\ cos(x)=cos(-x),\ sin(-x)=-sin(x)$}
\item{$sin(x)^2+cos(x)^2\overset{Def.}{=}|exp(ix)|^2=1$}
\item{$$exp(i(x+y)) = exp(i \cdot x) \cdot exp(i \cdot y)$$
$$\Rarr cos(x+y) + i \cdot sin(x+y) = (cos(x) + i \cdot sin(x))(cos(y) + i \cdot sin(y))$$
$$=cos(x) \cdot cos(y) - sin(x) \cdot sin(y) + i \cdot (sin(x) \cdot cos(y) + cos(x) \cdot sin(y))$$
Vergleich der Realteile / Imaginärteile \Rarr{} Behauptung 4\qed}
\end{enumerate}
\bem
	Die Gleichung $$cos(x)^2 + sin(x)^2 = 1$$
	impliziert $0 \leq cos(x)^2 \leq 1$, $0 \leq sin(x)^2 \leq 1$ somit $-1 \leq cos(x) \leq 1$, $-1 \leq sin(x) \leq 1$.

\sS{Definition}
	\begin{enumerate}
	\item{Eine Abbildung $f: \C→\C$ heißt stetig in $z \in \C$ wenn gilt:\\*
	Für jedes $\e > 0$ gibt es ein $\delta > 0$ so dass für jedes $w \in \C$ mit $|z - w| < \delta$ ist $|f(z) - f(w)| < \e$}
	\item{$f: \C \to \C$ heißt \ul{stetig}, wenn $f$ in jedem $z \in \C$ stetig ist.}
	\end{enumerate}

\sS{Satz}
Eine Abbildung $f:\C→\C$ ist stetig in $z\eC$ \equ\ Für jede Folge komplexer Zahlen $(z_n)_{n\geq 0}$ mit $z_n→z$ für \nif\ gilt auch $f(z_n)→f(z)$ für \nif
\bew
Wörtlich wie bei reeller Funktion (Satz 6.4)\qed

\sS{Satz}
Die komplexe Exponentialfunktion $exp:\C→\C$ ist stetig
\bew
Verwende Folgenstetigkeit
\begin{enumerate}
\item{Stetigkeit in $z=0\ exp (0)=1$\\*
Sie $z\eC$ (nahe 0)
$$\left|exp(z)-1\right|=\left|1+z+\frac{z^2}{2}+\frac{z^3}{6}+…-1\right|=\left|\sum_{n=1}^{∞}\frac{z^n}{n!}\right|\underset{unendliche Dreiecksungleichung 7.13}{\leq}exp(|z|)-1$$
Wenn $z_n→0$ in \C\\*
dann $|z_n|→0$ in \R\\*
dann $exp(|z_n|) \to exp(0) = 1$ (weil $exp: \R \to \R$ steig)\\*
d.h. $exp(|z_n|) -1 \to 0$\\*
$\Rarr|exp(z_n) - 1| \to 0$\\
Somit $exp$ stetig in $z = 0$ }
\item{Sei $z\eC$ beliebig, $z_n→z$
$$exp(z_n)-exp(z)=exp(z_n-z+z)-exp(z)=exp(z_n-z)-exp(z)-1·exp(z)=(exp(z_n-z)-1)·exp(z)$$
Es gilt: $z_n→z\ \equ\ z_n-z→0$\\*
\alg{&\underset{1)}{\Rarr}\ exp(z_n-z)→1\\
&\equ\ exp(z_n-z)-1→0\\
&\Rarr\ (exp(z_n-z)-1)·exp(z)→0·exp(z)=0
&\underset{(*))}{\Rarr}\ (exp(z_n)-exp(z)→0}
d.h. $exp(z_n)→exp(z)$\qed}
\end{enumerate}

\sS{Satz}
	Die Funktionen $sin: \R \to \R$ und $cos: \R \to \R$ sind stetig.\\
\bew
Mittels Folgenstetigkeit.\\
Sei $x_n \to x$ mit $x_n \in \R,\ x \in \R$\\*
$\Rarr i \cdot x_n \to i \cdot x$ in $\C$\\*
$\underset{7.23}{\Rarr} exp(i \cdot x_n) \to exp(i \cdot x)$\\*
d.h. $cos(x_n) + i \cdot sin(x_n) \to cos(x_n) + i \cdot sin(x_n)$\\*
$\equ cos(x_n) \to cos(x)$ und $sin(x_n) \to sin(x)$\\*
Somit sind $sin$ und $cos$ stetig in $x$ also stetig.\newpage
% Kopfzeile beim Kapitelanfang:
\fancypagestyle{plain}{
%Kopfzeile links bzw. innen
\fancyhead[L]{\calligra\Large Vorlesung Nr. 17}
%Kopfzeile rechts bzw. außen
\fancyhead[R]{\calligra\Large 06.12.2012}
}
%Kopfzeile links bzw. innen
\fancyhead[L]{\calligra\Large Vorlesung Nr. 17}
%Kopfzeile rechts bzw. außen
\fancyhead[R]{\calligra\Large 06.12.2012}
% **************************************************
\wdh
\begin{itemize}
\item{Komplexe Exponentialfunktion: $exp:\C→\C,\ z\mapsto exp(z)\ds \sum_{n\geq 0}\frac{z^n}{n!}$ stetig, Funktionalgleichung: $e^{z+w}=e^z·e^w, z,w\eC$
Additionstheoreme: $cos(x+y)=Re(e^{i(x+y)})=Re(e^{i(x)}·e^{i(y)})=cos(x)·cos(y)-sin(x)·sin(y)$}
\item{Sinus, Cosinus: $sin,cos:\R→\R,\ sin(x):=Im(e^{ix}),\ cos(x):=Re(e^{ix}),\ e^{ix}=cos(x)+i·sin(x)$}
\item{Weil $exp:\C→\C$ stetig \Rarr\ $sin, cos:\R→\R$ stetig
\usetikzlibrary{decorations.pathreplacing}
\begin{tikzpicture}[domain=-1.2:1.2, scale=2]
    \draw[very thin,color=gray] (-1.49,-1.49) grid (1.49,1.49);
    \draw[->] (-1.2,0) -- (1.2,0) node[right] {$x$};
    \draw[->] (0,-1.2) -- (0,1.2) node[above] {$i$};
	\draw (0,0) circle (1cm);
	\draw[thick, blue](0,0.8775) -- (0.4794, 0.8775);
	\draw[thick, red](0.4794,0) -- (0.4794, 0.8775);
	\draw[thick, black](0,0) -- (0.4794, 0.8775);
	\draw[decorate,decoration=brace,red] (0.4794, 0.8775) -- (0.4794,0);
	\draw[decorate,decoration=brace,blue] (0,0.8775) -- (0.4794, 0.8775);
	\draw (0.4794, 0.43) node[right, red] {$sin(x)$};
	\draw (0.24, 0.8775) node[above, blue] {$cos(x)$};
\end{tikzpicture}
\desc{Problem: zu zeigen}{$\~{sin}$ und $sin$ aus Vorlesung\\*
$\~{cos}$ und $cos$ aus Vorlesung}
\tikz[scale=2,domain=-0.49:2.5, samples=200,prefix=plots/,smooth]{
      \draw[very thin, color=gray!50] (-0.49,-0.25) grid (2.49,2.49);
      \draw[->] (-0.5,0) -- (2.5,0) node[right] {$x$};
      \draw[->] (0,-0.5) -- (0,2.5) node[above] {$y$};
      \clip (-0.49,-0.5) rectangle (2.5,2.5);
      \draw[color=red] plot[id=17.1] function{x} node[below]{\footnotesize $f_1(x) =x$};
      \draw[color=blue] plot[id=17.2,sharp plot] function{sin(x)} node {\footnotesize $f_2(x) = sin(x)$};
      \draw[color=cyan] plot[id=17.3] function{x - (x**3 / 6)} node [below] {\footnotesize $f_2(x) = x - \frac{x^3}{6}$};
}
}
\end{itemize}
Potenzreihen von $sin$ und $cos$:
Für $x \in \R$ gilt:
$$cos(x) + i \cdot sin(x) = exp(i \cdot x) = \frac{1}{0!}+\frac{i \cdot x}+{1!}+\frac{(i \cdot x)^2}{1!}+{2!}+\frac{(i \cdot x)^3}{3!}+…$$
$$=(\frac{1}{0!} + \frac{(i \cdot x)^2}{2!} + \frac{(i \cdot x)^4}{4!}) + \frac{(i \cdot x)^6}{6!} + …) + i (\frac{(i \cdot x)}{1!} + \frac{(i \cdot x)^3}{3!}) + \frac{(i \cdot x)^5}{5!}) + …)$$
\sS{Satz}
Für $x \in R$ gilt:
$$cos(x) = \ds\sum_{k \geq 0} \frac{(-1)^k}{(2k)!} \cdot x^{2k},\ sin(x) =\sum_{k \geq 0} \frac{(-1)^k}{(2k +1)!} \cdot x^{2k+1} $$
\sS{Bemerkung}
Siehe Übung\\*
$$cos(x)-cos(y)=2sin \ldots$$

% Satz 7.26


\uS{Analytische Definition der Zahl \pi}
\sS{Lemma}
Für $0<x\leq 2$ gilt: $0<x-\frac{x^3}{6}<sin(x)<x$.\\*
\begin{tikzpicture}[domain=-2:7,prefix=plots/, smooth]
    \draw[very thin,color=gray] (-2,-1.2) grid (7,1.2);
    \draw[->] (-2,0) -- (7,0) node[right] {$x$};
    \draw[->] (0,-1.2) -- (0,1.2) node[above] {$i$};
	\draw[color=red] plot[id=sin1] function{sin(x)} node[above, pos=0.5] {\footnotesize $f_1(x) = sin(x)$};
	\draw[color=blue] plot[id=cos1] function{cos(x)} node[below, midway] {\footnotesize $f_2(x) = cos(x)$};
\end{tikzpicture}

\bew
Schreibe $sin(x)=\sum(-1)^na_n$ mit $a_n\dfrac{x^{2n+1}}{(2n+1)!}>0$\\*[4pt]
Für $n\geq 1$ gilt: $$\frac{a_{n+1}}{a_n}=\dfrac{x^{2(n+1)+1}}{(2(n+1)+1)!}·\dfrac{(2n+1)!}{x^{2n+1}}=\frac{x^2}{(2n+3)·(2n+2)}<1$$
also: $a_1>a_2>a_3>a_4>…$\\*
Damit: $$x-sin(x)=\underset{>0}{(a_1-a_2)}+\underset{>0}{(a_3-a_4)}+\underset{>0}{(a_5-a_6)}…>0,\text{ d.h. $sin(x)<x$}$$
$$sin(x)-(x-\frac{x^3}{6})=\underset{>0}{(a_2-a_3)}+\underset{>0}{(a_4-a_5)}…>0,\text{ d.h. $sin(x)>x-\frac{x^3}{6}$}$$
Schließlich gilt für $0<x\leq 2$:
$$0<x-\frac{x^3}{6}, \text{ denn } \frac{x^3}{6}=\frac{x·x^2}{6}\leq x·\frac{4}{6}<x$$ \qed

\subsection*{Lemma} es gilt $cos(2) < 0$ und $cos(1) > 0$
\bew
	Es gilt $cos(2) = \sum (-1)^n \cdot b_n$, $b_n = \frac{2^{2n}}{(2n)!}$. Für $n \geq 1$ gilt: 
	$$\frac{b_{n+1}}{b_n} = \frac{2^2}{(2n+1)(2n+2)} < 1$$
	Also $b_1 > b_2 > b_3 > b_4 >…$ \\*[8pt]
	Somit:\\
	$cos(2) = b_0 - b_1 + b_2 - b_3 + b_4 - …$\\*
	$= b_0 - b_1 + b_2 \underbrace{-(b_3 + b_4)}_{<0} \underbrace{-(b_5 + b_6)}_{< 0}… < b_0 - b_1 + b_2$ \\*
	$= 1 - 2 + \frac{2}{3} = - \frac{1}{3}$ \qed\\
	Analog $cos(1) > 1-\frac{1}{2}$

\sS{Lemma}
Die Funktion $cos:[0,2]→\R$ ist streng monoton fallend im Intervall\\*
 \tikz[scale=2,domain=-0.49:2.5, samples=200,prefix=plots/,smooth]{
      \draw[very thin, color=gray!50] (-0.2,-0.25) grid (2,1.2);
      \draw[->] (-0.5,0) -- (2,0) node[right] {$x$};
      \draw[->] (0,-0.5) -- (0,1.2) node[above] {$y$};
      \draw[color=red] plot[id=x] function{x/1.4} node[below]{\footnotesize $cos$};
      \draw[color=red] plot[id=x] function{1 - x/(1.4)} node[below]{\footnotesize $sin$};
}
\bew
Sei $2\geq x>x\geq 0$, dann gilt $cos(x)-cos(y)\underset{\overset{\uparrow}{Additionstheoreme}}{=}-2·sin(\frac{x+y}{2})·sin(\frac{x-y}{2})$\\*
Weil $\frac{x+y}{2}, \frac{x-y}{2}\in (0,2]$ gilt mit Lemma 7.27\\*
$cos(x)-cos(y)<0$, d.h. $cos(x)<cos(y)\qed$

\sS{Satz}
Die Funktion $cos:[0,2]→\R$ hat genau eine Nullstelle $x$, und es gilt $x>1$
\bew
\begin{itemize}
\item{$cos(1)>0,\ cos(2)<0,\ cos$ stetig $\underset{Zwischenwertsatz}{\Rarr}\ cos$ \ul{hat} Nullstelle $x\in(1,2)\ (1<x<2)$}
\item{Da $cos:[0,2]→\R$ streng monoton fallend, hat $cos$ \ul{genau} eine Nullstelle \qed}
\end{itemize}

\sS{Definition}
Es sei $\pi\eR$ die eindeutige Zahl, so dass $cos(\frac{\pi}{2})$ und $1\leq\frac{\pi}{2}\leq 2$
\bem
$2\leq \pi \leq 4$, tatsächlich: $\pi=3,14159…$ (Kreiszahl)
\ul{es gilt}
Es gilt:\\*
\begin{tabular}{l|c|c|c|c|c|l}
$x$ & $0$ & $\frac{\pi}{2}$ & $\pi$ & $\frac{3 \cdot \pi}{2}$ &\\\hline
$cos(x)$ & $1$ & $0$ & $-1$ & $0$ & $1$ & \\\hline
$sin(x)$ & $0$ & $1$ & $0$ & $-1$ & $0$ & \\\hline
\end{tabular}
\hfill\\*
dazu:
\begin{enumerate}
\item{$sin(x)^2 + cos(x)^2 = 1$\\*
$sin(\frac{\pi}{2})^2 = 1$ also $sin(\frac{\pi}{2} = \pm 1$ aber $sin(\frac{\pi}{2}) > 0$\\*
d.h.:\\*
$e^{i \frac{\pi}{2}} = cos(\frac{\pi}{2}) + i \cdot sin(\frac{\pi}{2}) = i$}
\item{$$e^{i\cdot \pi} = (e^{i\cdot \frac{\pi}{2}})^2 = i^2 = -1 = cos(\pi) + i \cdot sin(\pi)$$}
\item{$$e^{i\frac{3 \cdot \pi}{2}} = e^{i\pi} \cdot e^{i\frac{\pi}{2}} = -1 \cdot i = -i = cos(\frac{3\pi}{2}) + i \cdot sin(\frac{3\pi}{2})$$}
\item{…}
\end{enumerate}

\sS{Satz}
Für $x\eR$ gilt:\\*
\begin{enumerate}
\item{$\ds cos(2\pi+x)=cos(x),\ sin(2\pi+x)=sin(x)$}
\item{$\ds cos(\pi+x)=-cos(x),\ sin(\pi+x)=-sin(x)$}
\item{$\ds cos(\pi-x)=cos(-x)=-cos(x),\ sin(\pi-x)=sin(x)$}
\item{$\ds cos(\frac{\pi}{2}-x)=sin(x)$}
\end{enumerate}
\Bew{Additionstheoreme anwenden}
\begin{enumerate}
\item{$\ds cos(2\pi+x)=\underset{=1}{cos(2\pi)}·cos(x)-\underset{=0}{sin(2\pi)}·sin(x)=cos(x)$}
\setcounter{enumi}{3}
\item{$\ds cos(\frac{\pi}{2}-x)=\underset{=0}{cos(\frac{\pi}{2})}·cos(-x)-\underset{=1}{sin(\frac{\pi}{2})}·\underset{=-sin(x)}{sin(-x)}=sin(x)$}
\end{enumerate}
\bem
$cos,sin$ sind periodisch mit Periode $2\pi$, $sin,cos$ sind durch $cos:[0,\frac{\pi}{2}]→\R$ eindeutig bestimmt.
\bem
$cos(x),sin(x)$ kann für $x\in \{0,\frac{\pi}{6},\frac{\pi}{4},\frac{\pi}{3},\frac{\pi}{2}\}$ explizit bestimmt werden
\bsp
$cos(\frac{\pi}{3})=\frac{1}{2},\ sin(\frac{\pi}{3})=\frac{\sqrt{3}}{2}$
\bew
	Sei $x = cos (\frac{\pi}{3}),\ y = sin(\frac{\pi}{3}), z = x + i\cdot y = e^{i\frac{\pi}{3}}$\\*
	Dann gilt:\\*
	$$z^2 = e^{2 \cdot i\frac{\pi}{3}} = e^{i\pi \cdot - \frac{\pi}{3}} = -1 \cdot e^{i\frac{\pi}{3}} = -\bar{z}$$\\*
	Also $(x + iy)^2 = -x + iy$ d.h. $x^2 - y^2 = -x$, $2xy = y$ und $x^2 + y^2 = 1$\\*
	Auflösen liefert Beh.\newpage
% Kopfzeile beim Kapitelanfang:
\fancypagestyle{plain}{
%Kopfzeile links bzw. innen
\fancyhead[L]{\calligra\Large Vorlesung Nr. 18}
%Kopfzeile rechts bzw. außen
\fancyhead[R]{\calligra\Large 10.12.2012}
}
%Kopfzeile links bzw. innen
\fancyhead[L]{\calligra\Large Vorlesung Nr. 18}
%Kopfzeile rechts bzw. außen
\fancyhead[R]{\calligra\Large 10.12.2012}
% **************************************************
\wdh
\section*{Definition}
$$cos(x) + i\cdot sin(x) = exp(i\cdot x) = e^{i\cdot x}$$
$cos\ [0,2] \to \R$ :streng monoton fallend,\\*
$cos(0) = 1$, $cos(1) > 0, cos(2) < 0$\\*
% Nach links floaten
\begin{tikzpicture}[domain=0.2:2.5,prefix=plots/, smooth]
    \draw[very thin,color=gray] (-0.2,-1.2) grid (2.5,1.2);
    \draw[->] (-0.2,0) -- (2.5,0) node[right] {$x$};
    \draw[->] (0,-1.2) -- (0,1.2) node[above] {$i$};
	\draw[color=blue] plot[id=18.1] function{cos(x)} node[below, midway] {\footnotesize $cos(x)$};
\end{tikzpicture}
\hfill\\
\Rarr{} cos hat in [1,2] eine eindeutige Nullstelle.\\*
\ul{Definiere} $\pi \in \R$ sei die Zahl mit $1 \leq \frac{\pi}{2} \leq 2$, $cos(\frac{\pi}{2}) = 0$\\*

\begin{tikzpicture}[domain=0.2:6.5,prefix=plots/, smooth]
    \draw[very thin,color=gray] (-0.2,-1.2) grid (6.5,1.2);
    \draw[->] (-0.2,0) -- (6.7,0) node[right] {$x$};
    \draw[->] (0,-1.2) -- (0,1.2) node[above] {$i$};
	\draw[color=blue] plot[id=18.2] function{cos(x)} node[below, midway] {\footnotesize $cos(x)$};
\end{tikzpicture}
\hfill\\*
Verschiebungsregeln (folgt aus Additionstheorem)\\*
$cos(2\pi + x) = cos(x)$\\*
$cos(2\pi - x) = cos(x)$\\*
$cos(\pi - x) = -cos(x)$\\*

\begin{tabular}{l|c|c|c|c|c}
$x$ & $0$ & $\frac{\pi}{2}$ & $\pi$ & $\frac{3\pi}{2}$ & $2\pi$\\\hline
$cos(x)$ & 1 & 0 & -1 & 0 & 1
\end{tabular}
\sS{Satz}
%WDH
Die Funktion $cos:[0,\pi]→[0,1]$ ist stetig, streng monoton fallend, bijektiv
\bew
$cos:[0,2]→\R$ streng monoton fallend \Rarr\ $cos:[0,\frac{\pi}{2}]→\R$ streng monoton fallend\\*
$cos(\pi-x)=-cos(x)$ \Rarr\ $cos:[\frac{\pi}{2},\pi]→\R$ streng monoton fallend SKIZZE\\*
\Rarr\ $cos:[0,\pi]→\R$ streng monoton fallend\\*
$cos(0)=1,\ cos[\pi]=-1$ \Rarr\ $cos:[0,\pi]→[-1,1]$ surjektiv, somit bijektiv\qed\\*
\ssss{Folge} es gibt eine Umkehrfunktion:
Arcuscosinus: $arccos=cos^{-1}:[-1,1]→[0,\pi]$SKIZZE ARCCOS\\*
\alg{cos(0)&=1\quad arccos(1)=0\\*
cos\left(\frac{\pi}{2}\right)&=0\quad arccos(0)=\frac{\pi}{2}\\*
cos(\pi)&=-1\quad arccos(1)=\pi}
\bem
Die Wahl des Intervalls $[0,\pi]$ ist willkührlich. Auch bijektiv:\\*
$cos:[\pi,2\pi]→[-1,1],\ cos:[-\pi,0]→[-1,1]$
\bem
Sei $x\eR$. Es gilt $cos(x)=1\ \equ\ x=2\pi·n$ mit $n\eZ$\\*
(Anschaulich: klar, Beweis: Übung)

\uS{Polarzerlegung}
\sS{Satz}
Jede komplexe Zahl $z \in \C$ hat eine Darstellung \fbox{$z = r \cdot e^{i\phi}$} mit $r \in \R,\ r \geq 0,\ \phi \in \R$. Es gilt $r= |z|$, Man kann $\phi$ so wählen, dass $\phi \in [0, 2\pi)$. Wenn $z \neq 0$, dann ist $\phi \in [0, 2\pi)$ eindeutig.\\*
Bezeichnung: $z = r \cdot e^{i\phi}$\\*
Polarzerlegung von $z$, $\phi \in [0, 2\pi)$ , Argument von $z$ (wenn $z \neq 0$)
SKIZZE
\bew
Wenn $z = 0$ \Rarr{} $|z| = |r| \cdot |e^{i\phi}| = r \cdot 1 = r$\\*
Wenn $z = 0$: $z = 0 \cdot e^{i \phi}$ für alle $\phi$\\*
Sei $z \neq 0$. $r := |z| > 0$\\*
$w:= \frac{z}{r} \in \C$. $|w| = \frac{|z|}{r} = \frac{r}{r} = 1$\\*
Suche $\phi$ mit $w = e^{i \phi}$. Sei $w = x + i \cdot y$, $x,y \in \R$\\*
$cos(\phi) = x,\ sin(\phi) = y$\\*
Setze $\til{\phi} := arcos(x)$ und $\til{y} = \sin(\til{\phi})$\\*
Dann $\til{y}^2 = sin(\til{\phi})^2 = 1 - cos(\til{\phi})^2$\\*
$= 1 - x^2 =  y^2$, denn $x^2 + y^2 = |w|^2 = 1$\\*
2 Fälle:\\*
$\til{y} = y$ oder $\til{y} = -y$\\*
Wenn $\til{y} = y$ dann $\phi = \til{\phi}$ Lösung: $e^{i\phi} = w$\\*
Wenn $\til{y} = -y$ dann $\phi := 2\pi 2\pi - \til{\phi}$\\*
$cos(\phi) = cos(\til{\phi}) = x$ \ok\\
$sin(\phi) = sin(2\pi - \til{\phi}) = sin(\til{phi}) = -\til{y} = y$ \ok\\
\Rarr{} $e^{i\phi} = w \Rarr z = r \cdot w = r \cdot e^{i\phi}$\\
Aber:
$$|\phi-\phi'| < 2 \pi \Rarr\ \phi-\phi'<0$$
Das zeigt \ul{Eindeutigkeit} der Polarzerlegung\qed
\bem
(Multiplikation komplexer Zahlen in Polarzerlegung)
$$(r·e^{i\phi})·(r·e^{i\phi'})=(r·r')·e^{i\phi+i\phi'}=(r·r')·e^{i(\phi+\phi')}$$
Multiplikation in $\C$ entspricht $\case{\text{Multiplikation der Beträge}\\ \text{Addition der Argumente}}$ SKIZZE

\sS{Satz (Einheitswurzel)}
Sei \nN\\*
Die Gleichung $z^n=1,\ z\eC$\\*
Hat genau $n$ Lösungen, nämlich $z=e^{2\pi\frac{k}{n}}$ mit $k\eZ,\ 0\leq k<n$
\bew
Wenn $z^n=1$, dann $|z|^n=|1|=1\ \Rarr\ |z|=1$\\*
Sei $z=e^{i\phi}$ mit $0\leq \phi <2 \pi\ z^n=1\ \equ\ (e^{i\phi})^n=1\ \equ\ e^{in\phi}=1$
\alg{&\equ n·\phi= k·2\pi \text{ mit } k\eZ\\*
&\equ \phi= 2\pi k/n \text{ mit } k\eZ\\*
&\equ z= e^{2\pi i\frac{k}{n}} \text{ mit } k\eZ}
\sss{Bedeutung}
$0\leq \phi<2\pi\ \equ\ 0\leq 2\pi k/n < 2\pi\ \equ\ 0\leq k<n\qed$\\*
SKIZZE $\case{e^0=1\\
e^{2\pi i/6}=e^{\pi i/3}=\frac{1}{2}+\frac{\sqrt{3}}{2}i\\
e^{2\pi i 2/6}=e^{\pi i 2/3}=-\frac{1}{2}+\frac{\sqrt{3}}{2}i\\
e^{2\pi i 3/6}=e^{\pi i}=-1\\
e^{2\pi i 4/6}=…=…\\
e^{2\pi i 5/6}=…}$
\sss{Verhalten von $exp(z)$ nahe Null}
Erinnerung: $exp:\C→\C$ stetig, dass heißt wenn $z→0$ dann $exp(z)→exp(0) =1$\\*
Betrachte $\dfrac{exp (z)-1}{z}$ für $z\eC,\ z\neq 0$

\sS{Satz}
Es gilt $\lim_{z \to 0} \frac{exp(z) - 1}{z} = 1$\\*
Das heißt: \\*
Wenn $(z_n)$ Folge in $\C$ mit $z_n \neq 0$, $z_n \to 0$ dann gilt $\frac{exp(z) -1}{z} \to 1$
\bew
$$exp(z) = 1 + \frac{z}{1!} + \frac{z^2}{2!} + \frac{z^3}{3!} + \ldots$$
$$\frac{exp(z) - 1}{z} =\frac{\frac{z}{1!} + \frac{z^2}{2!} + \frac{z^3}{3!} + \ldots}{z}$$
$$= \frac{1}{1!} + \frac{z}{2!} + \frac{z^2}{3!} + \frac{z^3}{4!} + \ldots$$
Also $|\frac{exp(z) - 1}{z} - 1| = |\frac{z}{2!} + \frac{z^2}{3!} + \ldots| \leq |\frac{z}{2!}| + |\frac{z^2}{3!}| + |\frac{z^3}{4!}| + \ldots$\\*
$\leq \frac{|z|}{1!} + \frac{|z|^2}{2!} + \frac{|z|^3}{3!} + \ldots = exo(|z|) -1$\\*
Wenn $z_n \to 0$ dann $|z_n| \to 0$\\*
\Rarr{} $(exp(|z|)) \to 0$\\*
\Rarr{} $\frac{exp(z_n) - 1}{z_n} \to 1$ \qed

\bem \begin{enumerate}
\item{Beschränkung auf $z=x\eR\ \leadsto\ \lim_{x \to 0}\ \overset{x \to 0}{x\neq 0}\quad \frac{e^x-1}{x}=1\quad x\eR$}
\item{Beschränkung auf $$z=ix,\ x\eR\ \leadsto\ \lim_{x \to 0}\frac{e^ix-1}{ix}=1$$ $$\lim_{x→0}\frac{cos(x)+i·sin(x)-1}{ix}=1$$ $$\lim_{x→0}\left(\frac{sin(x)}{x}-i·\frac{cos(x)-1}{x}\right)=1+0i\ (*)$$
$$(*) \underset{Realteil}{\Rarr} \lim_{x→0}\frac{sin(x)}{x}=1$$
$$(*) \underset{Imaginärteil}{\Rarr} \lim_{x→0}\frac{cos(x)-1}{x}=0$$}
\end{enumerate}

\uS{Geometrische Bedeutung von $\pi$?}
\sss{Frage} Was ist die Länge des Kreisbogens von $1$ bis $e^{ix}$? SKIZZE
\begin{enumerate}
\item{Wie ist diese Länge definiert?}
\item{Berechnen}
\end{enumerate}
Zerteilung in kleine Strecken
Wähle $n \in \N$ groß:
$$l_n = |e^{ix/n} - 1| + |e^{2ix/n} - e^{ix/n}| + ... + |e^{ix} - e^{(n-1)ix/n}|$$
$$= \sum_{k=0}^{n-1} \left| |e^{(k+1)ix/n} - e^{kix/n}| \right|$$

\sS{Satz}
Es gilt $\lim_\nif  l_n=|x|$ Interpretation der Länge des Bogens ist $|x|$\\*
SKIZZE bogenlänge
\bew
$$|e^{(k+1)ix/n}-e^{kix/n}|=|e^{kix/n}|·|e^{ix/n}-1|=|e^{ix/n}-1|$$
$$(**)\ \text{Satz 7.36: } \lim_{\nif}\left|\frac{e^{ix/n}-1}{ix/n}\right|=1$$
$$\lim_{n→∞}l_n=\lim_{\nif}n·\left|e^{ix/n}-1\right|=\frac{\left|e^{ix/n}-1\right|}{\frac{1}{n}}\underset{(**)}{=}|x|$$\qed\newpage
% Kopfzeile beim Kapitelanfang:
\fancypagestyle{plain}{
%Kopfzeile links bzw. innen
\fancyhead[L]{\calligra\Large Vorlesung Nr. 19}
%Kopfzeile rechts bzw. außen
\fancyhead[R]{\calligra\Large 17.12.2012}
}
%Kopfzeile links bzw. innen
\fancyhead[L]{\calligra\Large Vorlesung Nr. 19}
%Kopfzeile rechts bzw. außen
\fancyhead[R]{\calligra\Large 17.12.2012}
% **************************************************
%
\chapter{Differenzialrechnung}
\sS{Definition Differenzialrechnung}
Sei $I$ ein Intervall:\\*
Eine Funkion $f:I→\R $ heißt \ul{$x_0\in I$ differenzierbar}, wenn die Grenzwert exstiert $$f'(x_0):=\lim_{x→x_0}\frac{f(x)-f(x_0)}{x-x_0}$$ und $f'(x_0)$ heißt \ul{Ableitung} von $f$ in $x_0$\\*
$f$ heißt differnzierbar, wenn $f$ in jedem $x_0\in I$ differenzierbar ist.
\sss{Andere Bezeichnung}
$$f'(x_0)=\frac{df}{dx}(x_0)=Df(x_0).$$\\*
\sss{Geometrische Interpretation}
Der Differenzialquotient $$ \frac{f(x)-f(x_0)}{(x-x_0)} $$ ist Steigung der Geraden durch die Punkte $ (x,f(x)),\ (x_0,f(x_0)) $ (Sekante)\\
\begin{tikzpicture}[scale=2,domain=-0.5:2, samples=200,prefix=plots/,smooth]
\draw[very thin, color=gray!50] (0,0) grid (2,4.49);
\draw[->] (-0.5,0) -- (2.5,0) node[right] {$x$};
\draw[->] (0,-0.5) -- (0,4.5) node[above] {$y$};
\draw[color=black] plot[id=19.1] function{x*x} node[right] {$f_1(x) =x^2$};
\draw[color=blue] (0,0) -- (1.5, 0) node[midway,below] {$x-x_0$};
\draw[color=red] (1.5,0) -- (1.5, 2.25) node[midway,right] {$f(x)-f(x_0)$};
\draw[color=black] (0,0) -- (1.5, 2.25);
\end{tikzpicture}\\*
$f'(x_0)$ (wenn existiert) ist die Steigung der \ul{Tangente} an $Γ_f$ im Punkt $(x_0,f(x_0))$
\bem
\desc{Schreibe}{$x=x_0+h$\\$h=x-x_0$}
$$\leadsto\ f'(x_0):=\lim_{h→x_0}\frac{f(x_0+h)-f(x_0)}{h}$$
\bsp
\enum{
\item{$f:\R→\R,\ f(x)=c$ konstante Funktion
$$f'(x_0):=\lim_{x→x_0}\frac{c-c}{\underbrace{x-x_0}_0}=0$$}
\item{$f:\R→\R,\ f(x)=a·x, a\eR$
$$f'(x_0):=\lim_{x→x_0}\frac{a·x-a·x_0}{x-x_0}=\lim_{x→x_0}a=a\ \Rarr f\text{ differenzierbar}$$}
\item{$f:\R→\R,\ f(x)=x^2$
$$f'(x_0):=\lim_{h→0}\frac{(x_0+h)^2-x_0^2}{h}=\lim_{h→0}\frac{2x_0h+h^2}{h}=\lim_{h→0}\frac{2x_0+h}=2x_0\ \Rarr f\text{ differenzierbar}$$}
\item{$f:\R\\\{0\}→\R,\ f(x)=\frac{1}{x}$ Sei $x_0≠0$
$$f'(x_0):=\lim_{x→x_0}\frac{\frac{1}{x}+\frac{1}{x_0}}{x-x_0}=\lim_{x→x_0}\frac{\frac{x_0-x}{x·x_0}}{x-x_0}=\lim_{x→x_0}\frac{\cancel{x_0-x}}{\cancel{(x-x_0)}·x·x_0}=\lim_{x→x_0}\frac{-1}{x·x_0}=-\frac{1}{x_0^2}\ \Rarr f\text{ differenzierbar},\ \left(\frac{1}{x}\right)'=-\frac{1}{x^2}$$}
\item{$f: \R\to\R,\ f(x) = |x|$\\*
\begin{tikzpicture}[scale=2,domain=-2:2, samples=200,prefix=plots/,smooth]
\draw[very thin, color=gray!50] (-2,0) grid (2,2.5);
\draw[->] (-2.5,0) -- (2.5,0) node[right] {$x$};
\draw[->] (0,-0.5) -- (0,2.5) node[above] {$y$};
\draw[color=black] plot[id=19.2] function{abs(x)} node[right] (1,1) {\footnotesize $f_2(x) =|x|$};
\end{tikzpicture}
$x_0 = 0$\\*
$$f'(x_0):=\lim_{h→0}\frac{|h| - |0|}{h} = \frac{|h|}{h} \text{ existiert nicht, denn }\case{1 \ \ h>0 \\ -1 \ \ h < 0}$$
\Rarr{} $f$ ist nicht in $0$ differenzierbar.}
\item{$exp: \R \to \R$ bekannt aus Satz 7.36\\*
Sei $x_0 \in \R$.\\*
$exp(x_0) = \lim_{h \to 0} \frac{exp(x_0 +h) - exp(x_0)}{h}$\\*
$\lim_{x \to 0} \frac{exp{x} - 1}{x} = 1 = \frac{exp(x) -exp(0)}{x-0}$
Das heißt: $exp'(0) = 1$. Insbesondere ist $exp$ differenzierbar in $0$}
\item{$sin:\R\bs\{0\}→\R$ Sei $x_0\eR$
$$\lim_{h→0}\frac{sin(x_0+h)-sin(x_0)}{h}=\frac{1}{h}(sin(x_0)·cos(h)+cos(x_0)·sin(h)-sin(x_0))=\frac{1}{h}·sin(x_0)·(cos(h)-1)+cos(x_0)·\frac{sin(h)}{h}$$
$$=sin(x_0)·\underbrace{\frac{(cos(h)-1)}{h}}_{→0\ für\ h→0}+cos(x_0)·\underbrace{\frac{sin(h)}{h}}_{→1\ für\ h→0} \text{ (Korollar zu Satz 7.36)} $$
Somit $sin'(x_0) = \lim_{h \to 0} \frac{sin(x_0 + h) - sin(x_0)}{h} = cos(x_0)$
\Rarr{} $sin$ ist differenzierbar, $sin' = cos$.}
\item{$cos: \R \to \R $ analog... cos' = -sin \\
\begin{tikzpicture}[domain=-2:7,prefix=plots/, smooth]
\draw[very thin,color=gray!50] (-2,-1.2) grid (7,1.2);
\draw[->] (-2,0) -- (7,0) node[right] {$x$};
\draw[->] (0,-1.2) -- (0,1.2) node[above] {$i$};
\draw[color=red] plot[id=sin1] function{sin(x)} node[above, pos=0.5] {\footnotesize $f_1(x) = sin(x)$};
\draw[color=blue] plot[id=cos1] function{cos(x)} node[below, midway] {\footnotesize $f_2(x) = cos(x)$};
\end{tikzpicture}\\*
$ sin' = cos $ \\*
$ cos' = -sin $ }
}

\sS{Lemma}
Eine Funktion $f:I→\R$ ist genau dann in $x_0$ differenzierbar, wenn eine ander Funktion $\phi:I→\R$ existiert, sodass
\enum{
\item{
$f(x)-f(x_0)=\phi(x)·(x-x_0)$ für alle $x\in I$
}
\item{
$\phi$ ist stetig in $x_0$
}
}
Dann gilt $\phi(x_0) =f'(x_0)$
\bew
Definiere notwendig\\*
$$\phi_0:I\bs\{x_0\}→\R,\ \phi(x)=\frac{f(x)-f(x_0)}{x-x_0}$$
Folgenstetigkeit: $ \phi_0$ hat eine Fortsetzung $\phi:I→\R$, die in $x_0$ stetig ist\ \equ\ $\ds\lim_{x→x_0}\phi_0(x)$ existiert, dann ist 
$$\phi(x_0)=\lim_{x→x_0}\phi_0(x)\ \equ\ \lim_{x→x_0}\frac{f(x)-f(x_0)}{x-x_0}$$
existiert, dann ist $\phi(x_0)=f'(x_0)\qed$
\sS{Satz}
Sei $f I \to \R eine Funktion$\\*
\begin{enumerate}
\item{$f$ in $x_0$ differenzierbar \Rarr{} $f$ in $x_0$ stetig.}
\item{$f$ differenzierbar \Rarr{} $f$ stetig.}
\end{enumerate}
\bew
\begin{enumerate}
\item{Sei \phi wie im Lemma \Rarr{} $f(x) = f(x_0) + \phi(x)(x-x_0)$\\*
$\phi$ stetig in $x_0$ \Rarr{} $f$ stetig in $x_0$ \qed}
\item{folg aus 1.)}
\end{enumerate}

\uS{Berechnung der Ableitung}
\sS{Satz}
Seien $f,\ g: I \to \R$ differenzierbar in $x_0 \in I$, dann sind auch $f + g$, $a \cdot f$, $f \cdot g: I \to \R$ in $x_0$ differenzierbar. ($a \in R$) und:
\begin{enumerate}
\item{$(f + g)'(x_0) = f'(x_0) + g'(x_0)$}
\item{$a \cot f)'(x_0) = a \cdot f'(x_0)$}
\item{$(f \cdot g)'(x_0) = f'(x_0) \cdot g(x_0) + f(x_0) \cdot g'(x_0)$}
\end{enumerate}
\bew
Zeige 3), 1) und 2) analog.\\
$$ \lim_{x \to x_0} \frac{f(x) \cdot g(x) - f(x_0) \cdot g(x_0)}{x - x_0}$$
$$= \lim_{x \to x_0} \frac{f(x) \cdot g(x) - f(x_0) \cdot g(x_0)}{x - x_0} + \frac{f(x) \cdot g(x_0) - f(x_0) \cdot g(x_0)}{x - x_0}$$
$$= \lim_{x \to x_0} \left( \frac{f(x) \cdot g(x) - f(x_0) \cdot g(x_0)}{x - x_0} \right) + \lim_{x \to x_0}\left( \frac{f(x) \cdot g(x_0) - f(x_0) \cdot g(x_0)}{x - x_0} \right)$$
$$=f(x_0)g'(x_0) + f'(x_0)g(x_0)$$
Weil $f$ stetig in $x_0$ und nach Definition der Ableitung. \qed{}\\*
\ul{Folge:} Für $n \geq 1$ $(x^n)' = n \cdot x^{n-1}$
\bew
mit vollständiger Induktion:
$(x^1) = 1 =1 \cdot x^0$ \ok\\*
$n \to n + 1$\\*
$(x^n+1) = n\cdot x^{n-1}$
$$(x^{n+1})' = x' \cdot x^n + x \cdot (x^n)' = 1 + x^n + x \cdot n \cdot x^{n-1} = (n+1) x^n$$ \qed

\sS{Satz Kettenregel}
Sei $I,J$ Intervalle, $f:I→\R,\ g:J→\R$ Funktionen\\*
mit $f(I)\subseteq J \leadsto g\circ f:I→\R$ ist definiert\\*
$I\stackrel{f}{\longrightarrow}J\stackrel{g}{\longrightarrow}\R$\\*
$x_0\longmapsto f(x_0)$\\
\desc{Wenn}{$f$ in $x_0$ differenzierbar und\\$g$ in $f(x_0)$ differenzierbar,}
dann ist $g\circ f$ in $x_0$ differenzierbar,
und $(g\circ f)'(x_0)=g'(f(x_0))·f'(x_0)$\\
\ul{Beweisidee} $\frac{g(f(x)) - g(f(x_0))}{x-x_0} = \frac{g(f(x)) - g(f(x_0))}{f(x)-f(x_0)} \cdot \frac{f(x) - f(x_0)}{x-x_0}$\newpage
% Kopfzeile beim Kapitelanfang:
\fancypagestyle{plain}{
%Kopfzeile links bzw. innen
\fancyhead[L]{\calligra\Large Vorlesung Nr. 20}
%Kopfzeile rechts bzw. außen
\fancyhead[R]{\calligra\Large 20.12.2012}
}
%Kopfzeile links bzw. innen
\fancyhead[L]{\calligra\Large Vorlesung Nr. 20}
%Kopfzeile rechts bzw. außen
\fancyhead[R]{\calligra\Large 20.12.2012}
% **************************************************
%
\wdh
$I$ Intervall\\*
$f:I→\R$ ist in $x_0\in I$ differenzierbar wenn $$f'(x_0)=\lim_{x→x_0}\frac{f(x)-f(x_0)}{x-x_0}$$
$f'(x_0)$ Ableitung von $f$ an $x_0$
\bsp
$n\geq 0:$ $$(x^n)'=n·x^{n-1}$$
$$(\frac{1}{x})'=-\frac{1}{x^2},\ exp'=exp,\text{ d.h. }(e^x)'=e^x,\ cos'=-sin,\ sin'=cos$$
\sss{Produktregel}
$$(f·g)'=f'·g+f·g'$$
\sss{Kettenregel}
$$(g\circ f)'(x)=g'(f(x))·f'(x)$$
\bsp
$$(e^{x^2})'=(exp(x^2))'=exp'(x^2)·(x^2)'=2x·e^(x^2)$$
$$((cos (x))^2)' = f(cos (x))' = f'(cos (x)) \cdot cos'(x) = 2 \cdot cos(x) \cdot sin(x)$$

\sS{Satz Quotientenregel}
Seien $f, g: I \to \R$ in $x_0$ differenzierbar, $g(x) \neq 0$ für alle $x \in I$.\\*
Dann ist $\frac{f}{g}: I \to \R$, $\frac{f}{g}(x) := \frac{f(x)}{g(x)}$ differenzierbar in $x_0$
$$\left(\frac{f}{g}\right)'(x) = \frac{f'(x_0)\cdot g(x_0) - f(x_0) \cdot g'(x_0)}{g(x_0)^2}$$
\bew
Fall $f = 1$:\\*
Kettenregel:
$\frac{1}{g} = \frac{1}{x} \cdot g$\\*
Sei $h(x) = \frac{1}{x}\ \ \frac{1}{g}(x) = h(g(x))$ \\*
$(\frac{1}{g})'(x) = h'(g(x)) \cdot g'(x) = - \frac{1}{g(x^2)} \cdot g'(x) \approx Beh.$\\*
Insbesondere: $(\frac{1}{g})' = -\frac{1}{g^2} \cdot g' = -\frac{g'}{g^2}$
Allgemeiner Fall:
$\frac{f}{g} = f \cdot \frac{1}{g}$\\*
Produktregel \Rarr{} $(\frac{f}{g})' = (f \cdot \frac{1}{g})' = f'\frac{1}{g} + f \cdot (\frac{1}{g})'$\\*
$= \frac{f' \cdot g}{g^2} - f\cdot \frac{g'}{g^2} = \frac{f'\cdot g - f \cdot g'}{g^2} \qed$
\bsp
Sei $n<0$, $f:\R\bs\{0\}→\R,\ f(x)=x^n$\\*
Sei $m=-n>0$ $f(x)=\frac{1}{x^m}$
$$f'(x)=-\frac{(x^m)'}{(x^n)'}=-m\frac{x^{m-1}}{x^{2m}}=-mx^{-m-1}=nx^{n-1}$$
$$-m-1=n-1$$ $$-m=n$$
\sss{Folge}
$(x^n)'=nx^{n-1}$ gilt für alle $n\eZ$!
$$(x^{-3})'=-3x^{-4}$$

\sS{Satz (Ableitung der Umkehrfunktion)}
Sei $f:I→\R$ stetig, streng monoton\\*
Sei $J=f(I),\ g=f^{-1}:J→I$ die Umkehrfunkion von $f$\\*
Angenommen, $f$ ist $x_0\in I$ differenzierbar und $f'(x_0)≠0$\\*
Dann ist $g$ in $y_0:=f(x_0)$ differenzierbar und $g'(y_0)=\frac{1}{f'(x_0)}$\\*
\begin{tikzpicture}[domain=-1:4,prefix=plots/, smooth]
\draw[very thin,color=gray] (-0.99,-0.49) grid (3.99,2.99);
\draw[->] (-1,0) -- (4,0) node[right] {$x$};
\draw[->] (0,-1) -- (0,3) node[above] {$y$};
% Some fancy function
% Tangente zu dem Fancy Graphen bei x_0
\draw[color=blue] plot[id=cos1] function{cos(x)} node[below, midway] {\footnotesize $f_2(x) = cos(x)$};
\end{tikzpicture}
\bew
Sei $(y_n)_{n\geq 1}$ Folge in $J$ ist mit $y_n→y_0$
$$g'(y_0)=\lim_{\nif}\frac{g(y_n)-g(y_0)}{y_n-y_0}$$
\hfill(soll unabhängig von (y_n) sein)
$y_n→y_0\ (\nif)$ Sei $x_n=g(y_n)\ →\ x_n→x_0\ (\nif)$ da $g$ stetig.\\*
$x_n=g(y_n)\ \equ\ f(x_n)=y_n$
$$\lim_{\nif}\frac{g(y_n)-g(y_0)}{y_n-y_0}=\lim_{\nif}\frac{x_n-x_0}{f(x_n)-f(x_0)}=\left(\lim_{\nif}\frac{f(x_n)-f(x_0)}{x_n-x_0}\right)^{-1}=f'(x_0)^{-1}$$
Rechnung ok weil $f'(x_0)≠0$ \qed\\*
\ul{Folge} $log: \R_{>0}\to \R$ ist differenzierbar, $log'(x) = \frac{1}{x}$
\bew
$log(x) = exp(x)^{-1}$ Umkehrfunktion $exp'(x) = exp(x) \neq 0$ für alle $x$\\*
\Rarr{} 8.7 anwendbar. Sei $y = exp(x)$, $x = log(y)$.
$log'(y) = \frac{1}{exp'(x)} = \frac{1}{exp(x)} = \frac{1}{y}\qed$\\*
$log'(x)=\frac{1}{x}$
\sss{Anwendung}
$x=1\ log(1)=0$\\*
$log'(1)=\frac{1}{1}=1$
$$\lim_{\nif}\frac{log\left(1+\frac{1}{n}\right)-log(1)}{\frac{1}{n}}=log'(1)=1\ \Rarr\ 1=\lim_{\nif}\left(n·log\left(1+\frac{1}{n}\right)\right))$$
$exp$ anwenden $\underset{exp\ stetig}{\Rarr}$
$$exp(1)=\lim_{\nif}exp\left(n·log\left(1+\frac{1}{n}\right)\right))$$
$$e=exp(1)\underset{Def}{=}\lim_{\nif}\left(1+\frac{1}{n}\right)^n$$

\sS{Höhere Ableitungen}
Idee: Wenn $f: I \to \R$ differenzierbar $ \leadsto\ f':I \to \R$ Funktion\\*
Wenn $f'$ differenzierbar $ \leadsto\ (f')' = f'' = \frac{d^2f}{dx^2}$\\*
2. Ableitung weiter:\\*
$f'' = f^{(2)}$\\*
$f^{(n+1)} = (f^{(n)})'$ wenn differenzierbar\\*
$f^{(n)}$: n-te Ableitung von $f$.
\bsp
$$(x^5)^{(2)} = ((x^5)')' = (5x^4)' = 20x^3$$
$$cos'' = -sin' = -cos$$
$$sin'' = -cos' = -sin$$

\sS{Formale Definition der höheren Ableitung}
Rekursive Definition:\\*
Sei $n\geq 1$\\*
Eine Funktion $f:I→\R$ ist $n+1$ mal differenzierbar in $x_0\in I$, wenn ein $ε>0$ existiert, so dass $f$ auf $(x_0-ε,x_0+ε)$ $n$-mal differenzierbar und $f^{(n)}:(x_0-ε,x_0+ε)→\R$ in $x_0$ differenzierbar ist, dann setzte $f^{(n+1)}(x_0):=(f^{(n)})'(x_0)$

\uS{Lokale Extrema und Mittelwertsatz}
\sS{Definition Lokale Extrema}
Sei $f: I \to \R$ Funktion\\*
$f$ hat ein \ul{lokales Maximum} in $x_0 \in I$ wenn gilt: $\case{
\text{es gibt ein $\e > 0$ s.d.}\\
\text{Für alle $x \in I$ mit $|x-x_0| < \e$}\\
\text{gilt $f(x) \leq f(x_0)$}
}$\\*
Analog: Lokales Minimum.
\bem
Lokale Minima von $f$ = lokale Maxima von $-f$

\sS{Satz (Mittelwertsatz)}
Sei $I=(a,b),\ f:I→\R$ Funktion\\*
Wenn $f$ in $x_0\in(a,b)$ ein lokales Extremum hat, und wenn $f$ in $x_0$ differenzierbar ist, dann ist $f'(x_0)=0$ (Extremum = Maxium oder Minimum)
\bew
$$f'(x_0)=\lim_{x\searrow x_0}\underbrace{\frac{f(x)-f(x_0)}{x-x_0}}_{\geq 0}=\lim_{x\nearrow x_0}\underbrace{\frac{f(x)-f(x_0)}{x-x_0}}_{\leq 0}$$
Angenommen $f$ hat in $x_0$ lokales Minimum \Rarr\ $f(x)-f(x_0)\geq 0$ wenn $|x-x_0|<ε,\ ε$ wie oben\\*
Somit $f'(x_0)\leq 0,\ f'(x_0)\geq 0\ \Rarr\ f'(x_0)= 0\qed$

\sS{Satz von Rolle}
Sei $a < b$, $f:[a, b] \to \R$ stetig auf $(a, b)$ differenzierbar.\\*
Sei $f(a) = f(b)$.\\*
Dann gibt es ein $x_0 \in (a, b)$ mit $f'(x_0) = 0$\\*
GRAPH
\bew
Wenn $f$ konstant, d.h. $f(x) = f(a)$ für alle $x \in (a, b),$ dann $f'(x) = 0 $ für alle $x$ \Rarr{} Satz stimmt.\\*
Sei $f$ nicht konstant, gibt es $x_1 \in (a, b)$ mit $f(x_1) \neq f(a)$\\*
Angenommen $f(x_1) > f(a)$  (sonst Betrag $-f$)\\*
sei $x_0 \in I$ mit $f(x_0) \geq f(x)$ für alle $x \in I$\\*
$f(x_0) \geq f(x_1) > f(a) = f(b) \Rarr x_0 \neq a,\ x_0 \neq b$\\*
$f$ hat in $x_0$ ein \ul{lokales} Maximum $\overset{8.19}{\Rarr{}}\ f'(x_0) = 0$\newpage
% Kopfzeile beim Kapitelanfang:
\fancypagestyle{plain}{
%Kopfzeile links bzw. innen
\fancyhead[L]{\calligra\Large Vorlesung Nr. 21}
%Kopfzeile rechts bzw. außen
\fancyhead[R]{\calligra\Large 07.01.2013}
}
%Kopfzeile links bzw. innen
\fancyhead[L]{\calligra\Large Vorlesung Nr. 21}
%Kopfzeile rechts bzw. außen
\fancyhead[R]{\calligra\Large 07.01.2013}
% **************************************************
%
\wdh
Eine Funktion $f: I \to \R$ heißt differenzierbar in $x_0 \in I$ wenn der Limes
$$f(x) = \lim_{x\to x_0} \frac{f(x) - f(x0)}{x-x_0}\text{ existiert}$$
Ableitungsregeln: Produkt, Kettenregel, Umkehrfunktion $\leadsto$ Kann "alle" Ableitungen ausrechnen
8.11 $f: I = (a,b) \to \R$ differenzierbar, $f$ hat ein lokales extremum in $x_0 \in (a, b)$ \Larr{} $f'(x) = 0$\\*
\begin{tikzpicture}[domain=-0.3:2,prefix=plots/, smooth]
\draw[very thin,color=gray] (-0.3,-0.25) grid (1.99,1.99);
\draw[->] (0.3,0) -- (2.5,0) node[right] {$x$};
\draw[->] (0,0.3) -- (0,2) node[above] {$y$};
% Graphen beschriftung
\draw[color=blue] plot[id=21.1_cosp1] function{cos(x) + 1} node[below, midway] {};
\end{tikzpicture}\\*
8.12 (Satz von Rolle)\\*
Sei $f: [a, b] \to \R$  diff'bar\\*
\begin{tikzpicture}[domain=-0.3:2,prefix=plots/, smooth]
\draw[very thin,color=gray] (-0.3,-0.25) grid (1.99,1.99);
\draw[->] (0.3,0) -- (2.5,0) node[right] {$x$};
\draw[->] (0,0.3) -- (0,2) node[above] {$y$};
% Graphen beschriftung
\draw[color=blue] plot[id=21.2_cos_tiefer] function{cos(x) * 1.5 + 1};
\end{tikzpicture}
\\*
\begin{tikzpicture}[domain=-0.3:2,prefix=plots/, smooth]
\draw[very thin,color=gray] (-0.3,-0.25) grid (1.99,1.99);
\draw[->] (0.3,0) -- (2.5,0) node[right] {$x$};
\draw[->] (0,0.3) -- (0,2) node[above] {$y$};
% Graphen beschriftung
% Graph ausdenken, oder als Linie zeichnen.
\end{tikzpicture}
$f(a) = f(b)$ dann existiert $x_0 \in (a, b)$ mit $f'(x) = 0$

\sS{Satz (Mittelwertsatz der Differenzialrechnung)}
Sei $f:[a,b]→\R$ stetig, auf (a,b) differenzierbar, dann gibt es ein $x_0\in (a,b)$ mit.
$$f'(x_0)=\frac{f(b)-f(a)}{b-a}=\lambda$$
GRAPH
\bew
Sei $g:(a,b)→\R,\ g(x)=f(x)-\lambda·x$
$$\text{Rechne }g(a)-g(b)=f(a)-\lambda·a-f(b)+\lambda·b = f(a)-f(b)-\lambda(a-b)=f(a)-f(b)-\frac{f(b)-f(a)}{b-a}(a-b)=0$$
Satz von Rolle auf $g$ anwendbar \Rarr\ es gibt $x_0\in (a,b),\ g'(x_0)=0$
$$f(x)=g(x)+\lambda x\ \Rarr\ f'(x_0)=g'(x_0)+\lambda=\lambda$$ \qed

\sS{Folge}
Sei $f: T \to \R$ diffbar, $f'(x) = 0$ für alle $x$ dann ist $f$ konstant.
\bew
Sei $x_1 < x_2$ in $I$\\*
Es gilt $x_0$ mit $x_1 < x_0 < x_2$ mit $f(x_1) - f(x_2) = f(x_0) \cdot f(x_1 - x_2) = 0$\\*
\Rarr{} $f(x_1) = f(x_2) \Rarr f$ konstant. \qed{}\\*
Mittelwertsatz \qed

\sS{Satz (Monotonie)}
Sei $f:[a,b]→\R$ stetig, diff'bar auf $(a,b)$\\*
$f'(x)\geq 0$ für alle $x\in (a,b)$ \equ\ $f$ monoton wachsend
$f'(x)\leq 0$ für alle $x\in (a,b)$ \equ\ $f$ monoton fallend
$f'(x)> 0$ für alle $x\in (a,b)$ \Rarr\ $f$ streng monoton wachsend
$f'(x)< 0$ für alle $x\in (a,b)$ \Rarr\ $f$ streng monoton fallend
\bew
Angenommen $f'(x)\geq 0$ für alle $x$\\*
Gegeben sei $a<x_1<x_2<b$
\sss{Zeige}
$f(x_1)\leq f(x_2)$\\*
Mittelwertsatz: es gibt $x_0$ mit $x_1<x_0<x_2$ und $f(x_2)-f(x_1)=\underbrace{f'(x_0)}_{\geq 0}\underbrace{x_2-x_1}_{>0}$ \Rarr\ $f(x_2)\geq f(x_1)$, also monoton wachsend.\\*
Analog folgen alle "\Rarr" des Satzes.\\*
Angenommen $f$ monoton wachsend\\*
Sei $x_0 \in (a, b)$\\*
Zeige: $f'(x) \geq 0$\\*
% TOFIX Richtigen Arrow suchen
$f'(x_0) = \lim_{x \searrow x_0}\frac{f(x) - f(x_0)}{x - x_0}$\\*
$x > x_0 \Rarr x - x_0 > 0,\ f(x) - f(x_0) \geq 0$\\*
Analog: für monoton fallend \Rarr\ $f'(x)\leq 0$ für alle $x$\qed
\bsp
\enum{
	\item{$cos:[0,\pi]→\R$ streng monoton fallend\\*
\begin{tikzpicture}[domain=-0.3:3.5,prefix=plots/, smooth]
\draw[very thin,color=gray] (-0.3,-1.25) grid (3.49,1.25);
\draw[->] (0.3,0) -- (3.5,0) node[right] {$x$};
\draw[->] (0,-1.5) -- (0,1.5) node[above] {$y$};
% Graphenbeschriftung bei PI
\draw[color=blue] plot[id=21.4_cos] function{cos(x)};
\end{tikzpicture}
\\*
\begin{tikzpicture}[domain=-0.3:3.5,prefix=plots/, smooth]
\draw[very thin,color=gray] (-0.3,-1.25) grid (3.49,0.25);
\draw[->] (0.3,0) -- (3.5,0) node[right] {$x$};
\draw[->] (0,-1.5) -- (0,5) node[above] {$y$};
% Graphenbeschriftung bei PI
\draw[color=blue] plot[id=21.5_sin] function{sin(x)};
\end{tikzpicture}
	$cos'=-sin,\ -sin(x)<0$ für alle $x\in (0,\pi)$.
	}
	\item{$f:\R→\R,\ f(x)=x^3$
\begin{tikzpicture}[domain=-1.5:1.5,prefix=plots/, smooth]
\draw[very thin,color=gray] (-1.49,-3.99) grid (1.49,1.24);
\draw[->] (-1.6,0) -- (1.6,0) node[right] {$x$};
\draw[->] (0,-3) -- (0,3) node[above] {$y$};
\draw[color=blue] plot[id=21.4_x3] function{pow(x, 3)} node[below, midway] {$f(x) = x^3$};
\draw[color=red] plot[id=21.4_x3] function{pow(3x, 2)} node[below, midway] {$f'(x) = 3x^2$};
\end{tikzpicture}\\*
	$f'(x)\geq 0$ für alle $x$ $f'(0)=0$ trotzdem $f$ streng monoton wachsend
	}
}

\sS{Satz}
Sei $f:(a,b)→\Re$ in $x_0$ zweimal differenzierbar mit $f'(x_0)=0$, dann gilt:
\enum{
	\item{Wenn $f''(x_0)<0$ dann hat $f$ in $x_0$ ein lokales Maximum}
	\item{Wenn $f''(x_0)>0$ dann hat $f$ in $x_0$ ein lokales Minimum}
}
(Wenn $f''(x_0=0)$, dann keine Aussage)
\bew
Sei $f''(x_0)<0)$.
$$f''(x_0)=\lim_{x→∞}\frac{f'(x)-f'(x_0)}{x-x_0}$$
\Rarr Es gibt ein $ε>0$, so dass 
$$|x-x_0|<ε\ \Rarr\ \frac{f'(x)-f'(x_0)}{x-x_0}<0$$
% rest bis pause
d.h.
\begin{itemize}
\item[a]{$x_0 < x < x_0 + \e$ \Rarr{} $f'(x) - f'(x_0) < 0$ \Rarr $f'(x) < 0$}
\item[b]{$x_0 - \e < x < x_0$ \Rarr{} $f'(x) - f'(x_0) > 0$ \Rarr $f'(x) > 0$}
\end{itemize}
8.15 \Rarr{} $f$ streng monoton fallend auf $[x_0, x_0 + \e]$ wegen a)
			$f$ streng monoton steigend auf $[x_0 - \e, x_0]$ wegen b)
\bsp $f(x) x^3 - 3x$ $f: \R \to \R$\\*
$f'(x) = 3x^2 - 3$\\*
$f''(x) = 6x$\\*
Nullstelle (NST) von $f'$: $f'(x) = 0 \equ 3x^2 - 3 = 0 \equ x^2 = 1 \equ x \in \{1, -1\}$\\*
Anwendung von $f''$ an NST von $f'$: $f(1) = 6$

\uS{Regeln von de l' Hospital}
Ziel: Berechnung eines Limes $\lim_{x→a}\frac{f(x)}{g(x)}$ wenn $\lim_{x→a} f(x)=0=\lim_{x→a} g(x)$ oder $\lim_{x→a}g(x)=\pm ∞$

\sS{Satz}
Sei $I=(a,b)$ mit $-∞\leq a<b\leq ∞$\\*
Seien $f,g:I→\R$ differenzierbare Funktionen
\sss{Annahme}
Der Limes
$$\lim_{x→a}\frac{f'(x)}{g'(x)}=c\eR\text{ existiert}$$
\enum{
	\item{Wenn $\lim_{x→a}f(x)=\lim_{x→a}g(x)=0$, dann gilt $\lim_{x→a}\frac{f(x)}{g(x)}=c$}
	\item{Wenn $\lim_{x→a}g(x)=∞$ oder $-∞$, dann $\lim_{x→a}\frac{f(x)}{g(x)}=c$}
}
Analog für $x \to b$ (ohne Beweis)
\bsp
\begin{enumerate}
\item{$\lim_{x\to 0} \frac{sin(x)}{x} = ?$ \\*
$\lim_{x\to 0} x = 0$, $\lim_{x\to 0} \frac{sin(x)} = 0$\\*
$x' = 1, sin' = cos$\\*
$\leadsto$ Berechne\\*
$\lim_{x\to0} \frac{cos(x)}{1} = cos(0) = 1$ existiert.\\*
l'Hospital \Rarr{} $lim_{x \to 0} \frac{sin(x)}{x} = 1$}
\item{$\lim_{x \to \infty} \frac{log(x)}{x}$\\*
$\lim_{x \to \infty} x = \infty$\\*
$log(x)' = \frac{1}{x}$, $x' = 1$\\*
$\leadsto$ Berechne $$\lim_{x \to \infty} \frac{\frac{1}{x}}{1} = \lim_{x \to \infty} \frac{1}{x} = 0$$
8.17 \Rarr $\lim_{x \to \infty} \frac{log(x)}{x} = 0$}
\item{Rationale Funktion\\*
$\lim_{x \to \infty} \frac{x^3 + x}{2x^2 + 5}$\\*
$f(x) = x^3 + x$, $g(x) = 2x^2 + 5$\\*
$\lim_{x \to \infty} g(x) = \infty$\\*
$\leadsto \ f'(x) = 2x + 1$, $g(x) = 4x$\\*
$\leadsto$ Rechne\\*
$\lim_{x \to \infty} \frac{2x + 1}{4x}$\\*
$= \lim_{x \to \infty} (\frac{1}{2} + \frac{1}{4x}) = \frac{1}{2}$
$\Rarr \lim_{x \to \infty} \frac{x^2 + x}{2x^2 + 5} = \frac{1}{2}$
}
\item{$$\lim_{x→0}\left(\frac{1}{sin(x)}-\frac{1}{x}\right)$$
$$\frac{1}{sin(x)}-\frac{1}{x}=\frac{x-sin(x)}{x·sin(x)}=\frac{f(x)}{g(x)}$$
$$f(x)=x-sin(x),\ g(x)=x·sin(x)$$
$$\lim_{x→0}\left(x-sin(x)\right)=0=\lim_{x→0}\left(x·sin(x)\right)$$
$$f'(x)=1-cos(x),\ g'(x)=sin(x)+x·cos(x)$$
$$\lim_{x→0}\frac{1-cos(x)}{sin(x)+x·cos(x)}=?$$
$$\lim_{x→0}(1-cos(x))=0=\lim_{x→0}sin(x)+x·cos(x)$$
Wende 8.17 nochmal an
$$f''(x)=sin(x),\ g''(x)=cos(x)+cos(x)-x.sin(x)$$
$$lim_{x→0}\frac{f''(x)}{g''(x)}=lim_{x→0}\frac{sin(x)}{2cos(x)-x·sin(x)}=\footnote{$lim_{x→0}2cos(x)-x·sin(x)=2$\\*
$lim_{x→0}sin(x)=0$}\frac{0}{2}=0$$
$$\underset{8.17}{\Rarr} lim_{x→0}\frac{f'(x)}{g'(x)=0}\underset{8.17}{\Rarr} lim_{x→0}\frac{f(x)}{g(x)=0}$$
}
\end{enumerate}\newpage
% Kopfzeile beim Kapitelanfang:
\fancypagestyle{plain}{
%Kopfzeile links bzw. innen
\fancyhead[L]{\calligra\Large Vorlesung Nr. 22}
%Kopfzeile rechts bzw. außen
\fancyhead[R]{\calligra\Large 10.01.2013}
}
%Kopfzeile links bzw. innen
\fancyhead[L]{\calligra\Large Vorlesung Nr. 22}
%Kopfzeile rechts bzw. außen
\fancyhead[R]{\calligra\Large 10.01.2013}
% **************************************************
%
\chapter{Integration}
\sss{Idee} Sei $f:[a,b]→\R_{\geq 0}$\\*
\begin{tikzpicture}[domain=0.5:2,prefix=plots/, smooth]
\draw[very thin,color=gray] (-0.3,0.0) grid (2.5,2.0);
\draw[->] (-0.3,0) -- (2.5,0) node[right] {$x$};
\draw[->] (0,-0.3) -- (0,2) node[above] {$y$};
\draw (0.5,0) node[anchor=north] {$a$};
\draw (2,0) node[anchor=north] {$b$};
\draw[color=blue] plot[id=22.1_int1] function{sin(x)} node[below, midway] {};
\end{tikzpicture}\\*
$\int_a^b f(x)dx=Fläche$ zwischen Graphen von $f$ und $x$-Achse\\*
Wenn allgemeiner $f:[a,b]→\R$,\\*
\begin{tikzpicture}[domain=0.5:2,prefix=plots/, smooth]
\draw[very thin,color=gray] (-0.3,0.0) grid (2.5,2.0);
\draw[->] (-0.3,0) -- (2.5,0) node[right] {$x$};
\draw[->] (0,-0.3) -- (0,2) node[above] {$y$};
\draw (0.5,0) node[anchor=north] {$a$};
\draw (2,0) node[anchor=north] {$b$};
% Label zwischen Graph und X-Achse einfügen
\draw[color=blue] plot[id=22.2_int2] function{sin(2x + 0.5)} node[below, midway] {};
\end{tikzpicture}\\*
dann zählen Flächen unterhalb der $x$-Achse negativ
$$\int_a^b f(x)dx =F_1-F_2+F_3$$
\sss{Fragen}
Formale Definition des Intervalls? Welche Funktionen sind interpretierbar? Eigenschaften, Berechnung des Integrals.

\uS{Treppenfunktion}
\sS{Definition der Treppenfunktion}
Sei $a,b \eR$, $a<b$
\enum{
\item Eine Funktion $f:[a,b]→\R$ heißt Treppenfunktion, wenn es eine Unterteilung $a=x_0<x_1<x_2<x_3<…<x_n=b$ gibt, so dass $f$ auf $(x_{i-1},x_{i})$ konstant ist, dass heißt $f(x)=c_i$ für alle $x$ mit $ x_{i-1}<x<x_{i} $\\*
\begin{tikzpicture}
\draw[very thin,color=gray] (-0.3,0.0) grid (4.5,2.0);
\draw[->] (-0.3,0) -- (2.5,0) node[right] {$x$};
\draw[->] (0,-0.3) -- (0,2) node[above] {$y$};
% Schrafierung einfügen
\draw (0.2, 1) -- (0.2, 1);
\draw (0.2, 0) - - (0.2, 1);
\draw (1, 0) - - (1, 1);
\draw (1, 0.3) -- (2, 0.3);
\draw (2, 0) - - (2, 1.2);
\draw (2, 1.2) -- (3, 1.2);
\draw (3, 0) - - (3, 1.2);
\end{tikzpicture}
\item In diesem Fall definiere
$$\int_a^b f(x)dx=\sum_{i=1}^n c_i(x_{i}-x_{i-1})$$
"Summe der Rechtecke"
\bem
Die Definition eines Integrals für die Treppenfunktion ist unabhängig von der Unterteilung
\bsp
\begin{tikzpicture}
\draw[very thin,color=gray] (-0.3,0.0) grid (4.5,2.0);
\draw[->] (-0.3,0) -- (2.5,0) node[right] {$x$};
\draw[->] (0,-0.3) -- (0,2) node[above] {$y$};
\draw (0.5,0) node[anchor=north] {$a$};
\draw (2,0) node[anchor=north] {$b$};
% Schrafierung einfügen
\draw (0.2, 1) -- (0.2, 1);
\draw (0.2, 0) - - (0.2, 1);
\draw (1, 0) - - (1, 1);
\draw (1, 0.3) -- (2, 0.3);
\end{tikzpicture}\\*
(ohne formalen Beweis)
}

\sS{Lemma}
Seien $f,g:[a,b]→\R$ Treppenfuntionen\\*
Dann gilt:
\enum{
\item $\ds\int_a^b (f+g)(x)dx=\int_a^b f(x)dx+\int_a^b g(x)dx$
\item $\ds c\eR \int_a^b c·f(x)dx=c·\int_a^b f(x)dx$
\item Wenn $f\leq g$, dass heißt $f(x)\leq g(x)\ ∀x$, dann $\ds \int_a^b f(x)dx\leq \int_a^b g(x)dx$\\*
\begin{tikzpicture}[domain=0.5:2,prefix=plots/, samples=10]
\draw[very thin,color=gray] (-0.3,0.0) grid (2.5,2.0);
\draw[->] (-0.3,0) -- (2.5,0) node[right] {$x$};
\draw[->] (0,-0.3) -- (0,2) node[above] {$y$};
\draw[color=blue] plot[id=22.5_x3const, const plot] function{-(x-2)**3} node[below, midway] {};
\draw[color=red] plot[id=22.5_x3] function{-(x-2)**3} node[below, midway] {};
\end{tikzpicture}\\*
(ohne formalen Beweiß)
}

\uS{Das Riemannsche Integral}
\ul{Idee} Sei $f: [a, b] \to \R$ beliebige Funktion.\\*
Wenn $g \leq f$ und $g$ Treppenfunktion dann sollte $\int g(x)dx < \int f(x)dx$
Wenn $f \leq h$ und $f$ Treppenfunktion dann sollte $\int f(x)dx < \int h(x)dx$
Wenn $\int^b_a f(x)dx$ durch diese ($\infty$-vielen) Bedingungen festgelegt wird, nennen wir $f$ integrierbar und $\int_a^b f(x) dx$ ist definiert.

\sS{Definition des Riemannschen Integral}
Sei $f:[a,b]→\R$ beschränkte Funktion\\*
Unterintegral:
$$sup \left\{\int_a^b g(x)dx\mid g:[a,b]\text{ Treppenfunktion mit }g\leq f\right\}=:\int_a^b{}_* f(x)dx$$
Oberintegral:
$$inf \left\{\int_a^b h(x)dx\mid h:[a,b]\text{ Treppenfunktion mit }f\leq h\right\}=:\int_a^b{}^* f(x)dx$$
(\ul{Idee} Wenn $\int_a^b f(x)$ definiert, sollte $ \int_a^b{}_* f(x)dx\leq \int_a^b f(x)dx \leq \int_a^b{}^* f(x)dx $)
\sss{Definition}
$f$ heißt integrierbar, wenn $\int_a^b{}_* f(x)dx=\int_a^b{}^* f(x)dx$\\*
Dann setzte $\int_a^b f(x)dx:=\int_a^b{}_* f(x)dx$

\sS{Bemerkung}
$f:[a,b] \to \R$ ist integrierbar \equ{} es gilt: Für jedes \e > 0 gibt es eine Treppenfunktion $g,\ h$ mit $g \leq f \leq h$ mit $\int_a^b h(x)dx - \int_a^b g(x)dx < \e$. Damit ist $\int_a^b f(x)$ bist auf $\e$ festgelegt.

\sS{Satz Eigenschaften des Integrals}
Seien $f, g: [a, b] \to \R$ integrierbar, dann sind auch $f + g$ und $c \cdot f$ integrierbar und
\enum{
\item $\int_a^b (f + g)(x) dx = \int_a^b f(x) + \int_a^b g(x)$
\item $\int_a^b (c \cdot f)(x) dx = c \cdot \int_a^b f(x)$
\item wenn $f \leq g$ dann $\int f(x)dx \leq \int g(x)dx$
}
\bew
\notat{$I(f) = \int f(x)dx$}
Sei $\e > 0$ gegeben.\\*
Wähle Treppenfunktion $f_1, f_2, g_1, g_2$ mit $f_1 < f < f_2$ und $g_1 < g < g_2$\\*
$I(f_2) - I(f_1) < \e$, $I(g_2) - I(g_1) < \e$\\*
\Larr{} $f_1 + g_1 < f + g < f_2 + g_2$\\*
%Lustige Pfeile
$I(f_2 + g_2) - I(f_1 + g_1) = I(f_2) - I(f_1) + I(g_2) - I(g_1) < \e + \e = 2\e$\\*
Das für jedes $\e > 0$ \\*
$|I(f + g) - I(f) - I(g)| \leq |I(f + g) - I(f_1 + g_1)| + |I(f) - I(f_1)| + |I(f) - I(f_1)| = 2\e + \e + \e = 4\e$\\*
(Dreiecksungleichung)\\*
\Rarr{} $I(f + g) - I(f) - I(g) = 0$\\
Rest des Satzes analog. \qed{}

\sS{Satz}
Sei $f:[a,b]→\R$ stetig, dann:
\enum{
\item Für jedes $ε>0$ gibt es eine Treppenfuntion. $g:[a,b]→\R$ mit $|f(x)-g(x)|<ε$ für alle $x\in[a,b]$\\*
\begin{tikzpicture}[domain=0.5:2,prefix=plots/, smooth]
\draw[very thin,color=gray] (-0.3,0.0) grid (2.5,2.0);
\draw[->] (-0.3,0) -- (2.5,0) node[right] {$x$};
\draw[->] (0,-0.3) -- (0,2) node[above] {$y$};
\draw[color=blue] plot[id=22.6_x3const, const plot, samples=10] function{(0.3x-2)**2} node[below, midway] {};
\draw[color=red] plot[id=22.6_x3] function{(0.3x-2)**2} node[below, midway] {};
\end{tikzpicture}
\item $f$ ist integrierbar
}
\bew
Zeige 2) unter Annahme von 1).\\*
Gegeben $ε>0$. Setze $ε'=\frac{1}{2(b-a)}ε$\\*
Wegen 1) gibt es eine Treppenfunktion $g:[a,b]→\R$ mit $|f(x)-g(x)|<ε'$.
$$g_1(x)=g(x)-ε',\ g_2(x)=g(x)+ε'\ \Rarr\ g_1\leq f \leq g_2$$
\alg{&\int_a^b g_2(x)dx-\int_a^b g_1(x)dx=\int_a^b (g_2-g_1)(x)dx=\int_a^b \underset{\overset{\uparrow}{konstante\ Funktion}}{2ε'}(x)dx\\*
&={2ε'}(b-a)=\frac{1}{(b-a)}·ε·2(b-a)=ε\ \underset{9.4}{\Rarr} f\text{integrierbar}}
Zeige \enum{
\item Gegeben sei $\e > 0$\\
6.24 \Rarr{} $f$ gleichmäßig aber stetig. d.h. es gibt $\delta > 0$ so dass gilt:\\*
Wenn $|x-y|<\delta$ dann $|f(x) - f(y)|<\e$\\*
Wähle Unterteilung $a = x_0 < x_1 < x_2 < ... < x_n = b$ mit $x_i - x_{i-1} < \delta$
%Graph
Sei $c := f(x_i)$ \\*
Definiere Treppenfunktion $g:[a,b] \to \R$\\*
$x_{i-1} < x < x_i$ \Rarr{} $g(x) = c_i = f(x)\ (1 \leq i \leq n)$\\*
$g(x_0) = f(x_0)$ dann $|f(x) - g(x)| < \e$ für alle $x$.\qed{}
}

\sS{Satz (Mittelwertsatz der Integralrechnung)}
Sei $f:[a,b]→\R$ stetig (somit integrierbar)\\*
Dann gibt es ein $x_0\in [a,b]$ mit $\ds\int_a^b f(x)dx=f(x_0)(b-a)$\\*
GRAPH
\bew
\desc{Sei}{$m=inf{f(x)\mid x\in [a,b]}$\\$M=sup{f(x)\mid X\in [a,b]}$}
6.11 \Rarr\ \ul{Bekannt} es gibt $x_1,x_2\in [a,b]$ mit $f(x_1)=m,f(x_2)=M$\\*
$f(x_1)\leq f(x_2)$ für alle $f(x)\leq f(x_2)$
\Rarr{} $f(x_1)(b-a) = \int_a^b f(x) \leq \int_a^b dx \leq_a^bf(x_2) dx = f(x_2)(b-a)$\\*
$f(x_1) \leq \frac{1}{b-a} \int_a^b f(x)dx \leq f(x)$\\*
Zwischenwertsatz \Rarr{} es gibt auch $x_0 \in [a, b]$ mit $f(x_0) = y$ \Rarr{} $f(x_0)(b-a) = \int_a^b f(x)dx$

(nachtrag)
\sS{Definition Mittelwertsatz}
Sei $f:[a,b]→\R$ integrierbar, $a<b$
$$\int_b^a f(x)dx=-\int_a^b f(x)dx$$
\sss{Konsequenz}
Sei $f:I→\R$ stetig $a,b,c\in I$, dann $$\int_a^b f(x)dx+\int_b^c f(x)dx=\int_a^c f(x)dx$$ egal wie $a,b,c$ liegen!\\*
WEITERE Graphen\newpage
% Kopfzeile beim Kapitelanfang:
\fancypagestyle{plain}{
%Kopfzeile links bzw. innen
\fancyhead[L]{\calligra\Large Vorlesung Nr. 23}
%Kopfzeile rechts bzw. außen
\fancyhead[R]{\calligra\Large 14.01.2013}
}
%Kopfzeile links bzw. innen
\fancyhead[L]{\calligra\Large Vorlesung Nr. 23}
%Kopfzeile rechts bzw. außen
\fancyhead[R]{\calligra\Large 14.01.2013}
% **************************************************
%
\wdh
\enum{
\item Integration der Treppenfunktion (leicht)
\item $f: [a, b] \to \R$ beschränkt
$$\int_a^b {}_* f(x) = sup \left\{ \int_a^b g(x)dx \mid g: [a,b] \to \R\text{ Treppenfunktion,} g \leq f \right\}$$
$$inf \left\{\int_a^b h(x)dx \mid h: f\ \leq h \right\} = \int_a^b {}^* f(x)$$
\begin{tikzpicture}[domain=0:2,prefix=plots/, smooth]
\draw[very thin,color=gray] (-0.3,0.0) grid (2.5,2.0);
\draw[->] (-0.3,0) -- (2.5,0) node[right] {$x$};
\draw[->] (0,-0.3) -- (0,2) node[above] {$y$};
\draw[color=black] plot[id=23.2_int] function{cos(x)+1} node[below, midway] {$f(x)$};
\draw[color=green] (0, 2) -- (0.5, 2);
\draw[color=green] (0.5, 2) -- (0.5, 1.87);
\draw[color=green] (0.5, 1.87) -- (1, 1.87);
\draw[color=green] (1, 1.87) -- (1, 1.54);
\draw[color=green] (1, 1.54) -- (1.5, 1.54);
\draw[color=green] (1.5, 1.54) -- (1.5, 1.07);
\draw[color=green] (1.5, 1.07) -- (2, 1.07);
\draw[color=green] (2, 1.07) -- (2.5, 1.07);
\draw[color=green] (2.5, 1.07) -- (2.5, 0.2);
\draw[color=green] (2.5, 0.2) -- (3, 0.2);
\draw[color=green] (3, 0.2) -- (3.5, 0.1);
\end{tikzpicture}\\*
$f$ \ul{integrierbar} wenn $\int_a^b{}_* f(x)dx=\int_a^b{}^* f(x)dx$, dann $\int_a^b f(x)dx:=\int_a^b{}_* f(x)dx$
\enum{
\item $f:[a,b]→\R$ stetig \Rarr\ integrierbar
\item Mittelwertsatz: Wenn $f:[a,b]→\R$ stetig, dann gibt es ein $x_0 \in [a,b]$ mit $\int_a^b f(x)dx=f(x_0)·(b-a)$\\*
(Grundlage aller Berechnungen) %Graph mit fläche
}
}

\uS{Hauptsatz der Differential und Integralrechnung}
\sS{Satz}
Sei $I\subseteq\R$ Intervall, $f:I→\R$ stetige Funktion, $a\in I$ feste Zahl.\\*
Definiere: $$F(x):=\int_a^x f(t)dt$$
(Erinnerung: Wenn $x<a$, dann $\int_a^x\underset{Def}{=}-\int_x^a$)\\*
Dann ist $F:I→\R$ \ul{differenzierbar} und $F'(x)=f(x)$.
\bew
Sei $h≠0$ $$\frac{F(x+h)-F(x)}{h}=\frac{1}{h}\left(\int_a^{x+h}f(t)dt-\int_a^xf(t)dt\right)=\frac{1}{h}\int_a^{x+h}f(t)dt$$
% Treppengraph, erste Tafel
\begin{tikzpicture}
\draw[very thin,color=gray] (-0.3,0.0) grid (4.5,2.0);
\draw[->] (-0.3,0) -- (2.5,0) node[right] {$x$};
\draw[->] (0,-0.3) -- (0,2) node[above] {$y$};
\draw (0.5,0) node[anchor=north] {$a$};
\draw (2,0) node[anchor=north] {$b$};
% Schrafierung einfügen
\draw (0.2, 1) -- (0.2, 1);
\draw (0.2, 0) -- (0.2, 1);
\draw (1, 0) -- (1, 1);
\draw (1, 0.3) -- (2, 0.3);
\end{tikzpicture}\\*
Mittelwertsatz \Rarr\ es gibt $x_h\in[x,x+h]$ (wenn $h>0$) bzw. $x_h\in[x+h,x]$ (wenn $h<0$), so dass
\alg{&\int_a^{x+h}f(t)dt=f(x_n)·h\ \Rarr\ (*) = \frac{f(x_n)·h}{h}=f(x_n)\\
&\Rarr\ F'(x)=\lim_{h→0}\frac{F(x+h)-F(x)}{h}=\lim_{h→0}f(x_n)\underset{\footnote{\text{Für $h→0$ ist $x_n→x$, denn $|x-x_n|\leq h$, $f$ \ul{stetig}}}}{=}f(x)\ \Rarr \text{Behauptung}\qed}

\sS{Definition Stammfunktion}
Sei $f:I→\R$ Funktion. Eine Funktion $F:I→\R$ heißt Stammfunktion von $f$ wenn $F$ differenzierbar und $F'=f$
\bem
9.9 \Rarr\ Jede stetige Funktion $f$ hat eine Stammfunktion

\sS{Satz}
Sei $F$ Stammfunktion von $f$\\*
Eine Funktion $G:I→\R$ ist Stammfunktion von $f$ \equ\ $F-G$ konstant, dass heißt $G=F+c$ mit $c\eR$
\bew
$G$ differenzierbar mit $G'=f$ \equ\ $G-F$ differenzierbar mit $(G-F)'=f-f=0$ \equ\ $G-F$ konstant (bekannt)\qed

\sS{Satz (Hauptsatz der Differenzial und Integralrechnung)}
Sei $f:[a,b]→\R$ stetig, $F:[a,b]→\R$ Stammfunktion von $f$, dann $$\int_a^{b} f(x)dx=F(b)-F(a)=:F(x) \left|\ary{b\\a}\right.$$
\bew
Sei $G(x):=\int_a^xf(t)dt$, $G:[a,b]→\R$.\\*
9.9 \Rarr\ $G'=f\ \underset{9.11}{\Rarr}\ G-F=c$ konstant, $c\eR$. $G=F+c$
$$\int_a^bf(x)dx=G(b)=G(b)-\underbrace{G(a)}_{=0}=F(b)+c-(F(a)+c)=F(b)-F(a)$$\qed
\sss{Folge} Berechnung von Integralen \equ\ Finden von Stammfunktionen = Umkehrung des Ableitens
\notat{"$\int f(x)dx=F(x)$"(*)}
soll heißen: $F$ ist Stammfunktion von $f$, dass heißt $F'=f$\\*
Vorsicht: (*) ist keine echte Gleichung, bestimmt $F(x)$ nur bis auf Addition einer Konstante
\bsp
Sei $s\in\R\setminus\{-1\}$ $\ds\int_a^b x^s dx$ Erlaubter Integrationsbereich:\\*
% Graph an der dritten Tafel mit schrafur.
\begin{tikzpicture}[domain=0.5:2,prefix=plots/, smooth]
\draw[very thin,color=gray] (-0.3,0.0) grid (2.5,2.0);
\draw[->] (-0.3,0) -- (2.5,0) node[right] {$x$};
\draw[->] (0,-0.3) -- (0,2) node[above] {$y$};
\draw (0.5,0) node[anchor=north] {$a$};
\draw (2,0) node[anchor=north] {$x-h$};
\draw (0.5, 0) -- (0.5, 1.5);
\draw (2,0) -- (2, 0.6);
\draw[color=blue] plot[id=23.1_int2] function{sin(2x+0.5)} node[below, midway] {};
\end{tikzpicture}
\enum{
\item $s \eN$: $a,b$ beliebig
\item $s \eZ$: $s\leq -2:\ x=0$ ausschließen $x^s = \frac{1}{x^{-s}}$ entweder $a,b < 0$ oder $a,b > 0$
\item $s \in \R \setminus \Z$\\*
$x^3 := e^{3 \cdot log (x)}$ nur definiert für $x > 0$
$a, b > 0$
Suche $F$ mit $F' = x^s$\\*
$F=\frac{1}{s+1} x^{s+1}$ $F' = (s+1) \frac{1}{s + 1} x^2 = s^2$\\*
$s \neq -1 \Rarr s+1 = 0$\\*
$\int_a^b $ % NOOB
Für uns nur die Reste.
}
\bsp
\enum{
\setcounter{enumi}{1}
\item{$\int e^x dx = e^x$, denn $(e^x)'=e^x$}
\item $\int sin(x)dx=-cos(x) denn (-cos(x))'=sin(x)$\\*
$\int cos(x)dx=sin(x) denn (sin(x))'=cos(x)$
(Unbestimmte Integrale) $\underset{z.B.}{\Rarr}\ \Int_a^b e^xdx=e^x \left|\ary{b\\a}\right.=e^b-e^a$ etc.
$$\int e^{cx}dx=\frac{1}{c}e^{cx}\qquad\frac{1}{c}(e^{cx})'=\frac{1}{c}·c·e^{cx}=e^{cx}$$
$$\int x^sdx, s≠1 …\text{ bekannt aus 1)}$$
\item $$\int_a^b x^{-1} dx = \int_a^b \frac{1}{x} dx$$
 Graph f(x) = 1/x
Erlaubte Grenzen: $x \neq 0$\\*
d.h. $a, b > 0$ oder $a, b < 0$
\itm{
\item Sei $a, b > 0\quad$. $log'(x) = \frac{1}{x}\quad$ $\log: \R_0 \to \R$\\*
\Rarr{} $\int_a^b \frac{1}{x}dx = log(x) \mid^b_a$ wenn $a, b > 0$
\item Sei $a, b < 0$ Sei $g: \R_{<0} \to \R$, $g(x) = log(-x) = log(|x|)$\\*
$g'(x) = \frac{1}{-x} \cdot (-1) = \frac{1}{x}$\\*
$\int_a^b \frac{1}{x} = log(-x) \mid_a^b = log(|x|) \mid_a^b$
}
In beiden Fällen:
$\int \frac{1}{x}dx = log(|x|)$ wenn $x \neq 0$
\item $\int\frac{1}{1+x^2}dx=arctan(x)$\\*
GRAPH fehlt noch
\bew
$tan=\frac{sin}{cos}:[-\frac{\pi}{2},\frac{\pi}{2}]→\R$\\*
$(tan(x))'=\frac{1}{cos(x)^2}$\\*
% GRAPH
Wenn $y=tan(x)\qquad arctan'(y) = \frac{1}{tan'(x)}=cos(x)^2 \overset{!}{=} \frac{1}{1+y^2}$
$$cos(x)^2 \overset{!}{=} \frac{1}{1+\frac{sin(x)^2}{cos(x)^2}}=\frac{cos(x)^2}{cos(x)^2+sin(x)^2}=cos(x)^2$$\ok
}
Grundprinzip:\\*
Jede Ableitungsregel gibt eine Integrationsregel:
\itm{
\item Kettenregel $\to$ Substitutionsregel
\item Produktregel $\to$ Partielle Integration
}

\uS{Substitutionsregeln}
\sS{Satz}
Sei $f:I→\R$ stetig, $\phi:[a,b]→I$ differenzierbar, dann $$\int_a^b f(\phi(t))·\phi'(t)dt=\int_{\phi(x)}^{\phi(b)}f(x)dx$$
\bew
Sei $F:I→\R$ Stammfunktion von $f$, dass heißt $F'=f$
$$(F·\phi)'(x)=F'(\phi(x))·\phi'(x)=f(\phi(x))·\phi'(x)\ \Rarr \int_{\phi(x)}^{\phi(b)}f(x)dx=F(x)\left|\ary{\phi(b)\\\phi(a)}\right.
=F(\phi(b)-F(\phi'(a))$$
$F(\phi(X)) \mid_a^b = \int_a^b (F(\phi \circ F)'(x)) dx = \int_a^b f(\phi (x)) \cdot \phi'(x)dx$
\bsp
\enum{
\item $\int_a^b f(x + c)dx = \int_a^b \underbrace{f(\phi(t))}_f(t+c) \cdot \underbrace{\phi'(t)}_{=1} dt) = \int_{a + c}^{b + c} f(x) dx$\\*
$\phi(t) = t + c$
$\phi'(t) = 1$
\item $\int_a^b f(c \cdot x) = \int_a^b f(\phi(t)) \cdot \frac{\phi'(t)}{c}dt = \frac{1}{c} \cdot \int_a^b f(x) dx$
\item $$\Int_a^b t:f(t^2)dt=\footnote{$\phi(t)=t^2\quad\phi'(t)=2t$}\frac{1}{2}\Int_a^b\underbrace{\phi'(t)}_{2t}·f(\phi(t))=\frac{1}{2}\Int_{a^2}^{b^2}f(x)dx$$
z.B. $\Int_0^1xe^{x^2}dx=\frac{1}{2}\int_{0^2}^{1^2}e^xdx$ \\*
$f(x)=e^x=\frac{1}{2}e^2\left|\ary{1\\0}\right.=\frac{e-1}{2}$
$F(\phi(X)) \mid_a^b = \int_a^b (F(\phi \circ F)'(x)) dx = \int_a^b f(\phi (x)) \cdot \phi'(x)dx$
}
\bsp
\enum{
\item $\int_a^b f(x + c)dx = \int_a^b \underbrace{f(\phi(t))}_f(t+c) \cdot \underbrace{\phi'(t)}_{=1} dt) = \int_{a + c}^{b + c} f(x) dx$\\*
$\phi(t) = t + c$
$\phi'(t) = 1$
\item $\int_a^b f(c \cdot x) = \int_a^b f(\phi(t)) \cdot \frac{\phi'(t)}{c}dt = \frac{1}{c} \cdot \int_a^b f(x) dx$
}
$f(x)=e^x=\frac{1}{2}e^2\left|\ary{1\\0}\right.=\frac{e-1}{2}$
\newpage
% Kopfzeile beim Kapitelanfang:
\fancypagestyle{plain}{
%Kopfzeile links bzw. innen
\fancyhead[L]{\calligra\Large Vorlesung Nr. 24}
%Kopfzeile rechts bzw. außen
\fancyhead[R]{\calligra\Large 17.01.2013}
}
%Kopfzeile links bzw. innen
\fancyhead[L]{\calligra\Large Vorlesung Nr. 24}
%Kopfzeile rechts bzw. außen
\fancyhead[R]{\calligra\Large 17.01.2013}
% **************************************************
%
\wdh
\ul{Hauptsatz}
Wenn $F: [a, b] \to \R$ eine Stammfunktion der stetigen Funktion $f: [a, b] \to \R$ ist, (d.h. $F' = f$) dann $\int_a^b f(x) dx = f(x) |_a^b$\\*
\ul{Substitutionsregel}
$$F' = f \Rarr (F \circ \phi)' = (F' \circ \phi) \cdot \phi' = (f \circ) \cdot \phi$$
$$\int_a^b f(\phi(x)) \cdot \phi(x) dx = F(\phi(b)) - F(\phi(a)) = \int_{\phi(a)}^{\phi(b)} f(x) dx$$
\bsp
\enum{
\setcounter{enumi}{3}
\item Sei $\varphi : [a, b] \to \R$ differenzierbar, $\phi(x) \neq 0$ für alle x.
$$\int_a^b \frac{phi'(x)}{\phi(x)} = \int_a^b f(\phi(x)) \cdot \phi(x) = \int_{\phi(a)}^{\phi(b)} \frac{1}{x} dx = log(|x|)|^b_a$$
$$= log(|\phi(b)|) - log(|\phi(a)|)$$
\item Fläche unterm Halbkreis\\*
GRAPH Halbkreis
$$(*) = \int_a^b \sqrt{1-x^2}dx$$
$x^2 + y^2 = 1$ (Pythagoras)
$y = \sqrt{1-x^2}$
Substituiere $x = sin(t)$
$\sqrt{1 - sin(t)^2} = \sqrt{cos(t)} = cos(t)$\\*
(Wenn $cos(t) \geq 0$, d.h. z.B.$-\frac{\pi}{2} \leq t \leq \frac{\pi}{2}$)
GRAPH cos(x) Intervall -pi/2 -> pi/2\\*
$\phi(t) = sin(t)$\\*
$\phi'(t) = cos(t)$\\*
$a = sin(u) \qquad b = sin(v)$\\*
$u:= arcsin(a) \qquad v:= arcsin(b)$\\*
$(*) = \int_{sin(u)}^{sin(v)} \sqrt1-x^2 dx$\\*
$=\int_u^v \sqrt{1-sin(t)^2} \cdot cos(t)dt$\\*
$=\int cos(t)^2 dt$\\*
$\leadsto$ Siehe Übung
}

\uS{Partielle Induktion}
\sss{Produktregel} $(f·g)'=f'g+fg'$
\sS{Satz (Partielle Induktion )}
Seien $f,g:[a,b]→\R$ stetig, differenzierbar, dass heißt $f',g'$ stetig\\*
Dann gilt $\Int_a^b f(x)g'(x)dx=f(x)g(x)\left|\ary{b\\a}\right.-\int_a^b f'(x)·g(x)dx$
\bew
$$\int_a^b f'(x)·g(x)+\int_a^b f(x)·g'(x)\underset{Produktregel}{=}\int_a^b (f·g)'(x)\underset{Hauptsatz}{=}f(x)·g(x)\left|\ary{b\\a}\right.\ \Rarr\ \text{Behauptung}$$\qed
\bsp
\enum{
\item $\Int_a^b log(x)dx = (*)$\\*
Sei $g(x)=x, g'(x)=1, f(x)=log(x)$
\alg{(*)&=\Int_a^b log(x)g'(x)dx = log(x)\left|\ary{b\\a}\right.-\underbrace{\int_a^b log(x)·x\ dx}_{\int_a^b \frac{x}{x} dx= b-a=x \left|\ary{b\\a}\right.}\\*
&=(log(x)-x)\left|\ary{b\\a}\right.=x(log(x)-1)\left|\ary{b\\a}\right.
}
\sss{Probe}
x(log(x)-1)'=…=log(x)
\item \alg{\int_a^b cos^2(x)dx&=\int_a^b cos(x)·sin'(x)dx=cos(x)·sin(x)\left|\ary{b\\a}\right.\int_a^b cos'(x)·sin(x)dx\\
&=cos(x)·sin(x)\left|\ary{b\\a}\right.+\int_a^b \underbrace{sin(x)·sin(x)}_{sin^2(x)=1-cos^2(x)}dx\\
&=cos(x)·sin(x)\left|\ary{b\\a}\right.+x\left|\ary{b\\a}\right.-\int_a^b cos^2(x)dx\\
&\Rarr\ 2\int_a^b cos^2(x)dx=(cos(x)sin(x)+x)\left|\ary{b\\a}\right.\ \Rarr\ 2\int_a^b cos^2(x)dx=\frac{1}{2}(…)}
\item $\int_a^b e^x cos(x) dx = \int_a^b e^x sin'(x) dx $\\*
$= e^x sin(x) \left|\ary{b\\a}\right. - \int_a^b e^x sin(x) dx$\\*
$= e^x sin(x) \left|\ary{b\\a}\right. + \int_a^b e^x cos'(x) dx$\\*
$= e^x sin(x) \left|\ary{b\\a}\right. + e^x cos(x) \left|\ary{b\\a}\right. - \int_a^b e^x cos(x) dx$\\*
$\Rarr \int_a^b e^x cos'(x) dx = \frac{1}{2}\left(e^x (sin(x) + cos(x))\right) \left|\ary{b\\a}\right.$
}

\uS{Uneigentliche Intregrale}
\sS{Definition Uneigentliche Integrale}
Sei $f:[a,∞)→\R$ Funktion, die auf jedem Intervall $[a,R]$ mit $a\leq R<∞$ integrierbar ist. Setzte $$\int_a^{∞}f(x)dx:=\lim_{R→∞}\int_a^{R}f(x)dx$$
(Wenn der Limes existiert), dann nennt man $\Int_a^{∞}f(x)dx$ \ul{konvergent}\\*
Analog für $f:(-∞,b]→\R$
\bsp
\enum{
\item $$f(x)=\frac{1}{x^2}\qquad \int_1^{∞}\frac{1}{x^2}dx=?$$
Graph
$$\int_1^{R}\frac{1}{x^2}dx=-\frac{1}{x}\left|\ary{R\\1}\right.=\frac{1}{1}-\frac{1}{R}=1-\frac{1}{R}$$
$$\int_1^{∞}\frac{1}{x^2}dx=\lim_{R→∞}(1-\frac{1}{R})=1$$
\item $$f(x)=\frac{1}{x}m\qquad \int_1^\infty \frac{1}{x} dx$$
$$\int_1^{R}\frac{1}{x}dx=-log(x)\left|\ary{R\\1}\right.=log(R)-\underbrace{log(1)}_{=0}=1$$
$$\int_1^{∞}\frac{1}{x^2}dx=\lim_{R→∞}log(R) \text{ existiert nicht}$$
(bzw. lim()=∞)
}

\sS{Definition}
Sei $f: [a, b) \to \R$ eine Funktion, die auf einem Intervall $[a, R]$ mit $a \leq R \leq b$ integrierbar ist.\\*
Setze $\Int_a^b f(x)dx = \lim_{R \to b} \int_a^b f(x)dx $ (wenn der Grenzwert existiert.)
Dann heißt $\int_a^b f(x)dx$ konvergent.\\*
Analog für $f: (a, b] \to \R$
\bsp
\enum{
\item $\int_0^1 \frac{1}{x}dx = ?$\\*
GRAPH des Integrals\\*
$f(x)= \frac{1}{x}$,  $f:(0, 1] \to \R$\\*
$\int_0^1 \frac{1}{x}dx = \lim_{R \to 0} \int_a^b \frac{1}{x}dx = \lim_{R \to 0} \left(\underbrace{log(1)}_{= 0} - log(R)\right)$
\bem
für $\R \to 0$ ist $log(R) \to -\infty$\\*
GRAPH log(x)
\Rarr{} $\int_a^b \frac{1}{x}dx$ divergiert.
\item $\int_0^1 \frac{1}{sqrt{x}}dx = \lim_{R\to 0} \int_R^1 x^{-\frac{1}{2}} dx$\\*
$= \lim_{R\to 0} \left( 2x^{\frac{1}{2}} \left|\ary{b\\R}\right. \right) = \lim_{R\to 0} \left(2\sqrt{1} - 2\sqrt{R} \right) = 2$\\*
GRAPHEN
$= F_1 + F_2 = F_3 + 1 = 2$
}

\sS{Definition}
Sei $-∞\leq a\leq b \leq ∞$, $f:(a,b)→\R$ eine Funktion, die auf jedem Interavall $[R,S]$ mit $a<R\leq S<b$ integrierbar ist.\\*
Wähle $c\in(a,b)$. Setzte $\Int_a^bf(x)dx=\Int_a^cf(x)dx+\Int_c^bf(x)dx$\\*
Wenn beinde Integrale konvergieren (Nach Definition 9.16, 9.15)
\bem
Unabhängig von $c$
GRAPH
\bsp
$\Int_{-∞}^{∞}e^{-x^2} dx =\sqrt{\pi}$
%gaussche normalverteilung Graph\newpage
% Kopfzeile beim Kapitelanfang:
\fancypagestyle{plain}{
%Kopfzeile links bzw. innen
\fancyhead[L]{\calligra\Large Vorlesung Nr. 25}
%Kopfzeile rechts bzw. außen
\fancyhead[R]{\calligra\Large 21.01.2013}
}
%Kopfzeile links bzw. innen
\fancyhead[L]{\calligra\Large Vorlesung Nr. 25}
%Kopfzeile rechts bzw. außen
\fancyhead[R]{\calligra\Large 21.01.2013}
% **************************************************
%
\sss{Uneigentliche Integrale}
zum Beispiel:
$$\int_a^{∞}f(x)dx:=\lim_{b→∞}\int_a^bdx$$
(wenn der lim existiert)

\uS{Integrale mit Reihen}
Beobachtung: eine Reihe $\Sum_{k=0}^{∞}a_k$ ist das unbestimmte Integral einer Treppenfunktion:\\*
\begin{tikzpicture}[domain=0:5,prefix=plots/, samples=5, const plot]
\draw[very thin,color=gray] (-0.3,0.0) grid (5,2.0);
\draw[->] (-0.3,0) -- (5.2,0) node[right] {$x$};
\draw[->] (0,-0.3) -- (0,2) node[above] {$y$};
\draw[color=black] plot[id=23.2_int] function{sin(x)+1.2} node[below, midway] {$f(x)$};
\end{tikzpicture}\\*
$$\Sum_{k=0}^{∞}a_k=\int_0^{∞}f(x)dx$$

\sS{Satz (Integralkriterium für Reihen)}
Sei $f:[1,∞)→\R$ monoton fallend mit $f(x)\geq 0$ für alle $x$. Für $n\geq 1$, sei
$$a_n=\sum_{k=1}^{n}f(k)-\int_1^{n+1}f(x)dx$$
Graph 1/x Treppenfunktion über dem graphen, fester abstand, schraffur treppenfunktion ohne graph
\enum{
\item die Folge $(a_n)$ konvergiert
\item die Reihe $\Sum_{k=1}^{∞}f(x)$ konvergiert \equ\ $\Int_1^{∞}f(x)dx$ konvergiert
}
\bew
$f$ monoton: $k\leq x\leq k+1\ \Rarr\ f(k)\geq f(x)\geq f(k+1)$
\Rarr
$$f(k)=\int_k^{k+1}f(k)dx\geq \int_k^{k+1}f(x)dx\geq \int_k^{k+1}f(k+1)dx=f(k+1)$$
\alg{a_n&=\Sum_{k=1}^{n}\left(\underbrace{f(k)-\int_k^{k+1}f(x)dx}_{\geq 0}\right) \underset{(b)}{\leq}\Sum_{k=1}^{n}\left(f(k)-f(k+1)dx\right)\\
&=f(1)-f(2)+f(2)-f(3)+…-f(n+1)=f(1)-f(n+1)\leq f(1)}
\Rarr\ $(a_n)$ monoton wachsend, beschränkt \Rarr\ konvergent \Rarr\ (1).\\
Sei $\gamma=\lim_{n→∞}a_n$\\*
\enum{
\setcounter{enumi}{1}
\item Angenommen $\int_0^{\infty} f(x) dx$ konvergent.\\*
$\sum_{k=1}^\infty f(k)= \lim_{n \to \infty} \sum_{k=1}^n f(k)$\\*
$=\lim_{n \to \infty} \left( \underbrace{f(x) - \int_1^{n+1} f(x)dx}_{\gamma} \right) + \underbrace{\int_1{n + 1} f(x)dx}_{\text{konvergiert}}$
}
\Rarr\ $\lim$ existiert (auch $\Sum_{k\geq 1}f(x)=\gamma+\int_1^{∞}f(x)dx$)\\
Richtung: $\int$ konvergiert \Rarr\ $\sum$ konvergiert ähnlich\qed
\bsp 
$f(x)=\frac{1}{x}$\\*
\begin{tikzpicture}
		[smooth]
		\pgfmathsetmacro\minx{-2}
		\pgfmathsetmacro\maxx{5}
		\pgfmathsetmacro\miny{-2}
		\pgfmathsetmacro\maxy{2}
		\draw[very thin,color=gray!40] (\minx,\miny) grid (\maxx,\maxy);
		\draw[->] (\minx-0.3,0) -- (\maxx+0.3,0) node[right] {$x$};
		\foreach \x in {\minx,...,-1}{\draw (\x cm,2pt) -- (\x cm,-2pt) node[below] {\tiny $\x$};}
		\foreach \x in {1,...,\maxx}{\draw (\x cm,2pt) -- (\x cm,-2pt) node[below] {\tiny $\x$};}
		\draw[->] (0,\miny-0.3) -- (0,\maxy+0.3) node[above] {$y$};
		\foreach \y in {\miny,...,-1}{\draw (2pt,\y cm) -- (-2pt,\y cm) node[left] {\tiny $\y$};}
		\foreach \y in {1,...,\maxy}{\draw (2pt,\y cm) -- (-2pt,\y cm) node[left] {\tiny $\y$};}
		\clip (\minx,\miny) rectangle (\maxx,\maxy);
		\draw[domain=1:5,color=black] plot function{1/x} node[right] {$\frac{1}{x}$};
\end{tikzpicture}%-2.0 ist in der beschriftung ein fehler
\sss{Folge}
$\Sum_{k=1}^{∞}$ konvergiert \equ\ $\Int_1^{∞}\frac{1}{x}dx$ konvergiert (nicht der Fall)
$$\left(\int_1^{∞}\frac{1}{x}dx=\lim_{b→∞}log(b)=∞\right)$$
\ul{Bsp} sei $s > 1$ $$\sum_{k=1}^{\infty} \frac{1}{k^s} \text{konvergiert} \equ \int_1^{\infty} \frac{1}{x^s} \text{konvergiert}$$

\sS{Beispiel}
Berechnung der Reihe $1 - \frac{1}{2} + \frac{1}{3} + \frac{1}{4} + \frac{1}{5} ... = \sum_{k=1}^{\infty} (-1)^{k+1}\frac{1}{k}$\\*
(konvergiert nach Leibniz)
$\sum_{k=1}^{infty} (-1)^{k+1} \frac{1}{k} = \lim_{n \to \infty} (-1)^{k+1} \frac{1}{k}$\\*
$=\lim_{n \to \infty} (1 + \frac{1}{2} + \frac{1}{3} + \frac{1}{4} + ... + \frac{1}{2n}) - 2(\frac{1}{2} + \frac{1}{4} + \frac{1}{6} + ... + \frac{1}{2n}$\\*
Sei $cn = \sum_{k=1}^n \frac{1}{k} =\footnote{NR $2(\frac{1}{2} + \frac{1}{4} + \frac{1}{6} + ... \frac{1}{2n}  )$} \lim{n \to \infty} (c_{zn} - c_n)$
\alg{(n = 2\quad &1- \frac{1}{2} + \frac{1}{3} + \frac{1}{4}\\
&1 + \frac{1}{2} + \frac{1}{3} + \frac{1}{4} - 2·\frac{1}{2} - 2·\frac{1}{4})}
Sei $a_n:=\sum_{k=1}^n \frac{1}{k} -\int_1^{n+1} \frac{1}{x}dx = c_n - \int_1^{n+1} \frac{1}{x}dx\underset{Satz\ (1)}{\Rarr}\lim_{\nif} (a_n) = \gamma$ existiert!

\alg{b_n &:= \int_1^{n+1} \frac{1}{x}dx\\
a_n &= c_n - b_n\quad (c_n = a_n + b_n)}
MISSING STUFF
%\alg{&\lim_{\nif}(c_{2n} - c_n) = \lim_{\nif} (a_{2n} - a_n + b_{2n} - b_n)\\
%\underbrace{&\lim_{\nif} a_{2n}}_{\gamma} - \underbrace{\lim_{\nif} a_n}_{\gamma} + \lim_{\nif} (b_{2n} - b_n) \\ \lim_{\nif} (b_{2n} - b_n)}
%\ary{&=\lim_{n→∞}\int_{n+1}^{2n+1}\frac{1}{x}dx=\lim_{n→∞}\left(log(2n+1)-log(n+1)\right)\\
%&=\lim_{n→∞}\left(log\frac{2n+1}{n+1}\right)=log\left(\lim_{n→∞}\frac{2n+1}{n+1}\right)=log(2)\to 2(n→∞)\text{, log stetig.}}

\chapter{Potenzreihen}
\sS{Definition Potenzreihen}
Eine Potenzreihe in der Variablen $z$ ist eine Reihe der Form
$$P(z)=\Sum_{k=0}^{∞}a_kz11k\quad\text{mit $a_k\eC$}$$
(reelle Potenzreihe: $a_k\eR$)
\bsp
Exponentialreihe
$$exp(z)=\sum_{k=0}^{∞}\frac{1}{k!}z^k\quad a_k=\frac{1}{k!}$$

\uS{Lemma}
Wenn $P(z_0)$ für ein $z_0\eC$ konvergiert, dann konvergiert $P(z)$ für jedes $z\eC$ mit $|z|<|z_0|$ absolut.
\bew
$P(z_0)=\sum  a_kz_0^k$ konvergiert \Rarr\ es gibt $C\eR$ mit $|a_kz_0^k|\leq C$ für alle $k$\\*
Sei $|z|<|z_0|$, dass heißt $q=\frac{|z|}{|z_0|}<1$
$$|a_kz^k|=|a_kz_o^k\left(\frac{z}{z_0}\right)|=|a_kz_0^k|·q^k\leq C·q^k$$
\Rarr\ Die Reihe $P(z) = \sum_k a_k z^k$ hat eine Majorante $\sum_k C\cdot q^k$, letztere konvergiert (Geometrische Reihe)\\*
Majorantenkriterium \Rarr\ $P(z)$ konvergiert absolut\qed

\sS{Defintion Konvergenzradius}
Der Konvergenzradius von $P(z)$ ist
$$R:=sup\left\{r\eR_{\geq 0}\mid P(r)\text{ konvergiert} \right\}\eR_{\geq 0}\cup\{∞\}$$
\bem
ERROREOS STUCTURES
%10.2 $\Rarr\ \ary{|z| < R\ \Rarr\ P(z)\text{ konvergiert absolut}\\|z| > R\ \Rarr\ P(z)\text{ divergiert}\\|z| = R\ \Rarr\ ?}$
					% GRAPH Konvergenzradius
\bsp
\enum{
\item $exp(z)$ konvergiert absolut für jedes $z \in \C$\\*
$R = \infty$
\item $\sum_{n=0}^2\infty 2^w\ z^w = 1 + 2z + 4z^2 +... = \sum_{n=0}^{\infty} (2z)^n$ geometrische Reihe.\\
$|z| \geq \frac{1}{2} \equ |2z| \geq 1$: divergiert\\*
$|z| < \frac{1}{2} \equ |2z| < 1$: konvergiert
}
Also $R=\frac{1}{2}$
\bsp
$$P(z)=\Sum_{n=0}^{∞}\frac{1}{n+1}z^w=1+\frac{z}{2}+\frac{z^2}{3}+\frac{z^3}{4}+…$$
$$R=1\ \Leftarrow\ \left\{\ary{z=1:\ P(1)=1+\frac{1}{2}+\frac{1}{3}+…\text{ divergiert}\\
z=-1:\ P(1)=1-\frac{1}{2}+\frac{1}{3}-\frac{1}{4}+…=log(2)\text{ konvergiert}}\right\}$$
\sss{Folge}
Methoden zur Berechnung des Konvergenzradius

\sS{Definition}
Sei $(a_n)_{n\geq0}$ reelle Folge.\\*
Bilde $b_m = sup(a_n)_{n \geq m} = sup \{a_m, a_{m+1},...\} \eR \cup \{\infty\}$\\*
Dann: $b_0 \geq b_1 \geq b_2 \geq ...$ $(b_n)$ monoton fallend \Rarr $\Lim_{n\to\infty} (b_n) =: \lim_{n\to\infty}sup(a_n) \in \R \cup \{\pm \infty\}$ existiert
\bsp
\alg{(a_n)&=(1,-1,\frac{1}{2},-1,\frac{1}{3},-1,\frac{1}{4},-1,\frac{1}{5}…)\\
(b_n)&=(1,\frac{1}{2},\frac{1}{2},\frac{1}{3},\frac{1}{3},\frac{1}{4},\frac{1}{4},\frac{1}{5},\frac{1}{5}…)}
$$\limsup(a_n)=\lim(b_n)=0$$
\alg{(a_n)&=(0,1,0,2,0,3,0,4…)\\
(b_n)&=(∞,∞,∞,∞,∞,∞,∞,…)}
$$\limsup(a_n)=∞$$
\alg{(a_n)&=(0,-1,-2,-3,-4,…)\\
(b_n)&=(0,-1,-2,-3,-4,…)}
$$\limsup(a_n)=\lim(b_n)=-∞$$
\bem
$C=\limsup_{n→∞}(a_n)$ ist durch folgende Eigenschaft eindeutig bestimmt:
Für jedes $ε>0$ gibt es
\enum{\item unendlich viele $n\eN$ mit $a_n\geq C-ε$
\item unendlich viele $n\eN$ mit $a_n> C+ε$}
SKIZZE\\*
(zumindest wenn C≠-∞)\\*
(ohne Beweis)

\sS{Satz}
Der Konvergenzradius einer Potenzreihe $P(z)=\sum  a_kz_0^k$ ist $R=\left(\limsup_{n→∞}\left(\sqrt[n]{|a_n|}\right)\right)^{-1} \eR_{\geq 0} \cup \{\infty\}$
(Setze hier $0^{-1} = \infty, \infty^{-1} =0$)\newpage
\end{document}
