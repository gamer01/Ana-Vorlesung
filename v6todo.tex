%Kopfzeile links bzw. innen
\fancyhead[L]{\calligra {\Large Vorlesung Nr. 6}}
%Kopfzeile rechts bzw. außen
\fancyhead[R]{\calligra 25.10.2012}



\setcounter{chapter}{3}
\setcounter{section}{9}

\subsection*{Wiederholung / Ergänzung}
Eine Folge reeler Zahlen $(a_n)$ konvergiert uneigentlich gegen ∞ wenn gilt:\\
Für jedes $C \in \R$ gilbt es ein $n \in\N$ mit $a_n > C$ für jedes $n \in\N$\\
\\
$(a_n)$ konvergiert uneigentlich gegen $- \infty$ wenn $(-a_n)$ gegen $\infty$ konvergiert.\\

\notat{
$a_n \to \infty \qquad \text{ für } n \to \infty$\\
$a_n \to - \infty \qquad \text{ für } n \to \infty$
}

\bsp
$a_n = n^2 \to \infty$\\
$a_n = -n^2 \to -\infty$\\
$a_n = (-1)^n \cdot n^2$\\
$(0, -1, 4, -9)$ konvergiert weder gegen $\infty$ noch gegen $ - \infty$

\sss{Rechenregeln:}
Angenommen $(a_n), (b_n)$ sind konvergente Folgen.\\
\begin{enumerate}
\item{$(a_n + b_n) \to a + b$}
\item{$(a_n \cdot b_n) \to ab$}
\item{$\ds\frac{1}{b_n} \to \frac{1}{b}$}
\item{$c \cdot a_n \to c \cdot a$}
\item{$a_n - b_n \to a - b$}
\item{$\ds\frac{a_n}{b_n} \to \frac{a}{b}$}
\end{enumerate}

\noindent \underline{Beweis 6):}\\
3) $\Rightarrow \displaystyle\frac{1}{b_n} \to \displaystyle\frac{1}{b}$\\
$\displaystyle\frac{a_n}{b_n} = a_n \cdot \displaystyle\frac{1}{b}$\\
2) $\Rightarrow a_n \cdot displaystyle\frac{1}{b_n} \to a \cdot \displaystyle\frac{1}{b} = \displaystyle\frac{a}{b} \phantom{XXX} q.e.d.$\\ \\
\underline{Beispiel}\\\\

\begin{tabular}{l|c|c|c|c|c|r}
$n$   & 0 & 1 & 2 & 3 & 10 & 100\\\hline
$a_n$ & 0 & 0 & $\frac{2}{9}$ & $\frac{6}{19}$ & $\frac{90}{201}$ & $\frac{9900}{20001}$ \\
\end{tabular}
\vspace{5mm}\\
Vermutung: $a_n \to \displaystyle\frac{1}{2}$ für $n \to \infty$\\
Rechenregel 6 anwenden:\\
\begin{itemize}
\item[1.]{Versuch:\\
$a_n = \frac{b_n}{c_n}$\\ 
\\
$b_n = n^2 -n; c_n = 2n^2 + 1$\\
$(b_n) und (c_n)$ sind divergend. Schlecht.}
\item[2.]{Versuch:\\
$\displaystyle\frac{n^2 - n}{2n^2 + 1} = \displaystyle\frac{n^2(1 - \displaystyle\frac{1}{n})}{n^2(2 + \displaystyle\frac{1}{n^2}}$ für $n \geq 1$ \\ \\
$= \displaystyle\frac{1 - \frac{1}{n}}{2 + \frac{1}{n^2}} = \displaystyle\frac{b_n}{c_n}$ mit $b_n := 1 - \displaystyle\frac{1}{n}; c_n = 2 + \displaystyle\frac{1}{n^2}$\\ \\
$\displaystyle\frac{1}{n} \to 0$ für $n \to \infty $
\\ \\ $\Rightarrow 1 - \displaystyle\frac{1}{n} \to 1 - 0 = 1$ für $n \to \infty$
\\ \\ $\Rightarrow 2 + \displaystyle\frac{1}{n^2} \to 2 + 0 = 2$ für $n \to \infty$}
\end{itemize}

$\Rightarrow a_n \to \frac{1}{2}$ für $n \to \infty$\\

\section{Satz}
Seien $a_n \to a$, $b_n \to b$ zwei konvergente Folgen reeler Zahlen.\\
wenn $a_n \leq b_n$ für unendlich viele $n \in \mathbb{N}$ dann ist $a \leq b$. \\
\underline{Beweis:}\\
Angenommen: $a > b$\\

Wähle $\epsilon := \displaystyle\frac{a - b}{2} > 0$\\
Es gibt $N \in \mathbb{N}$ so dass:
$
\left.
\begin{array}{ll}
\mid a_n - a \mid  < \epsilon \\
\mid b_n - b \mid  < \epsilon
\end{array} \right\rbrace$ für $n \geq N$\\
$\Rightarrow a_n > a - \epsilon$\\ \\
$= a - \frac{a - b}{2} = \displaystyle\frac{a + b}{2} = b + \displaystyle\frac{a - b}{2}\\
\\
= b + \epsilon > b_n \Rightarrow a_n > b_n$ für $n \geq \mathbb{N}$\\
Widerspruch zur Annahme.\\
$a_n \leq b_n$ für unendlich viele $n \in \mathbb{N} \hfill q.e.d.$

\section{Definition: Reihen}
Sei $(a_n)_{n \geq 0}$ eine Folge reeler Zahlen.\\
Bilde eine Folge:
\begin{align*}
s_0 &= a_0\\
s_1 &= a_0 + a_1\\
s_2 &= a_0 + a_1 + a_2\\
\vdots
s_n &= a_0 + a_1 + a_n = \sum\limits_{k = 0}^{n} a_k
\end{align*}
Die Folge $(s_n)_{n \geq 0}$ heißt Reihe mit den Gliedern $a_n$.\\
$s_n$ heißen die \underline{Partialsummen} der Reihe.\\
Bezeichnung:\\
$\sum\limits_{k = 0}^{\infty} a_k$ oder $a_0 + a_1 + a_2 + a_3 + ...$\\ \\
Wenn $s_n \to s \in \mathbb{R}$ für $n \to \infty$ dann schreiben wir:\\
$\sum\limits_{k = 0}^{\infty} a_k = s$\\
Summe der Reihe.\\
\\
\underline{Achtung:} Symbol $\sum\limits_{k = 0}^{\infty} a_k$ hat \underline{zwei} Bedeutungen:
\begin{enumerate}
\item{die Folge $(s_n)$} \\
oder 
\item{deren Grenzwert}
\end{enumerate}
\noindent \underline{Beispiele:}\\
\begin{enumerate}
\item{$\sum\limits_{k = 1}^{\infty} 1 = 1+1+1+...$\\
ist die Folge $(1, 2, 3, 4,...) = (n + 1)_{n \in \mathbb{N}_{0}}$}
\item{$\sum\limits_{k = 1}^{\infty} k = 0 + 1 + 2 + 3+ ...$ \\
ist die Folge $(1, 3, 6, 10,...) = (\displaystyle\frac{n(n - 1)}{2})_{n \in \mathbb{N}}$ }
\item{$\sum\limits_{k = 1}^{\infty} \displaystyle\frac{1}{k(k+1)} = \displaystyle\frac{1}{2} + \displaystyle\frac{1}{6} + \displaystyle\frac{1}{12} + ...$\\
ist die Folge $(\displaystyle\frac{1}{2}, \displaystyle\frac{2}{3}, \displaystyle\frac{3}{4})$}
\end{enumerate}
\vspace{5mm}
Vorüberlegung:\\
$\displaystyle\frac{1}{k(k+1)} = \displaystyle\frac{(k+1) - k}{k(k+1)} = \frac{1}{k} - \displaystyle\frac{1}{k + 1}$\\ \\
$s_n := \sum\limits_{k = 1}^{\infty} \displaystyle\frac{1}{k(k+1)}
= (\displaystyle\frac{1}{1} - \displaystyle\frac{1}{2}) + (\displaystyle\frac{1}{2} - \displaystyle\frac{1}{3}) + ... + (\displaystyle\frac{1}{n} - \displaystyle\frac{1}{n + 1})\\ \\
= 1 - \displaystyle\frac{1}{n + 1}\\ $ Teleskopsumme \\ \\
$\displaystyle\frac{1}{n + 1} \to 0$ für $n \to \infty$\\ \\
Summe der Reihe:\\ \\
$\sum\limits_{k = 1}^{\infty} \displaystyle\frac{1}{k(k+1)} = \lim_{n \to \infty}(1 - \displaystyle\frac{1}{n + 1}) = 1 \phantom{XXX} q.e.d.$\\ 
\\
\underline{Bemerkung:} Jede Folge kann man auch als Reihe Schreiben. (Differenzen bilden)\\
z.B.: die Folge der Primzahlen:\\
$(2, 3, 5, 7, 11, 13, 17, 19)$\\
ist die Reihe:\\
$(2 + 1 + 2+ 4+2+4+2+...)$\\
Goldbachsche Vermutung: in dieser Reihe kommt die Zahl 2 unendlich oft vor.\\
\section{Satz, Die geometrische Reihe}
Sei $x \in \mathbb{R}$\\
a) $ \sum\limits_{k = 0}^{\infty} x^k = 1 + x^1 + x^2 + x^3 + ... = \frac{1}{1-x} \text{ wenn } \mid x \mid < 1$\\
b) $ \sum\limits_{k = 0}^{\infty} x^k \text{ divergiert wenn } \mid x \mid \geq 1$\\
\begin{itemize}
\item[a] {wenn $|x| < 1$\\
dann folgt $\sum{k=0}{\infty} a_k = \displaystyle\lim_{n \to \infty}(\frac{1}{1 - x} - \frac{x}{1-x} \cdot x^n) = \frac{1}{1 - x}$}
\item[b]{wenn $|x| > 1$\\
dann $(x^n)$ divergent $\Rightarrow (\frac{x}{1-x} \cdot x^n)$ divergent\\
denn $\frac{x}{1-x} \neq 0$\\
$\Rightarrow (\frac{?}{?})$}
\end{itemize}
Beweis:\\
\begin{quote}
$x = 1 \phantom{xxx} \sum_{k = 0}^{\infty} x^k = (1 + 1 + 1 +...)\text{ divergiert, ok}\\
\text{Sei nun }x \neq\\
\text{Bekannt aus der Übung: } \displaystyle\sum_{k = 0}^{\infty} x^k = 1 + x + x^2 +x^3 ... +x^n = \displaystyle\frac{1 -x^{n+1}}{1 - x} = \displaystyle\frac{1}{1 - x} - \displaystyle\frac{x}{1 - x} \cdot x^n \\ $
\end{quote}
Potenzenwachstum\\
$x^n \to 0$ für $ n \to \infty$ \underline{wenn} $|x| < 1$\\
$(x^n)$ divergiert, wenn $(|x| \geq 1 \text{ und } x \neq 1)$\\

\section{Satz}
Wenn die Reihe $\displaystyle\sum\limits_{k=0}^{\infty} a_k $ kovergiert, dann ist $(a_n)_{n \in \mathbb{N}}$ eine Nullfolge.\\ \\
\underline{Beweis:} Gegeben sei $\epsilon > 0$\\
Sei $a = \displaystyle\sum\limits_{k = 0}^{\infty} a_k = $ $\displaystyle\lim_{n \to \infty}(s_n)$ mit $s_n = a_0 + ... + a_n$\\
Es gibt $ N \ in \mathbb{N}$ mit $|s_n - a| < \displaystyle\frac{\epsilon}{2}$ für $n \geq N$\\
$|a_n| = |s_n - s_{n-1}|$\\
\phantom{$|a_n| $} = $|s_n - a + a - s_{n-1}|$\\
\phantom{$|a_n| $} $\leq |s_n - a| + |a - s_{n-1}| < \displaystyle\frac{\epsilon}{2} + \displaystyle\frac{\epsilon}{2} = \epsilon$\\
für $n \geq N + 1$\\
$\Rightarrow a_n \to 0$ für $n \to \infty$\\

\section{Satz, die harmonische Reihe}
$\displaystyle\sum\limits_{k = 1}^{\infty} \frac{1}{k}= 1 + \frac{1}{2} + \frac{1}{3} + ...$ divergiert\\
\\
\underline{Beweisidee:}\\
\\
$\phantom{= }1 + \displaystyle\frac{1}{2} + \displaystyle\frac{1}{3} + \displaystyle\frac{1}{4} + \displaystyle\frac{1}{5} + \displaystyle\frac{1}{6} + \displaystyle\frac{1}{7} + \displaystyle\frac{1}{8} + \displaystyle\frac{1}{9} + ...$\\
$\phantom{\geq }1 + \displaystyle\frac{1}{2} + \displaystyle\frac{1}{4} + \displaystyle\frac{1}{4} + \displaystyle\frac{1}{8} + \displaystyle\frac{1}{8} + \displaystyle\frac{1}{8} + \displaystyle\frac{1}{8} + \displaystyle\frac{1}{16} + ...$\\
$\phantom{= }1 + \displaystyle\frac{1}{2} + \displaystyle\frac{2}{4} + \displaystyle\frac{4}{8} + \displaystyle\frac{8}{16} + ...$\\
$\phantom{= }1 + \displaystyle\frac{1}{2} + \displaystyle\frac{1}{2} + \displaystyle\frac{1}{2} + \displaystyle\frac{1}{2} + ... = \infty$\\
\end{document}














