% Kopfzeile beim Kapitelanfang:
\fancypagestyle{plain}{
%Kopfzeile links bzw. innen
\fancyhead[L]{\calligra\Large Vorlesung Nr. 1}
%Kopfzeile rechts bzw. außen
\fancyhead[R]{\calligra\Large 8.10.2012}
}
%Kopfzeile links bzw. innen
\fancyhead[L]{\calligra\Large Vorlesung Nr. 1}
%Kopfzeile rechts bzw. außen
\fancyhead[R]{\calligra\Large 8.10.2012}
% **************************************************
\chapter{Mengen}
\sS{Definition:}
\begin{enumerate}
\item Eine Menge ist eine Ansammlung verschiedener Objekte
\item Die Objekte in einer Menge heißen \underline{Elemente}\\
%
\notat{
a $\in$ M heißt a ist Element der Menge M\\
a ${\not\in}$ M heißt a ist kein Element der Menge M}
%
\item Sei M eine Menge. Eine Menge U heißt Teilmenge von M, von der jedes Element von U auch Element von M ist\\
%
\notat{
U $\subseteq$ M heißt U ist Teilmenge von M\\
U ${\not\subseteq}$ M heißt U ist keine Teilmenge von M}
\end{enumerate}
%
\sS{Beispiele}
\begin{enumerate}
\item {\einruck{Sei}{M die Menge aller Studierenden in L1\\W  die Menge aller weiblichen Studierenden in L1\\F die Menge aller Frauen}
Dann gilt: W $\subseteq$ M, W $\subseteq$ F, M ${\not\subseteq}$ F, F ${\not\subseteq}$ M}
\item {Die Menge der natürlichen Zahlen
$\N = \{1,2,3,4 …\}$
G sei die Menge der geraden natürlichen Zahlen
$G := \{n \in \N | $n ist gerade$\} = \{2m | m \in \N\} = \{2,4,6,8 …\}$
Es gilt G $\subseteq \N, \N \subseteq$ G}
\item {Die Menge der ganzen Zahlen
$\Z = \{0,1,-1,2,-2,3,-3, …\}$}
\item {Die Menge der rationalen Zahlen
$\Q{} = \{a/b | a, b \in \Z{}, b \neq 0\}$}
\item {Die Menge ohne Element heißt die leere Menge
Symbol: $\emptyset = \{\}$}
\end{enumerate}
%
\bem
\begin{enumerate}
\item Für jede Menge M gilt $\setminus \subseteq M$
\item $\N \subseteq \Z \subseteq \Q$
\end{enumerate}

\sS{Definition: Sei M eine Menge und U,V $\subseteq$ M Teilmengen}
\begin{enumerate}
\item Die Vereinbarung von U und V ist $U \cup V := \{x \in M \mid x \in U oder x \in V\}$
\item Der Durchschnitt von U und V ist $U \cap V := \{x \in M \mid x \in U oder x \in V\}$
U und V heißen disjunkt, wenn $U \cap V = \emptyset$
\item Die Differenzmenge von U und V ist $U \setminus V := \{x \in U \mid x \in V\}$
\item Das Komplement von U ist $U^C = M \setminus U = \{x \in M \mid x {\not\in} U\}$
%
\einruck{Bsp: }{Sei M = N \\
$\{1,3\} \cup \{3,5\} = \{1,3,5\}$\\
$\{1,3\} \cap \{3,5\} = \{3\}$\\
$\{1,3\} \cap \{2,4,7\} = \emptyset \leftarrow$ disjunkt\\
$\{1,2,3\} \setminus \{3,4,5\} = \{1,2\}$\\
$\{1,3,5\}^C = \{2,4,6,7,8,…\}$}
\end{enumerate}
%
\sS{Satz (de Morgan'sche Regeln)}
Sei M eine Menge, U,V $\subseteq$ M Teilmengen\\
Dann:
\begin{enumerate}
\item $(U \cap V)^C = U^C \cup V^C$
\item $(U \cup V)^C = U^C \cap V^C$
\end{enumerate}
%
\bew
\begin{enumerate}
\item{Sei $x \in M$\\Es gilt: $x \in (U \cap V)^C \Leftrightarrow x {\not\in} U \cap V \Leftrightarrow x {\not\in} U$ oder $x {\not\in} V\Leftrightarrow x \in U^C$ oder $x \in V^C \Leftrightarrow x\in U^C \cup V^C$}
\item{ Sei $x \in M$\\ Es gilt: $x \in (U \cup V)^C \Leftrightarrow x {\not\in} U \cup V \Leftrightarrow x {\not\in} U$ und $x {\not\in} V\Leftrightarrow x \in U^C$ und $x \in V^C \Leftrightarrow x\in U^C \cap V^C$}
\end{enumerate}
%
\section{Prinzip der Vollständigen Induktion}
Für jedes $n \in \N$ sei eine Aussage A(n) gegeben\\
Ziel: Beweisen, Dass A(n) für jedes $n \in \N$ mehr ist dafür reicht es zu zeigen
\begin{enumerate}
\item{Induktionsanfang (IA): A(1) ist wahr}
\item{Induktionsschrit (IS): Wenn für ein $n \in \N$ A(n) wahr ist, dann ist auch A(n+1) wahr}
\end{enumerate}
%
\sS{Satz:}
Für jede natürliche Zahl n gilt:
$$1+2+3+4+5+…+n=\frac{n(n+1)}{2}$$
Probe:\\
\begin{tabular}{r|c|c|c|c}
n & 1 & 2 & 3 & 4\\ \hline\hline
1+2+3…+n & 1 & 3 & 6 & 10\\ \hline
$\ds \frac{n(n+1)}{2}$ & 1 & 3 & 6 & 10\\
\end{tabular}
\sss{Beweis des Satzes mit Induktion}
Abkürzung: $S(n) := 1+2+3+…+n$
Aussage: A(n): $\ds S(n) = \frac{n(n+1)}{2}$
\begin{enumerate}
\item {Induktionsanfang (IA): n=1 $S(1) = 1 = \dfrac{1·2}{2}$\marginpar{ok!}}
\item {Induktionsschritt (IS): $n → n+1$\\
Annahme: A(n) gilt: $\ds S(n) = \frac{n(n+1)}{2}$\\
Zu zeigen: A(n+1) gilt: $S(n+1)=\frac{(n+1)·(n+2)}{2}$\\
$\ds S(n+1)=S(n)+n+1=\frac{n(n+1)}{2}+\frac{2(n+1)}{2}=\frac{(n+2)(n+1)}{2}$\\
Das beendet den Beweis} \qed
\end{enumerate}
Zur Vereinfachung der Notation:\\
Seien $a_1,a_2,a_3,…,a_n$ Zahlen $n \in \N$\\
Setze: $\sum_{k=1}^n a_k := a_1+a_2+a_3+…+a_n$\\
\begin{tabbing}
Allgemeiner: \=Sei $l,m \in \N$, $l \le m \le n$\\
\>$\sum_{k=l}^m a_k = a_l+a_{l+1}+…+a_m$\\
\end{tabbing}
Aussage des Satzes:
\[ \sum_{k=1}^n k = \frac{n(n+1)}{2}\]\footnote{\ul{Kombinatorik} (mathematisches Zählen)}

\sS{Definition}
Seien A, B Mengen. Das kartesische Produkt von A und B ist definiert als $A × B := \{(a,b)|a\in A, b \in B\}$ Die Elemente von $A × B$ heißen geordnete Paare\\
Bsp.: $\{1,7\}\times \{2,3\}=\{(1,2),(1,3),(7,2),(7,3)\}$\\
Allgemeiner: Gegeben seien Mengen
$A_1,…,A_k$ mit $k \in \N$. Das kartesische Produkt von $A_1,…,A_k$ ist $A_1\times …\times A_k = \{(a_1,…,a_k)|a\in A, $für $i=1,…,k\}$\\
Elemente von $A_1 × … × A_k$ heißen k-Tupel\\
Falls $A_1=A_2=…=A_k=A$, schreibe $\underbrace{A\times…\times A}_{k-mal}=A^k$

\section{Definition}
Eine Menge A ist endlich, wenn A nur endlich viele Elemente hat. Dann bezeichnet
$\#A = \{|A|\}$ die Anzahl der Elemente von A und somit dessen Kardinalit\"at
oder M\"achtigkeit. Wenn A nicht endlich ist, so schreibe: $\# A= \infty$\\
Bsp.: $\#\emptyset = 0, \#\N=\infty, \# \{1,3,5\} = 3$

\section{Bemerkung}
\begin{enumerate}
\item Sei A endliche Menge. $U,V\subseteq A$ disjunkte Teilmengen\\
Dann $\#(U\cup V)=\# U + \# V$ 
\item Seien $A_1,…,A_k$ endliche Mengen $k \in \N$\\
Dann: $\#(A_1 \times … \times A_k)=(\#A_1)(\#A_2)…(\#A_k)$
\end{enumerate}

\section{Definition}
\begin{enumerate}
\item Für $n\in \N$ setze $n!=1· 2· 3· … · n=\prod_{k=i}^n k$
Setze $0!=1$
\item Für $k,n\in \Z$ mit $0\le k \le n$ sei $\binom{n}{k}:= \frac{n!}{k!·(n-1)!}$ \Rarr{} Binomialkoeffizient\\
\begin{tabular}{r|c|c|c|c|c|c|c}
n & 0 & 1 & 2 & 3 & 4 & 5 & 6\\ \hline
n! & 1 & 1 & 2 & 6 & 24 & 120 & 720
\end{tabular}\\
\bsp
$\binom{5}{2} := \frac{5!}{2!· 3!} = \frac{5· 4 ·\cancel{3 · 2 · 1}}{2· 1· \cancel{3· 2}· 1 } = \frac{20}{2}=10$\\
Bemerkung: $\binom{n}{0}= 1 = \binom{n}{n}$
\end{enumerate}