%Kopfzeile links bzw. innen
\fancyhead[L]{\calligra {\Large Vorlesung Nr. 12}}
%Kopfzeile rechts bzw. außen
\fancyhead[R]{\calligra \Large{19.11.2012}}
% **************************************************
\wdh
Sei $D \subseteq \R$. eine Funktion $f: D \to \R$ ist stetig in $x \in D$ wenn gilt:\\
$\begin{array}{ll}
\text{Für jedes $\epsilon > 0$ gibt es ein $\delta > 0$ so dass gilt:}\\
\text{wenn $y \in D$ mit $|x - y| < \delta$ dann $|f(x) - f(y)| < \epsilon$}
\end{array}$
\bsp
$exp:R→\R$ ist stetig (d.h. stetig an jedem $x\eR$)\\
$[·]:R→\R$ ist stetig an $x \equ x\not\in\Z$\\
%Diagramm, Gausklammern
\sS{Satz Folgenstetigkeit}
Eine Funktion $f: D \to \R$ ist stetig in $x \in D$ $\equ$ Für jede Folge $(x_n)$ mit $x_n \in D$ für alle $n$ und $x_n \to x$ gilt auch $f(x_n) \to f(x)$.\\
(d.h. f erhält Konvergenz)\\
\bew
"\Rarr" Angenommen $f$ ist stetig in $x$, $x_n→x$ mit $x_n\in D$. Zeige $f(x_n)→f(x)$\\
Gegeben $\e>0$. Stetigkeit \Rarr{} es gibt $\delta>0$ mit:\\
Wenn $y\in D$ mit $|x-y|<\delta$ dann $|f(x)-f(y)|<\e$\\
Wähle $N\eN$ so dass gilt:\\
Für $n\geq N$ ist $|x-x_n|<\delta \Rarr |f(x)-f(x_n)|<\e$\\
Also gilt $f(x_n)→f(x)$\\
"\Larr" Angenommen $f$ ist nicht stetig.\\
Zeige: Es gibt eine Folge $(x_n)$ mit $x_n\in D$ und $x_n→x$ aber nicht $f(x_n)→f(x)$ für nicht stetig in $x$ \Rarr{} es gibt ein $\e>0$ so dass für jedes $\delta>0$ ein $y\in D$ existiert mit $|x-y|<\delta$ und $|f(x)-f(y)|\geq \e$\\
Wähle für $\delta=\frac{1}{n}$ ein $x_n\in D$ mit $|x_n-x|<\delta,\ |f(x_n)-f(x)|\geq\e$\\
Dann gilt $x_n→x$ aber \underline{nicht} $f(x_1)→f(x)\qed$
%
\wdh
\sS{Satz 6.5}
Wenn $f, g: D \to \R$ stetig in $x \in D$ dann auch $f + g,\ f \cdot g$ und $\frac{1}{g}$ (falls $g(y) \neq 0$ für alle $y \in D$) \\
Und $a \cdot f$ für $a \in \R$
\sS{Korollar} 
Polynomfunktionen und rationale Funktionen sind stetig.
\bew
\begin{enumerate}
\item{$id_{\R}:\R→\R,\ id_{\R}(x)=x$ ist stetig}
\item{6.5 und Induktion \Rarr{} für $\R→\R,\ f(x)=x^n$ stetig für jedes $n\quad (x^n=x·x^{n-1}$}
\item{6.5 \Rarr{} Jede Funktion $f(x)=a_n·x^{n}+a_{n-1}·x^{n-1}+…+a_0,\ f:\R→\R$ ist stetig}
\item{6.5 \Rarr{} wenn $f,g:\R→\R$ Polynomfunktionen, dann $\frac{f}{g}:D→\R$ stetig, wobei $D=\{x\eR|g(x)\neq 0\}$ (denn $(\frac{f}{g}=f·\frac{1}{g})$\qed}
\end{enumerate}
\sS{Satz Stetigkeit der Komposition}
Sei $f: C \to \R$, $g: D \to \R$ Funktionen mit $f(C) \subseteq D$. Wenn $f$ stetig in $x \in D$ und g stetig in $f(x)$ dann ist:\\
$g \circ f: C \to \R$ stetig in x.\\
\bew mit Folgenstetigkeit:
Sei $x_n \to x$ mit $x_n \in C$\\
$f$ stetig in $x \Rarr \ f(x_n) \to f(x)$\\
$g$ stetig in $f(x) \Rarr g(f(x_n)) \to g(f(x))$\\
d.h. $(g \circ f)(x_n) \to (g \circ f)(x)$\\
also ist $g \circ f: C \to \R$ stetig in $x$ \qed
\sS{Definition (Konvergenz bei Funktionen)}
Sei $D\subseteq\R$ und $f:D→\R$ Funktion\\
\begin{enumerate}
\item{Ein $a\eR\cup\{-∞,∞\}$ heißt Berührpunkt von $D$ wenn es eine Folge $(x_n)$ mit $x_n\in D$ und $x_n→a$ gibt
\bem
Jedes $a\in D$ ist Berührpunkt von $D$ (wähle konstante Folge $x_n=a$)}
\item{Angenommen, $a$ ist ein Berührpunkt von $D$\\
Schreibe $\ds\lim_{x→a}f(x)=y$ wenn gilt:\\
Für jede Folge $(x_n)$ mit $x_n→a$ und $x_n\in D$ gibt $f(x_n)→y$}
\item{Angenommen, $a\neq ∞$ ist eine Berührpunkt von $D\cap(a,∞)$ SKIZZE \\% SKIZZE
Schreibe $\ds\lim_{x\searrow a}f(x)=y$ wenn gilt:\\
Für jede Folge $(x_n)$ mit $x_n→a$ und $x_n\in D$ und $x_n>a$ gilt $f(x_n)→y$}
\item{Angenommen, $a\neq -∞$ ist eine Berührpunkt von $D\cap(-∞,a)$ Schreibe $\ds\lim_{x\nearrow a}f(x)=y$ wenn für jede Folge $(x_n)$ mit $x_n→a$ und $x_n\in D$ und $x_n<a$ gilt $f(x_n)→y$
\bsp
$D= \R \setminus \{0\}\ f:D→\R,\ f(x)=\frac{|x|}{x}$ SKIZZE\\
\underline{Bemerke} f(x)=$\left\{\begin{array}{lcl}1 & \text{wenn} & x>0\\-1 & \text{wenn} & x<0\end{array}\right.$\\
$\ds\lim_{x\nearrow a}f(x)=-1,\ \lim_{x\searrow a}f(x)=1,\ \lim_{x→0}f(x)existiert nicht$\\
\ul{Vorher} $a = 0$ ist Berührpunkt vpn $D$ und $D \cap (0, \infty)$ und $D \cap (- \infty , 0)$, denn $\frac{1}{n} \to 0$ und $-\frac{1}{n} \to 0$}
\end{enumerate}
Sei $g: \R \to \R$, $g(x) = x^3$\\
$\infty , - \infty$ sind Berührpunkte von $D = \R$\\
$\ds\lim_{x \to \infty} g(x) = \infty$\\
$\ds\lim_{x \to -\infty} g(x) = -\infty$\\ 
\bem{Umformulierung von Satz 6.4}
Eine Funktion $f: D \to \R$ ist stetig in $a \in D \\ $
$\equ \quad \ds\lim_{x \to a} f(x) = f(a)$
\uS{Sätze über stetige Funktionen}
\Def
Eine Funktion $f:D→\Re$ heißt nach oben \ul{beschränkt} wenn die Menge $f(D)$ nach oben beschränkt ist, d.h. es gibt $C\eR$ mit $f(x)\leq C$ für alle $x\in D$\\
$f$ heißt nach unten beschränkt, wenn $f(D)$ nach unten beschränkt\\
$f$ heißt \ul{beschränkt}, wenn $f)$ nach oben und nach unten beschränkt\\
\sS{Definition}
Sei $M \in \R$ eine nicht-leere Teilmenge wenn $M$ nach oben unbeschränkt, schreibe $sub(M) = \infty$ (Sprich: uneigentliches Supremum)\\
\Satz
Sei $a,b\eR$ mit $a\leq b$ und $f:[a,b]→\R$ eine \ul{stetige} Funktion, dann ist $f$ beschränkt und nimmt ihr Maximum und Minimum an, d.h. es gibt $x_{min},x_{max}\in[a,b]$ mit: $f(x_{min}\leq f(x) \leq (x_{max})$ für jedes $x\eR$
\bsp
\begin{enumerate}
\item{$f: (0, 1) \to \R, \quad f(x) = \frac{1}{x}$ % Graph
$\ds\lim_{x \to 0} f(x) = \infty$\\
Somit $f$ nicht nach oben beschränkt}
\item{$g(0,1) \to \R, \ g(x) = x$\\
$g$ beschränkt. $g((0,1)) = (0,1)$\\
$sup \{g(x) | x \in (0, 1) \} = 1$\\
Aber $g(x) < 1$ für alle $x \in (0, 1)$. Also nimmt g nicht ihr Maximum an.}
\end{enumerate}
\sss{Beweis von 6.11}
Sei $y := sup\{ f(x)| x \in D\} \eR \cup \{ ∞ \}$\\
Wähle eine Folge $(x_n)$ mit $x_n\in D$ und $f(x_n)→y$ für $n→∞$\\
Bolzano-Weierstraß \Rarr{} Es gibt eine konvergente Teilfolge $(x_{n_k})_{k\geq 0}$ von $(x_{n})_{k\geq 0}$ \marginpar{Die Folge $(x_n)$ ist beschränkt $D=[a,b]$}\\
Sei $x_{n_k}→x$ für $k→∞$ $a\leq x_n\leq b$ für alle $n$ \Rarr $a\leq x\leq b,\ x\in D=[a,b]$\\
%\einruck{Und dann gilt:}{$f()x_{n_k})→y$ für $k→∞$ (Teilfolge einer konvergenten Folge)\\$f()x_{n_k})→f(x)$ für $k→∞$ weil f stetig}
% EINRÜCKUNG 
Und dann gilt: $f({x_n}_k) \to y$ für $k \to \infty$ (Teilfolge einer Konvergenten Folge)\\
\phantom{Und dann gilt: }$f({x_n}_k)) \to y$ für $k \to \infty$ weil $f$ stetig\\
Also $y = f(x)$\\
Insbesondere $y \neq \infty$ also $f$ beschränkt\\
Für alle $x' \in D$ gilt $f(x') \leq sup \{f(D)\} = y = f(x)$\\
Setze $x_{max} := x$. Dann $f(x') \leq f(x_{max})$ für alle $x' \in D$\\
Anfang findet man $x_{min}$ \qed
\sS{Satz (Zwischenwertsatz)}
Sei $a\leq b$ und $f:[a,b]→\R$ stetig.\\
Wenn $y\eR$ zwischen $f(a)$ und $f(b)$  liegt, d.h. $f(a)\leq y \leq f(b)$ oder $f(a)\geq y \geq f(b)$\\
Dann gibt es ein $x\in[a,b]$ mit $f(x)=y$ GRAPH\\