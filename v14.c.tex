% Kopfzeile beim Kapitelanfang:
\fancypagestyle{plain}{
%Kopfzeile links bzw. innen
\fancyhead[L]{\calligra\Large Vorlesung Nr. 1}
%Kopfzeile rechts bzw. außen
\fancyhead[R]{\calligra\Large 8.10.2012}
}
%Kopfzeile links bzw. innen
\fancyhead[L]{\calligra\Large Vorlesung Nr. 1}
%Kopfzeile rechts bzw. außen
\fancyhead[R]{\calligra\Large 8.10.2012}
% **************************************************
\wdh{Logarithmus und allgemeine Potenzen}
Die Funktion $exp: \R \to \R_{\geq 0}$ ist stetig, streng monoton wachsend, bijektiv.\\*
Die Umkehrfunktion heißt \ul{Logarithmus},\\*
$$log = exp^{-1}: \R_{\geq 0} \to \R$$
explizit definiert durch $log(x) = y \equ x = exp(y)$\\*
\Rarr_{Satz 6.15} log ist stetig, streng monoton wachsend, bijektiv.
% GRAPH exp(x) | GRAPH log(x)
%
% Stefan
%
\sS{Lemma}
Es gibt $a^{\frac{n}{m}} =  exp(\frac{n}{m} \cdot log(a)$\\*
\bew
	\begin{enumerate}
	\item{Sei $n \geq 0$:\\*
	$$exp(n \cdot log(a)) = exp(\overbrace{log(a) + log(a) + ... + log(a)}{n}) = exp(log(a)) \cdot ... \cdot exp(log(a))$$}
	\item{Sei $n < 0$\\*
	 $$exp(n \cdot log(a)) = exp(\underbrace{-n}{-n > 0} \cdot log (a)) = (a^{-1})^{-1} = a^n$$}
	 \item{Rechne: $exp(\frac{n}{m} log(a))^m = exp(m \cdot \frac{n}{m} \cdot log(a)) = exp(n \log(a)) = a^n$}
	\end{enumerate}	 
	$\sqrt[m]{} \Rarr exp(\frac{n}{m} log(a)) = \sqrt[m]{a^n} = a^\frac{n}{m}$
	
% Stefan

\sS{Definition Logarithmusbasis}
Sei $a > 1 \qquad x \in \R$ 
$$log_a (x) = \frac{log(x)}{log(a)}$$
\bem $a \neq 0 \Rarr log(a) \neq 0$\\*
Dann $a^{log_a (x)} = exp(log_a (x) \cdot log(a)) = exp(\frac{log(x)}{log(a)} \cdot log(a)) = exp(log(x)) = x$
\uS{Gleichmäßige Stetigkeit}
\wdh
$f: D \to \R$ stetig an $x \in D$ wenn gilt:\\*
Für jedes $\e > 0$ gilt $\delta > 0$ mit wenn $y \in D$ mit $|x - y| < \delta$ dann $|f(x) - f(y)| < \e$\\*
Hier hängt $\delta$ im allgemeinen von $\e$ und $x$ ab!
% Stefan
% Graph für Stefan:
\sS{Satz}
Seien $a \leq b$ reelle Zahlen\\*
Jede stetige Funktion $f: [a, b] \to \R$ ist gleichmäßig stetig.\\*
\bew
	Angenommen, $f$ ist nicht gleichmäßig stetig:
	Es gibt ein $\e > 0$ sodass für jedes $\delta > 0$ zwei Zahlen $x, y \in D$ existieren, mit $|x - y| < \delta$ und $|f(x) - f(y)| \geq \e$\\*
	Wähle zu $\delta = \frac{1}{n}$ Zahlen $x_n, y_n \in D$ mit $|x_n - y_n| \frac{1}{n}, \ |f(x) - f(y)| \geq \e$ (*)\\*
	Bolzano-Weierstraß \Rarr Es gibt eine Teilfolge $({x_n}_k)_{k\geq 0}$, die konvergiert. Sei $x = \ds\lim_{k\to \infty}({x_n}_k)_{k\geq 0} \in  [a, b] = D$\\*
	Dann $\ds\lim_{k \to \infty} ({y_n}_k)_{k\geq 0} = \ds\lim_{k \to \infty} (({y_n}_k)_{k\geq 0} - ({x_n}_k)_{k\geq 0}) = 0 + x = x$\\*
	${x_n}_k)_{k\geq 0} \to x, \ {y_n}_k)_{k\geq 0} \to x \ f: k \to \infty $\\*
	$f$ stetig \Rarr $f({x_n}_k)_{k} \to f(x), f({y_n}_k)_{k} \to f(x) \ f \ k \to \infty $\\*
	\Rarr $(f({x_n}_k)_{k}) - f({y_n}_k)_{k})  \to f(x) - f(x) = 0$
	Widerspruch zu (*) Also ist $f$ gleichmäßig stetig. \qed
% Stefan
\item{Existenz vom Inversen:\\*
Sei $z = (x, y) \in \C, \ x, y \in \R, \ z \neq 0$ \\*
Zeige es gibt ein $z^{-1} \in \C$ mit $z \cdot z^{-1} = (1,\ 0)$\\*
$z \neq 0 \Rarr x \neq 0$ oder $y \neq 0 \equ x^2 + y^2 > 0$\\
$$w:= \left(\frac{x}{x^2 + y^2}, \frac{-y}{x^2 + y^2}\right)$$
Rechne $z \codt w = (x, y) \cdot \left(\frac{x}{x^2 + y^2}, \frac{-y}{x^2 + y^2}\right)$
$=\left(\frac{x^2}{x^2 + y^2} - \frac{-y^2}{x^2 + y^2}, \frac{-yx}{x^2 + y^2} + \frac{yx}{x^2 + y^2}\right) = (1, 0)$ }
Also $w = z^{-1}$ \qed \\*
Weitere Axiome ähnlich.
% Stefan
\sss{Definition}
Sei $z = x + iy \in \C$\\*
Realteil: Re(z) := x \\*
Imagnärteil: Im(z) := y\\*
Komplex konjugierte Zahl $\={z} = x - iy$\\*
Komplex konjugation = Spiegelung an der x-Achse.\\*
\ul{Definition:} Der Betrag von $z = x + iy \in \C$ ist $|z| = \sqrt{x^2 + y^2}$ Abstand von $0 = (0, 0)$ zu $z$
% Stefan