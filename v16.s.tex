% Christophter

\uS{Die komplexe Exponentialfunktion}

\sS{Satz}
Für $z\eC$ konvergiert die "Exponentialreihe"
$$\sum_{n=0}^{∞}\frac{z^n}{n!}=1+\frac{z}{1}+\frac{z^2}{2}+\frac{z^3}{6}+…$$absolut (Somit konvergiert sie)
\bew
$$\sum_{n=0}^{∞}\left|\frac{z^n}{n!}\right|=\sum_{n=0}^{∞}\frac{|z^n|}{n!}=exp(|z|)$$
Bekannt: $exp(|z|)$ konvergiert\qed

\sS{Definition komplexe Exponentialfunktion}
Die komplexe Exponentialfunktion ist die Abbildung $\ds exp:\C →\C\quad exp(z)=\sum_{n=0}^{∞}\frac{z^n}{n!}$

\uS{Eigenschaften}
\sS{Satz}
Seien $z,w\eC$
\begin{enumerate}
\item{$exp(0)=1$ (klar)}
\item{$exp(z+w)=exp(z)+exp(w)$}
\item{$exp(z)≠0,\ exp(z)^{-1}=exp(-z)$}
\item{$exp(\ol{z})=\ol{exp{z}}$ (Komplexe Konjugation)}
\item{Für $x\eR$ ist $|exp(ix)|=1$}
\end{enumerate}
\bew
\begin{enumerate}
\setcounter{enumi}{1}
\item{Wie bei der reellen Exponentialfunktion:\\*
Die Reihe $exp(z+w)$ ist das Cauchy-Produkt der Reihen $exp(z)$ und $exp(w)$, dann folgt $(z)$ aus 7.15}
\item{$exp(z)·exp(-z)\underset{2)}{=}exp(z-z)=exp(0)\underset{1)}{=}1$}
\item{Sei $\ds s_n=\sum_{k=0}^{n}\frac{z^k}{k!}$ somit nach Definition $exp(z)=\lim\limits_{n→∞}s_n$\\*[4pt]
Sei $\ds s_n'=\sum_{k=0}^{n}\frac{\ol{z}^k}{k!}$ somit $exp(\ol{z})=\lim\limits_{n→∞}s_n'$\\*[4pt]
Es gilt $$s_n'=\ol{\sum_{k=0}^{n}\frac{z^k}{k!}}=\sum_{k=0}^{n}\ol{(\frac{z^k}{k!})}=\sum_{k=0}^{n}(\frac{\ol{z}^k}{k!})=s_n'$$
Somit $\ol{exp(z)}=\lim\limits_{\nif}(\ol{s_n})=\lim\limits_{\nif}s_n'=exp(\ol{z})$}
\item{$$|exp(ix)|^2=exp(ix)·\ol{exp(ix)}\underset{4)}{=}exp(ix)·exp(\ol{ix})=exp(ix)·exp(-ix)\underset{2)}{=}exp(ix-ix)=exp(0)\underset{1)}{=}1\ \overset{\sqrt{}}{\Rarr}\ |exp(ix)|=1$$
\end{enumerate}

%uS Trigonometrische funktion

%Definition

\sS{Satz}
\begin{enumerate}
\item{$cos(0)=1,\ sin(0)=0$}
\item{$cos(-x)=cos(x),\ sin(-x)=-sin(x)$}
\item{$sin(x)^2+cos(x)^2=1$}
\item{\desc{Additionstheoreme:}{$sin(x+y)=sin(x)·cos(y)+cos(x)·sin(y)$
eingerückt $cos(x+y)=cos(x)·cos(y)-sin(x)·sin(y)$}}
\end{enumerate}
\bew
\begin{enumerate}
\item{$exp(0i)=1=1+0i\ \Rarr\ cos(0)=1, sin(0)=0$}
\item{$exp(-ix)=exp(\ol{ix})=\ol{exp(ix)}=cos(x)-i·sin(x)$\\*
$exp(-ix)=cos(-x)+i·sin{-x}\ \Rarr\ cos(x)=cos(-x),\ sin(-x)=-sin(x)$}
\item{$sin(x)^2+cos(x)^2\overset{Def.}{=}|exp(ix)|^2=1$}
\item{$$exp(i(x+y)) = exp(i \cdot x) \cdot exp(i \cdot y)$$
$$\Rarr cos(x+y) + i \cdot sin(x+y) = (cos(x) + i \cdot sin(x))(cos(y) + i \cdot sin(y))$$
$$=cos(x) \cdot cos(y) - sin(x) \cdot sin(y) + i \cdot (sin(x) \cdot cos(y) + cos(x) \cdot sin(y))$$
Vergleich der Realteile / Imaginärteile \Rarr{} Behauptung 4\qed}
\end{enumerate}
%bemerkung

% % %\sS{def}

\sS{Satz}
Eine Abbildung $f:\C→\C$ ist stetig in $z\eC$ \equ\ Für jede Folge komplexer Zahlen $(z_n)_{n\geq 0}$ mit $z_n→z$ für \nif\ gilt auch $f(z_1)→f(z)$ für \nif
\bew
Wörtlich wie bei reeller Funktion (Satz 6.4)\qed

\sS{Satz}
Die komplexe Exponentialfunktion $exp:\C→\C$ ist stetig
\bew
Verwende Folgenstetigkeit
\begin{enumerate}
\item{Stetigkeit in $z=0\ exp (0)=1$\\*
Sie $z\eC$ (nahe 0)
$$\left|exp(z)-1\right|=\left|1+z+\frac{z^2}{2}+\frac{z^3}{6}+…-1\right|=\left|\sum_{n=1}^{∞}\frac{z^n}{n!}\right|\underset{unendliche Dreiecksungleichung 7.13}{\leq}exp(|z|)-1$$
Wenn $z_n→0$ in \C\\*
dann $|z_n|→0$ in \R\\*
dann $exp(|z_n|) \to exp(0) = 1$ (weil $exp: \R \to \R$ steig)\\*
d.h. $exp(|z_n|) -1 \to 0$\\*
$\Rarr|exp(z_n) - 1| \to 0$\\
Somit $exp$ stetig in $z = 0$ }
\item{Sei $z\eC$ beliebig, $z_n→z$
$$exp(z_n)-exp(z)=exp(z_n-z+z)-exp(z)=exp(z_n-z)-exp(z)-1·exp(z)=(exp(z_n-z)-1)·exp(z)$$
Es gilt: $z_n→z\ \equ\ z_n-z→0$\\*
\alg{&\underset{1)}{\Rarr}\ exp(z_n-z)→1\\
&\equ\ exp(z_n-z)-1→0\\
&\Rarr\ (exp(z_n-z)-1)·exp(z)→0·exp(z)=0
&\underset{(*))}{\Rarr}\ (exp(z_n)-exp(z)→0}
d.h. $exp(z_n)→exp(z)$\qed}
\end{enumerate}

%Christopher Satz

