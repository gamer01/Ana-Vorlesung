% Kopfzeile beim Kapitelanfang:
\fancypagestyle{plain}{
%Kopfzeile links bzw. innen
\fancyhead[L]{\calligra\Large Vorlesung Nr. 2}
%Kopfzeile rechts bzw. außen
\fancyhead[R]{\calligra\Large 11.10.2012}
}
%Kopfzeile links bzw. innen
\fancyhead[L]{\calligra\Large Vorlesung Nr. 2}
%Kopfzeile rechts bzw. außen
\fancyhead[R]{\calligra\Large 11.10.2012}
% **************************************************
%
\wdh
Sei $M$ Menge.\\*
Wenn $M$ endlich: $\#M=Anzahl$ $Elemente\in M$\\*
Wenn $M$ unendlich: $\#M=\infty$\\*
Für $n\in \N:=\{1,2,3,\ldots\}$\\*
$$n!=1 · 2 · 3 · 4 · … · n \qquad 0!=1$$
Binomialkoeffizient: Für $0\leq k\leq n$\\*
$$\bino{n}{k} \qquad\qquad\qquad \bino{n}{0}=\bino{n}{n}=1$$

\sS{Lemma}
Für $0<k< n$ gilt:
$$\binom{n}{k} = \binom{n -1}{k-1} + \binom{n-1}{k}$$\\*
%
\bew 
$$\binom{n-1}{k-1}+ \binom{n-1}{k}=\frac{(n-1)!}{(k-1)!·(n-k)!} +\frac{(n-1)!}{(k-1)!·(n-1-k)!} = \frac{k(n-1)!+(n-k)\cdot(n-1)!}{k! (n-k)!}=\frac{n(n-1)!}{k!(n-k)!}$$

\sS{Geometrische Anordnung (Pascalsches Dreieck)}
\parbox{0.4\textwidth}{\centering
$\binom{0}{0}$\\*
$\binom{1}{0} \binom{1}{1}$\\*
$\binom{2}{0} \binom{2}{1} \binom{2}{2}$\\*
$\binom{3}{0} \binom{3}{1} \binom{3}{2} \binom{3}{3}$}\hfill
\parbox{0.4\textwidth}{\centering
1\\*
1 1\\*
1 2 1\\*
1 3 3 1}\\*[5mm]
Folge $\binom{n}{k}\in \N$ für alle $0\leq k\leq n$
%

\sS{Satz: Anzahl von Teilmengen}
Sei $A$ endliche Menge. $\#A=n$\\*[4pt]
Sei $k\in\Z$ mit $0\leq k\leq n$\\*[4pt]
$P_k(A):=\{U\subseteq A\mid \#U=k\}$ (Menge aller $k$-elementigen Teilmengen von $A$)\\*[4pt]
Dann gilt $\#P_k(A)=\binom{n}{k}$\\*
\bsp
$A=\{1,2,3,4\}$ $n=4$ $k=2$\\*[4pt]
2-elementige Teilmengen von $A$:
$\{1,2\}, \{1,3\}, \{1,4\}, \{2,3\}, \{2,4\}, \{3,4\} \to 6\qquad \binom{4}{2}=6$ \ok
%
\bew
Vorüberlegung: Sei $k=0 \vee k=n$\\*
$P_0(A)=1=\binom{n}{0}$ $\#P_n(A)=1=\binom{n}{n}$\ok\\*
Jetzt: Induktionsbeweis nach n\\*[4pt]
\ind{$n=0$ Dann $k=0$}{Sei $\#A=n+1 \Rarr 0 \leq k \leq (n+1)$
Falls $k = 0\vee k = n + 1$\\*
Sei also: $o < k < n + 1$\\*
Wähle $a\in A$\\*
Sei $B=A\bs\{a\}$\\*
Dann $A=B\cup\{a\}, \#B=n$\\*
Man kann die Wahl einer $k$-elementigen Teilmenge von $A$ so strukturieren:
\begin{enumerate}
\item{Entscheiden, ob $a\in U \vee a\notin U$}
\item{\begin{enumerate}
\item{Wenn $a\notin U$: Wähle $k$ Elemente aus $B$}
\item{Wenn $a\in U$: Wähle $k-1$ Elemente aus $B$}
\end{enumerate}}
\end{enumerate}
$$\Rarr\ \#P_k(A)=\#P_k(B)+\#P_{k-1} (B) \stackrel{IV}{=} \binom{n}{k} + \binom{e}{ -1} \stackrel{1.11}{=} \binom{n+1}{k}$$
}

\sS{Satz (Binomische Formel)}
Seien $a,b$ Zahlen, $n\in\N$\\*
Dann $(a+b)^n=a^n+\binom{n}{1} a^{n-1} b+\binom{n}{2}a^{n-2}b^2+…+b^n$
%
\bsp
$(a+b)^4=a^4+4a^3b+6a^2b^2+4ab^3+b^4$\\*
$(a+b)^2=a^2+2ab+b^2$
%
\bew
Schreibe $(a+b)^n=\underbrace{(a+b)(a+b)(a+b)(a+b)…(a+b)}_{n-Faktoren}$
%
\sss{Ausmultiplizieren}
Halte Terme der Form $a^{n-k}b^k$ mit $0\leq k\leq n$\\*
Häufigkeit von $a^{n-k}b^k$ = Anzahl der Möglichkeiten aus n-Faktoren $k$ mal $b$ zu wählen.\\*
Das ist $\binom{n}{k}$ (Satz 1.13)
%
\sss{Folgerung}
Setze $a=b=1\qquad a^{n-k}b^k=1$\\*
$(a+b)^n=2^n=\binom{n}{0}+\binom{n}{1}+\binom{n}{2}+…+\binom{n}{n}$\\*
%
\bsp
$1+4+6+4+1=16=2^4$

\sS{Definition: Anordnung}
Sei $A$ endliche Menge\\*
Eine Anordnung von $A$ ist ein $n$-Tupel\\*
$(a_1,a_2,a_3,a_4,…,a_n)$ mit $a\in A$ für alle $i$ und $a_i\neq a_j$ wenn $i\neq j$\\*
%
\bsp
Anordnung von $\{1,2,3\}=(1,2,3)(1,3,2)(2,1,3)(2,3,1)(3,1,2)(3,2,1)→6$

\sS{Satz: Anzahl von Anordnungen}
Sei $A$ endliche Menge, $\#A=n\geq 1$\\*
Dann ist die Anzahl der Anordnungen von $A$ gleich $n!$
\bew
Induktion nach $n$\\*
\ind{n=1}{Sei $\#A=n+1$\\*
Wahl einer Anordnung von $A$ kann man so unterteilen:\\*
\begin{enumerate}
\item{Wähle 1 Element $a_1\in A$ ($n+1$ Möglichkeiten)}
\item{Wähle Anordnungen von $A\bs\{a_1\}$\\*
$\#(A\bs\{a_1\})=n$ \Rarr $n!$ Möglichkeiten bei 2\\*
Insgesamt $(n+1)·n!=(n+1)!$}
\end{enumerate}}
%
\bem
(Zusammenhang zwischen Anordnung und Teilmengen)\\*
Sei $A$ endliche Menge, $\#A=n,\ 0\leq k\leq n$\\*
Sei $(a_1,…,a_n)$ Anordnung von $A$\\*
$\leadsto$ Teilmenge $U:=\{a_1,…,a_n\}$\\*
Dann $U\subseteq A,\ \#U=k\qquad U\in P_k(A)$\\*
Jedes $U\in P_k(A)$ entsteht so, aber mehrfach:\\*
\[\underset{\overset{\uparrow}{Anordnungen\ von\ U}}{k!}·\underset{\overset{\uparrow}{Anordnungen\ von\ A\backslash U}}{(n-k)!}-mal\]
$\#$ Anordnungen von $A=n!=\#P_k(A)·k!(n-k)!\Rarr\#P_k(A)=\frac{n!}{k!·(n-k)!)}=\binom{n}{k}$\\*
%
\chapter{Die reellen Zahlen}
Was sind die reellen Zahlen?\\*
Präzise Konstruktion ist umfangreich, daher Axiomatischer Zugang\\*
Beschreibung der reellen Zahlen durch ihre Eigenschaften (Axiome):\\*
\begin{enumerate}
\item{Grundrechenarten → Körper}
\item{Ungleichungen → angeordneter Körper}
\item{Lückenlosigkeit → Vollständigkeit}
\end{enumerate}
%
\uS{Körper}
%2.1
\Def
Ein Körper ist eine Menge $K$ mit 2 Rechenoperationen:\\*
Addition (+) und Multiplikation (·), so dass folgende 9 Eigenschaften erfüllt sind:\\*[8pt]
\ul{Addition}\\*[-15pt]
\begin{enumerate}
\item{$(a+b)+c=a+(b+c)$ für alle $a,b,c\in K$ (Assotiativgesetz)}
\item{$a+b=b+a$ für alle $a,b\in K$ (Kommutativgesetz)}
\item{Es gibt ein $0\in K$ so dass $0+a=a$}
\item{Für jedes $a\in K$ gibt es ein $b\in K$ mit $a+b=0$}
\bem
$0\in K$ ist eindeutig
\bew
Wenn $0'\in K$ mit $0'+a=a$, dann $0=0'+0=0+0'=0'$\qed
\bem
Das $b$ in 4. ist auch eindeutig.\\*
\notat{$b=-a$ (Negatives von $a$)}
\bew
Angenommen $b'+a=0$\\*
$b=b+0=b+(a+b')=(b+a)+b'=0+b'=b'$\qed
\end{enumerate}
\ul{Multiplikation}\\*[-15pt]
\begin{enumerate}
\setcounter{enumi}{4}
\item{$a(b·c)=(a·b)c\qquad ∀a,b,c\in K$}
\item{$a·b=b·a\qquad ∀a,b\in K$}
\item{Es gibt ein $1\in K$ mit $1\neq 0$, so dass $1·a=a\qquad ∀a\in K$}
\item{Für alle $a\in K,\ a\neq 0$, gibt es ein $b\in K$ mit $a·b=1$}
\bem
$1\in K$ ist eindeutig, $b$ in 8. ist eindeutig\\*
Beziehung $b=a^{-1}$
\bew
Wie eben\qed
\item{$a(a+c)=a·b+a·c\qquad ∀a,b,c\in K$ (Distributivgesetz)}
\end{enumerate}
Weitere Bezeichnungen:\\*
$a-b:=a+(-b),\ \frac{a}{b}=a·b^{-1}$, wenn $b≠0$
\bem
Die üblichen Rechenregeln folgen aus diesen Axiomen 1.-9.
\bsp
$$-(-a)=a,\ a(b-c)=a·b-a·c,\ a(-b)=-(a·b)$$

\sS{Beispiele bekannter Körper}
\Q\ ist ein Körper\\*
\Z\ ist kein Körper (8. nicht erfüllt)

\sS{Beispiel für einen Körper}
$\mathbb{F}_z=\{0,1\}$\\*
\ul{Definitionen von + und · :}\\*[8pt]
\parbox{.2\textwidth}{\begin{tabular}{c|cc}
+&0&1\\*[2pt]\hline
0&0&1\\*[2pt]
1&1&0
\end{tabular}}
\parbox{.2\textwidth}{\begin{tabular}{c|cc}
·&0&1\\*[2pt]\hline
0&0&0\\*[2pt]
1&0&1
\end{tabular}}
$1+1=0$\\*[4pt]
\fbox{\ul{Übung:} Prüfe alle Körperaxiome}\\*
\bem
Sei $K$ \ul{endlicher} Körper\\*
Dann gilt $\#K=p^r$ wobei $p$ Primzahl, $r\in\N$\\*
Für jede solche Zahl $q=p^r$ gibt es genau einen Körper
