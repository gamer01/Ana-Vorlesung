% Kopfzeile beim Kapitelanfang:
\fancypagestyle{plain}{
%Kopfzeile links bzw. innen
\fancyhead[L]{\calligra\Large Vorlesung Nr. 21}
%Kopfzeile rechts bzw. außen
\fancyhead[R]{\calligra\Large 07.01.2013}
}
%Kopfzeile links bzw. innen
\fancyhead[L]{\calligra\Large Vorlesung Nr. 21}
%Kopfzeile rechts bzw. außen
\fancyhead[R]{\calligra\Large 07.01.2013}
% **************************************************
%
\wdh
Eine Funktion $f: I \to \R$ heißt differenzierbar in $x_0 \in I$ wenn der Limes
$$f(x) = \lim_{x\to x_0} \frac{f(x) - f(x0)}{x-x_0}$$ existiert.\\*
Ableitungsregeln: Produkt, Kettenregel, Umkehrfunktion $\leadsto$ Kann "alle" Ableitungen ausrechnen
8.11 $f: I = (a,b) \to \R}$ diffbar,\\*
$f$ hat ein lokales extremum in $x_0 \in (a, b)$ \Larr{} $f'(x) = 0$\\*
\begin{tikzpicture}[domain=-0.3:2,prefix=plots/, smooth]
\draw[very thin,color=gray] (-0.3,-0.25) grid (1.99,1.99);
\draw[->] (0.3,0) -- (2.5,0) node[right] {$x$};
\draw[->] (0,0.3) -- (0,2) node[above] {$y$};
% Graphen beschriftung
\draw[color=blue] plot[id=21.1_cosp1] function{cos(x) + 1} node[below, midway] {};
\end{tikzpicture}
8.12 (Satz von Rolle)\\*
Sei $f: [a, b] \to \R$  diffbar\\*
\begin{tikzpicture}[domain=-0.3:2,prefix=plots/, smooth]
\draw[very thin,color=gray] (-0.3,-0.25) grid (1.99,1.99);
\draw[->] (0.3,0) -- (2.5,0) node[right] {$x$};
\draw[->] (0,0.3) -- (0,2) node[above] {$y$};
% Graphen beschriftung
\draw[color=blue] plot[id=21.2_cos_tiefer] function{cos(x) * 1.5 + 1};
\end{tikzpicture}
\\*
\begin{tikzpicture}[domain=-0.3:2,prefix=plots/, smooth]
\draw[very thin,color=gray] (-0.3,-0.25) grid (1.99,1.99);
\draw[->] (0.3,0) -- (2.5,0) node[right] {$x$};
\draw[->] (0,0.3) -- (0,2) node[above] {$y$};
% Graphen beschriftung
% Graph ausdenken, oder als Linie zeichnen.
\end{tikzpicture}
$f(a) = f(b)$ dann existiert $x_0 \in (a, b)$ mit $f'(x) = 0$
% Stefan Mittelwertsatz
\sS{Folge}
Sei $f: T \to \R$ diffbar, $f'(x) = 0$ für alle $x$ dann ist $f$ konstant.
\bew
Sei $x_1 < x_2$ in $I$\\*
Es gilt $x_0$ mit $x_1 < x_0 < x_2$ mit $f(x_1) - f(x_2) = f(x_0) \cdot f(x_1 - x_2) = 0$\\*
\Rarr{} $f(x_1) = f(x_2) \Rarr f$ konstant. \qed{}\\*
Mittelwertsatz \qed

% Stefan Monotonie

Angenommen $f$ monoton wachsend\\*
Sei $x_0 \in (a, b)$\\*
Zeige: $f'(x) \geq 0$\\*
% TOFIX Richtigen Arrow suchen
$f'(x_0) = \lim_{x \searrow x_0}\frac{f(x) - f(x_0)}{x - x_0}$\\*
$x > x_0 \Rarr x - x_0 > 0,\ f(x) - f(x_0) \geq 0$

% Stefan BSP1 
% graph für Stefan:
\begin{tikzpicture}[domain=-0.3:3.5,prefix=plots/, smooth]
\draw[very thin,color=gray] (-0.3,-1.25) grid (3.49,1.25);
\draw[->] (0.3,0) -- (3.5,0) node[right] {$x$};
\draw[->] (0,-1.5) -- (0,1.5) node[above] {$y$};
% Graphenbeschriftung bei PI
\draw[color=blue] plot[id=21.4_cos] function{cos(x)};
\end{tikzpicture}
\\*
\begin{tikzpicture}[domain=-0.3:3.5,prefix=plots/, smooth]
\draw[very thin,color=gray] (-0.3,-1.25) grid (3.49,0.25);
\draw[->] (0.3,0) -- (3.5,0) node[right] {$x$};
\draw[->] (0,-1.5) -- (0,5) node[above] {$y$};
% Graphenbeschriftung bei PI
\draw[color=blue] plot[id=21.5_sin] function{sin(x)};
\end{tikzpicture}

% Beispiel 2 Graphen:

\begin{tikzpicture}[domain=-1.5:1.5,prefix=plots/, smooth]
\draw[very thin,color=gray] (-1.49,-3.99) grid (1.49,1.24);
\draw[->] (-1.6,0) -- (1.6,0) node[right] {$x$};
\draw[->] (0,-3) -- (0,3) node[above] {$y$};
\draw[color=blue] plot[id=21.4_x3] function{pow(x, 3)} node[below, midway] {$f(x) = x^3$};
\draw[color=red] plot[id=21.4_x3] function{pow(3x, 2)} node[below, midway] {$f'(x) = 3x^2$};
\end{tikzpicture}

d.h.
\begin{itemize}
\item[a]{$x_0 < x < x_0 + \e$ \Rarr{} $f'(x) - f'(x_0) < 0$ \Rarr $f'(x) < 0$}
\item[b]{$x_0 - \e < x < x_0$ \Rarr{} $f'(x) - f'(x_0) > 0$ \Rarr $f'(x) > 0$}
\end{itemize}
8.15 \Rarr{} $f$ streng monoton fallend auf $[x_0, x_0 + \e]$ wegen a)
			$f$ streng monoton steigend auf $[x_0 - \e, x_0]$ wegen b)
\bsp $f(x) x^3 - 3x$ $f: \R \to \R$\\*
$f'(x) = 3x^2 - 3$\\*
$f''(x) = 6x$\\*
Nullstelle (NST) von $f'$: $f'(x) = 0 \equ 3x^2 - 3 = 0 \equ x^2 = 1 \equ x \in \{1, -1\}$\\*
Anwendung von $f''$ an NST von $f'$: $f(1) = 6$

%Stefan

Analog für $x \to b$ (ohne Beweis)
\bsp
\begin{enumerate}
\item{$\lim_{x\to 0} \frac{sin(x)}{x} = ?$ \\*
$\lim_{x\to 0} x = 0$, $\lim_{x\to 0} \frac{sin(x)} = 0$\\*
$x' = 1, sin' = cos$\\*
$\leadsto$ Berechne\\*
$\lim_{x\to0} \frac{cos(x)}{1} = cos(0) = 1$ existiert.\\*
l'Hospital \Rarr{} $lim_{x \to 0} \frac{sin(x)}{x} = 1$}
\item{$\lim_{x \to \infty} \frac{log(x)}{x}$\\*
$\lim_{x \to \infty} x = \infty$\\*
$log(x)' = \frac{1}{x}$, $x' = 1$\\*
$\leadsto$ Berechne $$\lim_{x \to \infty} \frac{\frac{1}{x}}{1} = \lim_{x \to \infty} \frac{1}{x} = 0$
8.17 \Rarr $\lim_{x \to \infty} \frac{log(x)}{x} = 0$}
\item{Rationale Funktion\\*
$\lim_{x \to \infty} \frac{x^3 + x}{2x^2 + 5}$\\*
$f(x) = x^3 + x$, $g(x) = 2x^2 + 5$\\*
$\lim_{x \to \infty} g(x) = \infty$\\*
$\leadsto \ f'(x) = 2x + 1$, $g(x) = 4x$\\*
$\leadsto$ Rechne\\*
$\lim_{x \to \infty} \frac{2x + 1}{4x}$\\*
$= \lim_{x \to \infty} (\frac{1}{2} + \frac{1}{4x}) = \frac{1}{2}$
$\Rarr \lim_{x \to \infty} \frac{x^2 + x}{2x^2 + 5} = \frac{1}{2}$
}
\end{enumerate}