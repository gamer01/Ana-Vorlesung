%Kopfzeile links bzw. innen
\fancyhead[L]{\calligra {\Large Vorlesung Nr. 10}}
%Kopfzeile rechts bzw. außen
\fancyhead[R]{\calligra \Large{12.11.2012}}
% **************************************************
%
\setcounter{chapter}{4}
\chapter{Abbildungen und Funktionen}
\sS{Definition}
Seien $A, B$ Mengen. Eine Abbildung von $A$ nach $B$ ist eine Vorschrift, die jedem Element von $A$ ein Element von $B$ zuordnet.\\
\notat{$f: A \ B,\  a \mapsto f(a) \  a\in A$}
%
A heißt Definitionsbereich von $f$
B heißt Wertebereich von $f$
%
\bsp
\begin{enumerate}
\item {Alle Personen in $L1 \mapsto \N$\\
$P \mapsto$ Geburtsjahr von $P$}
%
\item{$f:\R → \R, \ f(x)=x^2$\\
$g:\R→\R_{\geq 0}=\{x\in\R|x\geq 0\}, \ g(x)=x^2$\\
$h: \R_{\geq 0} \to \R_{\geq 0} h(x) = x^2$}
\bem 
\item{
$f,g,h$ sind verschieden\\
Sei $M$ Menge. Die Identität von $M$ ist die Abbildung $id_{M}:M→M, id_M(x)=x$}
\end{enumerate}
%
\sS{Definition}
%
Eine Abbildung $f: A \to B$ heißt:
%
\begin{enumerate}
\item{\underline{injektiv} wenn gilt: Für alle $a, a' \in A$ mit $f(a) = f(a')$ ist auch $a = a'$}
\item{\underline{surjektiv} wenn gilt: Für jede $b\in B$ gibt es ein $a\in A$ mit $f(a)=b$}
\item{\underline{bijektiv} wenn $f$ injektiv und surjektiv ist}
\end{enumerate}
%
% Tafel 2.2
% Bild zeichnen
%
% Tafel 3.1
% Beispielbild
%
\bem
$f$ ist $\left\{
\begin{array}{lll}
\text{injektiv}\\
\text{surjektiv}\\
\text{bijektiv}
\end{array}
 \right\}$ genau dann wenn für jedes $b \in B$ $\left\{\begin{array}{llll} \text{höchstens}\\ \text{mindestens}\\ \text{genau} \end{array} \right\}$ ein $a \in A$ mit $f(a) = b$\\
%
\bsp
$f,g,h$ wie oben\\
\begin{description}
\item[f]{nicht surjektiv: es gibt kein $a\in\R$ mit $f(a)=-1$\\
nicht injektiv: $f(-2)=4=f(2), 2\neq -2.$}
%
\item[g]{ist surjektiv}, denn für jedes $b \in \R_{geq 0}$ gilt $f( \sqrt{b} ) = b$ also gibt es $b \in \R_a$
\item[g]{ist nicht injektiv (wie $f$)}
%
\item[h] {surjektiv wie g. $(\sqrt{b} \geq 0)$}
\item[h] {injektiv, denn: Wenn $a, a' \geq 0$ und $a^2 = (a')^2$ dann $a = a'$ also $h$ bijektiv.}
\end{description}
\sS{Definition}
Sei $f:A→B$, $g:B→C$ Abbildungen\\
Die Komposition von $f$ und $g$ ist die Abbildung\\
$g \circ  f: A→C$, $(g \circ f)(a):=g(f(a))$\\
$\circ$ = "nach"
%
%
\sS{Satz} 
Eine Abbildung $f: A \to B$ ist bijektiv \equ \ es gibt eine Abbildung $g: B \to A$ mit $f \circ g = id_B$\\
(d.h. f(g(b)) = b für alle $b \in B$\\
      g(f(a)) = a für alle $a \in A$)\\
%
\sss{Definition} %ohne nummer
Wenn $f:A→B$ bijektiv ist, heißt die eindeutige Abbildung $g:B→A$ wie oben die Umkehrabbildung (inverse Abbildung) von $f$
Bezeichnung: $g=f^{-1}$.
%
\bew von 5.4  % /circ war früher /irct
Angenommen, $g: B \to A$ gegeben mit $f \circ g = id_B, g \circ f = id_A$\\
$f$ surjektiv: Sei $b \in B$. $b = f(g(b)) = f(a)$ mit $a = g(b)$ ok.\\
$f$ injektiv: Sei $a, a'$ mit $f(a) = f(a')$ zeige $a = a'$ \\
$a = g(f(a)) = g(f(a')) = a' $\ok \\ \\
%
Angenommen, $f$ ist bijektiv, zeige g existiert.\\
Gegeben sei $b \in B$ f bijektiv $\Rightarrow $ es gibt genau ein $a \in A $ mit $f(a) = b$ 
Setze $g(b):=a$ Das definiert Abbildung $g:B→A$\\
Zeige $g \circ f=id; f \circ g= id$\\
$(f\circ g)(b)=f(g(b))=f(a)=b$ wobei a wie eben\\ \\
%
Zeige: $(g \circ f) (a) $ für alle $a \in A$\\
f injektiv: Reicht $f(g(f)a))) = f(a)$\\
Das gilt weil $f \circ g = id_B$ \ok \\ \\
Eindeutigkeit von $g$:\\
Angenommen,\\
$g^* : B \to A$ erfüllt $g^* \circ f = id_A$,
$f \circ g^* = id_B$ \\
%
Dann gilt: $g=g\circ id_B=g\circ f\circ g^*=id_A\circ g^* = g^*$ \qed
\bsp
werden zeigen:\\
\begin{itemize}
\item{$f: \R_{\geq 0} \to \R_{\geq 0}, f(x) = x^k$ bijektiv ($k \geq 1$)\\
Die Umkehrabbildung $f^{-1}$ heißt k-te Wurzelabbildung $f^{-1}(x) = \sqrt[k]{x}$ }
%
\item{exp: $\R→\R_{>0}$ $exp(x) = \sum_{k=0}^{\infty}$ (Absolutkonvergente Reihe) ist bijektiv. Die Umkehrabbildung heißt Logarithmus. bew.
$log = exp()^{-1} \R_{\geq } \to \R_a$ }
\end{itemize}
%
\section*{Bild und Urbild}
\sS{Definition}
Sei $f:A→B$ Abbildung\\
\begin{enumerate}
\item{Für eine Teilmenge $X \subset A$ ist \\
$f(x) := \{f(x) | x \in X\} \subseteq B$ \\
das Bild von X unter f}
\item{Für eine Teilmenge $Y \subseteq B$ ist $f^{-1}:=\{a\in A|f(a)\in Y\}\subseteq A$ das Urbild von $Y$ unter $f$}
\end{enumerate}
\underline{\underline{Vorsicht}} nicht Urbild und Umkehrabbildung verwechseln.\\
\bsp
$f:\R→\R, f(x)=x^2$\\
$f(\{1, 2, -2\}) = \{1, 4\}$\\
$f^{-1}(\{1,-2,4\})=\{1,-1,2,-2\}=f^{-1}(\{1,4\})$\\
$f^{-1}(\{9\})=\{3,-3\,f^{-1}(\{-5\})=\emptyset$\\
%
\section*{Funktionen}
\sS{Definition}
Sei $D\subseteq\R$ Teilmenge. Eine reele Funktion auf D ist eine Abbildung $f:D→\R$\\
%
Der \underline{Graf} von f ist die Menge $\Gamma_f = \/(x, f(x) | x \in D \}$) \\
$ \Gamma_f \subseteq D \times \R$ 
%
\bem Oft ist D ein Intervall
%
\sS{Definition Intervalle}
seien $a, b \in \R$ \\
$[a, b] = \{x \in \R| a \leq x \leq b\}$ (abgeschlossen)\\
$(a, b] = \{x \in \R| a < x \leq b\}$(halboffen)\\
$[a, b) = \{x \in \R| a \leq x < b\}$(halboffen)\\ %mit klammer zu pberer zeile\\
$(a, b) = \{x \in \R| a < x < b\}$ (offen)\\
%
Uneigentliche Intervalle: \\
$[a, \infty) = \{x \in \R | a \leq x\} = \R_{\geq a}$\\
$(a, \infty) = \{x \in \R | a < x\} = \R_{> a}$\\
$(- \infty, a] = \{x \in \R | x \leq a\} = \R_{\leq a}$\\
$(- \infty, a) = \{x \in \R | x < a\} = \R_{< a}$\\
$(- \infty, \infty) = \R$\\
%
\Bsp{Beispiel für Funktionen}
\begin{enumerate}
\item{$f:[0,2]→\R, f(x)=x^2., \Gamma_f \leq [0,2] x\R$
% Graphenpackage suchen Tafel 10.2
%
}
\item{Betragsfunktionen: $|\ |: \R→\R, x\mapsto|x|$
%noch mehr graphen aaaahahhahahah
}
An dieser Stelle fehlen noch Graphen.
\item{$g:\R\bs\{0\}→\R, g(x)=\dfrac{1}{x}$
%graph
Hier auch.
}
\item{$exp:\R→\R$.}
\item{[.] : $\R \to \R$ Gausklammer\\
$[x] := max\{n \in \Z | n \leq x \}$
\bsp
[5] = 5 [5 , 78] = 5}
\item{Sei $h:\R→\R$ definiert durch $h(x)=\begin{cases}0\ wenn\ x\in\Q\\ 1\ wenn\ x\notin\Q\end{cases}$}
\item{$h(\sqrt{2} = 1, h (\frac{3}{7}) = 0$ }
\end{enumerate}
%
\sS{Definition (Rechnen mit Funktionen)}
Sei $D\subseteq \R , \ f,g: D→\R$ Funktionen auf D.\\
Definiere
\begin{itemize}
\item{$f+g: D \to R$ durch $(f + g)(x) := f(x) + g(x)$}
\item{$(f \cdot  g) (x) = f(x) g(x)$}
\item{Für $a\in\R$ setze $a·f: D→\R, (a·f)(x):=a·f(x)$}
\item{Angenommen, $f(x) \neq 0$ für alle $x \in D$ \\
$$\frac{1}{f}: D \to R, \frac{1}{f}(x) := \frac{1}{f(x)} = f(x)^{-1}$$
\underline{\underline{Vorsicht}} nicht $\frac{1}{f}$ mit Umkehrbild oder Urbild verwechseln}
\end{itemize}
%
\sS{Definition}
\begin{itemize}
\item{Eine \underline{Polinomfunktion} ist eine Funktion der For\\
$f:\R→\R,\ f(x)=a_nX^n+a_{n-1}X^{n-1}+…+a_0=\ds\sum_{k=0}^na_kX^k $\\
wobei $a_0,…,a_n\in\R$ fest
%
\item{Seien $f, g : \R \to \R $ Polymonfunktionen
Sei $D = {x \in \R | g(x) \geq 0}\leadsto \dfrac{f}{g} : D \to , Rx \mapsto \frac{f(x)}{g(x)}$
Solche Funktionen heißen rationale Funktionen.}
\bsp
$f:\R\bs\{0,1\}→\R, \ f(x)=\dfrac{x^7+5x^2}{x(x-1)}$}
\end{itemize}
\sS{Definition}
Seien $f: D \to \R, g: D \to \R$ Funktionen sodass $f(C) \subseteq D$
Eine Komposition von f und g ist 
%
$g \circ f : C \to \R, (g \circ f \ (x) = g(f(x)))$