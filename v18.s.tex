\sS{Satz}
%WDH
Die Funktion $cos:[0,\pi]?[0,1]$ ist stetig, streng monoton fallend, bijektiv
\bew
$cos:[0,2]?\R$ streng monoton fallend \Rarr\ $cos:[0,\frac{\pi}{2}]?\R$ streng monoton fallend\\*
$cos(\pi-x)=-cos(x)$ \Rarr\ $cos:[\frac{\pi}{2},\pi]?\R$ streng monoton fallend SKIZZE\\*
\Rarr\ $cos:[0,\pi]?\R$ streng monoton fallend\\*
$cos(0)=1,\ cos[\pi]=-1$ \Rarr\ $cos:[0,\pi]?[-1,1]$ surjektiv, somit bijektiv\qed\\*
\ssss{Folge} es gibt eine Umkehrfunktion:
Arcuscosinus: $arccos=cos^{-1}:[-1,1]?[0,\pi]$SKIZZE ARCCOS\\*
\alg{cos(0)&=1\quad arccos(1)=0\\*
cos(\frac{\pi}{2})&=0\quad arccos(0)=\frac{\pi}{2}\\*
cos(\pi)&=-1\quad arccos(1)=\pi}
\bem
Die Wahl des Intervalls $[0,\pi]$ ist willk�hrlich. Auch bijektiv:\\*
$cos:[\pi,2\pi]?[-1,1],\ cos:[-\pi,0]?[-1,1]$
\bem
Sei $x\eR$. Es gilt $cos(x)=1\ \equ\ x=2\pi�n$ mit $n\eZ$\\*
(Anschaulich: klar, Beweis: �bung)
%Christopher uS Polarzerlegung
%Satz
Aber:
$$|\phi-\phi'| < 2 \pi \Rarr\ \phi-\phi'<0$$
Das zeigt \ul{Eindeutigkeit} der Polarzerlegung\qed
% christpher
\sS{Satz (Einheitswurzel)}
Sei \nN\\*
Die Gleichung $z^n=1,\ z\eC$\\*
Hat genau $n$ L�sungen, n�mlich $z=e^{2\pii\frac{k}{n}}$ mit $k\eZ,\ 0\leq k<n$
\bew
Wenn $z^n=1$, dann $|z|^n=|1|=1\ \Rarr\ |z|=1$\\*
Sei $z=e^{i\phi}$ mit $0\leq \phi <2 \pi\ z^n=1\ \equ\ (e^{i\phi})^n=1\ \equ\ e^{in\phi}=1$
\alg{&\equ n�\phi= k�2\pi \text{ mit } k\eZ\\*
&\equ \phi= 2\pi k/n \text{ mit } k\eZ\\*
&\equ z= e^{2\pi i\frac{k}{n}} \text{ mit } k\eZ}
\sss{Bedeutung:}
$0\leq \phi<2\pi\ \equ\ 0\leq 2\pi k/n < 2\pi\ \equ\ 0\leq k<n\qed$\\*
SKIZZE $\case{e^0=1\\
e^{2\pi i/6}=e^{\pi i/3}=\frac{1}{2}+\frac{\sqrt{3}}{2}i\\
e^{2\pi i 2/6}=e^{\pi i 2/3}=-\frac{1}{2}+\frac{\sqrt{3}}{2}i\\
e^{2\pi i 3/6}=e^{\pi i}=-1\\
e^0=1\\
e^0=1\\
e^0=1\\}$
\sss{Verhalten von $exp(z)$ nahe Null}
Erinnerung: $exp:\C?\C$ stetig, dass hei�t wenn $z?0$ dann $exp(z)?exp(0) =1$\\*
Betrachte $\dfrac{exp (z)-1}{z}$ f�r $z\eC,\ z\neq 0$
%chris
\bem\enum{\item{Beschr�nkung auf $z=x\eR\ \leadsto\ \lim_\overset{x?0}{x\neq 0} \frac{e^x-1}{x}=1 (x\eR)$}
\item{Beschr�nkung auf $$z=ix,\ x\eR\ \leadsto\ \lim_{x?0}\frac{e^ix-1}{ix}=1$$ $$\lim_{x?0}\frac{cos(x)+i�sin(x)-1}{ix}=1$$ $$\lim_{x?0}\left(\frac{sin(x)}{x}-i�\frac{cos(x)-1}{x}\right)=1+0i\ (*)$$
$$(*) \underset{Realteil}{\Rarr} \lim_{x?0}\frac{sin(x)}{x}=1$$
$$(*) \underset{Imagin�rteil}{\Rarr} \lim_{x?0}\frac{cos(x)-1}{x}=0$$}}
\uS{Geometrische Bedeutung von $\pi$?}
\sss{Frage} Was ist die L�nge des Kreisbogens von $1$ bis $e^{ix}$? SKIZZE
\enum{\item{Wie ist diese L�nge definiert?}\item{Berechnen}}
%zerteilung

\sS{Satz}
Es gilt $\lim_\nif  l_n=|x|$ Interpretation der L�nge des Bogens ist $|x|$\\*
\bew
$$|e^{(k+1)ix/n}-e^{kix/n}|=|e^{kix/n}|�|e^{ix/n}-1|=|e^{ix/n}-1|$$
%letzter Tafelrest