% Kopfzeile beim Kapitelanfang:
\fancypagestyle{plain}{
%Kopfzeile links bzw. innen
\fancyhead[L]{\calligra\Large Vorlesung Nr. 6}
%Kopfzeile rechts bzw. außen
\fancyhead[R]{\calligra\Large 25.10.2012}
}
%Kopfzeile links bzw. innen
\fancyhead[L]{\calligra\Large Vorlesung Nr. 6}
%Kopfzeile rechts bzw. außen
\fancyhead[R]{\calligra\Large 25.10.2012}
% *****************************************
%
%\setcounter{chapter}{3}
%\setcounter{section}{9}
%
\wdh
Eine Folge reeller Zahlen $(a_n)$ konvergiert uneigentlich gegen ∞ wenn gilt:\\*
Für jedes $C\eR$ gibt es ein \nN{} mit $a_n > C$ für jedes \nN\\*[4pt]
$(a_n)$ konvergiert uneigentlich gegen $- ∞$ wenn $(-a_n)$ gegen $∞$ konvergiert.
\notat{
$a_n \to ∞ \qquad \text{ für } n \to ∞$\\*
$a_n \to - ∞ \qquad \text{ für } n \to ∞$
}
%
\bsp
$a_n = n^2 \to ∞$\\*
$a_n = -n^2 \to -∞$\\*
$a_n = (-1)^n · n^2$\\*
$(0, -1, 4, -9)$ konvergiert weder gegen $∞$ noch gegen $ - ∞$
%
\sss{Rechenregeln}
Angenommen $(a_n), (b_n)$ sind konvergente Folgen.
\begin{enumerate}
\item{$(a_n + b_n) \to a + b$}
\item{$(a_n · b_n) \to ab$}
\item{$\ds\frac{1}{b_n} \to \frac{1}{b}$}
\item{$c · a_n \to c · a$}
\item{$a_n - b_n \to a - b$}
\item{$\ds\frac{a_n}{b_n} \to \frac{a}{b}$}
\end{enumerate}
%
\bew
\begin{enumerate}
\setcounter{enumi}{5}
\item{3) $\Rightarrow\ds\frac{1}{b_n}→\frac{1}{b}$\\*
$\ds\frac{a_n}{b_n} = a_n ·\frac{1}{b}$\\*
2) $\ds\Rightarrow a_n · \frac{1}{b_n} \to a ·\frac{1}{b} = \frac{a}{b}$\qed}
\end{enumerate}
\bsp
\begin{tabular}{l|c|c|c|c|c|r}
$n$   & 0 & 1 & 2 & 3 & 10 & 100\\*\hline
$a_n$ & 0 & 0 & $\frac{2}{9}$ & $\frac{6}{19}$ & $\frac{90}{201}$ & $\frac{9900}{20001}$
\end{tabular}
\quad Vermutung: $a_n \to \ds\frac{1}{2}$ für $n \to ∞$\\*
Rechenregel 6 anwenden:
\begin{itemize}
\item[1.]{Versuch:\\*
$a_n = \frac{b_n}{c_n}$\\*[4pt]
$b_n = n^2 -n; c_n = 2n^2 + 1$\\*
$(b_n)$ und $(c_n)$ sind divergend. Schlecht.}
\item[2.]{Versuch:
$$\frac{n^2 - n}{2n^2 + 1} = \frac{n^2(1 - \frac{1}{n})}{n^2(2 + \frac{1}{n^2}}\quad\text{ für }n \geq 1= \frac{1-\frac{1}{n}}{2 +\frac{1}{n^2}}=\frac{b_n}{c_n}\quad\text{ mit }b_n:=1-\frac{1}{n},\ c_n = 2 + \frac{1}{n^2}$$
$$\frac{1}{n} \to 0\qquad\text{ für }n \to ∞ $$
$$\Rightarrow 1 - \frac{1}{n} \to 1 - 0 = 1\qquad\text{ für }n \to ∞$$
$$\Rightarrow 2 +\frac{1}{n^2}\to 2 + 0 = 2\qquad\text{ für }n \to ∞$$}
\end{itemize}
$$\Rightarrow a_n \to \frac{1}{2}\qquad\text{ für }n \to ∞$$

\sS{Satz}
Seien $a_n \to a$, $b_n \to b$ zwei konvergente Folgen reeller Zahlen.\\*
wenn $a_n \leq b_n$ für unendlich viele $n \in \N{}$ dann ist $a \leq b$.
\bew
Angenommen: $a > b$\\*[4pt]
Wähle $\e := \ds\frac{a - b}{2} > 0$\\*
Es gibt $N \in \N{}$ so dass:
$\left.
\begin{array}{ll}
| a_n - a | < \e \\*
| b_n - b | < \e
\end{array} \right\rbrace$ für $n \geq N$\ $\Rightarrow a_n > a - \e$
$$= a - \frac{a - b}{2} = \ds\frac{a + b}{2} = b + \ds\frac{a - b}{2}= b + \e > b_n \Rightarrow a_n > b_n\qquad\text{ für }n \geq \N{}$$
Widerspruch zur Annahme.\\*
$a_n \leq b_n$ für unendlich viele $n \in \N$\qed

\sS{Definition Reihen}
Sei $(a_n)_{n \geq 0}$ eine Folge reeller Zahlen.\\*
Bilde eine Folge:
\begin{align*}
s_0 &= a_0\\*
s_1 &= a_0 + a_1\\*
s_2 &= a_0 + a_1 + a_2\\*
&\vdots\\*
s_n &= a_0 + a_1 + a_n = \sum\limits_{k = 0}^{n} a_k
\end{align*}
Die Folge $(s_n)_{n \geq 0}$ heißt Reihe mit den Gliedern $a_n$.\\*
$s_n$ heißen die \ul{Partialsummen} der Reihe.\\*
Bezeichnung:
$$\sum\limits_{k = 0}^{∞} a_k\text{ oder }a_0 + a_1 + a_2 + a_3 + …$$
Wenn $s_n \to s \in \R{}$ für $n \to ∞$ dann schreiben wir:
$$\sum\limits_{k = 0}^{∞} a_k = s$$
Summe der Reihe.\\*[4pt]
\ul{Achtung} Symbol $\ds\sum_{k = 0}^{∞} a_k$ hat \ul{zwei} Bedeutungen:
\begin{enumerate}
\item{die Folge $(s_n)$}\\*[8pt]
oder 
\item{deren Grenzwert}
\end{enumerate}
\bsp
\begin{enumerate}
\item{$$\sum\limits_{k = 1}^{∞} 1 = 1+1+1+…\text{ ist die Folge }(1, 2, 3, 4,…) = (n + 1)_{n \in \N{}_{0}}$$}
\item{$$\sum\limits_{k = 1}^{∞} k = 0 + 1 + 2 + 3+ …\text{ ist die Folge }(1, 3, 6, 10,…) = \left(\frac{n(n - 1)}{2}\right)_{n \in \N{}}$$}
\item{$$\sum\limits_{k = 1}^{∞} \frac{1}{k(k+1)} = \frac{1}{2} + \frac{1}{6} + \frac{1}{12} + …\text{ ist die Folge }\left(\frac{1}{2}, \frac{2}{3}, \frac{3}{4}\right)$$}
\end{enumerate}
\ul{Vorüberlegung}
$$\frac{1}{k\left(k+1\right)} = \frac{\left(k+1\right) - k}{k\left(k+1\right)} = \frac{1}{k} - \frac{1}{k + 1}$$
$$s_n := \sum\limits_{k = 1}^{∞} \frac{1}{k\left(k+1\right)}= \left(\frac{1}{1} - \frac{1}{2}\right) + \left(\frac{1}{2} - \frac{1}{3}\right) + … + \left(\frac{1}{n} - \frac{1}{n + 1}\right)
= 1 - \frac{1}{n + 1}$$
Teleskopsumme\\*
$\frac{1}{n + 1} \to 0$ für $n \to ∞$\\*
Summe der Reihe:
$$\sum\limits_{k = 1}^{∞}\frac{1}{k\left(k+1\right)} = \lim_{n \to ∞}\left(1 - \frac{1}{n + 1}\right) = 1$$\qed
\bem
Jede Folge kann man auch als Reihe Schreiben. (Differenzen bilden)\\*
z.B.: die Folge der Primzahlen:
$$(2, 3, 5, 7, 11, 13, 17, 19)$$
ist die Reihe:
$$(2 + 1 + 2+ 4+2+4+2+…)$$
Goldbachsche Vermutung: in dieser Reihe kommt die Zahl 2 unendlich oft vor.

\sS{Satz (Die geometrische Reihe)}
Sei $x \in \R{}$\\*
a) $\Sum\limits_{k = 0}^{∞} x^k = 1 + x^1 + x^2 + x^3 + … = \frac{1}{1-x} \text{ wenn } \mid x \mid < 1$\\*
b) $\ds\sum\limits_{k = 0}^{∞} x^k \text{ divergiert wenn } \mid x \mid \geq 1$
\begin{itemize}
\item[a] {wenn $|x| < 1$\\*
dann folgt $\sum{k=0}{∞} a_k = \lim_{n \to ∞}(\frac{1}{1 - x} - \frac{x}{1-x} · x^n) = \frac{1}{1 - x}$}
\item[b]{wenn $|x| > 1$\\*
dann $(x^n)$ divergent $\Rightarrow (\frac{x}{1-x} · x^n)$ divergent\\*
denn $\frac{x}{1-x} \neq 0\ \Rarr\ \left(\frac{?}{?}\right)$}
\end{itemize}
\bew
$$x = 1 \qquad \sum_{k = 0}^{∞} x^k = (1 + 1 + 1 +…)\text{ divergiert, ok}$$
Sei nun $x \neq 1$\\*
Bekannt aus der Übung:
$$\sum_{k = 0}^{∞} x^k = 1 + x + x^2 +x^3 … +x^n = \frac{1 -x^{n+1}}{1 - x} =\frac{1}{1 - x} -\frac{x}{1 - x} · x^n$$
Potenzenwachstum\\*
$x^n \to 0$ für $ n \to ∞$ \ul{wenn} $|x| < 1$\\*
$(x^n)$ divergiert, wenn $(|x| \geq 1 \text{ und } x \neq 1)$

\sS{Satz}
Wenn die Reihe $\ds\sum\limits_{k=0}^{∞} a_k $ konvergiert, dann ist $(a_n)_{n \in \N{}}$ eine Nullfolge.
\bew Gegeben sei $\e > 0$\\*
Sei $a = \ds\sum\limits_{k = 0}^{∞} a_k = $ $\ds\lim_{n \to ∞}(s_n)$ mit $s_n = a_0 + … + a_n$\\*
Es gibt $ N \ in \N{}$ mit $|s_n - a| < \ds\frac{\e}{2}$ für $n \geq N$\\*
$|a_n| = |s_n - s_{n-1}|$\\*
\phantom{$|a_n| $} = $|s_n - a + a - s_{n-1}|$\\*
\phantom{$|a_n| $} $\leq |s_n - a| + |a - s_{n-1}| < \ds\frac{\e}{2} + \frac{\e}{2} = \e$\\*
für $n \geq N + 1$\\*
$\Rarr a_n \to 0$ für $n \to ∞$
%
\sS{Satz, die harmonische Reihe}
$$\sum\limits_{k = 1}^{∞} \frac{1}{k}= 1 + \frac{1}{2} + \frac{1}{3} + …\text{ divergiert}$$
\sss{Beweisidee}
$$\phantom{= }1 + \frac{1}{2} + \frac{1}{3} +\frac{1}{4} +\frac{1}{5} +\frac{1}{6} +\frac{1}{7} + \frac{1}{8} +\frac{1}{9} +…$$
$$\phantom{\geq }1 + \frac{1}{2} +\frac{1}{4}+\frac{1}{4}+\frac{1}{8}+\frac{1}{8} + \frac{1}{8} + \frac{1}{8} + \frac{1}{16} + …$$
$$\phantom{= }1 + \frac{1}{2} + \frac{2}{4} + \frac{4}{8} + \frac{8}{16} + …$$
$$\phantom{= }1 + \frac{1}{2} + \frac{1}{2} + \frac{1}{2} + \frac{1}{2} + … = ∞$$