% Kopfzeile beim Kapitelanfang:
\fancypagestyle{plain}{
%Kopfzeile links bzw. innen
\fancyhead[L]{\calligra\Large Vorlesung Nr. 20}
%Kopfzeile rechts bzw. außen
\fancyhead[R]{\calligra\Large 20.12.2012}
}
%Kopfzeile links bzw. innen
\fancyhead[L]{\calligra\Large Vorlesung Nr. 20}
%Kopfzeile rechts bzw. außen
\fancyhead[R]{\calligra\Large 20.12.2012}
% **************************************************
%
\wdh
$I$ Intervall\\*
$f:I→\R$ ist in $x_0\in I$ differenzierbar wenn $$f'(x_0)=\lim_{x→x_0}\frac{f(x)-f(x_0)}{x-x_0}$$
$f'(x_0)$ Ableitung von $f$ an $x_0$
\bsp
$n\geq 0:$ $$(x^n)'=n·x^{n-1}$$
$$(\frac{1}{x})'=-\frac{1}{x^2},\ exp'=exp,\text{ d.h. }(e^x)'=e^x,\ cos'=-sin,\ sin'=cos$$
\sss{Produktregel}
$$(f·g)'=f'·g+f·g'$$
\sss{Kettenregel}
$$(g\circ f)'(x)=g'(f(x))·f'(x)$$
\bsp
$$(e^{x^2})'=(exp(x^2))'=exp'(x^2)·(x^2)'=2x·e^(x^2)$$
$((cos (x))^2)' = f(cos (x))' = f'(cos (x)) \cdot cos'(x) = 2 \cdot cos(x) \cdot sin(x)$
\sS{Satz Quotientenregel}
Seien $f, g: I \to \R$ in $x_0$ differenzierbar, $g(x) \neq 0$ für alle $x \in I$.\\*
Dann ist $\frac{f}{g}: I \to \R$, $\frac{f}{g}(x) := \frac{f(x)}{g(x)}$ differenzierbar in $x_0$
$$(\frac{f}{g})'(x) = \frac{f'(x_0)\cdot g(x_0) - f(x_0) \cdot g'(x_0)}{g(x_0^2)}$$
\bew
Fall $f = 1$:\\*
Kettenregel:
$\frac{1}{g} = \frac{1}{x} \cdot g$\\*
Sei $h(x) = \frac{1}{x}\ \ \frac{1}{g}(x) = h(g(x))$ \\*
$(\frac{1}{g})'(x) = h'(g(x)) \cdot g'(x) = - \frac{1}{g(x^2)} \cdot g'(x) \approx Beh.$\\*
Insbesondere: $(\frac{1}{g})' = -\frac{1}{g^2} \cdot g' = -\frac{g'}{g^2}$
Allgemeiner Fall:
$\frac{f}{g} = f \cdot \frac{1}{g}$\\*
Produktregel \Rarr{} $(\frac{f}{g})' = (f \cdot \frac{1}{g})' = f'\frac{1}{g} + f \cdot (\frac{1}{g})'$\\*
$= \frac{f' \cdot g}{g^2} - f\cdot \frac{g'}{g^2} = \frac{f'\cdot g - f \cdot g'}{g^2} \qed$
\bsp
Sei $n<0$, $f:\R\bs\{0\}→\R,\ f(x)=x^n$\\*
Sei $m=-n>0$ $f(x)=\frac{1}{x^m}$
$$f'(x)=-\frac{(x^m)'}{(x^n)'}=-m\frac{x^{m-1}}{x^{2m}}=-mx^{-m-1}=nx^{n-1}$$
$$-m-1=n-1$$ $$-m=n$$
\sss{Folge}
$(x^n)'=nx^{n-1}$ gilt für alle $n\eZ$!
$$(x^{-3})'=-3x^{-4}$$

\sS{Satz (Ableitung der Umkehrfunktion)}
Sei $f:I→\R$ stetig, streng monoton\\*
Sei $J=f(I),\ g=f^{-1}:J→I$ die Umkehrfunkion von $f$\\*
Angenommen, $f$ ist $x_0\in I$ differenzierbar und $f'(x_0)≠0$\\*
Dann ist $g$ in $y_0:=f(x_0)$ differenzierbar und $g'(y_0)≠\frac{1}{f'(x_0)}$\\*
\begin{tikzpicture}[domain=-1:4,prefix=plots/, smooth]
\draw[very thin,color=gray] (-0.99,-0.49) grid (3.99,2.99);
\draw[->] (-1,0) -- (4,0) node[right] {$x$};
\draw[->] (0,-1) -- (0,3) node[above] {$y$};
% Some fancy function
% Tangente zu dem Fancy Graphen bei x_0
\draw[color=blue] plot[id=cos1] function{cos(x)} node[below, midway] {\footnotesize $f_2(x) = cos(x)$};
\end{tikzpicture}
\bew
Sei $(y_n)_{n\geq 1}$ Folge in $J$ ist mit $y_n→y_0$
$$g'(y_0)=\lim_{\nif}\frac{g(y_n)-g(y_0)}{y_n-y_0}$$
\hfill(soll unabhängig von (y_n) sein)
$y_n→y_0\ (\nif)$ Sei $x_n=g(y_n)\ →\ x_n→x_0\ (\nif)$ da $g$ stetig.\\*
$x_n=g(y_n)\ \equ\ f(x_n)=y_n$
$$\lim_{\nif}\frac{g(y_n)-g(y_0)}{y_n-y_0}=\lim_{\nif}\frac{x_n-x_0}{f(x_n)-f(x_0)}=\left(\lim_{\nif}\frac{f(x_n)-f(x_0)}{x_n-x_0}\right)^{-1}=f'(x_0)^{-1}$$
Rechnung ok weil $f'(x_0)≠0$ \qed\\*
\ul{Folge} $log: \R_{>0}\to \R$ ist differenzierbar, $log'(x) = \frac{1}{x}$
\bew
$log(x) = exp(x)^{-1}$ Umkehrfunktion $exp'(x) = exp(x) \neq 0$ für alle $x$\\*
\Rarr{} 8.7 anwendbar. Sei $y = exp(x)$, $x = log(y)$.
$log'(y) = \frac{1}{exp'(x)} = \frac{1}{exp(x)} = \frac{1}{y}\qed$\\*
$log'(x)=\frac{1}{x}$
\sss{Anwendung}
$x=1\ log(1)=0$\\*
$log'(1)=\frac{1}{1}=1$
$$\lim_{\nif}\frac{log\left(1+\frac{1}{n}\right)-log(1)}{\frac{1}{n}}=log'(1)=1\ \Rarr\ 1=\lim_{\nif}\left(n·log\left(1+\frac{1}{n}\right)\right))$$
$exp$ anwenden $\underset{exp stetig}{\Rarr}$
$$exp(1)=\lim_{\nif}exp\left(n·log\left(1+\frac{1}{n}\right)\right))$$
$$e=exp(1)\underset{Def}{=}\lim_{\nif}\left(1+\frac{1}{n}\right)^n$$

\uS{Höhere Ableitungen}
Idee: Wenn $f: I \to \R$ differenzierbar\\*
\approx $f':I \to \R$ Fkt\\*
Wenn $f'$ diffbar $ \approx (f')' = f'' = \frac{d^2f}{dx^2}$\\*
2. Ableitung weiter:
$f'' = f^{(2)}$\\*
$f^{(n+1)} = (f^{n})'$ wenn differenzierbar\\*
$f^{(n)}$: n-te Ableitung von $f$.
\bsp
$$(x^5)^{(2)} = ((x^5)')' = (5x^4)' = 20x^3$$
$$cos'' = -sin' = -cos$$
$$sin'' = -cos' = -sin$$

\sS{Formale Definition der höheren Ableitung}
Rekursive Definition:\\*
Sei $n\geq 1$\\*
Eine Funktion $f:I→\R$ ist $n+1$ mal differenzierbar in $x_0\in I$ wenn ein $ε>0$ existiert, so dass $f$ auf $(x_0-ε,x_0+ε)$ $n$-mal differenzierbar und $f(n):(x_0-ε,x_0+ε)\R$ in $x_0$ differenzierbar sein, dann setzte $f^{(n+1)}(x_0):=(f^{(n)})'(x_0)$

\uS{Lokale Extrema und Mittelwertsatz}
\sS{Definition Lokales Maximum}
Sei $f: I \to \R$ Fkt.\\*
f hat ein \ul{lokales Maximum} in $x_0 \in I$ wenn gilt:\\*
$\case{
\text{es gibt ein $\e > 0$ s.d.}\\
\text{Für alle $x \in I$ mit $|x-x_0| < \e$}\\
\text{gilt $f(x) \leq f(x_0)$}
}$
Analog: Lokales Minimum.
\bem
Lokale Minima von $f$ = lokale Maxima von $-f$

\sS{Satz}
Sei $I=(a,b),\ f:I→\R$ Funktion\\*
Wenn $f$ in $x_0\in(a,b)$ ein lokales Extremum hat, und wenn $f$ in $x_0$ differenzierbar ist, dann ist $f'(x_0)=0$ (Extremum = Maxium oder Minimum)
\bew
$$f'(x_0)=\lim_{x\searrow x_0}\underbrace{\frac{f(x)-f(x_0)}{x-x_0}}_{\geq 0}=\lim_{x\nearrow x_0}\underbrace{\frac{f(x)-f(x_0)}{x-x_0}}_{\leq 0}$$
Angenommen $f$ hat in $x_0$ lokales Minimum \Rarr\ $f(x)-f(x_0)\geq 0$ wenn $|x-x_0|<ε,\ ε$ wie oben\\*
Somit $f'(x_0)\leq 0,\ f'(x_0)\geq 0\ \Rarr\ f'(x_0)= 0\qed$

\sS{Satz Rolle}
Sei $a < b$, $f:[a, b] \to \R$ stetig auf $(a, b)$ differenzierbar.\\*
Sei $f(a) = f(b)$.\\*
Dann gibt es ein $x_0 \in (a, b)$ mit $f'(x_0) = 0$\\*
GRAPH
\bew
Wenn $f$ konstant, d.h. $f(x) = f(a)$ für alle $x \in (a, b),$ dann $f'(x) = 0 $ für alle $x$ \Rarr{} Satz stimmt.\\*
Sei $f$ nicht konstant, gibt es $x_1 \in (a, b)$ mit $f(x_1) \neq f(a)$\\*
Angenommen $f(x_1) > f(a)$  (sonst Betrag $-f$)\\*
sei $x_0 \in I$ mit $f(x_0) \geq f(x)$ für alle $x \in I$\\*
$f(x_0) \geq f(x_1) > f(a) = f(b) \Rarr x_0 \neq a,\ x_0 \neq b$\\*
$f$ hat in $x_0$ ein \ul{lokales} Maximum $\overset{8.19}{\Rarr{}}\ f'(x_0) = 0$