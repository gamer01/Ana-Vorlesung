% Kopfzeile beim Kapitelanfang:
\fancypagestyle{plain}{
%Kopfzeile links bzw. innen
\fancyhead[L]{\calligra\Large Vorlesung Nr. 7}
%Kopfzeile rechts bzw. außen
\fancyhead[R]{\calligra\Large 29.10.2012}}
%Kopfzeile links bzw. innen
\fancyhead[L]{\calligra\Large Vorlesung Nr. 7}
%Kopfzeile rechts bzw. außen
\fancyhead[R]{\calligra\Large 29.10.2012}
%
% set chapters end sections
%\setcounter{chapter}3
\wdh
Sei $(n_n)$ eine Folge reeller Zahlen.\\*
Die Reihe mit den Gliedern $a_n$ ist die Folge $s_n = a_0 + a_1 + ... + a_n)_\nN$ \\*
Bezeichnung: $\ds\sum_{k=1}^{\infty} a_k$\\*
Wenn $S_n \to a$ für $n \to \infty$\\*
Schreibe: $\ds\sum_{k = 0}^{\infty} a_k = a$
\Bsp{Geometrische Reihe}
$\ds|x| = 1 \Rarr \sum\limits_{k = 0}^{\infty} x^k = \frac{1}{1-x}$ für $x = 0$ setzte $0^0 = 1$\\*[4pt]
Harmonische Reihe\\*
$\displaystyle\sum\limits_{k = 1}^{\infty} \frac{1}{k}$ Konvergiert nicht.\\*
\sS{Satz Rechenregeln für Reihen}
Seien $\sum\limits_{k = 0}^{\infty} a_k = a$ und $\sum\limits_{k = 0}^{\infty} b_k = b$ zwei konvergente Reihen. Dann:
\begin{enumerate}
\item{$\sum\limits_{k = 0}^{\infty} (a_k + b_k) = a + b$}
\item{Für $c \in \R{}$ ist $\sum\limits_{k = 0}^{\infty} c \cdot a_k = c \cdot a$}
\end{enumerate}
\bew
folgt aus 3.9.
\bem
Produkte von Reihen sind komplizierter.\\*
\ul{Korrektur:}
Primzahlen-Vermutung: es gibt ∞ viele Primzahlen $p$ so dass $p + 2$ auch Prim ist.\\*
Goldbach-Vermutung: Jede gerade natürliche Zahl ist die Summe von zwei Primzahlen.\\*
\chapter{Konvergenzsätze}
Erinnerung: \R{} ist Dedekind-vollständig. Das heißt, jede nicht-leere nach oben beschränkte Teilmenge $M \subset R$ hat eine kleinste obere Schranke $sup(M)$\ \Rarr{} Existenz von Grenzwerten

\sS{Definition Monotone Folgen}
\desc{Eine Folge $(a_n)_{n \geq 0}$ heißt}{monoton wachsend, wenn $a{n + 1} \geq a_n$ für alle $n \in \N_0$\\*
monoton fallend, wenn $a_{n + 1} \leq a_n$ für alle $n \in \N{}_0$\\*
streng monoton wachsend, wenn $a_{n + 1} > a_n$ für alle $n \in \N{}_0$\\*
streng monoton fallend, wenn $a_{n + 1} < a_n$ für alle $n \in \N_0$}
\bsp
$a_n = n$ ist streng monoton wachsend\\*
$a_n = \frac{1}{n}$ ist streng monoton fallend\\*

\sS{Satz}
\begin{enumerate}
\item{Jede nach oben beschränkte monoton wachsende Folge $(a_n)_{\nN}$ ist konvergent\\*
% Bild?
}
\item{Jede nach unten beschränkte monoton fallende Folge $(a_n)_\nN$ ist konvergent\\*
% Bild?
}
\end{enumerate}
\bew
Sei $(a_n)$ nach oben beschränkt, monoton wachsend\\*
Setze $a:= sup(\{a_n | n \in \N{}\})$\\*
dann \begin{enumerate}
\item{$a_n \leq a$ für alle $n$}
\item{Für jedes $\epsilon > 0$ ist $a - \epsilon$ \ul{keine} obere Schranke, d.h. es gibt $N \in N$ so dass $a_N > a - \epsilon$
\\*Für $n \geq N$ gilt\\*
$a - \epsilon < a_N \leq a_n \leq a$\\*
weil $(a_n)$ monoton wachsend\\*
$\Rightarrow a - \epsilon < a_N \leq a_n \leq a \Rightarrow |a_n -a| < \epsilon$\\*
Somit $a_n \to a$ für $n \to \infty$\phantom{XXX}$q.e.d.$\\*
Monoton fallend: analog}
\end{enumerate}

\uS{Reihen mit nicht-negativen Gliedern}
\bem
Sei $\ds\sum\limits_{k=0}^{\infty} a_k$ Reihe reeller Zahlen\\*
Die Folge der Partialsummen ist monoton wachsend $\Leftrightarrow a_n \geq 0$ für $n \geq 1$

\sS{Satz}
Eine Reihe $\ds\sum\limits_{k=0}^{\infty} a_k$ mit $a_k \geq 0$ für alle $k$ konvergiert, genau dann, wenn sie beschränkt ist (Das heißt die Folge der Partialsummen ist beschränkt)\qed

\sS{Definition}
Sei $\displaystyle\sum\limits_{k=0}^{\infty} a_k$ eine Reihe mit $a_n \geq 0$ für alle $k$\\*
Eine Reihe $\ds\sum\limits_{k=0}^{\infty} b_k$ heißt \ul{Majorante} von $\ds\sum a_k$ wenn $a_k \leq b_k$ für alle $k$
\sS{Satz Majorantenkriterium}
Wenn eine Reihe mit nicht-negativen Gliedern eine konvergente Majorante hat, dann konvergiert sie.
\bew
Sei $0 \leq a_k \leq b_k$ für alle $k \geq 0$\\*
Es gilt $a_0 + ... + a_n \leq b_0 + ... + b_n$ \\*
$\sum b$ konvergiert $\Rightarrow (b_0 + ... + b_n)_{n \geq 0}$  beschränkt\\*
$\Rightarrow ((a_0 + ... + a_n)_{n \geq 0})$ beschränkt $\Rightarrow \ds\sum_{k= 0}^{\infty} a_k$ konvergiert.
\Bsp{4.6:}
$$\sum\limits_{k=1}^{\infty} \frac{1}{k^2} = \left( 1 + \frac{1}{4} + \frac{1}{9}+ \frac{1}{16} + … \right)$$
$$\sum\limits_{k=1}^{\infty} \frac{1}{k^2} = 1 + \sum\limits_{k=1}^{\infty} \frac{1}{(k + 1)^2}$$
$$\frac{1}{(k + 1)^2} \leq \frac{1}{k \cdot (k + 1)}$$
$$\Rarr \sum\limits_{k=1}^{\infty} \frac{1}{k \cdot (k + 1)}\text{ ist Majorante von }\sum\limits_{k=1}^{\infty} \frac{1}{k^2}$$
$$\sum\limits_{k=1}^{\infty} \frac{1}{k \cdot (k + 1)}\text{ konvergiert (bekannt)}$$
\sS{Satz Quotientenkriterium}
Sei $C \in \R{}, (a_n)$ eine Folge reeller Zahlen mit $a_n \geq 0$ für alle $n$ \ul{und} $a_{n + 1} \leq C \cdot a_n$ für fast alle $n$\\*
$0 \leq C < 1$\\*
Dann konvergiert die Reihe $\ds\sum\limits_{k=0}^{\infty} a_k$
\bew
Konvergenz ändert sich nicht, wenn endlich viele $a_n$ geändert werden.\\*
Also kann man annehmen, dass $a_{n + 1} \leq C \cdot a_n$ für alle $n$ gilt.\\*
Dann gilt $a_1 < C \cdot a_0$\\*
$a_2 < C \cdot a_1 \leq C \cdot C \cdot a_0 = C^2 \cdot a_0$\\*
$a_3 < C \cdot a_2 \leq C \cdot C \cdot a_1 = C^3 \cdot a_0$\\*
etc. $\Rightarrow a_n \leq C^n \cdot a_0$\\*
Somit ist $\displaystyle\sum\limits_{k=0}^{\infty} C^k \cdot a_0$ konvergente Majorante von $\ds\sum_{k=0}^{\infty} a_k$ (Geometrische Reihe)
\sS{Beispiel Die Exponentialreihe}
$$exp(x) := \sum\limits_{k=0}^{\infty} \frac{x^k}{k!} \text{ für } x \in \R{}, x \geq 0$$
Setze $a_k = \frac{x^k}{k!}$\\*
$$a_n+1 = \frac{x^{n + 1}}{(n + 1)!} = \frac{x}{n+1} \cdot \frac{x^n}{n!} = \frac{x}{n+1} \cdot a_n \leq \frac{1}{2} a_n$$
\Rarr{} Quotientenregel ist erfüllt.\\*
Reihe $exp(x)$ konvergiert.\\*
Bezeichnung: $$exp(x) = \sum_{k=0}^{\infty} \frac{x^k}{k!} \eR$$
%
\uS{Bezeichnung:}
$$exp(1) = \sum\limits_{k=0}^{\infty} \frac{1}{k!} = e\text{ (Eulerische Zahl)}$$
\sS{Leibnitz-Kriterium}
Sei $(a_n)_{n \in \N{}_0}$ eine monoton fallende Nullfolge\footnote{$a_n → 0$ für $n→∞$} mit $a_n \geq 0$ für alle $n$\\*
Dann konvergiert die alternierende Reihe\\*
$$\sum_{k=0}^{\infty} (-1)^k · a_k$$
\bsp
$$a_k = \frac{1}{k + 1} \sum_{k=0}^{\infty} (-1)^k · a_k = 1 - \frac{1}{2} + \frac{1}{3} - \frac{1}{4} + \frac{1}{5}= log(2)$$
\bew
Sei $s_n = a_0 + ... + a_n$
\uS{Behauptung: }
$S_{2n + 1} \leq S_{2n + 3} \leq S_{2n + 2} \leq S_{2n}$ für jedes $n \in \N{}$\\*[4pt]
\ul{Rechne:}\\*
$S_{2n + 2} - S_{2n} = - a_{2n + 1} + a_{2n + 2} \leq 0 \Rightarrow (3)$\\*
$S_{2n + 3} - S_{2n + 1} = - a_{2n + 3} \leq 0 \Rightarrow (2)$\\*
$S_{2n + 3} - S_{2n + 1} = - a_{2n + 2} - a_{2n + 3} \leq 0 \Rightarrow (1)$\\*[8pt]
\desc{Die Folge }{$b_n = S_{2n}$\\*
$c_n = S_{2n + 1}$}
sind beschränkt und monoton (fallend bzw. steigend)\\*
$\Rightarrow b_n \text{ und } c_n$ konvergieren\\*[4pt]
Sei $$b = \lim_{n \to \infty} b_n \qquad c = \lim_{n \to \infty} c_n$$
$$c - b = \lim_{n \to \infty} (c_n - b_n) = \lim_{n \to \infty} (a_{2n + 1}) = 0$$
weil $(a_n)$ Nullfolge
\sS{Behauptung:} % uS stande dort ich vermute so
$S_n \to b$ für $n \to \infty$\\*[4pt]
Gegeben sei $\e > 0$. Es gibt $N \in \N{}$ so dass für $n \leq N$:\\*
$|b_n - b| < \e, |c_n - c| < \epsilon$\\*
Somit für $n \geq 2N+1 $\\*
$|S_n - b| < \e$ also $S_n \to b$\qed