% Kopfzeile beim Kapitelanfang:
\fancypagestyle{plain}{
%Kopfzeile links bzw. innen
\fancyhead[L]{\calligra\Large Vorlesung Nr. 13}
%Kopfzeile rechts bzw. außen
\fancyhead[R]{\calligra\Large 22.11.2012}
}
%Kopfzeile links bzw. innen
\fancyhead[L]{\calligra\Large Vorlesung Nr. 13}
%Kopfzeile rechts bzw. außen
\fancyhead[R]{\calligra\Large 22.11.2012}
% **************************************************
%
\wdh
Zwischenwertsatz\\*
Sei $a\leq b,\ f:[a,b]→\R$ stetig\\*
Sei $y\eR$ zwischen $f(a)$ und $f(b)$ d.h. $f(a)\leq y\leq f(b)$ oder $f(a)\geq y\geq f(b)$\\*
Dann gibt es ein $x\in[a,b]$ mit $f(x)=y$ SKIZZE
\Bew {Intervallschachtelung}
Starte mit $[a_0,b_0]=[a,b]$\\*
Definiere unendliche Kette von Intervallen\\*
$[a_0,b_0]\supseteq [a_1,b_1]\supseteq [a_2,b_2]\supseteq …$\\*
So dass $[b_n-a_n]=2^{-n}[b_0,a_0]$ und $y$ zwischen $f(a_n)$ und $f(b_n)$ liegt.\\*
Annahme: $f(a)\leq y\leq f(b)$ (Anderer Fall $f(a)\geq y\geq f(b)$ analog)\\*
Angenommen, $[a_n,b_n]$ ist konstruiert so dass $[b_n-a_n]=2^{-n}(b_0,a_0)$ und $f(a_n)\leq y\leq f(b_n)$\\*
Betrachte $m:=\frac{a_n+b_n}{2}$, Wenn $f(m)\geq y$ dann setze $[a_{n+1},b_{n+1}]:=[a_n,m]$\\*
Wenn $f(m)<y$ dann setze $[a_{n+1},b_{n+1}]:=[m,b_n]$\\*
Dann gilt in beiden Fällen:
$$b_{n+1}-a_{n+1}=\frac{1}{2}(b_n-a_n)=2^{-1}·2^{-n}(b_0-a_0)=2^{-n-1}(b_0-a_0)$$
und $f(a_{n+1})\leq y\leq f(b_{n+1})$\\*
%
\ul{Idee:} Folge von Intervallen $[a_n, b_n]$ "Konvergent" gegen gesuchtes $x$.\\*
	Die Folge $(a_n)_{n \geq 0}$ ist monoton wachsend und beschränkt, ($b$ ist obere Schranke) \Rarr $(a_n)$ konvergiert, sei $x:=\lim_{n \to \infty} a_n$\\*
	Die Folge $(b_n)_{n \geq 0}$ ist monoton fallen und beschränkt \Rarr konvergent nach 4.2\\*
	Sei $x' = \lim_{n \to \infty} b_n$\\
	$x' - x = \lim_{n \to \infty} (b_n - a_n)$\\*
	$= \lim_{n \to \infty} (2^{-n} \cdot (b_0 - a_0)) = 0$\\*
	also $x = x'$ $f(x) = ?$\\
	$f$ stetig $\Rarr f(x) = \lim_{n \to \infty} f(a_n) = \lim_{n \to \infty} f(b_n)$\\
	Wegen $f(a_n) \leq y \leq f(b_n)$ für alle $n$ gilt $f(x) = \lim_{n \to \infty} f(a_n) \leq y \leq \lim_{n \to \infty} f(b_n)$\\*
	$\Rarr f(x) = y$ 
%
\bem
Weil $a\leq a_n\leq b$ gilt $a\leq x\leq b$ d.h. $x\in[a,b]$\qed
\uS{Anwendung}
\sS{Satz Sichere Nullstellen}
Sei \nN{} \ul{ungerade}, $f:\R→\R$
$$f(x)=x^n+a_{n-1}·x^{n-1}+…+a_0$$
Dann hat $f$ eine Nullstelle, d.h. es gibt $x\eR$ mit $f(x)=0$
\bew
Für $x\neq 0$ betrachte
$$g(x)=\frac{1}{x^n} \cdot f(x)=1+\frac{a_{n-1}}{x}+\frac{a_{n-2}}{x^2}+…+\frac{a_0}{x^n}$$
Für $x→∞$ ist $g(x)→1$\\*
Für $x→-∞$ ist $g(x)→1$\\*
D.h. es gibt $a\eR$ mit $a>0$ und
$$x\geq a \Rarr g(x)>0$$
$$x\geq -a \Rarr g(x)>0$$
%
\ul{Also:} $x \geq a \Rarr f(x) = x^n \cdot g(x) > 0$\\*
	$x \geq a \Rarr f(x) = x^n \cdot g(x) < 0$\\* %Pfeil auf x^n mit {x^n < 0} % Pfeil auf g(x) {> 0}
	$f(-a) < 0 < f(a)$\\*
	Zwischenwertsatz \Rarr ergibt $x \in [-a, a]$ mit $f(x) = 0$\\*

\sS{Satz Ergänzung Zwischenwertsatz} % 6.14
	Sei $f: D \to \R$ \ul{stetig} und $D \subseteq \R$ ein nicht-leeres Intervall. Dann ist $f(D) = \{f(x) | x \in D\}$ auch ein Intervall (d.h. hat keine Lücken.)
\bem
	Hier sind auch uneigentliche Intervalle zugelassen. (z.B. $(0, \infty)$)
%
\bsp
$f(x)=x^3-x+20$ GRAPH
\bsp
Bedingung "$n$ ungerade" ist wesentlich, denn $f(x)=x^2+1$ hat keine Nullstelle
% Satz + bem
\bsp
$$f:(0,1)→\R,\ f(x)=\frac{1}{x}$$
$$f(D)=(1,∞)$$ GRAPH
\bsp
$$f:(-1,1)→\R,\ f(x)=x^2$$
$$f(D)=[0,1)$$ GRAPH
\bew
$f:D→\R$ stetig\\*
Sei $a:=inf(f(D))\eR\cup\{-∞\}$\\*
\phantom{Sei }$a:=inf(f(D))\eR\cup\{-∞\}$\\*
Angenommen $y\eR$ mit $a<y<b$ d.h. $x\in(a,b)$\\*
Es gibt $x_1,x_2\in D$ mit $a<f(x_1)<y<f(x_2)<b$\\*
Zwischenwertsatz \Rarr{} es gibt $x$ zwischen $x_1,x_2$\\*
(\Rarr $x\in D$ weil $D$ Intervall) mit $f(x)=y$\\*
Also $(a,b) \subseteq f(D)$\\*
Dann ist $f(D)$ eines der Intervalle $(a,b),[a,b),(a,b],\underset{\overset{\uparrow}{\text{nur wenn }a\neq -∞, b\neq ∞}}{[a,b]}$\qed
% 
\sS{Satz Umkehrfunktion}
	Sei $D$ ein Intervall, $f: D \to \R$, stetig, streng monoton wachsend oder fallend. Sei $D' = f(D)$ (Intervall nach 6.14)\\*
	Dann ist die Abbildung $f: D \to D'$ bijektiv und die Umkehrabbildung $f^{-1}: D' \to D$ ist stetig und streng monoton wachsend, bzw. fallen.
%
\bew
Die Abbildung $f:D→D'$ ist
\begin{itemize}
\item{surjektiv nach Definition von $D'$}
\item{streng monoton \Rarr{} injektiv}
\item{also bijektiv. Somit existiert $f^{-1}:D'→D$}
\end{itemize}
\sss{Annahme}
$f$ streng monoton wachsend (fallend analog)
%
\beh
$f^{-1}$ ist streng monoton wachsend, d.h. gegeben $x_1,x_2\in D'$ mit $x_1<x_2$ zeige:
$$f^{-1}(x_1)<f^{-1}(x_2)$$
Angenommen $f^{-1}(x_1)\geq f^{-1}(x_2)$ \Rarr{} $f$ monoton wachsend \Rarr{} $x_1=f(f^{-1}(x_1))\geq f(f^{-1}(x_2))=x_2$ \Rarr{} Widerspruch\\*
also $f^{-1}(x_1)<f^{-1}(x_2)$ \Rarr{} $f^{-1}$ streng monoton wachsend
%
\beh
	$f^{-1}$ ist stetig. Gegeben $x \in D$\\*
	Annahme $x$ ist kein Randpunkt des Intervalls $D'$\\*
	Gegeben sei $\e > 0$ Suche $\delta$ mit (Stetigkeitsdefinition)\\*
	$y := f^{-1}(x) \in D$ ist kein Randpunkt (weil $f,\ f^{-1}$ bijektiv und streng monoton.)\\*
	%Graph
	Wähle $\e'\leq \e$ mit $\e > 0$ sodass $[y-\e', y+\e'] \subseteq D'$\\*
	$f(y - \e') < f(y) = x \Larr y - \e' < y$\\*
	also $f(y - \e') = x - \delta_1 \qquad \delta_1 > 0$\\*
	$genauso f(y + \e') = x + \delta_2 \qquad \delta_2 > 0$ \\*
	%Graph
	Sei $\delta = min(\delta_1, \delta_2)$
\beh
	Wenn $z \in D'$ mit $|z - x| < \delta$ dann $|f^{-1}(z) - f^{-1}(x)| < \e$
%
\bew
$$x+δ_1\leq x-δ<z<x+δ\leq x+δ-2$$
$f^{-1}$ streng monoton wachsend \Rarr
$$f^{-1}(x)-ε'=f^{-1}(x-δ_1)<f^{-1}(z)<f^{-1}(x+δ-2)=y+ε'=f^{-1}(x)+ε'$$
$$\Rarr |f^{-1}(z)-f^{-1}(x)|<ε'\leq ε \text{\Rarr{} Behauptung}$$
Somit $f^{-1}$ stetig in $x$\\*
Falls $x$ Randpunkt: Betrachte $[x,x+δ]$ bzw. $[x-δ,x]$ wieder analog\qed
%
\uS{2 Anwendungen}
\sS{Beispiel}
	Sei $k \ in \N$\\*
	Die Abbildung $f: \R_{\geq 0} \to \R_{\geq 0} f(x) = x^k$\\*
	%Graph
	\ul{Bekannt:} $f$ ist stetig streng monoton wachsend.\\*
	$f(0) = 0, \qquad \lim_{x \to \infty} x^k = \infty$\\*
	$D:= [0, \infty) = \R_{\geq 0}$\\*
	$f(D) = D' = [0, \infty)$\\*
	6.15 \Rarr f hat \ul{stetige} und streng monoton wachsende Unterfunktionen\\*
	$f^{-1} : [0, \infty) \to [0, \infty)$\\*
	Bezeichnung: $f^{-1}(x) = \sqrt[k]{x}$
%
\uS{Logarithmus und allgemeine Potenzen}
\sS{Satz}
Die Exponentialfunktion $exp:\R→\R_{>0}=(0,∞)$ ist stetig, streng monoton wachsend und $exp(\R)=\R_{>0}$
\bew
Bekannt: $exp$ stetig, streng monoton wachsend.
Für $x>0$ ist $$exp(x)=1+x+\frac{x^2}{2}+…\geq 1+x$$
also gilt $$\lim_{x→∞} exp(x)=∞$$
$$exp(-x)=\frac{1}{exp(x)}\Rarr \lim_{x→-∞}exp(x)=\lim_{x→∞}\frac{1}{exp(x)}$$
Somit $exp(\R)=(0,∞)$\qed\\*
Folge mittels 6.15: $exp:\R→\R_{>0}$ ist bijektiv und die Umkehrfunktion $exp^{-1}:=log:\R_{>0}→\R$ ist stetig, streng monoton wachsend, bijektiv \footnote{$exp^{-1}=log$ heißt Logarithmusfunktion} konkret: $log(x)=y \equ x=exp(y)$ GRAPH
