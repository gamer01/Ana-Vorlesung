% Kopfzeile beim Kapitelanfang:
\fancypagestyle{plain}{
%Kopfzeile links bzw. innen
\fancyhead[L]{\calligra\Large Vorlesung Nr. 25}
%Kopfzeile rechts bzw. außen
\fancyhead[R]{\calligra\Large 21.01.2013}
}
%Kopfzeile links bzw. innen
\fancyhead[L]{\calligra\Large Vorlesung Nr. 25}
%Kopfzeile rechts bzw. außen
\fancyhead[R]{\calligra\Large 21.01.2013}
% **************************************************
%
\sss{Uneigentliche Integrale}
zum Beispiel:
$$\int_a^{∞}f(x)dx:=\lim_{b→∞}\int_a^bdx$$
(wenn der lim existiert)

\uS{Integrale mit Reihen}
Beobachtung: eine Reihe $\Sum_{k=0}^{∞}a_k$ ist das unbestimmte Integral einer Treppenfunktion:\\*
\begin{tikzpicture}[domain=0:5,prefix=plots/, samples=5, const plot]
\draw[very thin,color=gray] (-0.3,0.0) grid (5,2.0);
\draw[->] (-0.3,0) -- (5.2,0) node[right] {$x$};
\draw[->] (0,-0.3) -- (0,2) node[above] {$y$};
\draw[color=black] plot[id=23.2_int] function{sin(x)+1.2} node[below, midway] {$f(x)$};
\end{tikzpicture}\\*
$$\Sum_{k=0}^{∞}a_k=\int_0^{∞}f(x)dx$$

\sS{Satz (Integralkriterium für Reihen)}
Sei $f:[1,∞)→\R$ monoton fallend mit $f(x)\geq 0$ für alle $x$. Für $n\geq 1$, sei
$$a_n=\sum_{k=1}^{n}f(k)-\int_1^{n+1}f(x)dx$$
Graph 1/x Treppenfunktion über dem graphen, fester abstand, schraffur treppenfunktion ohne graph
\enum{
\item die Folge $(a_n)$ konvergiert
\item die Reihe $\Sum_{k=1}^{∞}f(x)$ konvergiert \equ\ $\Int_1^{∞}f(x)dx$ konvergiert
}
\bew
$f$ monoton: $k\leq x\leq k+1\ \Rarr\ f(k)\geq f(x)\geq f(k+1)$
\Rarr
$$f(k)=\int_k^{k+1}f(k)dx\geq \int_k^{k+1}f(x)dx\geq \int_k^{k+1}f(k+1)dx=f(k+1)$$
\alg{a_n&=\Sum_{k=1}^{n}\left(\underbrace{f(k)-\int_k^{k+1}f(x)dx}_{\geq 0}\right) \underset{(b)}{\leq}\Sum_{k=1}^{n}\left(f(k)-f(k+1)dx\right)\\
&=f(1)-f(2)+f(2)-f(3)+…-f(n+1)=f(1)-f(n+1)\leq f(1)}
\Rarr\ $(a_n)$ monoton wachsend, beschränkt \Rarr\ konvergent \Rarr\ (1).\\
Sei $\gamma=\lim_{n→∞}a_n$\\*
\enum{
\setcounter{enumi}{1}
\item Angenommen $\int_0^{\infty} f(x) dx$ konvergent.\\*
$\sum_{k=1}^\infty f(k)= \lim_{n \to \infty} \sum_{k=1}^n f(k)$\\*
$=\lim_{n \to \infty} \left( \underbrace{f(x) - \int_1^{n+1} f(x)dx}_{\gamma} \right) + \underbrace{\int_1{n + 1} f(x)dx}_{\text{konvergiert}}$
}
\Rarr\ $\lim$ existiert (auch $\Sum_{k\geq 1}f(x)=\gamma+\int_1^{∞}f(x)dx$)\\
Richtung: $\int$ konvergiert \Rarr\ $\sum$ konvergiert ähnlich\qed
\bsp 
$f(x)=\frac{1}{x}$\\*
\begin{tikzpicture}
		[smooth]
		\pgfmathsetmacro\minx{-2}
		\pgfmathsetmacro\maxx{5}
		\pgfmathsetmacro\miny{-2}
		\pgfmathsetmacro\maxy{2}
		\draw[very thin,color=gray!40] (\minx,\miny) grid (\maxx,\maxy);
		\draw[->] (\minx-0.3,0) -- (\maxx+0.3,0) node[right] {$x$};
		\foreach \x in {\minx,...,-1}{\draw (\x cm,2pt) -- (\x cm,-2pt) node[below] {\tiny $\x$};}
		\foreach \x in {1,...,\maxx}{\draw (\x cm,2pt) -- (\x cm,-2pt) node[below] {\tiny $\x$};}
		\draw[->] (0,\miny-0.3) -- (0,\maxy+0.3) node[above] {$y$};
		\foreach \y in {\miny,...,-1}{\draw (2pt,\y cm) -- (-2pt,\y cm) node[left] {\tiny $\y$};}
		\foreach \y in {1,...,\maxy}{\draw (2pt,\y cm) -- (-2pt,\y cm) node[left] {\tiny $\y$};}
		\clip (\minx,\miny) rectangle (\maxx,\maxy);
		\draw[domain=1:5,color=black] plot function{1/x} node[right] {$\frac{1}{x}$};
\end{tikzpicture}%-2.0 ist in der beschriftung ein fehler
\sss{Folge}
$\Sum_{k=1}^{∞}$ konvergiert \equ\ $\Int_1^{∞}\frac{1}{x}dx$ konvergiert (nicht der Fall)
$$\left(\int_1^{∞}\frac{1}{x}dx=\lim_{b→∞}log(b)=∞\right)$$
\ul{Bsp} sei $s > 1$ $$\sum_{k=1}^{\infty} \frac{1}{k^s} \text{konvergiert} \equ \int_1^{\infty} \frac{1}{x^s} \text{konvergiert}$$

\sS{Beispiel}
Berechnung der Reihe $1 - \frac{1}{2} + \frac{1}{3} + \frac{1}{4} + \frac{1}{5} ... = \sum_{k=1}^{\infty} (-1)^{k+1}\frac{1}{k}$\\*
(konvergiert nach Leibniz)
$\sum_{k=1}^{infty} (-1)^{k+1} \frac{1}{k} = \lim_{n \to \infty} (-1)^{k+1} \frac{1}{k}$\\*
$=\lim_{n \to \infty} (1 + \frac{1}{2} + \frac{1}{3} + \frac{1}{4} + ... + \frac{1}{2n}) - 2(\frac{1}{2} + \frac{1}{4} + \frac{1}{6} + ... + \frac{1}{2n}$\\*
Sei $cn = \sum_{k=1}^n \frac{1}{k} =\footnote{NR $2(\frac{1}{2} + \frac{1}{4} + \frac{1}{6} + ... \frac{1}{2n}  )$} \lim{n \to \infty} (c_{zn} - c_n)$
\alg{(n = 2\quad &1- \frac{1}{2} + \frac{1}{3} + \frac{1}{4}\\
&1 + \frac{1}{2} + \frac{1}{3} + \frac{1}{4} - 2·\frac{1}{2} - 2·\frac{1}{4})}
Sei $a_n:=\sum_{k=1}^n \frac{1}{k} -\int_1^{n+1} \frac{1}{x}dx = c_n - \int_1^{n+1} \frac{1}{x}dx\underset{Satz\ (1)}{\Rarr}\lim_{\nif} (a_n) = \gamma$ existiert!

\alg{b_n &:= \int_1^{n+1} \frac{1}{x}dx\\
a_n &= c_n - b_n\quad (c_n = a_n + b_n)}
MISSING STUFF
%\alg{&\lim_{\nif}(c_{2n} - c_n) = \lim_{\nif} (a_{2n} - a_n + b_{2n} - b_n)\\
%\underbrace{&\lim_{\nif} a_{2n}}_{\gamma} - \underbrace{\lim_{\nif} a_n}_{\gamma} + \lim_{\nif} (b_{2n} - b_n) \\ \lim_{\nif} (b_{2n} - b_n)}
%\ary{&=\lim_{n→∞}\int_{n+1}^{2n+1}\frac{1}{x}dx=\lim_{n→∞}\left(log(2n+1)-log(n+1)\right)\\
%&=\lim_{n→∞}\left(log\frac{2n+1}{n+1}\right)=log\left(\lim_{n→∞}\frac{2n+1}{n+1}\right)=log(2)\to 2(n→∞)\text{, log stetig.}}

\chapter{Potenzreihen}
\sS{Definition Potenzreihen}
Eine Potenzreihe in der Variablen $z$ ist eine Reihe der Form
$$P(z)=\Sum_{k=0}^{∞}a_kz11k\quad\text{mit $a_k\eC$}$$
(reelle Potenzreihe: $a_k\eR$)
\bsp
Exponentialreihe
$$exp(z)=\sum_{k=0}^{∞}\frac{1}{k!}z^k\quad a_k=\frac{1}{k!}$$

\uS{Lemma}
Wenn $P(z_0)$ für ein $z_0\eC$ konvergiert, dann konvergiert $P(z)$ für jedes $z\eC$ mit $|z|<|z_0|$ absolut.
\bew
$P(z_0)=\sum  a_kz_0^k$ konvergiert \Rarr\ es gibt $C\eR$ mit $|a_kz_0^k|\leq C$ für alle $k$\\*
Sei $|z|<|z_0|$, dass heißt $q=\frac{|z|}{|z_0|}<1$
$$|a_kz^k|=|a_kz_o^k\left(\frac{z}{z_0}\right)|=|a_kz_0^k|·q^k\leq C·q^k$$
\Rarr\ Die Reihe $P(z) = \sum_k a_k z^k$ hat eine Majorante $\sum_k C\cdot q^k$, letztere konvergiert (Geometrische Reihe)\\*
Majorantenkriterium \Rarr\ $P(z)$ konvergiert absolut\qed

\sS{Defintion Konvergenzradius}
Der Konvergenzradius von $P(z)$ ist
$$R:=sup\left\{r\eR_{\geq 0}\mid P(r)\text{ konvergiert} \right\}\eR_{\geq 0}\cup\{∞\}$$
\bem
ERROREOS STUCTURES
%10.2 $\Rarr\ \ary{|z| < R\ \Rarr\ P(z)\text{ konvergiert absolut}\\|z| > R\ \Rarr\ P(z)\text{ divergiert}\\|z| = R\ \Rarr\ ?}$
					% GRAPH Konvergenzradius
\bsp
\enum{
\item $exp(z)$ konvergiert absolut für jedes $z \in \C$\\*
$R = \infty$
\item $\sum_{n=0}^2\infty 2^w\ z^w = 1 + 2z + 4z^2 +... = \sum_{n=0}^{\infty} (2z)^n$ geometrische Reihe.\\
$|z| \geq \frac{1}{2} \equ |2z| \geq 1$: divergiert\\*
$|z| < \frac{1}{2} \equ |2z| < 1$: konvergiert
}
Also $R=\frac{1}{2}$
\bsp
$$P(z)=\Sum_{n=0}^{∞}\frac{1}{n+1}z^w=1+\frac{z}{2}+\frac{z^2}{3}+\frac{z^3}{4}+…$$
$$R=1\ \Leftarrow\ \left\{\ary{z=1:\ P(1)=1+\frac{1}{2}+\frac{1}{3}+…\text{ divergiert}\\
z=-1:\ P(1)=1-\frac{1}{2}+\frac{1}{3}-\frac{1}{4}+…=log(2)\text{ konvergiert}}\right\}$$
\sss{Folge}
Methoden zur Berechnung des Konvergenzradius

\sS{Definition}
Sei $(a_n)_{n\geq0}$ reelle Folge.\\*
Bilde $b_m = sup(a_n)_{n \geq m} = sup \{a_m, a_{m+1},...\} \eR \cup \{\infty\}$\\*
Dann: $b_0 \geq b_1 \geq b_2 \geq ...$ $(b_n)$ monoton fallend \Rarr $\Lim_{n\to\infty} (b_n) =: \lim_{n\to\infty}sup(a_n) \in \R \cup \{\pm \infty\}$ existiert
\bsp
\alg{(a_n)&=(1,-1,\frac{1}{2},-1,\frac{1}{3},-1,\frac{1}{4},-1,\frac{1}{5}…)\\
(b_n)&=(1,\frac{1}{2},\frac{1}{2},\frac{1}{3},\frac{1}{3},\frac{1}{4},\frac{1}{4},\frac{1}{5},\frac{1}{5}…)}
$$\limsup(a_n)=\lim(b_n)=0$$
\alg{(a_n)&=(0,1,0,2,0,3,0,4…)\\
(b_n)&=(∞,∞,∞,∞,∞,∞,∞,…)}
$$\limsup(a_n)=∞$$
\alg{(a_n)&=(0,-1,-2,-3,-4,…)\\
(b_n)&=(0,-1,-2,-3,-4,…)}
$$\limsup(a_n)=\lim(b_n)=-∞$$
\bem
$C=\limsup_{n→∞}(a_n)$ ist durch folgende Eigenschaft eindeutig bestimmt:
Für jedes $ε>0$ gibt es
\enum{\item unendlich viele $n\eN$ mit $a_n\geq C-ε$
\item unendlich viele $n\eN$ mit $a_n> C+ε$}
SKIZZE\\*
(zumindest wenn C≠-∞)\\*
(ohne Beweis)

\sS{Satz}
Der Konvergenzradius einer Potenzreihe $P(z)=\sum  a_kz_0^k$ ist $R=\left(\limsup_{n→∞}\left(\sqrt[n]{|a_n|}\right)\right)^{-1} \eR_{\geq 0} \cup \{\infty\}$
(Setze hier $0^{-1} = \infty, \infty^{-1} =0$)