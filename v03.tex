% Kopfzeile beim Kapitelanfang:
\fancypagestyle{plain}{
%Kopfzeile links bzw. innen
\fancyhead[L]{\calligra\Large Vorlesung Nr. 3}
%Kopfzeile rechts bzw. außen
\fancyhead[R]{\calligra\Large 15.10.2012}
}
%Kopfzeile links bzw. innen
\fancyhead[L]{\calligra {\Large Vorlesung Nr. 3}}
%Kopfzeile rechts bzw. außen
\fancyhead[R]{\calligra \Large{15.10.2012}}
% **************************************************
%
\wdh
Ein Körper $K$ ist eine Menge mit $+$ und $·$, sodass gewisse Eigenschaften erfüllt sind:
\bsp
$\ds\Q = \left\{\frac{a}{b} \mid a \in \Z, b \neq 0\right\}$\\*
$F_1 = \{0, 1\} \qquad 1 + 1 = 0$\\*
\notat{Setze $a^n = \underbrace{ a · a · a · a· … · a}_{n-Faktoren}$\\*
$\left.\begin{array}{lcc}
a^0 &=& 1\\*
a^{-n} &=& (a^{-1})^n
\end{array}\right\}$ wenn $a \neq 0$}
Daraus folgt $a^n$ ist definiert, wenn $a \neq 0$ und $n \in \Z$\\*
Regeln der Potenzgleichung:\\*
$a^{n+m} = a^n \cdot a^m$\\*
$a^{n \cdot m} = (a^{n})^m$\\*
\bew
Übung

\sS{Definition angeordneter Körper}
    Ein angeordneter Körper ist ein Körper $K$ für dessen Elemente eine "Kleiner als Beziehung" $<$ definiert ist, so dass folgende Eigenschaften erfüllt sind:\\*
    \begin{enumerate}
    \item{Für alle $a, b \in K$ gilt genau eine von drei Notationen:\\*
    $a < b$ oder $a = b$ oder $a > b$}
    \item{Für alle $a, b, c \in K$ gilt wenn $a < b$ und $b < c$ dann $a < c$\\* (Transitivität)}
    \item{Für alle $a, b, c \in K$ gilt wenn $a < b$ dann $a + c < b + c$}
    \item{für $a, b, c \in K$ gilt, wenn $a < b$ und $c \neq 0$ dann $a \cdot c < b \cdot c$}
    \end{enumerate}
	Weitere Beziehungen:\\*
	$a > b$ heißt $b < a$
	\begin{enumerate}
	\item{Wenn $a < 0$ dann $-a > 0$:\\*
	$a < 0 \Rarr a + (-a) > 0 + (-a) \Rarr 0 > -a$}
	\item{Für jedes $a \in K $ gilt wenn $a \neq 0$, dann $a^2 > 0$
	\begin{itemize}
	\item[(a)]{$\begin{array}{ccc}
	a &>& 0\\*
	a · a &>& 0 · a\\*
	a^2 &>& 0
	\end{array}$\qed}
	\item[(b)]{$\begin{array}{ccc}
		a &<& 0\\*
		-a &>& 0 · a\\*
		a^2 &=& (-a)^2 > 0
		\end{array}$\qed}
	\end{itemize}}
	\item{$1 > 0 $ denn $1 = 1^2$}
	\end{enumerate}
	Sei $K$ ein Angeordneter Körper:
	$$0 < 1 \Rarr 1 < 1 + 1 \Rarr 1 + 1 < 1 + 1 + 1 \text{ etc.}$$
	$$0 < 1 < 1 + 1 < 1 + 1 +1 \text{ etc.}$$
	Für $n \in \N$ setze $\underbrace{n:= 1 + 1 + 1 + … + 1}_{n-Faktoren}$\\*
	Dann $0 < 1 < 2 < 3 … $ in $K$\\*[4pt]
	\ul{Folge} Verschiedene natürliche Zahlen bleiben in $K$ verschieden.\\*
	%Falscher Pfeil, verbesserung kommt noch
	Fasse $\N$ als Teilmenge von $K$ auf.\\*
	\desc{Dann}{$\ds \Z = \left\lbrace a - b \mid a, b \in N \right\rbrace \subseteq K$\\*[4pt]
	$\ds \Q = \left\lbrace \frac{a}{b} \mid a, b \in N\right\rbrace \subseteq K$}
	Insbesondere ist $K$ unendlich.\\*[8pt]
	z.B. hat $F_z$ keine Anordnung.

\sS{Definition Absolutbetrag}
	Sei $K$ ein angeordneter Körper mit $a \in K$\\*
	Der Absolutbetrag von $a$ ist definiert als \\*
	$|a| = \left\lbrace \begin{array}{rcr}
	a&\text{ wenn }&a > 0\\*
	-a&\text{ wenn }&a < 0
	\end{array}\right.$

\sS{Satz (Dreiecksungleichung)}
	Sei $K$ ein angeordneter Körper $a, b, c \in K$\\*
	Dann gilt:
	\begin{enumerate}
	\item{$a = 0$ wenn $|a| = 0$}
	\item{$-|a| \leq a \leq |a|$}
	\item{Dreiecksungleichung: $\ds |a + b| \leq |a| + |b|$}
	\item{untere Dreiecksungleichung: $\ds |a - b| \geq |a| - |b|$}
	\end{enumerate}
\bew
	\begin{enumerate}
	\item{klar.}
	\item{wenn $a \geq 0$:\\*
	$|a| \geq 0$\\*
	$\Rarr\ -|a| \leq 0 \leq a \leq |a|$\\*
	wenn $a \leq 0$:
	$-|a| \leq a \leq 0 \leq |a|$}
	\item{Es gilt: $-|a| \leq a \leq |a|$, $-|b| \leq b \leq |b|$\\*
	wenn $a + b \geq 0$\\*
	$|a + b| = a + b \leq |a| + b \leq |a| + |b|$
	wenn $a + b < 0$\\*
	$|a + b| = -(a + b) = (-a) + (-b) \leq |a| + |b|$}
	\item{$(a - b) + b = a$\\*
	$\Rarr\ |a| = |(a - b) + b| \leq |a - b| + b$\\*
	$|a - b| \leq |a - b|$} % ??? Noch mal gegenlesen, mathematisch schwierig.
	\end{enumerate}

\sS{Satz Bernoulli'sche Ungleichungen}
	Sei $K$ ein angeordneter Körper $a, b \in K, a > -1$ und $n \in \N \{0, 1, 2, 3, 4, …\}$.\\*
	Dann gilt:
	$$(1 + a)^n \geq 1 + n \cdot a$$
	Beweis durch vollständige Induktion:\\*
	\ind{
		$$n = 0\\*
		(1 + a)^0 = 1 = 1 + 0 \cdot a$$
	}{
		Annahme:\\*
		$$(1 + a)^{n + 1} = (1 + a)(1 + a)^n \geq (1 + a)(1 + n \cdot a)$$\\*
		weil $1 + a > 0$\\*
		$= 1 + a + n \cdot a + n \cdot a^2$\\*
		$= 1 + (n + 1) \cdot a + n \cdot a^2$\\*
		weil $a^2 \geq 0\ \Rarr\ n \cdot a^2 \geq 0$
	}

\sS{Definition Beschränktheit}
	Sei $K$ ein angeordneter Körper, $M \subseteq K$ eine Teilmenge, $a \in K$.\\*
	\begin{enumerate}
	\item{$M \leq a$ bedeutet: $x \leq a$ für jedes $x \in M$}
	\item{$a$ heißt ",obere Schranke"' von $M$, wenn $M \leq a$.\\*
		$a$ heißt ",untere Schranke"' wenn $M \geq a$}
	\item{$M$ heißt nach oben beschränkt wenn $M$ eine obere Schranke hat.\\*
	Analog: nach unten beschränkt wenn $M$ eine untere Schranke hat.}
	\item{$a$ heißt Maximum von $M$, wenn $M \leq a$ \ul{und} $a \in M$. $a = max(M)$\\*
		$a$ heißt Minimum von $M$, wenn $M \geq a$ \ul{und} $a \in M$. $a = min(M)$}
	\end{enumerate}
\bew
	Sei $a, b \in M$\\*
	$M \leq a, M \leq b$\\*
	Dann $b \leq a$ und $b \leq a\ \Rarr\ a = b$ \qed
\bsp
	$K = \Q$
	\begin{enumerate}
	\item{$M = \N$\\*
	Sei $a \in \Q$\\*
	\desc{$a \leq N$}{$\equ a \leq n$ für alle $n \in N$\\*
	$\equ a \leq 1$}
	Wenn $N$ nach unten beschränkt $1 = min(N)$}
	\item{$M = \left\lbrace -\frac{1}{n} | n \in \N \right\rbrace \qquad 0 \notin M$
	\begin{tabbing}
	$-1 = min(M)$ \= $\Rarr\ M$ ist nach unten beschränkt.\\*
	$M \leq 0$	\> $\Rarr\ M$ ist nach oben beschränkt.\\*
	\end{tabbing}
	$M$ hat kein Maximum.\\*
	Sei $a \in M$ dann $a = -\frac{1}{n}, n \in N, -\frac{1}{n + 1} \in M$\\*
	$n + 1 > n\ \Rarr\ \frac{1}{n + 1} < \frac{1}{n}\ \Rarr\ -\frac{1}{n + 1} > -\frac{1}{n}$\\*
	$M \nleq -\frac{1}{n}$ $a$ ist keine obere Schranke.}
	\item{$M = \left\lbrace -\frac{1}{n} \mid n \in \N\right\rbrace \cup \{ 0 \}$\\*
	$min(M) = -1$\\*
	$max(M) = 0$}
	\item{$M = \emptyset$ hat weder ein $min(M)$ noch ein $max(M)$\\*
	Jedes $a \in \Q$ erfüllt $a \leq M$ und $M \leq a$}
	\end{enumerate}

\sS{Satz}
	\begin{enumerate}
	\item{Sei $K$ ein angeordneter Körper.\\*
	Wenn $M$ endlich und nicht leer, dann hat $M$ auch ein $max$ und ein $min$}
	\item{Wohlordnungsprinzip\\*
	Jede nicht leere Teilmenge $M \in \N$ hat ein Minimum.}
	\end{enumerate}
\bew
	\begin{enumerate}
	\item{klar.}
	\item{$M$ ist nicht leer, wähle $n \in M$\\*
	$\{1, 2, 3, 4, 5, … n\}$, endlich aber nicht leer.\\*
	Dann $min(\{1, 2, 3, 4, 5, … n\} \cap M) = min(M)$ \qed}
	\end{enumerate}

\sS{Definition Infimum Supremum}
	Sei $K$ ein angeordneter Körper und $M \subseteq K$, $a \in K$\\*
	$a$ heißt kleinste obere Schranke von $M$ oder Supremum.
	\begin{enumerate}
	\item{$M \leq a$}\\*[6pt]
	und
	\item{kein $b \in K$ mit $b < a$ erfüllt $M \leq b$}
	\end{enumerate}
	$a$ ist größte untere Schranke oder Infimum vom $M$, wenn
	\begin{enumerate}
	\item{$a \leq M$}\\*[6pt]
	und
	\item{Kein $b \in M$ mit $a < b$ erfüllt $b \leq M$}
	\end{enumerate}
\notat{$a = sup(M)$\\*$a = inf(M)$}
\bem
	Wenn $a = max(M)\ \Rarr\ a = sup(M)$ 
\bew
	Sei $a, b \in M$ und $a \nleq b$\\*
	$\Rarr\ M \nleq b\ \Rarr\ a$ ist Supremum
\bem
	Wenn ein Supremum existiert, ist es eindeutig.
\bew
	$a, b$ sind Supremum von $M$\\*
	$M \leq a$, $M \leq b\ \Rarr\ a \leq b $ und $b \leq a\ \Rarr\ a = b$ \qed
\bsp
	$sup(\{ -\frac{1}{n} \mid n \in \N \})$