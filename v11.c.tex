%v11
\item[2]{$h: \R \to \R, h(x) = [x]$}\\
% Stufenfunktion Diagramm
h ist monoton wachsend, aber nicht streng monoton.\\
Monoton wachsend: $x < y \Rarr [x] < [y]$\\
$x < y \not\Rarr [x] < [y]$\\
z. B.: $1,2 < 1,3 , [1,2] = 1 = [1,3] $}
\item[3]{Exponentialfunktion\\
$exp: \R \to \R, exp(x)= \ds\sum_{k=0}^{\infty} \frac{x^k}{k!}$\\
Ist streng monoton wachsend.\\
\bew
\begin{enumerate}
\item{$exp(0) = 1 + \frac{0}{1!} + \frac{0}{2!} + ... = 1$}
\item{Sei $a > 0$\\
$exp(a) = = 1 + \frac{a}{1!} + \frac{a}{2!} + ... > 1$}
\item{Sei $a > 0 exp(-a) \cdot exp(a) = exp(-a + a) = exp(0) = 1$\\
$\Rarr exp(-a) = \frac{1}{exp(a)} \Rarr 0 < exp(a) < 1$\\
%das kann ich nicht lesen... (iPhone Bild)
$exp(b) > 0$ für alle $b \in \R$}
\item{Sei $a > b$\\
$exp(a) = exp(a - b + b) = exp(a - b) \cdot exp(b)$ %overbraces über exp(a - b) => {> 1} über exp(b) => {> 0}
> exp(b) $\Rarr$ exp streng monoton wachsend \qed }
\end{enumerate}
}
%
% S 6.0 - 6.1 inkl bsp und bew.
%
% Nach der Pause...
%
\sS{Satz}
Die Exponentialfunktion $exp: \R \to \R$ ist stetig.\\
\bew
Verwende nur:
\begin{itemize}
\item{Funktionalgleichung: $exp(x + y) = exp(x) \cdot exp(y)$}
\item{exp ist streng monoton wachsend}
\item{exp(0) = 1}
\end{itemize}
\subsection*{Behauptung}
Für jedes $\epsilon > 0$ gibt es ein $n \in \N$ mit $exp(\frac{1}{n}) < 1 + \epsilon$\\
%Graf
Angenommen, $exp(\frac{1}{n}) \geq 1 + \epsilon$\\
Dann $exp(1) = \frac{1}{n} + ... \frac{1}{n}$ % underbrace unter den Brüchen {n}
\phantom{Dann $exp(1) $} $= exp(\frac{1}{n}) + ... + exp(\frac{1}{n}) = exp(\frac{1}{n})^n$ \\
\phantom{Dann $exp(1) $} $ \geq (1 + \epsilon)^n \geq 1 + n \epsilon $ %(Pfeil auf letztes geq Bernoulli)
\\
$exp(1) \geq 1 + n \epsilon$\\
$n \leq \frac{exp(1) - 1}{\epsilon}$\\
Das gilt nur für endliche viele $n \in \N$\\
$Rarr$ Beh.\\
\underline{Zeige:} exp ist stetig an 0. Gegeben sei $\epsilon > 0, OE ? \epsilon < 1$\\
Wähle $n \in \N$ mit $exp(\frac{1}{n}) < 1 + \epsilon$\\
$$Rarr exp(-\frac{1}{n}) = exp(\frac{1}{n})^{-1} < \frac{1}{1 + \epsilon} = \frac{1 - \epsilon}{(1+\epsilon)(1-\epsilon)} = \frac{1-\epsilon}{1 - \epsilon^2} > 1 - \epsilon$$\\
Sei $\delta \frac{1}{n}$\\
Sei $y \in \R, |0 - y| < \delta = \frac{1}{n}$\\
$|y| < \frac{1}{n}$ d.h.\\
$-\frac{1}{n} < y < \frac{1}{n}$\\
exp streng monoton wachsend.\\
$$Rarr 1 - \epsilon < exp(-\frac{1}{n}) < \exp(y) < exp(\frac{1}{n}) < 1 + \epsilon$$\\
$Rarr |exp(y) - exp(0)| < \epsilon$ Also exp stetig in 0\\
\underline{Zeige:} exp ist eine stetig in $x \in \R$. Gegeben sei \epsilon > 0\\
Sei $y = x + h$, $|h| < \delta$ ($\delta$ noch zu wählen)
$$|exp(y) - exp(x)| = |exp(x + h) - exp(x)| = |exp(x) \cdot exp(h) - exp(x)| = exp(x) \cdot exo(h) -1$$\\
$|exp(y) -exp(x) | < \epsilon$\\
$$\Leftrightarrow exp(x) \cdot |exp(h) - 1| < \epsilon \Leftrightarrow exp(h) - 1 < \frac{\epsilon}{exp(x)} = \epsilon '$$\\
Weil exp stetig in 0 ist gibt es ein $\delta > 0$ mit $|h| < \delta \Rarr |exp(h) -1| < \frac{\epsilon}{exp(x)}$\\
$Rarr$ exp ist stetig in x \qed\\
%
%
%% Letze Tafel
\Bew{mit Folgenstetigkeit}
Sei $x_n \to x$ für $n \to \infty$\\
mit $x_n \in D$\\
$f, g$ stetig $Rarr f(x_n) \to f(x)$ \\
\phantom{$f, g$ stetig $Rarr$} $g(x_n) \to g(x)$\\
$\Rarr f(x_n) + g(x_n) \to f(x) + g(x)$
$\phantom{\Rarr} f(x_n) \cdot g(x_n) \to f(x) \cdot g(x)$\\
Wenn also $f(x) \neq 0$\\
$f(x_n)^{-1} \to f(x)^{-1}$\\
$\Rarr f + g, f) \cdot g, \frac{1}{f}$ stetig in x \qed